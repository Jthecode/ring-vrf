\def\eprintsmallskip{\smallskip}{}%
%
Anonymized ring VRFs are ring signatures that prove correct evaluation
of some authorized signer's PRF while hiding the specific signer's
identity within some set of possible signers, known as the ring.
% \eprint{We propose ring VRFs as a natural fulcrum around which a diverse array of zkSNARK circuits turn, making them an ideal target for optimization and eventually standards.}{}

\eprintsmallskip
We demonstrate a reusable {\em zero-knowledge continuation} technique,
which works by adjusting a Groth16 trusted setup to hide public inputs
when rerandomizing the Groth16.  We then build ring VRFs that amortize
expensive ring membership proofs across many ring VRF signatures.
%
Incredibly, our ring VRF needs only eight $\mathcal{E}_1$ and two
$\mathcal{E}_2$ scalar multiplications, making it the only ring signature
with performance competitive with group signatures.

\eprintsmallskip
Ring VRFs produce a unique identity for any give context but remain
unlinkable between different contexts.  These unlinkable but unique
pseudonyms provide a far better balance between user privacy and service
provider or social interests than attribute based credentials like IRMA.

\eprintsmallskip
Ring VRFs support anonymously rationing or rate limiting resource
consumption that winds up vastly more efficient than purchases via money-like protocols.
