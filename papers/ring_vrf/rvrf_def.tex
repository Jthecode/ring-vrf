
\subsection{rVRF-AD security}
\label{sec:rvrf_def}

Ring VRFs are firstly ring signatures broadly interpreted, in that they
prove an associated public key lies inside some commitment \comring to
the plausible signer set,
 which anyone could construct from this set of public keys.
At the same time, ring VRFs prove correct output of a PRF keyed by
the signer's actual secret key, and evaluated on a supplied message \msg,
 which then links ring VRF signatures sharing the same \msg.

A {\em ring verifiable random function with auxiliary data} (rVRF-AD)
consists of the algorithms of a VRF-AD-KC, except with
 \compk and \openpk renamed to \comring and \openring,
 plus one additional algorithm:
\begin{itemize}
\item $\rVRF.\CommitRing : \ctx \mapsto \comring$ takes a set \ctx of
 public keys and returns a public key set commitment \comring.
\end{itemize}

\def\rSign{\Sign}
\def\rVerify{\Verify}

In this, we have renamed the commitment and opening to avoid confusion
when we build a rVRF-AD from a VRF-AD-KC.  This fresh notation leaves
$\rVRF.\KeyGen$ and $\rVRF.\Eval$ untouched, but
 changes the other methods' signatures:
\begin{itemize}
\item $\rVRF.\CommitKey : (\pk,\ctx) \mapsto (\comring,\openring)$
\item $\rVRF.\OpenKey : (\comring,\openring) \mapsto \pk$
\item $\rVRF.\rSign : (\sk,\openring,\msg,\aux) \mapsto \sigma$
\item $\rVRF.\rVerify : (\comring,\msg,\aux,\sigma) \mapsto \Out \, \lor \perp$
\end{itemize}

We extend the VRF-AD-KC commitment correctness condition for \CommitRing:

\begin{definition}
We say rVRF satisfies {\em ring commitment correctness} if
commitment correctness holds, and also $\rVRF.\CommitRing$ is 
 compatible with $\rVRF.\CommitKey$ in that
  $\rVRF.\CommitRing(\ctx) = \rVRF.\CommitKey(\pk,\ctx).0$.
\end{definition}

We lack anonymity against full key exposure ala
 \cite[pp. 6 Def. 4]{cryptoeprint:2005:304} of course, due to the VRF output,
but instead demand a weaker anonymity condition similar to
 \cite[pp. 5 Def. 3]{cryptoeprint:2005:304}:

\begin{definition}\label{def:rvrf_sign_oracle}
We let \ora{Sign} denote a ring signature CMA oracle, meaning
\begin{itemize}
\item $\ora{Sign}(\oramsg{keygen})$ creates and logs a fresh key pair
 $(\pk,\sk) \leftarrow \KeyGen$ and adds $\pk$ to the set $\ctx_0$, and then returns \pk.
\item $\ora{Sign}(\comring,\openring,\msg,\aux)$ returns
 $\Sign(\sk,\openring,\msg,\aux)$, provided it logged $(\OpenKey(\comring,\openring),\sk)$ previously.
\end{itemize}
\end{definition}

\begin{definition}
We say \rVRF satisfies {\em ring anonymity} if
any PPT adversary \adv has an advantage only
 negligible in $\secparam$ to win the game:
\begin{itemize}
\item[]
 Initially \adv outputs a message \msg, associated data \aux,
 two distinct public keys $\pk_0,\pk_1 \in \ora{Sign}.\ctx_0$ created by \ora{Sign},
 and a ring $\ctx \subset \ora{Sign}.\ctx_0$ containing $\pk_0,\pk_1$.
 Set $\comring = \CommitRing(\ctx)$.
 Next the challenger chooses $j=0$ or $j=1$ and gives
  \adv a signature $\sigma = \rSign(\sk_j,\openring,\msg,\aux)$ with $\OpenKey(\comring,\openring) = \pk_j$.
 %
 \adv calls \ora{Sign} of Definition \ref{def:rvrf_sign_oracle} throughout,
 except \adv loses if they ever query $\ora{Sign}(\comring',\openring',\msg,\cdot)$
 on \msg for some $\comring',\openring'$ with $\OpenKey(\comring',\openring') \in \{ \pk_0, \pk_1 \}$.
 Finally \adv guesses $j$ and wins if correct.
\end{itemize}
\end{definition}

We similarly want a ring uniqueness that limits adversaries who know secret keys,
as well as a ring unforgeability resembling \cite[pp. 7 Def. 7]{cryptoeprint:2005:304}. % their definition appears broken

\begin{definition}
We say \rVRF satisfies {\em ring uniqueness} (resp. ring unforgeability)if
any PPT adversary \adv has an advantage only
 negligible in $\secparam$ to win the game:
\begin{itemize}
\item[]
 \adv calls \ora{Sign} of Definition \ref{def:rvrf_sign_oracle} throughout,
 but also creates its own keys freely.
 %
 Finally \adv outputs a ring $\ctx$ as well as
 $k + |\ctx \setminus \ora{Sign}.\ctx_0|$ valid ring VRF signatures,
  each with distinct outputs,    % $\sigma_i$
 for $\ctx$ on a message \msg (resp. and also associated data \aux).
 \adv wins if they invoked $\ora{Sign}$ strictly fewer than $k$ times
 on $\msg$ (respt. and also $\aux$), and
  distinct $i$ with $\pk_i \in \ctx \cap \ora{Sign}.\ctx_0$.
\end{itemize}
\end{definition}
 
Any ring VRF becomes a non-anonymized VRF whenever
 the ring becomes a singleton $\ctx = \{ \pk \}$ of course.
In doing so this, our ring uniqueness reduces to
 the seperate unforgeability and uniqueness games for VRF-AD,
meaning our uniqueness with only a trivial key commitment.

We reuse the VRF-AD-KC pseudo-randomness definition for ring VRFs
because pseudo-randomness is strongest for singleton rings, i.e. $|\ctx| = 1$.

\smallskip 

Although our applications could mostly ignore key multiplicity. 
we briefly mention AML/KYC applications in \S\ref{subsec:AML_KYC},
which demands suspects prove non-involvement using ring VRFs.

\begin{definition}\label{def:rvrf_exculpability}
We say \rVRF is {\em exculpatory} if we have an efficient algorithm
for equivalence of public keys, but a PPT adversary \adv cannot
find non-equivalent public keys $\pk_0,\pk_1$ with colliding VRF outputs.
% (perfectly or computationally)
% (either ever or with advantage negligible advantage in $\secparam$)
\end{definition} 



