
\section{Security of Ring VRFs}

In this section, we model the security of ring VRF in both standard model and UC model. We show that our ring VRF protocol is UC secure but we also want to define the security of ring VRF in the standard model for potential protocols which may not be shown secure in the UC-model. 

\subsection{Standard model}
\label{subsec:rvrf_games}

We briefly give security games for a {\em ring verifiable random
function with auxiliary data} rVRF-AD constructions, which broadly
resemble existing VRF or ring signature definitions.  In this,
we include the full key commitment procedure and auxiliary data \aux,
as they impact implementers and applications, but we suppress multiple inputs for brevity.

% We need some basic correctness conditions of course.

\begin{definition}
We say an rVRF-AD satisfies {\em evaluation correctness} if
$(\pk,\sk) \leftarrow \KeyGen$ implies
\def\tmpC{\Eval(\sk,\msg) = \Verify(\pk,\msg,\aux,\Sign(\sk,\msg,\aux))}
\eprint{$\tmpC$, succeeds}{$$ \tmpC, $$}
and also {\em ring commitment correctness} if
$\pk \in \ring$ implies both
\def\tmpA{\OpenRing(\CommitRing(\ring,\pk)) = \pk}
\def\tmpB{\CommitRing(\ring) = \CommitRing(\ring,\pk).0} 
\eprint{$$ \tmpA \quad\textrm{and}\quad \tmpB \mathperiod $$}%
{$$ \begin{aligned}
   \tmpA \qquad\textrm{and} \\
   \tmpB \mathperiod
\end{aligned} $$}
\end{definition}

We lack anonymity against full key exposure ala
 \cite[pp. 6 Def. 4]{cryptoeprint:2005:304} of course, due to the VRF output,
but instead demand a weaker anonymity condition similar to
 \cite[pp. 5 Def. 3]{cryptoeprint:2005:304}:

\begin{definition}\label{def:rvrf_sign_oracle}
We let \ora{Sign} denote a ring signature CMA oracle, meaning
\begin{itemize}
\item $\ora{Sign}(\oramsg{keygen})$ \, 
 creates a fresh key pair $(\pk,\sk) \leftarrow \KeyGen$, logs it, and
 adds $\pk$ to a set $\ring_0$ it maintains, and then returns \pk.
\item $\ora{Sign}(\comring,\openring,\msg,\aux)$ \,
 returns the ring VRF signature $\Sign(\sk,\openring,\msg,\aux)$,
 provided it logged $(\pk,\sk)$ with $\pk = \OpenKey(\comring,\openring)$ previously.
\end{itemize}
\end{definition}

\begin{definition}
We say \rVRF satisfies {\em ring anonymity} if
any PPT adversary \adv has an advantage only
 negligible in $\secparam$ to win the game:
\begin{itemize}
\item[]
 Initially \adv outputs a message \msg, associated data \aux,
 two distinct public keys $\pk_0,\pk_1 \in \ora{Sign}.\ring_0$ created by \ora{Sign},
 and a ring $\ring \subset \ora{Sign}.\ring_0$ containing $\pk_0,\pk_1$.
 Set $\comring = \CommitRing(\ring)$.
 Next the challenger chooses $j=0$ or $j=1$ and gives
  \adv some signature $\sigma = \rSign(\sk_j,\openring,\msg,\aux)$ with $\OpenKey(\comring,\openring) = \pk_j$.
 %
 \adv calls \ora{Sign} of Definition \ref{def:rvrf_sign_oracle} throughout,
 except the adversary \adv loses if they ever query $\ora{Sign}(\comring',\openring',\msg,\cdot)$
 on \msg for some $\comring',\openring'$ with
 \def\tmp{\OpenKey(\comring',\openring') \in \{ \pk_0, \pk_1 \}}
 \eprint{$\tmp$.}{$$ \tmp \mathperiod $$}
 Finally \adv guesses $j$ and wins if correct.
\end{itemize}
\end{definition}

We similarly want a ring uniqueness that limits adversaries who know secret keys,
as well as a ring unforgeability resembling \cite[pp. 7 Def. 7]{cryptoeprint:2005:304}. % their definition appears broken

\begin{definition}
We say \rVRF satisfies {\em ring uniqueness} (resp. ring unforgeability)if
any PPT adversary \adv has an advantage only
 negligible in $\secparam$ to win the game:
\begin{itemize}
\item[]
 \adv calls \ora{Sign} of Definition \ref{def:rvrf_sign_oracle} throughout,
 but also creates its own keys freely.
 %
 Finally \adv outputs a ring $\ring$ as well as
 $k + |\ring \setminus \ora{Sign}.\ring_0|$ valid ring VRF signatures,
  each with distinct outputs,    % $\sigma_i$
 for $\ring$ on a message \msg (resp. and also associated data \aux).
 \adv wins if they invoked $\ora{Sign}$ strictly fewer than $k$ times
 on $\msg$ (resp. and also $\aux$), and
  distinct $i$ with $\pk_i \in \ring \cap \ora{Sign}.\ring_0$.
\end{itemize}
\end{definition}
 
Any ring VRF becomes a non-anonymized VRF whenever
 the ring becomes a singleton $\ring = \{ \pk \}$ of course.
We inherit ring VRF output properties from non-anonymized VRFs
because they wind up strongest for singleton rings, i.e. $|\ring| = 1$.
These include 
 residual pseudo-randomness \cite[Def. VRF (3) \S3.2, pp. 4]{vrf_micali} and
 residual unpredictability \cite[Def. VUF (3) \S3.2, pp. 5]{vrf_micali}
  (also \cite[Def. 4, pp. 8]{agg_dkg}).
% as well as collision resistance ??? and pseudo-randomness \cite{thin_vrf}.  % TODO





