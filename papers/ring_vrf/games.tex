
\section{{r}VRF-AD security}
\label{sec:rvrf_games}

% We need some basic correctness conditions of course.

\begin{definition}
We say an rVRF-AD satisfies {\em evaluation correctness} if
$(\pk,\sk) \leftarrow \KeyGen$ implies
$$ \Eval(\sk,\msg) = \Verify(\pk,\msg,\aux,\Sign(\sk,\msg,\aux)), $$  % succeeds
and also {\em ring commitment correctness} if
$\pk \in \ring$ implies both 
$$ \begin{aligned}
  \OpenRing(\CommitRing(\ring,\pk)) = \pk \qquad\textrm{and} \\
  \CommitRing(\ring) = \CommitRing(\ring,\pk).0  \mathperiod
\end{aligned} $$
\end{definition}

We lack anonymity against full key exposure ala
 \cite[pp. 6 Def. 4]{cryptoeprint:2005:304} of course, due to the VRF output,
but instead demand a weaker anonymity condition similar to
 \cite[pp. 5 Def. 3]{cryptoeprint:2005:304}:

\begin{definition}\label{def:rvrf_sign_oracle}
We let \ora{Sign} denote a ring signature CMA oracle, meaning
\begin{itemize}
\item $\ora{Sign}(\oramsg{keygen})$ \, 
 creates a fresh key pair $(\pk,\sk) \leftarrow \KeyGen$, logs it, and
 adds $\pk$ to the set $\ring_0$, and then returns \pk.
\item $\ora{Sign}(\comring,\openring,\msg,\aux)$ \,
 returns the signature $\Sign(\sk,\openring,\msg,\aux)$,
 provided it logged $(\pk,\sk)$ with $\pk = \OpenKey(\comring,\openring)$ previously.
\end{itemize}
\end{definition}

\begin{definition}
We say \rVRF satisfies {\em ring anonymity} if
any PPT adversary \adv has an advantage only
 negligible in $\secparam$ to win the game:
\begin{itemize}
\item[]
 Initially \adv outputs a message \msg, associated data \aux,
 two distinct public keys $\pk_0,\pk_1 \in \ora{Sign}.\ring_0$ created by \ora{Sign},
 and a ring $\ring \subset \ora{Sign}.\ring_0$ containing $\pk_0,\pk_1$.
 Set $\comring = \CommitRing(\ring)$.
 Next the challenger chooses $j=0$ or $j=1$ and gives
  \adv some signature $\sigma = \rSign(\sk_j,\openring,\msg,\aux)$ with $\OpenKey(\comring,\openring) = \pk_j$.
 %
 \adv calls \ora{Sign} of Definition \ref{def:rvrf_sign_oracle} throughout,
 except \adv loses if they ever query $\ora{Sign}(\comring',\openring',\msg,\cdot)$
 on \msg for some $\comring',\openring'$ with
 $$ \OpenKey(\comring',\openring') \in \{ \pk_0, \pk_1 \} \mathperiod $$
 Finally \adv guesses $j$ and wins if correct.
\end{itemize}
\end{definition}

We similarly want a ring uniqueness that limits adversaries who know secret keys,
as well as a ring unforgeability resembling \cite[pp. 7 Def. 7]{cryptoeprint:2005:304}. % their definition appears broken

\begin{definition}
We say \rVRF satisfies {\em ring uniqueness} (resp. ring unforgeability)if
any PPT adversary \adv has an advantage only
 negligible in $\secparam$ to win the game:
\begin{itemize}
\item[]
 \adv calls \ora{Sign} of Definition \ref{def:rvrf_sign_oracle} throughout,
 but also creates its own keys freely.
 %
 Finally \adv outputs a ring $\ring$ as well as
 $k + |\ring \setminus \ora{Sign}.\ring_0|$ valid ring VRF signatures,
  each with distinct outputs,    % $\sigma_i$
 for $\ring$ on a message \msg (resp. and also associated data \aux).
 \adv wins if they invoked $\ora{Sign}$ strictly fewer than $k$ times
 on $\msg$ (resp. and also $\aux$), and
  distinct $i$ with $\pk_i \in \ring \cap \ora{Sign}.\ring_0$.
\end{itemize}
\end{definition}
 
Any ring VRF becomes a non-anonymized VRF whenever
 the ring becomes a singleton $\ring = \{ \pk \}$ of course.
We inherit ring VRF output properties from non-anonymized VRFs
because they wind up strongest for singleton rings, i.e. $|\ctx| = 1$.
These include 
 residual pseudo-randomness \cite[Def. VRF (3) \S3.2, pp. 4]{vrf_micali} and
 residual unpredictability \cite[Def. VUF (3) \S3.2, pp. 5]{vrf_micali}
  (also \cite[Def. 4, pp. 8]{agg_dgk}).
% as well as collision resistance ??? and pseudo-randomness \cite{thin_vrf}.  % TODO





