\subsection{Related Works}


\noindent\textbf{Security Models:} The unique ring signature framework \cite{URCframework} is the closest model to our ring VRF framework particularly in terms of the presence of a deterministic component known as the unique identifier for the signed message. This identifier remains constant for the same signed message even when the ring changes.  Essentially, the unique identifier in the unique ring signature model and the ring VRF evaluation value function equivalently in both models. However, a fundamental distinction lies in the treatment of this identifier. In our ring VRF model, we impose the requirement of pseudorandomness, as defined in \cite{ucvrf,praos}, on this unique identifier, even in the case of malicious parties.
This requirement is crucial for applications such as lotteries or leader elections where the unique identifier plays a privileged or reward-based role based on predefined conditions. Another definitional difference is that a ring VRF  signature not only prove the correctness of the evaluation value of an input but also signs an auxiliary data independent from the input. This property is needed for anonymous access mechanisms to prevent replay attacks because auxiliary data can be used to effectively bind the ring VRF signature to e.g., a TLS session.
The signature size of unique ring signature schemes scales either linearly \cite{URCframework,URCfc} or logarithmically  \cite{URCblockchainprivacy,URClattice} with the size of the ring. In contrast, our ring VRF constructions maintain a constant signature size while providing stronger security guarantees. Also our signing and verification algorithms show better asymptotic scalability compared to existing unique ring signatures because they operate with a constant-size  commitment to the ring.

Other related models are linkable ring signature \cite{ring_linkable,ring_linkablee} and traceable ring signature \cite{traceable07,traceable_sub}. Linkable ring signatures allows a third party to link whether two ring signatures of two inputs are signed by the same party in the same ring without revealing the identity. This type of linkability property is valuable in applications that impose restrictions on authentications, such as preventing double spending or multiple voting. Akin to ring VRF and unique ring signatures, if a signer signs the same message twice for the same ring and issuer, it becomes evident that both signatures are produced by the same party, although the specific party's identity remains secret. Both ring VRF and unique ring signature schemes have this property in a single context through the unique identifier for each party.
Differently than ring VRF, traceable ring signatures  disclose the identity of the signer when the signer generates two signatures for two different inputs within the same ring and from the same issuer.

Another related informal design is Semaphore \cite{Semaphore}, which also provides a "nullifier", unique per identity and context but anonymous, (akin to a ring VRF output in our formalism) along with a signature on a message. However, the security properties of Semaphore are not fully formalized, and our constructions distinguishes themselves by offering more efficient proving times and the potential for proof reuse.

Anonymous VRF \cite{anonymousVRF} is a special type of VRF designed to enable verification of the VRF output without dependence on the party's key.  Differently than ring VRF, the verification is executed with another public key which is generated from the public key of the party.  A crucial distinction lies in their  uniqueness definitions, as anonymous VRFs ensure the uniqueness of VRF outputs for each (updated) public key and input. Consequently, anonymous VRFs are not suitable for identity applications where the VRF output serves as a unique and anonymous identifier, as each updated public key generates a different VRF output.
Another notable difference is related to the pseudorandomness definition, which does not guarantee pseudorandomness even when the key belongs to a malicious party. This limitation can pose challenges in applications like consensus mechanisms as described in \cite{anonymousVRF}, making their use potentially infeasible.



\noindent\textbf{Commit and Prove SNARKs:} ZK Continuations are an example of the commit and prove approach \cite{LegoSNARK}, linking  in a way similar to the ccGroth16 construction from LegoSNARK \cite{LegoSNARK}. Our work extends this concept by formalizing the reuse of previously generated proofs through simple transformations while maintaining the zero-knowledge property. Our protocol $ \SpecialG $ is very similar to the ccGroth16 construction from LegoSNARK \cite{LegoSNARK} with the additional feature of providing an interface for rerandomizing previously generated proofs, all while preserving the zero-knowledge property.