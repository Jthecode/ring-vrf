\section{Protocol overview}
\label{sec:overview}

As a beginning, we introduce the ring VRF interface, give a simple
unamortized NIZK that realizes the ring VRF properties discussed later,
and discuss our later amortized NIZKs.

Any VRF needs a \KeyGen algorithm, which returns a secret key \sk and
associated public key \pk, as well as an evaluation algorithm
$\Eval : (\sk,\msg) \mapsto \Out$. % for which many constructions exist.
%
Although many constructions exist,
 ours resemble EC VRFs like \cite{nsec5,VXEd25519,draft-irtf-cfrg-vrf-10}.
% In other words,
% internally we prove a VUF output $\PreOut = \sk H_{\grE}(\msg)$,
% with a hash-to-curve $H_{\grE}$, so then applying a PRF $H'$ yields a
% VRF output $\Out = H'(\msg, h \PreOut)$ ala \cite[Prop. 1]{vrf_micali},
% using a key pair like $\pk = \sk \genG$ for a generator $\genG$.

At their simplest, ring VRFs' algorithms could operate directly
upon the signer set \ring:
\begin{itemize}
\item $\rVRF.\rSign : (\sk,\ring,\msg) \mapsto \sigma$ \,
    returns a ring VRF signature $\sigma$ for an input \msg.
\item $\rVRF.\rVerify : (\ring,\msg,\sigma) \mapsto \Out \, \lor \perp$ \,
    returns either an output $\Out$ or else failure $\perp$.
\end{itemize}

A ring VRF like this could be instantiated using a NIZK for a relation like
$$ \Setst{ \Out, \msg, \ring }{
    \eprint{
        (\pk,\sk) \leftarrow \KeyGen, \quad %   \textrm{ and }
        \pk \in \ring, \quad
        \Out = \PRF(\sk,\msg)
    }{
        \begin{aligned}
        & (\pk,\sk) \leftarrow \KeyGen \\
        & \pk \in \ring \\
        & \Out = \PRF(\sk,\msg)
        \end{aligned}
    }
} $$

% Although convenient for security arguments, % formalization
We would need time $O(|\ring|)$ in \rSign and \rVerify merely to read
this \ring argument.  Instead, ring signatures run asymptotically faster
by replacing the \ring argument with some set commitment to \ring,
roughly like what ZCash does.
\begin{itemize}
% \item $\rVRF.\CheckRing : \ring \mapsto \comring$ takes a set \ring of public keys and returns a public key set commitment \comring.
\item $\rVRF.\CommitRing : (\ring,\pk) \mapsto (\comring,\openring)$ \,
    returns a commitment for a set \ring of public keys, and
    optionally the operning \openring for some $\pk \in \ring$ as well.
\item $\rVRF.\OpenRing : (\comring,\openring) \mapsto \pk \, \lor \perp$ \,
    returns a public key \pk, provided \openring correctly opens
    the ring commitment \comring, or failure $\perp$ otherwise.
\end{itemize}

We thus replace the membership condition $\pk \in \ring$ in the above
relation and NIZK by the by the opening condition
$$ \exists \openring \textrm{\ s.t.\ } \pk = \OpenRing(\comring,\openring) \mathperiod $$
% $\pk = \OpenRing(\comring,\openring)$.
%
% $$ \pi_0 = \NIZK \Setst{ \Out, \msg, \comring }{
%     \begin{aligned}
%         (\pk,\sk) &\leftarrow \KeyGen,  \quad
%           \Out = \PRF(\sk,\msg)  \\
%         \exists \openring \textrm{\ s.t.\ }
%           \pk &= \OpenRing(\comring,\openring)  \\      
%     \end{aligned}
% } $$

As EC VRF does, our ring VRFs signs auxiliary data \aux too.%
\footnote{We suppress multiple input-output pairs until \S\ref{subsec:multi_io} below, but they work like in \cite{PrivacyPass}.}
% and support multiple messages $\msg_1,\ldots,\msg_n$ ala .
Our algorithms for a {\em ring verifiable random function
with auxiliary data} (rVRF-AD) therefore really look like
\begin{itemize}
\item $\rVRF.\rSign : (\sk,\openring,\msg,\aux) \mapsto \sigma$, \quad and
\item $\rVRF.\rVerify : (\comring,\msg,\aux,\sigma) \mapsto \Out \,\, \lor \perp$.
\end{itemize}

Although asymptotically faster, our opening \OpenRing invariably
winds up being extremely heavy inside a zkSNARK.
We solve this obstacle in \S\ref{sec:rvrf_cont} below by introducing
{\em zero-knowledge continuations}, a new zkSNARK technique built from
rerandomizable Groth16s and designed for composition and reuse.

In this, we should split the relation describing our ring VRF into
a VRF evaluation relation \Reval and a reusable or continuable relation
\Rring, which enforces our heavy $\pk = \OpenRing(\comring,\openring)$
condition.  Anonymity requires we rerandomize a Groth16 SNARK for \Rring
ala \cite[Theorem 3, Appendix C, pp. 31]{RandomizationGroth16}.
%
Yet, we must join aka pass \pk from \Rring to \Reval somehow, which
demands some hiding commitment \compk of course.
%
$$ \Rring = \Setst{ \compk, \comring }{
    \cdots \, \eprint{ \pk =}{} \OpenRing(\comring,\openring) \, \cdots
} $$

We discovered this becomes incredibly efficent if one actually uses
a Pedersen commitment to \sk in the Groth16 itself.

$$ \Rring' = \Setst{ \sk, \comring }{
    \eprint{
    (\pk,\sk) \leftarrow \KeyGen, \quad % \textrm{ and }
    \pk = \OpenRing(\comring,\openring) 
    }{
    \begin{aligned}
        &(\pk,\sk) \leftarrow \KeyGen, \\
        % \exists \openring \textrm{\ s.t.\ }
        &\pk = \OpenRing(\comring,\openring)  \\      
    \end{aligned}
    }
} $$

$$ \Reval = \Setst{ \Out, \msg, \aux, \pk }{ \Out = \PRF(\sk,\msg) } $$

In \S\ref{sec:rvrf_cont} we rerandomize $\compk = \pk + b \, K$
without reproving the Groth16 for \Rring, but instead requires
expanding the Groth16 trusted setup into what becomes zero-knowledge continuation.
%
In \S\ref{sec:pederson_vrf}, we introduce an extremely efficient NIZK
for \Reval, which also provides an essential proof-of-knowledge for \compk.

% We first however discuss security definitions in \S\ref{sec:???}.



\endinput





% We define ring VRFs in \S\ref{sec:rvrf_games} and \S\ref{sec:rvrf_uc_fun} below, but
Ring VRFs are firstly ring signatures broadly interpreted, in that they
prove an involved public key lies inside some commitment \comring to
the plausible signer set, known as the ring.
Anyone could compute \comring from this set of public keys.
%
At the same time, ring VRFs prove correct output of a PRF keyed by
the signer's actual secret key, and evaluated on a supplied message \msg,
which then links ring VRF signatures on the same \msg.

\smallskip
