\section{Protocol overview}
\label{sec:overview}

As a beginning, we introduce the ring VRF interface, give a simple
unamortized non-interactive zero-knowledge (NIZK) protocol that
realizes the ring VRF properties discussed later,
and give some intuition for our later amortization trick.

As VRFs do \cite{vrf_micali}, all ring VRFs need: 
\begin{itemize}
\item $\rVRF.\KeyGen : (1^{\lambda}) \mapsto (\sk, \pk)$ algorithm,
 which creates a random secret key \sk and associated public key \pk;

\item $\rVRF.\Eval : (\sk,\msg) \mapsto \Out$ which deterministically computes the VRF output \Out from a secret key \sk and a message \msg.
\end{itemize}
%
% Although many constructions exist,
Our \rVRF.\KeyGen and \rVRF.\Eval initially resemble EC VRFs like \cite{nsec5,VXEd25519,draft-irtf-cfrg-vrf-10}.
% In other words,
% internally we prove a VUF output $\PreOut = \sk H_{\grE}(\msg)$,
% with a hash-to-curve $H_{\grE}$, so then applying a PRF $H'$ yields a
% VRF output $\Out = H'(\msg, h \PreOut)$ ala \cite[Prop. 1]{vrf_micali},
% using a key pair like $\pk = \sk \genG$ for a generator $\genG$.
We demand pseudo-randomness properties from \Eval, which could mirror
\cite{vrf_micali} if desired.  We provide a UC definition resembling
\cite{praos,ucvrf} which handles adversarial keys better however.

% TODO: Should this text be moved elsewhere?
Ring VRFs differ from VRFs in that they do not expose a specific signer,
and instead prove the signer's key lies in some plausible signer set \ring,
 much like how ring signatures differ from signatures.
Ring VRFs differ from ring signatures in that they prove a VRF output \Out.
% and prove it corresponds to $\rVRF.\Eval$ for some plausible signer.

% As an instructive but insecure over simplification, 
At their simplest,
ring VRFs' other algorithms operate directly upon
 the alleged signer set \ring, like:
\begin{itemize}
\item $\rVRF.\rSign : (\sk,\ring,\msg) \mapsto \sigma$ \,
    returns a ring VRF signature $\sigma$ for an input \msg.
\item $\rVRF.\rVerify : (\ring,\msg,\sigma) \mapsto \Out \, \lor \perp$ \,
    returns either an output $\Out$ or else failure $\perp$.
\end{itemize}

After success, our verifier should be convinced that $\pk \in \ring$, that
$\Out = \rVRF.\Eval(\sk,\msg)$ for some $(\pk,\sk) \leftarrow \rVRF.\KeyGen$,
 and that $\Out$ is pseudo-random.
In other words, this simplified ring VRF could be instantiated by making
\rVRF.\Eval a pseudo-random (hash) function, and using a NIZK for a language like
$$ \Lrvrf = \Setst{ \Out, \msg, \ring}{
    \eprint{
        \exists (\pk,\sk) \leftarrow \rVRF.\KeyGen, \quad %   \textrm{ and }
        \pk \in \ring, \quad
        \Out = \rVRF.\Eval(\sk,\msg)
    }{
        \begin{aligned}
        & \exists (\pk,\sk) \leftarrow \rVRF.\KeyGen \\
        & \pk \in \ring \\
        & \Out = \rVRF.\Eval(\sk,\msg)
        \end{aligned}
    }
} $$
% TODO:  \PRF vs \rVRF.\Eval here??
% Although convenient for security arguments, % formalization
The zero-knowledge property of the NIZK ensures that our verifier learns nothing about the specific
signer, except that their key is in the ring and maps $\msg$ to $\Out$.
Importantly, pseudo-randomness also says that \Out is an identity
for the specific signer, but only within the context of \msg.

% \smallskip

Aside from proving an evaluation using \rVRF.\Eval, 
we always need \rVRF.\Sign and \rVRF.\Verify to sign some associated data \aux,
as otherwise the ring VRF signature become unmoored and permits replay attacks.
%
As an example, our identity protocol below in \S\ref{sec:app_identity}
yields the same ring VRF outputs each time the same user logs into the
same site, which suffers replay attacks unless \aux binds the
ring VRF signature to the TLS session.

Indeed, regular (non-anonymous) VRF uses always encounter similar tension
with VRF inputs \msg being smaller than full message bodies $(\msg,\aux)$.
As an example, Praos \cite{praos} binds their VRF public key together
with a second public key for another (forward secure) signature scheme,
with which they sign their \aux, the block itself.
%
An EC VRF should expose an \aux parameter which it hashes when computing
its challenge hashes.  Aside from saving redundant signatures, exposing
\aux avoids user key handling mistakes that create replay attacks.

Ring VRFs cannot so easily be combined with another signatures, which
makes \aux essential,%
\footnote{If ring VRFs authorized creating blocks in an anonymous Praos blockchain then \aux must include the block being created, or else others could steal their block production turn.}
but thankfully our ring VRFs expose \aux exactly like EC VRFs do in \S\ref{sec:pederson_vrf}.%
\footnote{We suppress multiple input-output pairs until \S\ref{subsec:multi_io} below, but they work like in \cite{PrivacyPass} too.}

% \smallskip

Taking into account only the $\rVRF$ interface described above, we would need time $O(|\ring|)$ in \rVRF.\rSign and \rVRF.\rVerify merely to read
this \ring argument though, which severely limits applications.  Instead,
ring signatures run asymptotically faster by replacing the \ring argument
with a set commitment to \ring, roughly like what ZCash does \cite{zcash_protocol}.
\begin{itemize}
% \item $\rVRF.\CheckRing : \ring \mapsto \comring$ takes a set \ring of public keys and returns a public key set commitment \comring.
\item $\rVRF.\CommitRing : (\ring,\pk) \mapsto (\comring,\openring)$ \,
    returns a commitment for a set \ring of public keys, and
    optionally the opening \openring for some $\pk \in \ring$ as well.
\item $\rVRF.\OpenRing : (\comring,\openring) \mapsto \pk \, \lor \perp$ \,
    returns a public key \pk, provided \openring correctly opens
    the ring commitment \comring, or failure $\perp$ otherwise.
\end{itemize}

We thus replace the membership condition $\pk \in \ring$ in the above
language and NIZK by the opening condition
$$ \exists \openring \textrm{\ s.t.\ } \pk = \rVRF.\OpenRing(\comring,\openring) \mathperiod $$
% $\pk = \OpenRing(\comring,\openring)$.
%
% $$ \pi_0 = \NIZK \Setst{ \Out, \msg, \comring }{
%     \begin{aligned}
%         \exists (\pk,\sk) &\leftarrow \KeyGen,  \quad
%           \Out = \PRF(\sk,\msg)  \\
%         \exists \openring \textrm{\ s.t.\ }
%           \pk &= \OpenRing(\comring,\openring)  \\      
%     \end{aligned}
% } $$

% \smallskip

Having these concerns in mind, we name our new notion \emph{ring verifiable function with additional data} (rVRF-AD); 
it has the same api as the \rVRF notion described above with the mention that signing and verifying are now defined as per below. Note that for simplicity and consistency,
we continue to use the prefix \rVRF:
\begin{itemize}
\item $\rVRF.\rSign : (\sk,\openring,\msg,\aux) \mapsto \sigma$, \quad and
\item $\rVRF.\rVerify : (\comring,\msg,\aux,\sigma) \mapsto \Out \,\, \lor \perp$.
\end{itemize}

Although an asymptotic improvement, our opening \rVRF.\OpenRing based condition invariably
still winds up being computationally expensive to prove inside a zkSNARK.
We solve this obstacle in \S\ref{sec:rvrf_cont} below by introducing
{\em zero-knowledge continuations}, a new zkSNARK technique built from
rerandomizable Groth16s \cite{Groth16} and designed for SNARK composition and reuse.

In order to achieve this, we split the language \Lrvrf into
a language \Leval for \rVRF evaluation and a language
\Lring for what we call a reusable or continuable SNARK; in turn, \Lring enforces our computationally 
expensive $\pk = \rVRF.\OpenRing(\comring,\openring)$ condition.  Anonymity requires we rerandomize a Groth16 SNARK for \Lring
ala \cite[Theorem 3, Appendix C, pp. 31]{RandomizationGroth16}.
%
Yet, we must connect together the NIZKs for the two languages  \Leval and \Lring. We do this by passing \pk from \Lring to \Leval, which
demands some hiding commitment \compk to \pk.

%
\def\tmpAA{\Out = \rVRF.\Eval(\sk,\msg)}%
\def\tmpBB{\textrm{\compk commits to\ \sk}}%
$$ \Leval = \Setst{ \Out, \msg, \aux, \compk }{
	\eprint{
	\tmpAA, \, \tmpBB
    }{
	\begin{aligned}
	&\tmpAA, \\
	&\tmpBB \\
	\end{aligned}
    }
} $$
\def\tmpAA{\textrm{\compk commits to\ }}%
\def\tmpBB{\rVRF.\OpenRing(\comring,\openring)}%
$$ \Lring = \Setst{ \compk, \comring }{
	\eprint{
	\tmpAA \pk = \tmpBB
	}{
    \begin{aligned}
	&\tmpAA \\
	&\, \pk = \tmpBB \\
    \end{aligned}
	}
} $$

We discovered this becomes incredibly efficient if one specializes
the original Groth16 SNARK construction:  An inner true Groth16 SNARK for $\Lring^\inner$
handles the secret key \sk directly via its public inputs, but
\sk and even \pk remain secret by transforming the trusted setup to have
a rerandomizable Pedersen commitment \compk outside this SNARK.
$$ \Lring^\inner = \Setst{ \sk, \comring}{
    \eprint{
    (\pk,\sk) \leftarrow \rVRF.\KeyGen, \, % \textrm{\,and }
    \pk = \rVRF.\OpenRing(\comring,\openring) 
    }{
    \begin{aligned}
        &(\pk,\sk) \leftarrow \rVRF.\KeyGen, \\
        % \exists \openring \textrm{\ s.t.\ }
        &\pk = \rVRF.\OpenRing(\comring,\openring)  \\      
    \end{aligned}
    }
} $$

Our zero-knowledge continuation in \S\ref{sec:rvrf_cont} rerandomizes
$\compk = \pk + b \, K$ without reproving the Groth16 SNARK for $\Lring^\inner$.
For this, the secret key \sk must be a public input of $\Lring^\inner$, and
the Groth16 trusted setup must be expanded by a multiple of
 the otherwise independent point $K$.
%
In \S\ref{sec:pederson_vrf}, we introduce an extremely efficient NIZK
for \Leval, which also provides an essential proof-of-knowledge for \compk.


\endinput


% We define ring VRFs in \S\ref{sec:rvrf_games} and \S\ref{sec:rvrf_uc_fun} below, but
Ring VRFs are firstly ring signatures broadly interpreted, in that they
prove an involved public key lies inside some commitment \comring to
the plausible signer set, known as the ring.
Anyone could compute \comring from this set of public keys.
%
At the same time, ring VRFs prove correct output of a PRF keyed by
the signer's actual secret key, and evaluated on a supplied message \msg,
which then links ring VRF signatures on the same \msg.

\smallskip
