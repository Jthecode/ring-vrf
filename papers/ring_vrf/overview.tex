\section{Protocol overview}
\label{sec:overview}

\newcommand\PRF{\primalgo{PRF}}

% vrf_def.tex
\def\compk{\ensuremath{\mathsf{compk}}\xspace}
\def\openpk{\ensuremath{\mathsf{oppk}}\xspace}
\newcommand\CommitKey{\primalgo{CommitKey}}
\newcommand\OpenKey{\primalgo{OpenKey}}

%rvrf_def.tex
\def\comring{\ensuremath{\mathsf{comring}}\xspace}
\def\openring{\ensuremath{\mathsf{opring}}\xspace}
\newcommand\CommitRing{\primalgo{CommitRing}}
\newcommand\OpenRing{\primalgo{OpenRing}}
\newcommand\piring{\ensuremath{\pi_{\mathtt{ring}}}\xspace}
\newcommand\pieval{\ensuremath{\pi_{\mathtt{eval}}}\xspace}

We discuss formal definitions for ring VRFs later in \S\ref{sec:rvrf_def}.
At an intuitive level though, ring VRFs are ring signatures that prove
correct evaluation of some PRF determined by the actual key pair used in signing.
In other words, ring VRFs look like a NIZK employing several algorithms,
a pseudo-random function \PRF, some secret key commitment \CommitKey,
and an opening algorithm \OpenRing for some set commitment algorithm \CommitRing.

$$ \pi_0 = \NIZK \Setst{ \mathsf{out}, \msg, \comring }{
    \exists \sk, \openring \textrm{\ s.t.\ } 
    \genfrac{}{}{0pt}{}{
        \CommitKey(\sk) = \OpenRing(\comring,\openring)
    }{ \textrm{ and }
        \mathsf{out} = \PRF(\sk,\msg)
    }
} $$

As a rule, all ring signatures work via zero-knowledge execution of
some such \OpenRing, but invariably this winds up being extremely heavy.
In \S\ref{sec:rvrf_cont} we resolve this major performance
obstacle by using a {\em zero-knowledge continuation} for \OpenRing,
meaning we split $\pi_0$ into some zkSNARK $\piring$ running \OpenRing that
permits cheap rerandomization, and a seperate PRF evaluation NIZK $\pieval$.

$$ \piring = \NIZK \Setst{ \compk, \comring }{
    \exists \openring \textrm{\ s.t.\ } 
    \compk = \OpenRing(\comring,\openring)
} $$

$$ \pieval = \NIZK \Setst{ \mathsf{out}, \msg, \compk }{
    \exists \sk \textrm{\ s.t.\ }
    \compk = \CommitKey(\sk) \textrm{ and }
    \mathsf{out} = \PRF(\sk,\msg)
} $$

In this, we nauturally employ Groth16 in $\piring$ for its rerandomization,
except our \compk must be cheaply reblinded too, and thus cannot itself
be purely a Groth16 public input, or even exist inside the Groth16.
Instead in \S\ref{sec:rvrf_cont}, we build zero-knowledge continuations
by expanding a Groth16 trusted setup with independent blinding factors
for Pedersen commitments, which then creates a new zkSNARK type whose
public inputs are cheaply malleable in exactly the desired way.
We then exploit the zero-knowledge continuation by wiring these
Pedersen commitments into another NIZK.  

It's clear our NIZKs $\piring$ and $\pieval$ above wound up underspecified
in how they handle the Pedersen commitment \compk.  Among other issues
they never explain the proof-of-knowledge required for \compk.
As a rule, we found typical NIZK notation cannot distinguish the Groth16
itself from these Pedersen commitments, and thus cannot describe the
wiring of zero-knowledge continuations properly.
We therefore describe only the base Groth16 in NIZK notation, and then
seperately explain how the zero-knowledge continuation wires up its
Pedersen commitments.

We shall explain this zero-knowledge continuation $\piring$ in detail
in \S\ref{sec:rvrf_cont}.  Yet first in \S\ref{subsec:vrf_pederson}
we introduce an extremely efficent instantiation for $\pieval$, which
also provides the required proof-of-knowledge for \compk.

We shall discuss several variations on $\piring$, as alterations impact
deployment dramatically.  Yet we know interesting variations on $\pieval$
as well.  As an example, if one employs hash functions for \CommitKey
and \PRF, and certian zkSNARKs for $\pieval$, then one obtains post-quantum
anonymity, although not post-quantum soundness.

