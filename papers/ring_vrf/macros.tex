
\renewcommand{\msg}{\ensuremath{\mathsf{input}}\xspace}
\renewcommand{\aux}{\ensuremath{\mathsf{ass}}\xspace}


\newcommand{\PedVRF}{\primalgo{PedVRF}} 


\newcommand\crs{\ensuremath{\mathit{crs}}\xspace}
\newcommand\crspk{\ensuremath{\mathit{crs}_{\mathit{pk}}}\xspace}
\newcommand\crsvk{\ensuremath{\mathit{crs}_{\mathit{vk}}}\xspace}


\newcommand\barx{\ensuremath{\bar{x}}\xspace}
\newcommand\bary{\ensuremath{\bar{y}}\xspace}
\newcommand\barz{\ensuremath{\bar{z}}\xspace}
\newcommand\baromega{\ensuremath{\bar{w}}\xspace}
\newcommand\baromegap{\ensuremath{\bar{w'}}\xspace}
\newcommand\relone{\ensuremath{\mathcal{R}_1}\xspace}
\newcommand\reltwo{\ensuremath{\mathcal{R}_2}\xspace}

\newcommand\baseL{L}
\newcommand\Lrvrf{\ensuremath{\baseL_{\mathtt{rvrf}}}\xspace}
\newcommand\Leval{\ensuremath{\baseL_{\mathtt{eval}}}\xspace}
\newcommand\Lring{\ensuremath{\baseL_{\mathtt{ring}}}\xspace}
\newcommand\Lfast{\ensuremath{\baseL_{\mathtt{fast}}}\xspace}
\newcommand\Lsk{\ensuremath{\baseL_{\mathtt{sk}}}\xspace}
\newcommand\Lpk{\ensuremath{\baseL_{\mathtt{pk}}}\xspace}

\newcommand\rrSNARK{\primalgo{Groth16}\xspace}
\newcommand\rrSNARKweak{\primalgo{Groth16/KZG}\xspace}

\newcommand\negl{\ensuremath{\mathsf{negl}}\xspace}
\newcommand\pieval{\ensuremath{\pi_{\mathtt{eval}}}\xspace}
\newcommand\piring{\ensuremath{\pi_{\mathtt{ring}}}\xspace}

\newcommand\pifast{\ensuremath{\pi_{\mathtt{fast}}}\xspace}
% \newcommand\pifastdot{\ensuremath{\dot{\pi}_{\mathtt{fast}}}\xspace}
\newcommand\pisk{\ensuremath{\pi_{\mathtt{sk}}}\xspace}
\newcommand\pipk{\ensuremath{\pi_{\mathtt{pk}}}\xspace}


%\newcommand{\PoK}{\ensuremath{\primalgo{PoK}}\xspace}
\newcommand{\SpecialG}{\ensuremath{\primalgo{SpecialG}}\xspace}
\newcommand{\ZKCont}{\ensuremath{\primalgo{ZKCont}}\xspace}
\newcommand{\Preprove}{\ensuremath{\primalgo{Preprove}}\xspace}
\newcommand{\Reprove}{\ensuremath{\primalgo{Reprove}}\xspace}
\newcommand{\Setup}{\ensuremath{\primalgo{Setup}}\xspace}
%\newcommand{\Gen}{\ensuremath{\primalgo{KeyGen}}\xspace}

\newcommand{\inner}{\mathtt{inner}}

\def\maybestack#1#2{\eprint{ #1, #2 }{
    \begin{aligned}
        &#1, \\
        % \exists \openring \textrm{\ s.t.\ }
        &#2  \\      
    \end{aligned}
}}

