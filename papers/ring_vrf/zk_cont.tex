
\newcommand{\handan}[2]{{\sout{#1}}\textcolor{blue}{#2}}
%\newcommand{\handan}[2]{{#1}{}}

\newcommand{\qhandan}[2]{{\uline{#1}}\textcolor{pink}{(#2)}}
%TO REMOVE ALL QUESTIONS
%\newcommand{\qhandan}[2]{{#1}{}}

\section{Zero-knowledge Continuations}
\label{sec:rvrf_cont}

\noindent In the following, we describe a NIZK for a relation $\rel$ where
$$\rel = \{(\bary, \barz; \barx, \baromega_1, \baromega_2):  (\bary, \barx; \baromega_1) \in \relone, (\barz, \barx; \baromega_2) \in \reltwo \},$$
and $\relone$, $\reltwo$ are NP relations. 
At a high level, this is based on the commit-and-prove methodology~\cite{Kilian1990UsesOR,CLOS02,LegoSNARK} 
as relations $\relone$ and $\reltwo$ have input $\barx$ in common and the respective 
proofs or arguments can and will make use of a commitment $X$ to $\barx$. However, our proposed NIZK has a more specific functionality: 
it is designed to efficiently re-prove membership for relation $\relone$ via a new technique which we call \emph{zero-knowledge continuation}. 
In practice, using a NIZK that ensures a zero-knowledge continuation for a 
subcomponent relation (i.e., in our case $\relone$) means one essentially needs to create only once an otherwise expensive proof for that subcomponent 
relation; the initial proof can later be re-used multiple times (just after inexpensive re-randomisations) 
while preserving knowledge soundness and zero-knowledge of the entire NIZK. 
Below, we formally define zero-knowledge continuation. In section~\ref{sec:rvrf_groth16} we instantiate it via a \emph{special(ised) 
Groth16} or \SpecialG, and finally, in section~\ref{subsec:rvrf_faster} we use it to instantiate our $ \rVRF.\Sign $ algorithm in Section \ref{sec:pederson_vrf} with fast amortised prover time. \\

\noindent In addition, the anonymity property of our ring VRF demands we not only finalise multiple times a component of the zero-knowledge 
continuation but also each time the result remains unlinkable to previous finalisations, meaning our ring VRF stays zero-knowledge 
even with a continuation component being reused. We formalise such a more general zero-knowledge property in 
section~\ref{sec:rvrf_groth16} and give an instantiation of our NIZK fulfilling such a property in section~\ref{subsec:rvrf_faster}. 
%Moreover, the anonymity property of a ring VRF demands we finalise the amortized ``continuation'' multiple
%times, with each time being unlinkable to the others, meaning our rVRF
%stays zero-knowledge even with the continuation being reused.


%\begin{definition}[ZK Continuations] A zero-knowledge continuation $\SpecialG_\rel$ consists of four algorithms 
%($\SpecialG_\rel.\Preprove$, $\SpecialG_\rel.\Reprove$, $\ldots$, $\SpecialG_\rel.\Verify$) such that:
%\begin{itemize}
%\item $\SpecialG_\rel.\Preprove : (\bar{y}, \bar{x}; \baromega_1) \mapsto (X,\pi)$ \,
%constructs a commitment $X$ and a proof $\pi$ for relation $\rel$ from a vector 
%of inputs $\bar{y}$ (called \em{transparent}), a vector of inputs $\bar{x}$ (called \em{opaque}), and witnesses $\baromega_1$.
%\item $\SpecialG_\rel.\Reprove : (X,\pi) \mapsto ((X',\pi'); b)$ \,
%finalises the commitment $X'$ and proof $\pi'$ and returns an opening $b$ for the commitment. 
%\item $\SpecialG_\rel.\Verify(\bar{y}; (X',\pi') )$ \, 
%verifies the 
%\end{itemize}
%%TO DO: add an algorithm to the $\SpecialG_\rel$ such that: Our \Verify needs a separate proof-of-knowledge for $X'$, 
%%the production of which requires knowledge of $\bar{x}$, and occurs in parallel to \Reprove.

%We define (white-box) knowledge soundness for zero-knowledge continuations
%exactly like for zero-knowledge proofs, but with the composition 
%$\Prove : (\bar{y}, \bar{x}; \baromega_1) \mapsto \Reprove(\Preprove(\bar{y}, \bar{x}; \baromega_1))[0]$
%as well as this additional proof-of-knowledge.
%Zero-knowledge however should hold even if \Reprove gets invoked multiple
%times upon the same \Preprove results, again even with the additional proof-of-knowledge.
%\end{definition} 

%TODO: Check notation consistancy with NIZK in preliminaries
\begin{definition}[ZK Continuations]
\label{def:zk_cont}
 A zero-knowledge continuation $\ZKCont$ for a relation $\relone$ with 
input $(\bary, \barx)$ and witness $\baromega_1$ is a tuple of efficient algorithms 
($\ZKCont.\Setup$, $\ZKCont.\Preprove$, $\ZKCont.\Reprove$, $\ZKCont.\VerifyCom$, $\ZKCont.\Verify$, $\ZKCont.\Sim$) 
such that for implicit security parameter $\lambda$,
\begin{itemize}

\item $\ZKCont.\Setup: (1^{\lambda}, \relone) \mapsto (\crs, \tw, \pp)$ a setup algorithm that on input the security parameter 
outputs a common reference string $\crs$, a trapdoor $\tw$ and a list $\pp$ of public parameters, 

%\item $\ZKCont.\Gen: (\crs, \relone) \mapsto (\crspk, \crsvk, \pp)$ \, 
%outputs a list $\pp$ of public parameters and a pair of proving key $\crspk$ and verification key $\crsvk$, 

\item $\ZKCont.\Preprove: (\crs, \bar{y}, \bar{x}, \baromega_1, \relone) \mapsto (X', \pi', b')$ \,
constructs commitment $X'$ from a vector of inputs $\bar{x}$ (called \emph{opaque}) and 
constructs proof $\pi'$ from vector of inputs 
$\bar{y}$ (called \emph{transparent}), $\bar{x}$ and vector of witnesses $\baromega_1$, and it 
also outputs $b'$ as the opening for $X'$,

\item $\ZKCont.\Reprove: (\crs, X', \pi', b', \relone) \mapsto (X, \pi, b)$ \,
returns a new commitment $X$ and proof $\pi$ and returns an opening $b$ for the commitment, 

\item $\ZKCont.\VerifyCom: (\pp, X, \barx, b) \mapsto 0/1$ \, 
verifies that indeed $X$ is a commitment to $\barx$ with opening (e.g., randomness) $b$ and 
outputs 1 if indeed that is the case and 0 otherwise,
 
\item $\ZKCont.\Verify: (\crs, \bar{y}, X, \pi, \relone) \mapsto 0/1$ \, outputs $1$ in case it accepts  and $0$ otherwise,

\item $\ZKCont.\Sim: (\tw, \bary, \relone) \mapsto (\pi, X)$ takes as input a simulation trapdoor $\tw$ and statement $(\bary, \barx)$ and returns 
arguments $\pi$ and $X$,
\end{itemize}
and satisfies perfect completeness for $\Preprove$ and, for $\Reprove$, it satisfies knowledge soundness and zero-knowledge as defined below:\\
%TO DO: Re-write this: We define (white-box) knowledge soundness for zero-knowledge continuations
%exactly like for zero-knowledge proofs, but with the composition 
%$\Prove : (\bar{y}, \bar{x}; \baromega_1) \mapsto \Reprove(\Preprove(\bar{y}, \bar{x}; \baromega_1))[0]$
%as well as this additional proof-of-knowledge.
%Zero-knowledge however should hold even if \Reprove gets invoked multiple
%times upon the same \Preprove results,
%again even with the additional proof-of-knowledge.

\noindent We define perfect completeness for $\Preprove$ and $\Reprove$ algorithms separately and in the most general way possible,
(i.e., with inputs supplied by the adversary where possible).  

\noindent \textbf{Perfect Completeness for $\Preprove$} For all $\lambda \in \mathbb{N}$, for every $(\bary, \barx; \baromega_1) \in \relone$:
\begin{align*}
\mathit{Pr} (& \ZKCont.\Verify(\crs, \bar{y}, X, \pi, \relone) = 1 \ \wedge \ \ZKCont.\VerifyCom (\pp, X, \barx, b) = 1\  | \ \\ 
                   & (\crs, \tw, \pp) \leftarrow \ZKCont.\Setup (1^{\lambda}, \relone), \\ 
                   & (X, \pi, b) \leftarrow \ZKCont.\Preprove(\crs, \bar{y}, \bar{x}, \baromega_1, \relone)) = 1
\end{align*}

\noindent \textbf{Perfect Completeness for $\Reprove$} For all $\lambda \in \mathbb{N}$, for every PPT adversary $A$: 
\begin{align*}
\mathit{Pr} (& (\ZKCont.\Verify(\crs, \bar{y}, X', \pi', \relone) = 1  \Rightarrow \ZKCont.\Verify(\crs, \bar{y}, X, \pi, \relone) = 1)  \ \wedge \  \\
                   & \wedge \ (\ZKCont.\VerifyCom (\pp, X', \barx, b') = 1 \Rightarrow \ZKCont.\VerifyCom (\pp, X, \barx, b) = 1) \ | \\
                   & (\crs, \tw, \pp) \leftarrow \ZKCont.\Setup (1^{\lambda}, \relone),  (\bary, \barx, X', \pi', b') \leftarrow A(\crs,\relone), \\
                   & (X, \pi, b) \leftarrow \ZKCont.\Reprove(\crs, X', \pi', b', \relone)) = 1
\end{align*}

%TODO Check consistency of extractor notations in the paper
\begin{comment}
\begin{align*}
\mathit{Pr} (& (\ZKCont.\Verify(\crs, \bar{y}, X, \pi, \relone) = 1  = >  \ZKCont.\Verify(\crs, \bar{y}, X', \pi', \relone) = 1)  \ \wedge \  \\
                   & \wedge \ (\ZKCont.\VerifyCom (\pp, X, \barx, b) = 1 => \ZKCont.\VerifyCom (\pp, X', \barx, b') = 1) \ | \\
                   & (\crs, \tw, \pp) \leftarrow \ZKCont.\Setup (1^{\lambda}, \relone), \\ 
                   & (\bary, \barx, X, \pi, b) \leftarrow A(\crs,\pp, \relone) \\
                   & (X', \pi', b') \leftarrow \ZKCont.\Reprove(\crs, X, \pi, b, \relone)) = 1
\end{align*}
\end{comment}

\noindent \textbf{Knowledge Soundness} For all $\lambda \in \mathbb{N}$, for every benign auxiliary input $\realaux$ (as per~\cite{bening_auxiliary}) and 
every non-uniform efficient adversary $A$, there exists efficient non-uniform extractor $E$ such that:
\begin{align*}
\mathit{Pr} (& (\ZKCont.\Verify(\crs, \bar{y}, X, \pi, \relone) = 1) \ \wedge \ (\ZKCont.\VerifyCom(\pp, X, \bar{x}, b) = 1) \ \wedge\ \\
                   & \wedge \ ( (\bary, \barx; \baromega_1) \notin \relone) \ | \ (\crs, \tw, \pp) \leftarrow \ZKCont.\Setup (1^{\lambda}, \relone), \\
                   %& (\crspk, \crsvk, \pp) \leftarrow \ZKCont.\Gen(\crs, \relone), \\
                   & (\bary, \barx, X, \pi, b; \baromega_1 ) \leftarrow A || E (\crs, \realaux, \relone)) = \negl (\lambda),
\end{align*}

\begin{comment}
\begin{align*}
\mathit{Pr} (& (\ZKCont.\Verify(\crs, \bar{y}, X, \pi, \relone) = 1) \ \wedge \ (\ZKCont.\VerifyCom(\pp, X, \bar{x}, b) = 1) \ \wedge\ \\
                   & \wedge \ ( (\bary, \barx; \baromega_1) \notin \relone) \ | \ (\crs, \pp) \leftarrow \ZKCont.\Setup (1^{\lambda}), \\
                   & (\bary, \barx, X, \pi, b; \baromega_1 ) \leftarrow A || E (\crs, \realaux, \relone)) = \negl (\lambda),
\end{align*}
\end{comment}

\noindent %where $\ZKCont.\Preprove_{|X}(\crspk, \bar{y}, \bar{x}, \baromega_1, \relone; b)$ means running the part of algorithm 
%$\ZKCont.\Preprove$ that computes and outputs $X$ with its regular inputs and using $b$ when randomness is required; 
where by $(\mathit{output_{A}};\mathit{output_{B}}) \leftarrow A || B(\mathit{input})$ we denote algorithms $A$, $B$ running on the same 
$\mathit{input}$ and $B$ having access to the random coins of $A$. \\

\noindent Finally, we introduce a new flavour of zero-knowledge property. It allows us to formalise the intuition that if one calls once 
$\ZKCont.\Preprove$ on $((\bary, \barx), \baromega_1) \in \relone$ and 
then sequentially calls $ \ZKCont.\Reprove $ at least once but also (possibly) multiple times, 
then every time after the first call to $ \ZKCont.\Reprove $ the resulting proofs reveal nothing regarding either $\barx$ or $\baromega_1$. 
Hence, the proofs obtained via sequential use of $ \ZKCont.\Reprove $ as described above are not linkable, i.e., a property targeted  
in the preamble of this section. \\

\noindent \textbf{Perfect Zero-knowledge w.r.t. $\relone$} For all $\lambda \in \mathbb{N}$, for every benign auxiliary input $\realaux$, 
for all  $(\bary, \barx; \baromega_1) \in \relone$, for all $X'$, for all $\pi'$, for all $b'$, for every adversary $A$ there exists a PPT algorithm $ \Sim $ such that:
\begin{align*}
\mathit{Pr}(& A(\crs, \realaux, \pi, X, \relone) = 1 \ | \ (\crs, \tw, \pp) \leftarrow \ZKCont.\Setup (1^{\lambda}, \relone), \\
                  %& (\crspk, \crsvk, \pp) \leftarrow \ZKCont.\Gen(\crs, \relone), \\ 
                  & \ZKCont.\Verify(\crs, \bary, X', \pi', \relone) = 1, \\
                  & (\pi, X, \_) \leftarrow \ZKCont.\Reprove (\crs, X', \pi', b', \relone)) =  \\
= \mathit{Pr}(& A(\crs, \realaux, \pi, X, \relone) = 1 \ | \ (\crs, \tw, \pp) \leftarrow \ZKCont.\Setup (1^{\lambda}, \relone), \\ 
                     %& (\crspk, \crsvk, \pp) \leftarrow \ZKCont.\Gen(\crs, \relone),\\
                     & \ZKCont.\Verify(\crs, \bary, X', \pi', \relone) = 1, (\pi, X) \leftarrow \ZKCont.\Sim(\tw, \bary, \relone))
\end{align*} 
\end{definition} 

% $$ \Lring = \Setst{ \compk, \comring }{
%  \exists \openpk,\openring \textrm{\ s.t.\ } 
%  \genfrac{}{}{0pt}{}{\PedVRF.\OpenKey(\compk,\openpk) \quad}{\,\, = \rVRF.\OpenRing(\comring,\openring)}
% } \mathperiod $$

% \smallskip
\subsection{Specialised Groth16}
\label{sec:rvrf_groth16}

Below we instantiate our zero-knowledge continuation notion with a scheme based on Groth16~\cite{Groth16} SNARK;
hence, we call our instantiation \emph{specialised Groth16} or \emph{$\mathsf{SpecialG}$}. We start with a  
reminder of the definition of Quadratic Arithmetic Program (QAP)~\cite{LegoSNARK}, ~\cite{GGPR13}.

\begin{definition}[QAP] 
\label{def:QAP}
A Quadratic Arithmetic Program (QAP) $\cQ = (\cA, \cB, \cC, t(X))$ of size $m$ 
and degree $d$ over a finite field $\F_q$ is defined by three sets of polynomials $\cA = \{a_i(X)\}_{i=0}^m$, 
$\cB = \{b_i(X)\}_{i=0}^m$, $\cC = \{c_i(X)\}_{i=0}^m$, each of degree less than $d-1$ and a target degree $d$ polynomial $t(X)$. Given 
$\cQ$ we define $\relRQ$ as the set of pairs $((\bary, \barx); \baromega) \in \F_q^{l} \times \F_q^{n-l} \times \F_q^{m-n}$ for which it 
holds that there exist a polynomial $h(X)$ of degree at most $d-2$ such that:
$$(\sum_{k=0}^m v_k \cdot a_k(X)) \cdot (\sum_{k=0}^m v_k \cdot b_k(X)) = (\sum_{k=0}^m v_k \cdot c_k(X)) + h(X)t(X) \ \ (\ast)$$ 
where $\barv = (v_0, \ldots, v_m) = (1, x_1, \ldots, x_n, w_1, \ldots w_{m-n})$ and $\bary = (x_1, \ldots, x_l)$ and 
$\barx = (x_{l+1}, \ldots, x_n)$ and $\baromega = (w_1, \ldots, w_{m-n})$. 
\end{definition}

\noindent In summary, \SpecialG setup is an extension of original Groth16~\cite{Groth16} setup by two additional 
group elements $\Kgamma$ and $\Kdelta$, as defined below. $\SpecialG$ setup is identical to commit-carrying 
LegoSNARK $\ccgroth$~\cite[Fig.~22]{LegoSNARK} setup\footnote{$\ccgroth$ is, in turn, based on  Groth16.}.
Moreover, together the $\Preprove$ and the $\Reprove$ procedures of $\SpecialG$ are identical to the proving  
procedure in $\ccgroth$ with the difference that $\SpecialG$ also re-randomises the commitment to part 
of the public input. Additionally, the verification procedure and the commitment verification are identical to their 
counterparts in $\ccgroth$. 
%Second, our $\SpecialG.\Reprove$ algorithm uses a Groth16 re-randomisation 
%technique for the proof (see~\cite[Fig.~1]{RandomizationGroth16} or LegoSNARK~\cite[Fig.~22]{LegoSNARK}), 
%but, in addition, $\SpecialG.\Reprove$ also re-randomises $X$ which is a commitment to a slice of the public input;
%Moreover, in terms of security properties, we appropriately define the zero-knowledge for zk continuations such 
%that even after iteratively applying $\SpecialG.\Reprove$ zero-knowledge property is preserved for both the witness 
%as well as the public input committed to in $X$.  

\noindent Given notation provided in section~\ref{sec:background}, in particular elliptic curve $\ecE$, its pairing $e$ and 
the related source, target groups and generators, we introduce%Let $\mathbb{F}_q$ be a prime field, 
%let $G_1$, $G_2$, $G_T$ be as defined in section~\ref{??}, let $e$, $g$, $h$ be defined as $\ldots$. Let $t(X)$ and
%$\{u_i(X),v_i(X),w_i(X)\}_{i=0}^m$ be polynomials in $\F_q[X]$, let $\ldots$ be $\ldots$ such that there exists $h(X) \in \F_q[X]$ with
% $$ \sum_{i=0}^m a_iu_i(X) \cdot  \sum_{i=0}^m a_iv_i(X)  = \sum_{i=0}^m a_iw_i(X) + h(X)t(X)  \ (\ast)$$
%Then let $\relone = \{ (;) \ | \ (;)  (\ast) \}$

\begin{definition}[Specialised Groth16 ($\SpecialG$)]
\label{insta:sg16} Let $\relRQ$ be as in Definition~\ref{def:QAP}. We call 
specialised Groth16 for relation $\relRQ$ the following: %instantiation of the zero-knowledge continuation notion from Definition~\ref{def:zk_cont}:
\begin{itemize}
\item $\SpecialG.\Setup: (1^{\lambda}, \relRQ) \mapsto (\crs, \tw,\pp)$. \\ 
\noindent Let $\alpha, \beta, \gamma, \delta, \tau, \eta  \xleftarrow{\$} \F_q^{*}$. Let $\tw = (\alpha, \beta, \gamma, \delta, \tau, \eta)$. \\ 
Let $\crs = (\barsig_1, \barsig_2)$ where 
\begin{align*}
\barsig_1 = (&\alpha \cdot \gone, \beta \cdot \gone, \delta \cdot \gone, \{\tau_i \cdot \gone\}_{i=0}^{d-1}, \left\{\frac{\beta a_i(\tau)+ \alpha b_i(\tau)+ c_i(\tau)}{\gamma} \cdot \gone \right\}_{i=1}^n,  
\frac{\eta}{\gamma} \cdot \gone, \\ 
&\left\{\frac{\beta a_i(\tau)+ \alpha b_i(\tau)+ c_i(\tau)}{\delta} \cdot \gone \right\}_{i=n+1}^m, \left\{\frac{1}{\delta}\sigma^it(\sigma)\cdot \gone \right\}_{i=0}^{d-2}, 
\frac{\eta}{\delta}\cdot \gone), \\
\barsig_2 = (&\beta \cdot \gtwo, \gamma \cdot \gtwo, \delta \cdot \gtwo, \{\tau^i \cdot \gtwo\}_{i=0}^{d-1}). 
\end{align*} 

%\item $\SpecialG.\Gen: (\crs, \relRQ) \mapsto (\pp, \crspk, \crsvk)$ where \\
%$\crspk = \left([\barsig_1]_1, [\beta]_2, [\delta]_2, \left\{[\tau^i]_2\right\}_{i=0}^{d-1}\right)$  \\ 
%$\crsvk = \left([\alpha]_1, \left\{ \left[ \frac{\beta a_i(\tau)+ \alpha b_i(\tau)+ c_i(\tau)}{\gamma} \right]_1 \right\}_{i=1}^{l}, 
%[\beta]_2, [\gamma]_2, [\delta]_2\right)$  \\ 
$\pp = \left( \left \{ \frac{\beta a_i(\tau)+ \alpha b_i(\tau)+ c_i(\tau)}{\gamma} \cdot \gone \right \}_{i=l+1}^{n}, \frac{\eta}{\gamma} \cdot \gone \right)$. \\
\noindent Moreover, for simplicity and later use, we call $\Kgamma = \frac{\eta}{\gamma} \cdot \gone$  and $\Kdelta = \frac{\eta}{\delta} \cdot \gone$.

\begin{comment}
{We should say what is the difference from Groth16.Setup or it is the same. I think in general in $ \SpecialG $, you should tell which part is from Groth16 or Legosnark and where we change it while describing the algorithms. It will be much clear for the reader to verify. You have notes in the end but I think it is better to have it while describing since you can tell more right away from the algorithm than in the end of everything}
\end{comment}
\item $\SpecialG.\Preprove: (\crs, \bar{y}, \bar{x}, \baromega_1, \relRQ) \mapsto (X', \pi', b')$ such that \\
\begin{align*}
&b' = 0; r, s\xleftarrow{\$} \F_p; X' = \sum_{i=l+1}^{n} v_i \cdot  \frac{\beta a_i(\tau)+ \alpha b_i(\tau)+ c_i(\tau)}{\gamma} \cdot \gone;  \\
&o = \alpha + \sum_{i=0}^{m} v_i \cdot a_i(\tau) + r \cdot \delta; u = \beta + \sum_{i=0}^{m} v_i \cdot b_i(\tau) + s \cdot \delta; \\ 
&v = \frac{\sum_{i=n+1}^{m} (v_i (\beta a_i(\tau)+ \alpha b_i(\tau)+ c_i(\tau))) + h(\tau)t(\tau)}{\delta}   + o\cdot s + u \cdot r - r \cdot s \cdot \delta; \\
& \pi' = (o \cdot \gone, u \cdot \gtwo, v \cdot \gtwo), 
\end{align*}
where $\bary = (x_1, \ldots, x_l)$, $\barx = (x_{l+1}, \ldots, x_n)$, $\baromega = (w_1, \ldots, w_{m-n})$, \\
$\barv = (1, x_1, \ldots, x_n, w_1, \ldots, w_{m-n})$ (same as in Definition~\ref{def:QAP}).


\item $\SpecialG.\Reprove: (\crs, X', \pi', b', \relRQ) \mapsto (X, \pi, b)$  such that
\begin{align*}
&b , r_1, r_2  \xleftarrow{\$} \F_p, X = X' + (b- b') \Kgamma, \pi = (O, U, V), \\
&O = \frac{1}{r_1} O', U = r_1 U' + r_1r_2 \delta \gtwo, V = V' + r_2O'  - (b - b') \Kdelta \mathperiod
\end{align*}
\noindent where $\pi' = (O', U', V')$.
 
\item $\SpecialG.\VerifyCom: (\pp, X, \barx, b) \mapsto 0/1$ where the output is 1 iff the following holds
$$X = \sum_{i=l+1}^{n} x_i \cdot  \frac{\beta a_i(\tau)+ \alpha b_i(\tau)+ c_i(\tau)}{\gamma} \cdot \gone  + b \Kgamma,$$
where $\barx = (x_{l+1}, \ldots, x_n)$, $ 0 \leq l \leq n-1$. 

\item $\SpecialG.\Verify: (\crs, \bar{y}, X, \pi, \relRQ) \mapsto 0/1$ where the output is 1 iff the following holds 
$$e(O,U) = e(\alpha \cdot \gone, \beta \cdot \gtwo) \cdot e(X + Y, \gamma \cdot \gtwo) \cdot e(V, \delta \cdot \gtwo),$$
where $\pi = (O, U, V)$, $Y = \sum_{i=1}^{l} x_i \cdot \frac{\beta a_i(\tau)+ \alpha b_i(\tau)+ c_i(\tau)}{\gamma}  \cdot \gone$ 
and $\bary = (x_1, \ldots, x_l)$.

\item $\SpecialG.\Sim: (\tw, \bary, \relRQ) \mapsto (\pi, X)$ where $x, o, u \xleftarrow {\$} \F_p$ and let \\
$\pi = (o \cdot \gone, u  \cdot \gtwo , v \cdot \gone)$ where 
$v = \frac{o\cdot u - \alpha \beta - \sum_{i=1}^{l} x_i (\beta a_i(\tau)+ \alpha b_i(\tau)+ c_i(\tau))- x}{\delta}  $ and, 
by definition $\bary = (x_1, \ldots, x_l)$. Note that $\pi$ is a simulated proof for transparent input $\bary$ 
and commitment $X = x \cdot \gone$.
\end{itemize} 
\end{definition}

\begin{comment}
\noindent \paragraph{Notes:} First, the trusted setup required by \SpecialG is 
an extension of that required by original Groth16~\cite{Groth16} by two additional 
group elements $\Kgamma = [\frac{\eta}{\gamma}]_1$ and $\Kdelta = [\frac{\eta}{\delta}]_1$. 
An identical trusted setup to that used by \SpecialG was also used in LegoSNARK~\cite[Fig.~22]{LegoSNARK} which defines 
a commit-carrying SNARK based on Groth16. Second, our $\SpecialG.\Reprove$ algorithm uses a Groth16 re-randomisation 
technique for the proof (see~\cite[Fig.~1]{RandomizationGroth16} or LegoSNARK~\cite[Fig.~22]{LegoSNARK}), 
but, in addition, $\SpecialG.\Reprove$ also re-randomises $X$ which is a commitment to a slice of the public input; moreover, in terms of security 
properties, we appropriately define the zero-knowledge for zk continuations such that even after iteratively applying 
$\SpecialG.\Reprove$ zero-knowledge property is preserved for both the witness as well as the public input committed to in $X$.  \\
\end{comment}

%{\color{red}Note that, if a trusted setup is used (for example the one described in~\cite{subversion_zk}) such that there 
%exist a public and efficient procedure for verifying it, then, by extending it with $\Kgamma$ and $\Kdelta$ (which is our 
%extension of the standard Groth16 trusted setup), the resulting setup remains publicly verifiable (i.e., by using the additional 
%verification equation $e(\Kgamma, [\gamma]_2) = e(\Kdelta, [\delta]_2)$), and, hence, according to~\cite{subversion_zk}, 
%subversion zero-knowledge. }
%From page 3 of ~\cite{subversion_zk}"We change Groth?s zk-SNARK by adding extra elements to the CRS so that the CRS will become 
%publicly verifiable; this minimal step (clearly, some public verifiability of the CRS is needed in the case the CRS generator 
%cannot be trusted) will be sufficient to obtain Sub-ZK. However, choosing which elements to add to the CRS is not straightforward 
%since the zk-SNARK must remain knowledge-sound even given enlarged CRS; adding too many (or just ?wrong?) elements to the 
%CRS can break the knowledge-soundness."

\noindent Finally, we are ready to prove that $\SpecialG$ is a zero-knowledge continuation. We show that the knowledge soundness property 
of $\SpecialG$ (i.e., as defined for $\ZKCont$) is implied by the knowledge soundness property of commit-carrying SNARK with double 
binding (cc-SNARK with double binding, see Definition 3.4~\cite{LegoSNARK}); our notion of zero-knowledge for $\ZKCont$ is, in fact a 
new and stronger notion so we prove that directly. Formally, we have: 

\begin{theorem}
\label{sec_specialg}
Let $\relRQ$ be a relation as per Definition~\ref{def:QAP} such that additionally $\{a_k(X)\}_{k=0}^n$ are linearly independent polynomials. Then, in the 
AGM~\cite{Fuchs_AGM}, $\SpecialG$ is a zero-knowledge continuation as per Definition~\ref{def:zk_cont}. 
\end{theorem}
\begin{proof} It is straightforward to show that $\SpecialG$ has perfect completeness for $\Preprove$ and perfect completeness 
for $\Reprove$. \\

\noindent We prove knowledge-soundness (KS) an in Definition~\ref{def:zk_cont} by first arguing $\SpecialG$ is a cc-SNARK with double binding 
(see Definition 3.4~\cite{LegoSNARK}).  We use the fact that $\ccgroth$ as defined by the NILP detailed in Fig.22, Appendix H.5~\cite{LegoSNARK} 
satisfies that latter definition. Moreover, $\SpecialG$'s $\Setup$ on one hand, and $\ccgroth$'s $ \mathsf{KeyGen} $, on the other hand, are the same 
procedure. Also $\SpecialG$ and $\ccgroth$ share the same verification algorithm. Hence, translating the notation appropriately, $\SpecialG$ also 
satisfies KS of a cc-SNARK with double binding. \\

% we first argue that We prove knowledge-soundness by reducing it to the knowledge-soundness property of the Groth16 
%commit-carrying with double binding scheme (for short $\ccgroth$). This knowledge-soundness property has been formalised in 
%Definition 3.4~\cite{LegoSNARK} and the NILP corresponding to $\ccgroth$ has been detailed in Fig.22, Appendix H.5~\cite{LegoSNARK}. 
%Indeed, $\SpecialG$'s $\Setup$ together with $\Gen$ and $\ccgroth$'s $\mathit{KeyGen}$ are the same procedure. Moreover $\SpecialG$ 
%and $\ccgroth$ share the same verification algorithm. Since $\ccgroth$ satisfies the definition of a cc-SNARK with double binding, translating 
%the notation appropriately, $\SpecialG$ also satisfies the knowledge soundness properties for a cc-SNARK with double binding. 
\noindent Let $A_{\SpecialG}$ be an adversary for KS in Definition~\ref{def:zk_cont} and 
define adversary $A_{\ccgroth}$ for KS in Definition 3.4~\cite{LegoSNARK}:
\begin{align*}
&\mathit{If} \ A_{\SpecialG} (\crs, \pp, \realaux, \relRQ)\ \mathit{outputs} \  (\bary, \barx, X, \pi, b) \\
&\ \ \ \ \ \ \ \ \mathit{then}\ A_{\ccgroth} (\crs, \realaux, \relRQ)\ \mathit{outputs} \ (\bary, X, \pi). 
\end{align*}

\noindent Given extractor $E_{\ccgroth}$ fulfilling Definition 3.4~\cite{LegoSNARK} for $A_{\ccgroth}$, we
construct extractor $E_{\SpecialG}$ for $A_{\SpecialG}$
\begin{align*}
&\mathit{If} E_{\ccgroth} (\crs, \realaux, \relRQ)\ \mathit{outputs} \ (\barx^*, b^*, \baromega^*) \\
& \ \ \ \ \ \ \ \ \mathit{then} \ E_{\SpecialG} (\crs, \realaux, \relRQ) \ \mathit{outputs} \ \baromega^*; \\
& \mathit{Otherwise} \ E_{\ccgroth}(\crs, \realaux, \relRQ) \ \mathit{outputs} \ \bot.
\end{align*}

We show $E_{\SpecialG}$ fulfils Definition~\ref{def:zk_cont} for $A_{\SpecialG}$. Assume by contradiction that is not the case. 
This implies there exists an auxiliary input $\realaux$ such that each: 
\begin{align*}
\SpecialG.\Verify(\crs, \bary, X, \pi, &\relRQ) =1 \ (10)\ ; \ \SpecialG.\VerifyCom(\pp, X, \barx, b) =1 \ (20) \\
& (\bary, \barx; \baromega) \notin \relRQ  \ \ (30) 
\end{align*}
hold with non-negligible probability. Since $(20)$ holds with non-negligible probability and verification (for both proofs and commitments actually) is identical in $\SpecialG$ and $\ccgroth$ respectively, 
and since $E_{\ccgroth}$ is an extractor for $A_{\ccgroth}$ as per Definition 3.4~\cite{LegoSNARK},
 then each of the two events 
\begin{align*}
\mathit{\ccgroth.\mathit{VerCommit}^*}(\mathit{ck}, X, \barx^*, b^*) =1 \ (40) \ ; \ (\bary, \barx^*; \baromega^*) \in  \relRQ \ (50)
\end{align*}
holds with overwhelming probability. Since $(20)$ holds with non-negligible probability and $(40)$ holds with overwhelming probability and 
together with (ii) from Definition 3.4~\cite{LegoSNARK} we obtain that $\barx^* = \barx$. Since $(50)$ holds with overwhelming probability, it implies 
$(\bary, \barx; \baromega^*) \in \relRQ $ with overwhelming probability which contradicts our assumption, so our claim that $\SpecialG$ does not have 
KS as per Definition~\ref{def:zk_cont} is false. \\

\noindent Finally, regarding zero-knowledge, it is clear that if $\pi = (O, U, V)$ is part of the output of $\SpecialG.\Reprove$, 
then $O$ and $U$ are uniformly distributed as group elements in their respective groups. This holds, as long as the 
input to $\SpecialG.\Reprove$ is a verifying proof, even when the proof was maliciously generated. Hence, it is easy to check  
that the output $\pi'$ of $\SpecialG.\Sim$ is identically distributed to a proof $\pi$ output by $\SpecialG.\Reprove$ so the perfect 
zero-knowledge property holds for $\SpecialG$. 
\end{proof}

%TODO Check consistency with the preliminaries section
\subsection{Putting Together a NIZK and a $\ZKCont$  for Proving $\rel$}
%$$\rel = \{(\bary, \barz; \barx, \baromega_1, \baromega_2):  (\bary, \barx; \baromega_1) \in \relone, (\barz, \barx; \baromega_2) \in \reltwo \},$$
Let $\ZKCont$ be a zk continuation for $\relone$ (from preamble of Section~\ref{sec:rvrf_cont}) and 
let $\nizktwo$ be a NIZK for $\reltwo'(\pp)$ (for some public parameters $\pp$) defined by:
$$\reltwo'(\pp) = \{(X, \barz, \pp; \barx, b, \baromega_2): \ZKCont.\VerifyCom(\pp, X, \barx, b) =1 \ \wedge \ (\barz, \barx; \baromega_2) \in \reltwo \},$$
\noindent with $\reltwo$ from preamble of Section~\ref{sec:rvrf_cont}. Then we define the system $\nizkR$ for relation $\mathcal{R}$ 
from the preamble of Section~\ref{sec:rvrf_cont} as:
\begin{itemize}
\item $\nizkR.\Setup: (1^{\lambda}) \mapsto (\crsR = (\crs,\crstwo), \twR = (\tw, \twtwo), \pp)$ where
$(\crs, \tw, \pp) \leftarrow \ZKCont.\Setup(1^{\lambda}, \relone)$, \\ $(\crstwo, \twtwo) \leftarrow \nizktwo.\Setup(1^{\lambda})$

%\item $\nizkR.\Gen: (\crsR) \mapsto (\ppR = \pp, \crspkR = (\crspk,\crspktwo), \crsvkR = (\crsvk,\crsvktwo))$ where 
%$(\crspk, \crsvk, \pp) \leftarrow \ZKCont.\Gen(\crs, \relone)$, \\ $(\crspktwo, \crsvktwo) \leftarrow \nizktwo.\Gen(\crstwo)$ 

\item $\nizkR.\Prove: (\crsR,\bary, \barz; \barx, \baromega_1, \baromega_2 ) \mapsto (\pi_1, \pi_2, X)$ where \\
$(X', \pi'_1, b')\leftarrow \ZKCont.\Preprove(\crs, \bar{y}, \bar{x}, \baromega_1, \relone)$ \\
$(X, \pi_1, b) \leftarrow \ZKCont.\Reprove(\crs, X', \pi_1', b', \relone)$ \\
$ \pi_2 \leftarrow \nizktwo.\Prove(\crstwo, X, \barz; \barx, b, \baromega_2)$ 

\item $\nizkR.\Verify: (\crsR,(\bary, \barz), (\pi_1, \pi_2, X)) \mapsto 0/1$ where the output is $1$ iff 
$$\ZKCont.\Verify(\crs, \bar{y}, X, \pi_1, \relone) = 1 \  \wedge \ \nizktwo.\Verify(\crstwo, X, \barz, \pi_2) =1$$

\item $\nizkR.\Sim: (\twR, \bary, \barz) \mapsto (\pi_1, \pi_2, X)$ where \\
$(\pi_1, X) \leftarrow \ZKCont.\Sim(\tw, \bary, \relone)$, $\pi_2 \leftarrow \nizktwo.\Sim(\twtwo, X, \barz)$ 
 \end{itemize}
 
\begin{lemma}[Knowledge-soundness for $\nizkR$] 
\label{le:KS_for_nizkR}
If $\ZKCont$ is a zk continuation for $\relone$ and $\nizktwo$ is a NIZK for $\reltwo'(\pp)$ for some appropriately chosen public parameters $\pp$, 
then the $\nizkR$ construction described above has knowledge-soundness for $\rel$. 
\end{lemma} 
\begin{proof}This is easy to infer by linking together the extractors guaranteed for $\ZKCont$ and $\nizktwo$ due to their respective 
knowledge-soundness.
\end{proof}
 
\noindent Next, we define \\ 
%\noindent {\color{red}\textbf{Perfect Completeness after reproving a $\ZKCont$ Proof}}
\noindent \textbf{Special Perfect Completeness} For all $\lambda \in \mathbb{N}$, for every efficient adversary $A$, for every 
 $(\barz, \barx; \baromega_2) \in \reltwo$ it holds
%{It is a completeness definition for $ \reltwo $ but there is nothing related to $ \reltwo $ below. We want to define a new completeness definition for $ 
%\nizkR $. So, I suggest you to replace everything before given $ (|) $ with $ \nizkR.\Verify((\crsR,(\bary, \barz), (\pi_1, \pi_2, X)))\rightarrow 1 $ because %everything you have before $ | $are already satisfied from the $ \NIZK $ completeness and $ \ZKCont $ special completeness, so they are already %satisfied, they are not new statements. }: 
\begin{align*}
\mathit{Pr} (& (\ZKCont.\Verify(\crs, \bar{y}, X', \pi'_1, \relone) = 1 \ \wedge \ \ZKCont.\VerifyCom (\pp, X', \barx, b') = 1) ) \\
                   & \Rightarrow \ \nizkR.\Verify(\crsR, X, \barz, \pi_2) =1 \ | \ (\crs, \tw, \pp) \leftarrow \ZKCont.\Setup (1^{\lambda}, \relone), \\
                   & (\crstwo, \twtwo) \leftarrow \nizktwo.\Setup(1^{\lambda}), (\bary, X', \pi'_1, b') \leftarrow A(\crs,\relone), \\ 
                   & (X, \pi_1, b) \leftarrow \ZKCont.\Reprove(\crs, X', \pi'_1, b', \relone), \\
                   &\pi_2 \leftarrow \nizktwo.\Prove(\crstwo, X, \barz, \barx, b, \baromega_2)) = 1
\end{align*}

\begin{lemma}[Special Perfect Completeness] 
\label{le:specialCompl_for_nizkR}
If $\ZKCont$ is a zk continuation for $\relone$ and $\nizktwo$ is a NIZK for $\reltwo'(\pp)$ for some appropriately chosen public parameters $\pp$, 
then the $\nizkR$ construction described above has special perfect completeness.
\end{lemma} 
\begin{proof} This is easy to infer by combining the perfect completeness properties of $\nizktwo$ and perfect completeness 
for $\ZKCont.\Reprove$.

%  $\ZKCont.\Verify(\crs, \bar{y}, X', \pi'_1, \relone) = 1 \ \Rightarrow \ \ZKCont.\Verify(\crs, \bar{y}, X, \pi_1, \relone) = 1$
%  $\ZKCont.\VerifyCom (\pp, X', \barx, b') = 1 \ \Rightarrow \ \ZKCont.\VerifyCom(\pp, X, \barx, b) = 1$
%  $\pi_2 \leftarrow \nizktwo.\Prove(\crstwo, X, \barz, \barx, b, \baromega_2)) \ \Rightarrow \  \nizktwo.\Verify(\crstwo, X, \barz, \pi_2) =1$
\end{proof}

\noindent Finally, we define \\
\begin{comment} 
\noindent \textbf{Perfect Zero-knowledge after Reproving a $\ZKCont$ Proof} For all $\lambda \in \mathbb{N}$, for every benign auxiliary input $aux$, 
for all $\bary,\barx,\barz, \baromega_1, \baromega_2$ with $(\bary, \barx; \baromega_1) \in \relone$ and $(\barz, \barx; \baromega_2) \in \reltwo$, for all $X,\pi_1,\pi_2, b$, for every adversary $A$ it holds:
\begin{align*}
\mathit{Pr}(& A(\crs, \realaux, \pi'_1,\pi_2, X', \rel) = 1 \ | \ (\crs, \tw, \pp) \leftarrow \ZKCont.\Setup (1^{\lambda}, \relone), \\
                  & (\pi'_1, X', \_) \leftarrow \ZKCont.\Reprove (\crs, X, \pi, b, \relone), \\
                  & \pi_2 \leftarrow \nizktwo.\Prove(\crstwo, X, \barz, \barx, b, \baromega_2), \\
                  &  \ZKCont.\Verify(\crs, \bary, X, \pi, \relone) = 1 
                  \wedge \VerifyCom(\pp, X, \bar{x}, b) = 1) =  \\
= \mathit{Pr}(& A(\crs, \realaux, \pi',\pi_2, X', \rel) = 1 \ | \ (\crs, \tw, \pp) \leftarrow \ZKCont.\Setup (1^{\lambda}, \relone), \\ 
                     & (\pi'_1, X',\pi_2) \leftarrow \nizkR.\Sim(\tw, \bary, \relone) \\ 
                     &  \ZKCont.\Verify(\crs, \bary, X, \pi, \relone) = 1
                    \wedge \VerifyCom(\pp, X, \bar{x}, b) = 1)
\end{align*}
\end{comment} 

\noindent \textbf{Zero-knowledge after Reproving a $\ZKCont$ Proof} For all $\lambda \in \mathbb{N}$, for every benign auxiliary input $aux$, 
for all $\bary,\barx,\barz, \baromega_1, \baromega_2$ with $(\bary, \barx; \baromega_1) \in \relone$ and $(\barz, \barx; \baromega_2) \in \reltwo$, for all $X',\pi_1', b'$, for every adversary $A$ it holds:
\begin{align*}
| \mathit{Pr}(& A(\crs, \realaux, \pi_1,\pi_2, X, \rel) = 1 \ | \ (\crs, \tw, \pp) \leftarrow \ZKCont.\Setup (1^{\lambda}, \relone), \\
                  %& (\crspk, \crsvk, \pp) \leftarrow \ZKCont.\Gen(\crs, \relone), \\
                  &(\pi_1, X, \_) \leftarrow \ZKCont.\Reprove (\crs, X', \pi_1', b', \relone), \\
                  & \pi_2 \leftarrow \nizktwo.\Prove(\crstwo, X, \barz, \barx, b, \baromega_2), \\
                  &  \ZKCont.\Verify(\crs, \bary, X', \pi'_1, \relone) = 1 
                  \wedge \ZKCont.\VerifyCom(\pp, X', \barx', b') = 1)   \\
- \mathit{Pr}(& A(\crs, \realaux, \pi_1,\pi_2, X, \rel) = 1 \ | \ (\crs, \tw, \pp) \leftarrow \ZKCont.\Setup (1^{\lambda}, \relone), \\ 
                     %& (\crspk, \crsvk, \pp) \leftarrow \ZKCont.\Gen(\crs, \relone), 
                     & (\pi_1,\pi_2, X) \leftarrow \nizkR.\Sim(\tw, \bary, \relone) \\ 
                     &  \ZKCont.\Verify(\crs, \bary, X', \pi'_1, \relone) = 1 \wedge \ZKCont.\VerifyCom(\pp, X', \barx', b') = 1) | \\
                    \leq \negl(\lambda)
\end{align*}

\begin{lemma}[ZK after Reproving a $\ZKCont$ Proof] 
\label{le:specialZK_for_nizkR}
If $\ZKCont$ is a zk continuation for $\relone$ and $\nizktwo$ is a NIZK for $\reltwo'(\pp)$ for some appropriately chosen public parameters $\pp$, 
then the $\nizkR$ construction described above has zero-knowledge after reproving a $\ZKCont$ proof.
\end{lemma} 
\begin{proof} The statement follows from the perfect zero-knowledge w.r.t. $\relone$ for $\ZKCont$ and 
the zero-knowledge property of $\nizktwo$ w.r.t. $\reltwo'(\pp)$ .
\end{proof}
 
\begin{corollary}
If $\ZKCont$ is a zk continuation for $\relone$ and $\nizktwo$ is a NIZK for $\reltwo'(\pp)$ for some appropriately chosen public parameters $\pp$, 
then the $\nizkR$ construction described above is a NIZK for $\rel$.
 \end{corollary}
 
\begin{proof} Putting together the results of Lemma~\ref{le:KS_for_nizkR}, Lemma~\ref{le:specialCompl_for_nizkR}, 
Lemma~\ref{le:specialZK_for_nizkR} and  we obtain the above statement.
\end{proof} 

\subsection{Continuation}
\label{subsec:rvrf_faster}

% TODO \PedVRF.\OpenKey(\compk,\openpk)

\def\longeq{=\mathrel{\mkern-10mu}=}% {=\joinrel=} % https://tex.stackexchange.com/questions/35404/is-there-a-wider-equal-sign
We describe a much faster choice \pifast for \piring with
opaque inputs $x_1 \longeq \sk$ and transparent inputs $y_1 \longeq \comring$
 so that taking
 $\genG \longeq \chi_1$, $\genB \longeq \genB_\gamma$, and $\openpk \longeq b$
in \PedVRF yields an incredibly fast amortized ring VRF prover.
Also \PedVRF itself proves knowledge of $X' =  \sk\, \chi_1 + b \genB_\gamma $,
 as required by $\SpecialG.\Verify$.
% $$ X' + Y = \comring\, \Upsilon_1 + \sk\, \chi_1 + b \genB_\gamma $$

A priori, we do not know $\chi_1$ during the trusted setup for $\pifast$,
which prevents computing $\pk = \sk\, \chi_1$ inside $\pifast$.
Instead, we propose $\ring$ contain commitments to $\sk$ over
some Jubjub curve $\ecJ$, while $\sk \in \F_p$ remains a scale for $\grJ$.

We know the large subgroup $\grJ$ of $\ecJ$ typically has smaller prime
order $p_\grJ$ than $\grE$, itself due to $\ecJ$ being an Edwards curve.
%
We thus choose $\sk_0,\sk_1 < p_\grJ$ with at least $\lambda$ bits
so that
 $\PedVRF.\sk = \sk_0 + \sk_1 \, 2^{\lambda} \mod p_\grE$
becomes our secret key.
\footnote{If $\lambda \approx 128$ then $p, p_\grJ > 2^{2\lambda-3}$.}
Our $\rVRF.\KeyGen$ \eprint{returns}{shall now return}
a secret key of the form $\rVRF.\sk = (\sk_0,\sk_1,d)$
 with $d \leftsample \F_{p_\grJ}$ and
a public key of the form
 $\rVRF.\pk = \sk_0\, \genJ_0 + \sk_1\, \genJ_1 + d \genJ_2$,
for some independent $\genJ_0,\genJ_1,\genJ_2 \in \grJ$. % (see \S\ref{subsec:AML_KYC}).
\footnote{Interestingly we avoid range proofs for $\sk_1$ and $\sk_2$ by this independence.}
We thus have a fairly efficient instantiation for $\Lring^\inner$ give by

$$ \Lfast^\inner = \Setst{ \sk_0 + 2^{128} \sk_1, \comring }{
 \eprint{ \exists d,\openring \textrm{\ s.t.\ } }{}
 % 0 < \sk_0,\sk_1 < 2^{128} \textrm{\ and\ } 
 \genfrac{}{}{0pt}{}{ \eprint{\rVRF.}{}\OpenRing(\comring,\openring) }{ \,\, = \sk_0 \genJ_0 + \sk_1 \genJ_1 + d \genJ_2 }
} \mathperiod $$

Applying our rerandomization \Reprove to $\pifast^\inner$ with opaque input
yields a zkSNARK $\pifast$ with the extra $\PedVRF.\OpenKey$ arithmetic to
have exactly the form $\piring$.

We explain later in \S\ref{sec:ring_hiding} how one could
choose $\chi_1$ independent before doing the trusted setup,
 and then wire $\chi_1$ into $\pifast$ inside $C$.
We could then prove $\pk = \sk\, \chi_1$ directly inside $\pifast^\inner$,
but doing so here requires slow non-native field arithmetic.

At this point, $\PedVRF.\Sign$ requires two scalar multiplications on $\ecE_1$
 and two on the somewhat faster $\ecE'$,
so together with rerandomization costing four scalar multiplications
on $\ecE_1$ and two on $\ecE_2$, our amortized prover time
 runs faster than 12 scalar multiplications on typical $\ecE_1$ curves. 
We expect the three pairings dominate verifier time, but
 verifiers also need five scalar multiplications on $\ecE_1$.

As an aside, one could construct a second faster curve with the same
group order as $\grE$, which speeds up two scalar multiplications
 in both the prover and verifier. 

Importantly, our fast ring VRF' amortized prover time now rivals
group signature schemes' performance \cite{group_sig_survey,}.
We hope this ends the temptation to deploy group signature like
 constructions where the deanonymization vectors matter.

% BEGIN TODO: Oana

\begin{theorem}\label{thm:knowledge_soundness}
\rVRF instantiated with \pifast and \PedVRF satisfies knowledge soundness.
\end{theorem}

\begin{proof}[Proof stetch.]
An extractor for \PedVRF reveals the opening of $X$ for us,
so our result follows from Lemma \ref{lem:knowledge_soundness}.
\end{proof}

% \begin{corollary}\label{cor:???}
% Our Pedersen ring VRF instantiated with \pifast satisfies ring unforgability and uniqueness.
% \end{corollary}

% \begin{theorem}\label{thm:pifast_anonymity}
% \rVRF instantiated with \pifast and \PedVRF satisfies zero-knowledge.
% \end{theorem}
%
% \begin{proof}[Proof stetch.]
% Assuming the same \comring, we know the zero-knowledge continuations
% are identically distributed by Lemma \ref{lem:unlinkable},
% even when reusing a zero-knowledge continuation $(X,A,B,C)$.
% It follows the typical simulator for \PedVRF ... WHAT???
% \end{proof}

% \begin{corollary}\label{cor:???}
% Our Pedersen ring VRF instantiated with \pifast satisfies ring anonymity.
% \end{corollary}

% END TODO: Oana

