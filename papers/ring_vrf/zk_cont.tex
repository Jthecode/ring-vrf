
\section{Zero-knowledge continuations}
\label{sec:rvrf_cont}

We now construct ring VRFs which achieves fast amortized prover time
by using a heavy zero-knowledge continuation for $\rVRF.\OpenRing$ but
which permits updating \openpk in the $\PedVRF.\OpenKey$ invocation
without reproving \piring for \Rring.

% $$ \Rring = \Setst{ \compk, \comring }{
%  \exists \openpk,\openring \textrm{\ s.t.\ } 
%  \genfrac{}{}{0pt}{}{\PedVRF.\OpenKey(\compk,\openpk) \quad}{\,\, = \rVRF.\OpenRing(\comring,\openring)}
% } \mathperiod $$

% \smallskip
\subsection{Rerandomization}
% \label{sec:rvrf_groth16}

Zero-knowledge continuations need rerandomizable zkSNARKs,
meaning Groth16 \cite{Groth16}, but unlinkability requires more than merely rerandomization.

In Groth16 \cite{Groth16}, we have an SRS $S$ consisting of curve
points in $\grE_1$ and $\grE_2$ that encode the circuit being proven.
We follow \cite{Groth16} in discussing the SRS $S$ in terms of
its ``toxic waste''
 $(\alpha,\beta,\delta,\gamma,\tau^1,\tau^2,\ldots) \in \F_q^*$.
In other words, we could write say $[ f(\tau)/\delta ]_1$ or $[\cdots]_2$
to denote an element of our SRS $S$ in $\grE_1$ or $\grE_2$ respectively,
computed by scalar multiplication of the Groth16 generators from
the toxic waste $\tau$ and $\delta$,
 but for which nobody knows the underlying $\tau$ or $\delta$ anymore.

In the SRS $S$, we distinguish the verifiers' string of elements
 $\chi_1,\ldots,\chi_k, [\alpha]_1 \in \grE_1$ and
 $[\beta]_2, [\gamma]_2, [\delta]_2 \in \grE_2$.
% as separate from the provers' much longer string of elements in $\grE_1$ and $\grE_2$.
A Groth16 \cite{Groth16} proof then takes the form 
 $\pi = (A,B,C) \in \grE_1 \times \grE_2 \times \grE_1$.
A verifier then produces a $X = \sum_i^k x_i \chi_i \in \grE_1$ from
 the public inputs $x_i$ and then checks 
$$ e(A,B) = e([\alpha]_1, [\beta]_2) \cdot
 e(X, [\gamma]_2) \cdot e(C, [\delta]_2) \mathperiod $$

We need the rerandomization algorithm from \cite[Fig.~1]{RandomizationGroth16}:
% to build a zero-knowledge continuation:
% https://eprint.iacr.org/2020/811
% https://github.com/arkworks-rs/groth16/pull/16/files
% \algo{rerandomize}
An existing SNARK $(A,B,C)$ is transformed into a fresh
SNARK $(A',B',C')$ by sampling random $r_1,r_2 \in \F_p$ and computing
% $$
% A' = {1 \over r_1} A, \qquad
% B' = r_1 B + r_1 r_2 [\delta]_2, \qquad
% C' = C + r_2 A \mathperiod
% $$
$$ \begin{aligned}
A' &= {1 \over r_1} A \\
B' &= r_1 B + r_1 r_2 [\delta]_2 \\
C' &= C + r_2 A \mathperiod \\
\end{aligned} $$
At this point, our $x_i$ remain identical after rerandomization,
so $X$ links $(A,B,C)$ to $(A',B',C')$.
Alone rerandomization cannot alter public inputs $x_i$, so
we instead need an opaque public input point $X$, which then becomes
part of our proof and incurs its own separate proof of correctness.

We also need one fresh basepoint $\genB_\gamma$ independent from all others,
again perhaps created by applying $H_\grE$ to an input outside existing usages' domain.
We now give provers the additional SRS elements
$$ \genB_\delta := {\gamma\over\delta} \genB_\gamma $$
Although $\genB_\gamma$ is independent, 
we create $\genB_\delta$ during the trusted setup,
 so the toxic waste $\gamma$ and $\delta$ remain secret.
After this, subversion resistance could be checked like 
$$ e(\genB_\gamma, [\gamma]_2) = e(\genB_\delta, [\delta]_2) \mathperiod $$

We now have a zero-knowledge continuation $(X,A,B,C)$ from which
we produce an unlinkable instance $(X',A',B',C')$ by
 first sampling random $b,r_1,r_2 \in \F_p$ and then computing
$$ \begin{aligned}
X' &= X + b \genB_\gamma \\
A' &= {1 \over r_1} A \\
B' &= r_1 B + r_1 r_2 [\delta]_2 \\
C' &= C + r_2 A + b \genB_\delta \mathperiod \\
\end{aligned} $$
As our two $b$ terms cancel in the pairings, we wind up with the
standard Groth16 rerandomization construction above,
 except with $X$ an now opaque Pedersen commitment.

% TODO:  Should we be saying opaque less and Pedersen more below?

Along side hidden inputs in $X = \sum_i^k x_i \chi_i$,
our verifier would typically assemble some transparent inputs
$Y = \sum_i^l y_i \Upsilon$ themselves.  In particular, our ring VRFs
handle the Merkle root \comring of the ring $\ctx$ in this way below.

We then verify $(X' + Y,A',B',C')$ like Groth16 
$$ e(A',B') = e([\alpha]_1, [\beta]_2) \cdot
 e(X' + Y, [\gamma]_2) \cdot e(C', [\delta]_2) \mathcomma $$
As our verifier does not build $X'$ themselves, we proves nothing
with this pairing equation unless the verifier separately checks
 some proof-of-knowledge that $X' = \sum_i^k x_i \chi_i$.

\begin{lemma}\label{lem:unlinkable}
Our rerandomization procedure % $(X,A,B,C) \mapsto (X',A',B',C')$
transforms honestly generated zero-knowledge continuations $(X,A,B,C)$
into identically distributed zero-knowledge continuations $(X',A',B',C')$,
with identical transparent inputs $y_1,\ldots,y_l$.
\end{lemma}

\begin{proof}[Proof idea.]
Adapt the proof of Theorem 3 of \cite[Appendix C, pp. 31]{RandomizationGroth16}.
\end{proof}

% \begin{corollary}\label{cor:unlinkable}
%	If $\sigma'$ and $\sigma''$ are \PedVRF{}s then ???
% \end{corollary}

All told, our opaque rerandomization trick converts any conventional
Groth16 zkSNARK $\pi$ for $\rVRF.\OpenRing$ into a zkSNARK $\pi'$
with inputs split into a transparent part $Y$ vs opaque unlinkable part $X$.
% We explore two concrete $\pi$ proposals below.

Importantly, rerandomization requires only
 four scalar multiplications on $\ecE_1$ and
 two scalar multiplications on $\ecE_2$,
which  BLS12 curves make roughly equivalent to
 eight scalar multiplications on $\ecE_1$.

% TODO:
% \begin{definition}
% restrited knowledge soundness
% \end{definition}

\begin{lemma}\label{lem:knowledge_soundness}
Assuming AGM plus the $(2n-1,n-1)$-DLOG assumption,
any zero-knowledge continuation with circuit size less than $n$
satisfies knowledge soundness, restricted to challengers
 who learn the actual public input wire values and blinding factors.
\end{lemma}

\begin{proof}[Proof idea.]
Adapt the proof of Theorem 2 in \cite[\S3, pp. 9]{RandomizationGroth16},
observing that $K_\gamma$ and $K_\delta$ never interact with other elements. 
% TODO: Check this argument!
\end{proof}

% TODO:  Oy!
% We obtain knowledge soundness if we fold hiding Pedersen commitments
% into the witness too.  Yet, 
% we do not have ``knowledge soundness'' for the hidden Pedersen commitment 

In fact, one could prove zero-knowledge continuations satisfy
weak white-box simulation extractability under similar restrictions,
much like Theorem 1 in \cite[\S3, pp. 8 \& 11]{RandomizationGroth16}.
We depends upon the specific simulator though, which itself increases
our dependence upon the usage of the zero knowledge continuation.


\subsection{Single continuation}
\label{subsec:rvrf_faster}

We describe a much faster choice $\pifast$ for $\pi$
that sets $x_1 := \sk$ and $x_0 = \comring = \CommitRing(\ctx)$ so that
taking $\genG := \chi_1$, $\genB := \genB_\gamma$, and $\openpk := b$ in \PedVRF
yields and incredibly fast amortized ring VRF prover.
Also, \PedVRF itself proves knowledge of $X'$,
 or more precisely of $X' - \comring\, \chi_0$ since the verifier knows \comring.
$$ X' = \comring\, \chi_0 + \sk\, \chi_1 + b \genB_\gamma $$

% TODO: Do we mention that $\chi_0$ gets omitted in the proof-of-knowledge

A priori, we do not know $\chi_1$ during the trusted setup for $\pifast$,
which prevents computing $\pk = \sk\, \chi_1$ inside $\pifast$.
Instead, we propose $\ctx$ contain commitments to $\sk$ over
some Jubjub curve $\ecJ$.  

We know the large prime order group $\grJ$ of $\ecJ$ typically has
smaller order than $\grE$, itself due to $\ecJ$ being an Edwards curve. 
Yet, if $\sk = \sk_0 + \sk_1 \, 2^{128}$ then our public key commitments could
take the form $\sk_0\, \genJ_0 + \sk_1\, \genJ_1 + d \genJ_2$,
with independent $\genJ_0,\genJ_1,\genJ_2$.
Interestingly, we avoid range proofs for $\sk_1$ and $\sk_2$
by this independence. 

$$ \pifast = \rrSNARK \Setst{ \sk_0 + \sk_1 2^{128}, \comring }{
 \exists d,\openring \textrm{\ s.t.\ }
 % 0 < \sk_0,\sk_1 < 2^{128} \textrm{\ and\ } 
 \genfrac{}{}{0pt}{}{ \rVRF.\OpenRing(\comring,\openring) }{ \,\, = \sk_0 \genJ_0 + \sk_1 \genJ_1 + d \genJ_2 }
} $$ % \mathperiod 

Applying our rerandomization to $\pifast$ with opaque input yields
a zkSNARK $\pifast'$ of exactly the form $\piring$.

We explain later in \S\ref{sec:ring_hiding} how one could
choose $\chi_1$ independent before doing the trusted setup,
 and then wire $\chi_1$ into $\pifast$ inside $C$.
In ths case, we could prove $\pk = \sk\, \chi_1$ inside $\pifast$, but then
non-native arithmetic makes $\pifast$ far slower.

At this point, \PedVRF requires four scalar multiplications on $\ecE_1$,
so together with rerandomization costing four scalar multiplications
on $\ecE_1$ and two on $\ecE_2$, our amortized prover time
 approaches 12 scalar multiplications on typical $\ecE_1$ curves. 
We expect the three pairings dominate verifier time, but
 verifiers also need five scalar multiplications on $\ecE_1$.

As an aside, one could construct a second faster curve with the same
group order as $\grE$, which speeds up two scalar multiplications
 in both the prover and verifier. 

Importantly, our fast ring VRF' amortized prover time now rivals
group signature schemes' performance.  We hope this ends the temptation
to deploy group signature like constructions where the deanonymization vectors matter.

\begin{theorem}\label{thm:knowledge_soundness}
Our Pedersen ring VRF instantiated with \pifast satisfies knowledge soundness.
\end{theorem}

\begin{proof}[Proof stetch.]
An extractor for \PedVRF reveals the opening of $X$ for us,
so our result follows from Lemma \ref{lem:knowledge_soundness}.
\end{proof}

% \begin{corollary}\label{cor:???}
% Our Pedersen ring VRF instantiated with \pifast satisfies ring unforgability and uniqueness.
% \end{corollary}

% \begin{theorem}\label{thm:pifast_anonymity}
% Our Pedersen ring VRF \rVRF using \pifast satisfies zero-knowledge.
% \end{theorem}
%
% \begin{proof}[Proof stetch.]
% Assuming the same \comring, we know the zero-knowledge continuations
% are identically distributed by Lemma \ref{lem:unlinkable}.
% It follows the typical simulator for \PedVRF ... WHAT???
% \end{proof}

% \begin{corollary}\label{cor:???}
% Our Pedersen ring VRF instantiated with \pifast satisfies ring anonymity.
% \end{corollary}


\subsection{Side channel}
\label{subsec:rvrf_side_channel}

In this, we dislike exposing the secret key material inside
 the Groth16 prover for \pifast.
Adversaries could trigger \pifast recomputation only by updating the ring,
but this still presents a side channel risk.

If concerned, one could address this via a second zk continuation that
splits \pifast into two parts: % \pisk and \pipk.
%
$$ \pipk = \rrSNARKweak \Setst{ J_\pk, \comring }{
 \exists \openring \textrm{\ s.t.\ }
  J_\pk = \rVRF.\OpenRing(\comring,\openring)
} \mathcomma \quad\textrm{and} $$ 
%
$$ \pisk = \rrSNARK \Setst{ \sk_0 + \sk_1 2^{128}, J_\pk }{ 
 \exists d \textrm{\ s.t.\ }
 % 0 < \sk_0,\sk_1 < 2^{128} \textrm{\ and\ } 
 J_\pk = \sk_0 \genJ_0 + \sk_1 \genJ_1 + d \genJ_2
} \mathperiod $$

We now prove \pisk only once {\it ever} during secret key generation,
which largely eliminates any side channel risks.
We do ask verifiers compute more pairings, but nobody cares when
the VRF verifiers are few in number or institutional,
 as in many applications.
We also ask provers rerandomize both \pisk and \pipk, but this costs relatively little.
Assuming \pipk is Groth16 then we need a proof-of-knowledge for the desired structure of $J_pk$ too.
All totaled this almost doubles the size and complexity of our ring VRF signature.

There is no ``arrow of time'' among zk continuations per se, but
as \pisk bridges between the \PedVRF and \pipk,
one might consider this zk continuation to be time reversed.

Interestingly, our \pipk has become simple enough that opening
a KZG commitment \cite{KZG} at a secret point now suffices,
 ala Caulk+ \cite{caulk+}, or perhaps Caulk \cite{caulk}.

% \subsection{Ring blockchains}
% \label{subsec:rvrf_side_channel}
\smallskip

\def\comblock{\ensuremath{\mathsf{comblock}}\xspace}
\newcommand\pichain{\ensuremath{\pi_{\mathtt{chain}}}\xspace}

We could build our ring using a blockchain,
 not unlike how zcash handles their UTXOs.
If so, we suggest \pifast provide an inner per block or per epoch ring,
but then another proving system like Caulk+ \cite{caulk+}
 extend this ring to the growing blockchain, like
%
$$ \pichain = \rrSNARKweak \Setst{ \comblock, \comring }{
	\exists \openring \textrm{\ s.t.\ }
	J_\pk = \rVRF.\OpenRing(\comring,\openring)
} \mathcomma \quad\textrm{and} $$ 
%
$$ \pifast = \rrSNARK \Setst{ \comblock, \comring }{
	\exists d,\openring \textrm{\ s.t.\ }
	\genfrac{}{}{0pt}{}{ \rVRF.\OpenRing(\comblock,\openring) }{ \,\, = \sk_0 \genJ_0 + \sk_1 \genJ_1 + d \genJ_2 }
} \mathperiod $$  

Again we need a proof=of-knowledge for \comblock because it winds up opaque
 in both of these proofs. 

\smallskip

A priori, our JubJub representations $\sk_0 \genJ_0 + \sk_1 \genJ_1$
costs us exculpability from Definition \ref{def:rvrf_exculpability}.
If desired, exculpability could easily be repaired by asking users
provide $\pisk$ when joining the ring.

% \begin{proposition}
% \end{proposition}


\endinput

