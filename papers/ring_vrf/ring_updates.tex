\section{Ring commitments}

We now discuss strategies and tooling for the ring commitment scheme
$\rVRF.\{ \CommitRing, \CommitKey, \OpenKey \}$.
We mentioned public keys in $\grE$ being represetated in $\grJ$ already,
but otherwise treated \pifast and \pisafe largely like a black box,
instead focusing upon the implementation details of 
 $\PedVRF.\{ \CommitKey, \OpenKey \}$.

% \subsection{Ring updates}

Although $\rVRF.\rSign$ runs fast, all users must update their stored
SNARK \pifast every time the ring $\ctx$ changes.
Almost any circuit works for \pifast though,
 which permits diverse optimizations depending upon usecase.

\subsection{Updates: Vector commitments}

In either \pifast and \pisafe configurations, 
our $\rVRF.\{ \CommitRing, \CommitKey, \OpenKey \}$ could implement a
Merkle tree using SNARK friendly hash functions like Poseidon \cite{poseidon}.
%
All users need $O(\log |\ctx|)$ data with every update, which sounds
reasonable but not free.  There is a fast moving literature on securing
and optimizing SNARK friendly hash functions, with different techniques
being better suited to different SNARKs or even curves.

TODO: Arity 9 for 300 constraints?   % \cite{groth16} vs plookup \cite{plookup}.

Instead of Merkle trees, our SNARk $\pi$ could use polynomial based
vector commitments \cite{KZG} or so called ``Verkle trees'' \cite{??Verkle??} too.
%
Aggregatable subvector commitments \cite{aSVC} permit one server to
compute a KZG commitment \comring together with every users' ring opening
\openring,  and then send each user their \openring encrypted.
Instead of Groth16 \cite{groth16}, our \pisafe then consists
of blinding this KZG commitment and then open it in zero-knowledge.

TODO: Explain zero-knowledge KZG opening positions?!?

We cannot replace \pifast with a KZG commitment directly like this.
We could however replace $\pisafe'$ with a zero-knowledge continuation
that swaps the KZG blinding for the secret key blinding used by \pifast.

$$ \pifast' = \rrSNARK \Setst{ \sk_0 + \sk_1 2^128, \pk }{ 
 \exists b' \textrm{\ s.t.\ }
 % 0 < \sk_0,\sk_1 < 2^128 \textrm{\ and\ } 
 \pk = \sk_0 \genJ_0 + \sk_1 \genJ_1 + b' \genJ_2
} $$ % \mathperiod 

\subsection{Updates: Certificate}

If an authority grants ring membership, then ring membership proofs
could simply verify some certificate by the authority, likely using
a signature on JubJub.

In this, we prefer a SNARK friendly random oracle,
because conventional random oracles cost like 30k constraints.
We also need a variable base scalar multiplication, which costs like
4k constraints, as well as a couple fixed base scalar multiplication.
A priori, these fixed base scalar multiplications cost roughly 700
constraints each, but ocasionally they cost only half this.   

We conjecture one fixed based scalar multiplication could be replaced
by adapting implicit certificate scheme technqiues,
 instead of simply a signature on a user provided key.

 \subsection{Updates: Revokation}

We typically need expiration dates in certificates, likely demanding
a range proof and demanding that \pifast be recomputed ocasionally. 

A priori, we cannot revoke ring membership when using certificates,
but revokation of ring membership winds up rare in practice, especially
since ring membership cannot be traced to missbehavior at any particular site.

We could therefore employ a seperate revokation list and inside \pifast
do a non-membership proof like \cite{???}.
In this way, we update \pifast only when the revokation list updates,
far less often than the full ring \ctx itself updates, and likely
corresponding with expiration checks.

\subsection{Updates: Append only logs}

If an append only log grants ring membership, then a recursive SNARK
could validate ring membership with each recursive addition being
relatively inexpensive.

In this, we need a $\pifast_0$ similar to \pifast as well as a
$\pifast_n$ that proves some $\comring_{n-1}$ to be the ancestor of
its own $\comring_n$ and recursively proves some $\pifast_{n-1}$ with
its own $\comring_{n-1}$ and the same $\sk$.
We expect half pairing cycles fit this usage nicely, although they complicate provers.

Append only logs wind up still being updates whenever \ctx updates
however, so they benefit less than certificates from revokation lists.

A priori, we expect prover complexity plus update fequency makes
append only logs suboptimal, but
 they remain an interesting corner of the design space.

\subsection{Hiding rings} % ring membership circuits}
\label{subsec:hiding_rings}

At first, one imagines sites would accept few rings because each ring
gives some users multiple ``Sybil'' identities within the site.
In practice however, we think many sites benefit from accepting
multiple overlapping rings for convenience, reach, etc. but then
tollerating the resulting few ``Sybil'' users.

As sites accept more rings, we increase risks that each user's ring
\ctx reveals private user attributes, especially if
 users join many rings, sites accept many rings, and
 user agents manage the association poorly.
As a solution, we suggest tweaking \pifast to prove the ring itself
lies in some permitted set of rings, but hide the specific ring used.

We could achieve this using recursion inside \pifast of course,
but doing so lies out of scope.  We instead discuss using other
zk continuation techniques or similar.

As a first step, if all rings use the same circuit, then we hide the
ring through openning a blinded polynomial commitment \cite{KZG}: 
In \S\ref{subsec:rvrf_faster}, our \pifast takes public input
 $X = \comring\, Y_0 + \compk$ where $\compk = \sk\, Y_1 + b \genB_\gamma$.
Instead of revealing \comring, we prove correctness of \comring in
 $X'' = \comring\, Y_0 + b'' \genB_\gamma$.
We prove $X''$ has this structure by opening a polynomial commitment
\cite{KZG}, but over a larger slower recursive elliptic curve
 like BW6 \cite{BW6}.

TODO: More details?

If using $\pisafe$ anyways then we could prove correctness for \comring
using $\pisafe'$ too, which saves pairings over adding KZK.

\smallskip

Although $\gamma=1$ remains viable, all circuits wind up with
unique $\delta$ and hence unique $\ecE_2$ SRS element $[\delta]_2$.
We should again prove correctness of $[\delta]_2$ using a
blinded polynomial commitment \cite{KZG} over BW6 \cite{BW6},
except this time a multipicative blinding works better.

TODO: More details?

At this point, we have blinded and proven correct both the
ring commitment \comring and the circuit commitment $[\gamma]_2$.
A priori, \pifast chooses $\genG = Y_1$, which reveals the circuit too, like
$$ Y_1 = [{1\over\gamma} (\beta u_1(\tau) + \alpha v_1(\tau) + w_1(\tau))]_1 \mathperiod $$

Instead, we propose $Y_{1,\gamma}$ be chosen independent before the circuit.
We then merely add an SRS element $Y_{1,\delta}$, for usage in $C$, that binds
 our independent $Y_{1,\gamma}$ to the desired definition, so
$$ Y_{1,\delta} := [{1\over\delta} (\beta u_1(\tau) + \alpha v_1(\tau) + w_1(\tau) - \gamma Y_{1,\gamma})]_1 \mathperiod $$
At this point, we replace $Y_1$ by $Y_{1,\gamma}$ everywhere and
 include $\comring \, Y_{1,\delta}$ inside $C$.

In this way, all ring membership circuits could share identical
public input SRS points $Y_{1,\gamma}$, and similarly $Y_0$ if desired.

% Interestingly the SRS ceremony could safely output points for both forms

\subsection{SnarkPack}

TODO: Handle $\pi$ hashes?


