\section{Ring updates}
\label{sec:ring_updates}

% We discussed \pifast, or \pisk and \pipk, representing public keys
%  in $\grE$ in $\grJ$ already,
% % along with circuit implementation details of $\PedVRF.\{ \CommitKey, \OpenKey \}$,
% but otherwise mostly treated the ring commitment scheme
% $\rVRF.\{ \CommitRing, \OpenRing \}$ like a black box.

We now discuss the performance of \pifast.
Although our $\rVRF.\rSign$ runs fast, all users should update their
stored zkSNARK \pifast every time \ring changes, % which requires reproving \OpenRing.
but zero knowledge continuations help here too.


\subsection{Merkle trees} % {Poseidon}

Our $\rVRF.\{ \CommitRing, \OpenRing \}$ could implement a Merkle tree
using a zkSNARK friendly hash function like Poseidon \cite{poseidon},
giving $O(\log |\ring|)$ prover time.
%
At least one Poseidon \cite{poseidon} provides arity four with only
600 R1CS constraints.  We need roughly 700 R1CS constraints for each
fixed based scalar multiplication too, so the flavor of \pifast costs
under 12k R1CS constraints for a ring with four billion people.

% TODO: Arity 9 for 300 constraints?   % \cite{Groth16} vs plookup \cite{plookup}.


\subsection{Side channels}
\label{subsec:rvrf_side_channel}

In \pifast, one might dislike processing secret key material inside
the Groth16 prover for \pifast.
Adversaries could trigger \pifast recomputation only by updating the ring,
but this still presents a side channel risk.

If concerned, one could address this via a second zk continuation that
splits \pifast into a Groth16 \pisk and a Groth16 or KZG \pipk for two
respective languages:
%
$$ \Lpk^\inner = \Setst{ J_\pk, \comring }{
	\exists \openring \textrm{\ s.t.\ }
	J_\pk = \eprint{\rVRF.}{}\OpenRing(\comring,\openring)
} \mathcomma \eprint{\quad\textrm{and}}{} $$ 
%
$$ \Lsk^\inner = \Setst{ \sk_0 + \sk_1 2^{128}, J_\pk }{ 
	\exists d \textrm{\ s.t.\ }
	% 0 < \sk_0,\sk_1 < 2^{128} \textrm{\ and\ } 
	J_\pk = \sk_0 \genJ_0 + \sk_1 \genJ_1 + d \genJ_2
} \mathperiod $$

We now prove \pisk only once {\it ever} during secret key generation,
which largely eliminates any side channel risks.
We do ask verifiers compute more pairings, but nobody cares when
the VRF verifiers are few in number or institutional,
as in many applications.
We also ask provers rerandomize both \pisk and \pipk, but this costs relatively little.
Assuming \pipk is Groth16 then we need a proof-of-knowledge for the desired structure of $J_\pk$ too.
All totaled this almost doubles the size and complexity of our ring VRF signature.

There is no ``arrow of time'' among zk continuations per se, but
as \pisk bridges between the \PedVRF and \pipk,
one might consider the \pisk-to-\pipk continuation to be ``time reversed'',
 in that the ``middle'' continuation is proved first.


\subsection{Polynomial commitments}
\label{subsec:rvrf_caulk}

As \pipk became rather simple, % in \S\ref{subsec:rvrf_side_channel},
there exists an alternative formulation:  
\comring could be a KZG polynomial commitment \cite{KZG} to users' $J_\pk$s,
while \pipk itself becomes an opening at a secret location, like
Caulk+ \cite{caulk+} or Caulk \cite{caulk}.
We benefit from faster ring updates this way, but pay in
 increased verifier time and increased marginal prover time.


\subsection{Append only rings}

\newcommand\pichain{\ensuremath{\pi_{\mathtt{chain}}}\xspace}

As a slight variation,
we could build \ring using append only structures like some blockchains,
in which case we should split $\rVRF.\OpenRing$ differently between
an inner ring block or epoch proof $\baseL_{\mathtt{block}}$,
 which we only prove once like \pisk above, and
a chain state proof $\baseL_{\mathtt{chain}}$,
 which extends this inner ring to the growing blockchain.
Now our inner SNARKs pass a $\mathtt{blk}$ parameter, which 
our zero-knowledge continuation transforms into a opaque commitment
$\mathtt{comblk}$, thereby requiring a proof-of-knowledge.
%
$$ \baseL_{\mathtt{chain}}^\inner = \Setst{ \mathtt{blk}, \mathtt{chain} }{
	\mathtt{blk} \in \mathtt{chain}
} \mathcomma \quad\textrm{and} $$ 
%
$$ \baseL_{\mathtt{block}}^\inner = \Setst{ \sk_0 + \sk_1 2^{128}, \mathtt{blk} }{
	\eprint{\exists d,\openring \textrm{\ s.t.\ }}{}
	\genfrac{}{}{0pt}{}{ \OpenRing(\mathtt{blk},\openring) }{ \,\, = \sk_0 \genJ_0 + \sk_1 \genJ_1 + d \genJ_2 }
} \mathperiod $$  
%
We suggest appending $\mathtt{blk}$ to a polynomial commitment using
\cite{aSVC}, which then $\baseL_{\mathtt{chain}}$ blind opens
via Caulk+ \cite{caulk+} as above.


\subsection{Expiration and revocation}

We expect expiration and revocation would be required for append only
rings like blockchains, or say a zero-knowledge proof of a certificate.
% proof-of-personhood parties \cite{pop2008,pop2017}

For expiry, we suggest \pisk or $\baseL_{\mathtt{block}}$ commit to
the expiration date alongside the secret key in their $X$, and
then \pipk or $\baseL_{\mathtt{chain}}$ enforce expiration, but
 really even \PedVRF could enforce expiration.

A revocation list could be enforced by a non-membership proof in
\pipk or $\baseL_{\mathtt{chain}}$.
We expect a revocation list updates only rarely compared with \ring
itself though, which makes doing this non-membership proof inside some
separate zero-knowledge continuation tempting too.
A deployment faces should make this choice carefully.

% We already trust an authority with issuing certificates, so we trust them with managing the revocation list too.  
% As such, our revocation list non-membership proofs merely requires proving adjacency of the revoked public keys lexicographically before and after our own public key.
% If the revocation list requires secrecy, then VRFs could hide its ordering,
% similarly to NSEC5 \cite{nsec5}.

