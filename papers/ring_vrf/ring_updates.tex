\section{Ring updates}
\label{sec:ring_updates}

% We discussed \pifast, or \pisk and \pipk, representing public keys
%  in $\grE$ in $\grJ$ already,
% % along with circuit implementation details of $\PedVRF.\{ \CommitKey, \OpenKey \}$,
% but otherwise mostly treated the ring commitment scheme
% $\rVRF.\{ \CommitRing, \OpenRing \}$ like a black box.

Although our $\rVRF.\rSign$ runs fast, all users must update their
stored zkSNARK \pifast every time \ring changes, which requires reproving \OpenRing.


\subsection{Merkle trees} % {Poseidon}

Our $\rVRF.\{ \CommitRing, \OpenRing \}$ could implement a Merkle tree
using a zkSNARK friendly hash function like Poseidon \cite{poseidon},
giving $O(\log |\ring|)$ prover time.
%
At least one Poseidon \cite{poseidon} provides arity four with only
600 R1CS constraints.  We need roughly 700 r1CS constraints for each
fixed based scalar multiplication too, so the flavor of \pifast costs
under 12k R1CS constraints for a ring with four billion people.

% TODO: Arity 9 for 300 constraints?   % \cite{Groth16} vs plookup \cite{plookup}.


\subsection{Side channels}
\label{subsec:rvrf_side_channel}

In \pifast, one might dislike processing secret key material inside
the Groth16 prover for \pifast.
Adversaries could trigger \pifast recomputation only by updating the ring,
but this still presents a side channel risk.

If concerned, one could address this via a second zk continuation that
splits \pifast into a Groth16 \pisk and a Groth16 or KZG \pipk for two
respective relations:
%
$$ \Rpk = \Setst{ J_\pk, \comring }{
	\exists \openring \textrm{\ s.t.\ }
	J_\pk = \eprint{\rVRF.}{}\OpenRing(\comring,\openring)
} \mathcomma \eprint{\quad\textrm{and}}{} $$ 
%
$$ \Rsk = \Setst{ \sk_0 + \sk_1 2^{128}, J_\pk }{ 
	\exists d \textrm{\ s.t.\ }
	% 0 < \sk_0,\sk_1 < 2^{128} \textrm{\ and\ } 
	J_\pk = \sk_0 \genJ_0 + \sk_1 \genJ_1 + d \genJ_2
} \mathperiod $$

We now prove \pisk only once {\it ever} during secret key generation,
which largely eliminates any side channel risks.
We do ask verifiers compute more pairings, but nobody cares when
the VRF verifiers are few in number or institutional,
as in many applications.
We also ask provers rerandomize both \pisk and \pipk, but this costs relatively little.
Assuming \pipk is Groth16 then we need a proof-of-knowledge for the desired structure of $J_\pk$ too.
All totaled this almost doubles the size and complexity of our ring VRF signature.

There is no ``arrow of time'' among zk continuations per se, but
as \pisk bridges between the \PedVRF and \pipk,
one might consider the \pisk-to-\pipk continuation to be time reversed.


\subsection{Polynomial commitments}

As \pipk became rather simple, % in \S\ref{subsec:rvrf_side_channel},
there exists an alternative formulation:  
\comring could be a KZG polynomial commitment \cite{KZG} to users' $J_\pk$s,
while \pipk itself becomes an opening at a secret location, like
Caulk+ \cite{caulk+} or Caulk \cite{caulk}.
We clearly benefit from much faster ring updates this way, but pay in
 increased verifier time and increased marginal prover time.


\subsection{Append only blockchains}

\def\comblock{\ensuremath{\mathsf{comblock}}\xspace}
\newcommand\pichain{\ensuremath{\pi_{\mathtt{chain}}}\xspace}

We could build \ring using append only structures like some blockchains,
in which case we should split $\rVRF.\OpenRing$ differently between
an inner ring block or epoch proof $\baseR_{\mathtt{block}}$,
 which we only prove once like \pisk above, and
a chain state proof $\baseR_{\mathtt{chain}}$,
 which extends this inner ring to the growing blockchain, and
 admits alternatives like KZG and Caulk+ \cite{caulk+}.
%
$$ \baseR_{\mathtt{chain}} = \Setst{ \mathtt{blk}, \mathtt{chain} }{
	\mathtt{blk} \in \mathtt{chain}
} \mathcomma \quad\textrm{and} $$ 
%
$$ \baseR_{\mathtt{block}} = \Setst{ \sk_0 + \sk_1 2^{128}, \mathtt{blk} }{
	\eprint{\exists d,\openring \textrm{\ s.t.\ }}{}
	\genfrac{}{}{0pt}{}{ \OpenRing(\mathtt{blk},\openring) }{ \,\, = \sk_0 \genJ_0 + \sk_1 \genJ_1 + d \genJ_2 }
} \mathperiod $$  

Again here we need a proof-of-knowledge for $\mathtt{blk}$ because
it winds up opaque to verifiers.


\subsection{Expiration} % {Certificates} % \& revokation}

We could build \piring or \pipk that merely check a certificate or
credential issued by some authority too.
We do not explore this direction except to observe: 
All certificates need expiry dates because certificates are inherently
``append only''.  We think all append only \ring constructions similarly
benefit from expiration dates, including append only blockchains, and
also proof-of-personhood parties \cite{pop2008,pop2017}.
We suggest \pisk or $\baseR_{\mathtt{block}}$ emit the expiration date
alongside the secret key in $X$, and \pipk or $\baseR_{\mathtt{chain}}$
enforce expiration, but really even \PedVRF could enforce expiration.

% In this, we prefer a SNARK friendly random oracle, because conventional random oracles cost like 30k constraints.
% We also need a variable base scalar multiplication, which costs like 4k constraints, as well as a couple fixed base scalar multiplication.
% A priori, these fixed base scalar multiplications cost roughly 700 constraints each, but occasionally they cost only half this.   
%
% We conjecture one fixed based scalar multiplication could be replaced by adapting implicit certificate scheme techniques,  instead of simply a signature on a user provided key.


\subsection{Revocation}

As a rule, one needs some revocation path for certificates,
despite the underlying signature not being revocable, but again
revocation improves all append only \ring designs.
%
We suggest revocation list be maintained separately from an append only
blockchain and then do a non-membership proof from the revocation list
in \pipk or $\baseR_{\mathtt{chain}}$.  % perhaps via \cite{???}.
These non-revocation proofs provide more nuanced \ring membership
control, without reproving \pisk or $\baseR_{\mathtt{block}}$.
We expect a revocation list update only rarely compared with \ring itself.

% We already trust an authority with issuing certificates, so we trust them with managing the revocation list too.  
% As such, our revocation list non-membership proofs merely requires proving adjacency of the revoked public keys lexicographically before and after our own public key.
% If the revocation list requires secrecy, then VRFs could hide its ordering,
% similarly to NSEC5 \cite{nsec5}.



\endinput




\subsection{Append only logs}

If an append only log grants ring membership, then a recursive SNARK
could validate ring membership with each recursive addition being
relatively inexpensive.

In this, we need a $\pipk^0$ similar to \pipk as well as a
$\pipk^n$ that proves some $\comring_{n-1}$ to be the ancestor of
its own $\comring_n$ and recursively proves some $\pifast^{n-1}$ with
its own $\comring_{n-1}$ and the same $\sk$.
We expect half pairing cycles fit this usage nicely, although they complicate provers.

Append only logs still depend \pipk be reproven whenever
\ring updates however, so they only reduce bandwidth they not CPU usage.
We shy away from such append only log optimizations, due to this
prover complexity and our desire for revocation, but
 they remain an interesting corner of the design space.
