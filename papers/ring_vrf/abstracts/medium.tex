
\def\eprintsmallskip{\smallskip}{}%
We introduce a new cryptographic primitive, aptly named
 \emph{ring verifiable random functions (ring VRF)},
 which provides an array of uses, especially in anonymous credentials.
% Anonymized
%\eprint{Ring VRFs are}{We introduce ring VRFs, which are}
Ring VRFs are (anonymized) ring signatures that prove correct evaluation
of an authorized signer's PRF, while hiding the specific signer's
identity within some set of possible signers, known as the ring.

We discover a family of ring VRF protocols with surprisingly efficient
instantiations, thanks to our novel {\em zero-knowledge continuation} technique.
%
Intuitively our ring VRF signers generate two linked proofs,
 one for PRF evaluation and one for ring membership. 
An evaluation proof needs only a cheap Chaum-Pedersen DLEQ proof,
while ring membership proof depends only upon the ring itself.
We reuse this ring membership proof across multiple inputs
by expanding a Groth16 trusted setup to re-hide public inputs when
 rerandomizing the Groth16.
%
Incredibly, our fastest amortized ring VRF needs only eight $\mathcal{G}_1$
 and two $\mathcal{G}_2$ scalar multiplications, 
making it the only ring signature with performance competitive with group signatures.

\eprintsmallskip

We discuss applications that range across the anonymous credential space:
As in proof-of-personhood work by Bryan Ford, et al.,
a ring VRF output acts like a unique pseudo-anonymous identity
 within some desired context, given as the ring VRF input,
but remains unlinkable between different contexts. 
These unlinkable but unique pseudonyms provide a better balance between
user privacy and service provider or social interests than attribute
based credentials like IRMA (``I Reveal My Attributes'') credentials.

\eprintsmallskip
Ring VRFs support anonymously rationing or rate limiting resource
consumption that winds up vastly more flexible and efficient than
purchases via money-like protocols.

\eprintsmallskip
We define the security of ring VRFs in the universally composable (UC) model and show that our protocol is UC secure.
