
\def\eprintsmallskip{\smallskip}{}%
We introduce a new cryptographic primitive,  named
\emph{ring verifiable random function (ring VRF)}.
% which enables better anonymous credentials...
% Anonymized
%\eprint{Ring VRFs are}{We introduce ring VRFs, which are}
Ring VRF is a ring signature that proves correct evaluation
of some authorized signer's PRF, while hiding the signer's
identity within a ring, some set of possible signers. We design a ring VRF protocol which has efficient instantiations with our novel {\em zero-knowledge continuation} technique.
% \eprint{We propose ring VRFs as a natural fulcrum around which a diverse array of zkSNARK circuits turn, making them an ideal target for optimization and eventually standards.}{}
We demonstrate a {\em zero-knowledge continuation} technique,
which works by adjusting a Groth16 trusted setup to hide public inputs
when rerandomizing the Groth16, ensuring that muliple uses of a proof generated once are unlinkable.  We then build ring VRFs that amortizes
expensive ring membership proofs across many ring VRF signatures.
%
Our ring VRF needs only eight $\mathcal{G}_1$ and two
$\mathcal{G}_2$ scalar multiplications, making it the only ring signature
with performance competitive with group signatures.

A ring VRF can be used to obtain a unique pseudo-anonymous identity from a given a list of identities.
By using a different input for the ring VRF in different contexts, we can generate a pseudonym for each context that is unlinkable between different contexts. 
We discuss applications that range across the anonymous credential space.

%Ring VRFs produce a unique identity for any given context but remain
%unlinkable between different contexts.  These unlinkable but unique
%pseudonyms provide a far better balance between user privacy and service
%provider or social interests than attribute based credentials like IRMA credentials.
%Ring VRFs also support anonymously rationing or rate limiting resource
%consumption that winds up vastly more flexible and efficient than
%purchases via money-like protocols.

We define the security of ring VRFs in the universally composable (UC) model and show that our protocol is UC secure.
