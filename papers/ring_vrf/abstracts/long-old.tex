A {\it ring verifiable random function} (ring VRF) is a ring signature
that proves correct evaluation of some pseudo-random function (PRF)
determined by the actual key pair used in signing, but while hiding
the actual signer's identity within some set of possible signers,
known as the ring.

\smallskip

We show ring VRFs that amortize their ring membership proof across many
ring VRF signatures.
%
For this, we devised a {\em zero-knowledge continuation} technique that
reuses a previously proven Groth16 SNARK without reproving.
We wire this Groth16 SNARKs inputs into another signature, or another SNARK, 
but avoid linking the multiple reuses by adjusting the Groth16 trusted setup
to reblind the public inputs when rerandomizing the Groth16.
%
Incredibly, our ring VRF needs only eight $\mathcal{G}_1$ and two
$\mathcal{G}_2$ scalar multiplications, irregardless of the ring size!

\smallskip

Ring VRFs provide a flexible  but straightforward framework to address
diverse deployment and privacy problems, so they could truly bring
anonymous credentials into the main stream.

Ring VRFs produce a unique identity for any give context but remain
unlinkable between different contexts.  These unlinkable but unique
pseudonyms provide a far better balance between user privacy and service
provider or social interests than attribute based credentials like IRMA credentials.

Ring VRFs support anonymously rationing or rate limiting resource
consumption that winds up vastly more efficient than purchases via
money-like protocols.  Importantly ``rings'' are far cleaner to audit
than certificates, in principle reducing fraud and improving trust.

We discuss anonymous credentials face malicious verifier problems,
but how ring VRF inputs could include their root of trust, which
avoids third party CAs and complexities like certificate transparency.
