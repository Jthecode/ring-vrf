\section{Conclusion}
\label{sec:conclusion}

We introduced a novel cryptographic primitive ring VRF in this paper which combines the unique properties of VRFs  and ring signatures. Our new primitive has notable use cases in identity systems, where users can register their public keys and generate pseudonyms using Ring VRF outputs, ensuring privacy protection while preventing Sybil behaviour. Ring VRF finds applications in a wide range of other cases, including rate limiting systems, rationing, and leader elections. We presented two distinct Ring VRF constructions, one offering flexibility in instantiation and the other focusing on optimizing signature generation within the same ring. Moreover, we introduced the notion of ZK continuations enabling the efficient regeneration of proofs by preserving the ZK property.

\paragraph{Instantiation of our first protocol:}  Our instantiation commits to the ring using KZG commitments to the $ x $ and $ y $ coordinates of the public key. This means that we do not need constraints for opening the KZJ commitment as the pulic keys are directly available to use in constranits.  So the constraint system is simple and we can use a custom SNARK for $\Rring$ similar to the construction in \cite{accountable}, modified to obtain zero-knowledge.  For this protocol, the prover needs to know the entire ring, i.e. $\OpenRing$ is the entire ring rather than a KZG opening, which results in $O(n)$ proving time unlike in the second protocol and it does not allow fast reproving.
 Verification takes $O(1)$ time. Without needing opening constraints, it is concretely fast with proving time under a second for rings of size up to a few thousand (comparable to the benchmarks in \cite{accountable}).  



\paragraph{Instantiation of our second protocol with $ \SpecialG $:} Since $ \SpecialG $ is $ \ZKCont $, we can instantiate our second protocol with $ \SpecialG$. In this instantiation, we let $ \GG = \grone $ generated in $ \SpecialG.\Setup $.  
We present an appropriate $ \mathsf{Com}.\mathsf{Commit}(\sk) $ algorithm that together with $ \SpecialG $ efficiently instantiate the NIZK for $ \Rring^{\mathtt{inner}} $. To make this efficiently provable inside the SNARK,  we use the Jubjub Edwards curve $\ecJ$ which contains a large subgroup $\grJ$ of prime order $p_\grJ$. Here, $p_\grJ < p$ where $ p $ is the order of $\grE$ used in our ring VRF construction\eprint{\footnote{This condition can be satisfied if $\ecJ$ is an Edwards curve with a cofactor.}}{}. We let $\genJ_0,\genJ_1,\genJ_2 \in \grJ$ be independent generators. We also fix a parameter $ \kappa $ where $(\log_2 p)/2 < \kappa < \log_2 p_\grJ$. $ \mathsf{Com}.\mathsf{Commit}(\sk) $ first samples $\sk_1,\sk_2 \in 2^\kappa$  where $\sk = \sk_0 + \sk_1 \, 2^{\lambda} \mod p$ and samples a blinding factor $d \leftsample \F_{p_\grJ} $. In the end, it outputs $ \sk_0, \sk_1,d $ as an opening and the commitment $\pk=\sk_0\, \genJ_0 + \sk_1\, \genJ_1 + d \genJ_2$ as a public key of our ring VRF construction. This commitment scheme is binding and perfectly hiding as our ring VRF construction requires because $ \pk $ is, in fact, a Pedersen commitment. Indeed, $\pk$ is a Pedersen commitment to $\sk$ because we can represent $ \sk = \sk_0\, \genJ_0 + \sk_1 \mod p$ since we have selected $ \kappa $ accordingly.

We can instantiate $ \com^*$ with a Merkle tree hash by letting the leaves the public keys in the ring. Then, we instantiate $ \NARK_{\relcomring}.\Prove $ with inclusion proving of a key with respect to the Merkle tree root.
TODO:Why Merkle tree? What is the relation with Groth  16?

The first run of $\rVRF.\Sign$ for $\ring$ with $ \SpecialG $ runs =$\SpecialG.\Preprove$ and $ \SpecialG.\Reprove$ which consists of  7 multiplications in $\grone $ and $3$ multiplications in $\grtwo$ and then runs $\NIZK_{\rel_{eval}}.\Prove$ which need  3 multiplications in $ \grone $.
%TODO 2G1+1G actually
For the next signatures for the same ring,  $\rVRF.\Sign$  runs only  $\SpecialG.\Reprove$ which is 4 multiplications in $\grone $ and $2$ multiplications in $\grtwo$.




