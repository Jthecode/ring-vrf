%%
%% This is file `sample-sigconf.tex',
%% generated with the docstrip utility.
%%
%% The original source files were:
%%
%% samples.dtx  (with options: `sigconf')
%% 
%% IMPORTANT NOTICE:
%% 
%% For the copyright see the source file.
%% 
%% Any modified versions of this file must be renamed
%% with new filenames distinct from sample-sigconf.tex.
%% 
%% For distribution of the original source see the terms
%% for copying and modification in the file samples.dtx.
%% 
%% This generated file may be distributed as long as the
%% original source files, as listed above, are part of the
%% same distribution. (The sources need not necessarily be
%% in the same archive or directory.)
%%
%%
%% Commands for TeXCount
%TC:macro \cite [option:text,text]
%TC:macro \citep [option:text,text]
%TC:macro \citet [option:text,text]
%TC:envir table 0 1
%TC:envir table* 0 1
%TC:envir tabular [ignore] word
%TC:envir displaymath 0 word
%TC:envir math 0 word
%TC:envir comment 0 0
%%
%%
%% The first command in your LaTeX source must be the \documentclass
%% command.
%%
%% For submission and review of your manuscript please change the
%% command to \documentclass[manuscript, screen, review]{acmart}.
%%
%% When submitting camera ready or to TAPS, please change the command
%% to \documentclass[sigconf]{acmart} or whichever template is required
%% for your publication.
%%
%%
\documentclass[sigconf]{acmart}

%%
%% \BibTeX command to typeset BibTeX logo in the docs
\AtBeginDocument{%
  \providecommand\BibTeX{{%
    Bib\TeX}}}

%% Rights management information.  This information is sent to you
%% when you complete the rights form.  These commands have SAMPLE
%% values in them; it is your responsibility as an author to replace
%% the commands and values with those provided to you when you
%% complete the rights form.
\setcopyright{acmcopyright}
\copyrightyear{2018}
\acmYear{2018}
\acmDOI{XXXXXXX.XXXXXXX}

%% These commands are for a PROCEEDINGS abstract or paper.
\acmConference[Conference acronym 'XX]{Make sure to enter the correct
  conference title from your rights confirmation emai}{June 03--05,
  2018}{Woodstock, NY}
%%
%%  Uncomment \acmBooktitle if the title of the proceedings is different
%%  from ``Proceedings of ...''!
%%
%%\acmBooktitle{Woodstock '18: ACM Symposium on Neural Gaze Detection,
%%  June 03--05, 2018, Woodstock, NY}
\acmPrice{15.00}
\acmISBN{978-1-4503-XXXX-X/18/06}


%%
%% Submission ID.
%% Use this when submitting an article to a sponsored event. You'll
%% receive a unique submission ID from the organizers
%% of the event, and this ID should be used as the parameter to this command.
%%\acmSubmissionID{123-A56-BU3}

%%
%% For managing citations, it is recommended to use bibliography
%% files in BibTeX format.
%%
%% You can then either use BibTeX with the ACM-Reference-Format style,
%% or BibLaTeX with the acmnumeric or acmauthoryear sytles, that include
%% support for advanced citation of software artefact from the
%% biblatex-software package, also separately available on CTAN.
%%
%% Look at the sample-*-biblatex.tex files for templates showcasing
%% the biblatex styles.
%%

%%
%% The majority of ACM publications use numbered citations and
%% references.  The command \citestyle{authoryear} switches to the
%% "author year" style.
%%
%% If you are preparing content for an event
%% sponsored by ACM SIGGRAPH, you must use the "author year" style of
%% citations and references.
%% Uncommenting
%% the next command will enable that style.
%%\citestyle{acmauthoryear}

\def\eprint#1#2{#2} % eprint
%%
%% end of the preamble, start of the body of the document source.
% \usepackage{amsthm}
\usepackage{amsfonts}
\usepackage{amsmath}
\usepackage{mathtools}
\usepackage{algorithm}

\usepackage[noend]{algpseudocode}

\usepackage[utf8]{inputenc}
\usepackage{xspace}

\usepackage{url}
\usepackage{hyperref}
% \usepackage[capitalize,nameinlink]{cleveref}

% \usepackage{graphicx}
% \usepackage{xcolor}

% \usepackage{tikz}
% \usetikzlibrary{arrows,chains,matrix,positioning,scopes}

\usepackage{enumitem}
% \setlist[itemize]{leftmargin=*}

\usepackage{etoolbox}
\makeatletter
\patchcmd{\@maketitle}% <cmd>
{\end{center}}% <search>
{\bigskip\small\@date
\end{center}}% <replace>
{}{}% <success><failure>
\makeatother

% Italics should be prefered over underlines..
% \usepackage{ulem}

% Avoid boxes if pssible..
% \usepackage{framed}
% \usepackage{fancybox}
% \usepackage{tcolorbox}
% ..but if required mdframe splits across pages.
% \usepackage[linewidth=1pt]{mdframed}

% TODO: Clean up macros

% \newcommand\doubleplus{+\kern-1.3ex+\kern0.8ex}
\newcommand\doubleplus{\ensuremath{\mathbin{+\mkern-10mu+}}}


%% oracles
\newcommand{\ora}[1]{\ensuremath{\mathcal{O}\mathsf{#1}}\xspace}
\newcommand{\oramsg}[1]{\ensuremath{\mathsf{#1}}\xspace}
%% algorithm
\newcommand{\algo}[1]{{\textsc{#1}}}
%%primitive algo
\newcommand{\primalgo}[1]{{\ensuremath{\mathsf{#1}}}\xspace}
%%primitive
\newcommand{\prim}[1]{{\ensuremath{\mathsf{#1}}}\xspace}
%%set
\newcommand{\setsym}[1]{{\ensuremath{\mathcal{#1}}}}
%%array
\newcommand{\arraysym}[1]{{\ensuremath{\mathsf{#1}}}}

\newcommand\N{\mathbb{N}}
\newcommand\F{\mathbb{F}}
% \newcommand\Gr{\mathbb{G}}


\def\mathperiod{.}
\def\mathcomma{,}



\newcommand*\set[1]{\{ #1 \}} % in text, we don't want {} to grow
\newcommand*\Set[1]{\left\{ #1 \right\}}
\newcommand*\setst[2]{\{ #1 | #2 \}}
\newcommand*\Setst[2]%
        {\left\{\,#1\vphantom{#2} \;\right|\left. #2 \vphantom{#1}\,\right\}}
% ``set such that''; puts in a vertical bar of the right height


\providecommand{\bin}{\ensuremath{\{0,1\}}\xspace}


% https://tex.stackexchange.com/questions/471713/is-mathrel-always-needed
% https://tex.stackexchange.com/questions/418740/how-to-write-left-arrow-with-a-dollar-sign/626086#626086
\makeatletter
\providecommand\leftsample{\leftarrow\mathrel{\mkern-2.0mu}\pc@smalldollar}
\providecommand\rightsample{\pc@smalldollar\mathrel{\mkern-2.0mu}\rightarrow}
%
\newcommand{\pc@smalldollar}{\mathrel{\mathpalette\pc@small@dollar\relax}}
\newcommand{\pc@small@dollar}[2]{%
  \vcenter{\hbox{%
    $#1\textnormal{\fontsize{0.7\dimexpr\f@size pt}{0}\selectfont\$\hskip-0.05em
 plus 0.5em}$%
  }}%
}
\makeatother


\newcommand{\KeyGen}{\primalgo{KeyGen}}

% Why was this \prim before?
\newcommand{\VRF}{\primalgo{VRF}} 
\newcommand{\rVRF}{\primalgo{rVRF}} 

\newcommand{\Sign}{\primalgo{Sign}}
\newcommand{\Verify}{\primalgo{Verify}}
\newcommand{\Eval}{\primalgo{Eval}}
\newcommand{\Prove}{\primalgo{Prove}}
\newcommand{\Simulate}{\primalgo{Simulate}}
\newcommand{\Extract}{\primalgo{Extract}}


\newcommand{\In}{\primalgo{In}} 
\newcommand{\Out}{\primalgo{Out}} 
\newcommand{\PreOut}{\ensuremath{\primalgo{Out}_0}\xspace} 

\newcommand{\vk}{\ensuremath{\mathsf{vk}}\xspace}
\newcommand{\sk}{\ensuremath{\mathsf{sk}}\xspace}
\newcommand{\pk}{\ensuremath{\mathsf{pk}}\xspace}
\newcommand{\apk}{\ensuremath{\mathsf{apk}}\xspace}
\newcommand{\pkring}{\ensuremath{\setsym{PK}}}
\newcommand{\msg}{\ensuremath{\mathsf{msg}}\xspace}
\newcommand{\aux}{\ensuremath{\mathsf{aux}}\xspace}
\newcommand{\ctx}{\ensuremath{\mathsf{ctx}}\xspace}
\newcommand{\ringset}{\ensuremath{\mathsf{ring}}\xspace}



\newcommand\SNARK{\primalgo{SNARK}}
\newcommand\NIZK{\primalgo{NIZK}}


\newcommand{\adv}{\ensuremath{\mathcal{A}}\xspace}

\endinput

\newcommand{\evalprove}{\primalgo{EvalProve}}
\newcommand{\link}{{\primalgo{Link}}}
\newcommand{\update}{{\primalgo{Update}}}
\newcommand{\hashG}{\primalgo{H}_{\GG}}
\newcommand{\secreteval}{\primalgo{Secret}\eval}
\newcommand{\secretprove}{\primalgo{Secret}\prove}
\newcommand{\secretverify}{\primalgo{Secret}\verify}

\newcommand{\randsel}[0]{\ensuremath{\xleftarrow{\text{\$}}}}
\newcommand{\rel}{\ensuremath{\mathcal{R}}}


\endinput




\newcommand{\skvrf}{\ensuremath{\sk^{\mathsf{vrf}}}}
\newcommand{\pkvrf}{\ensuremath{\pk^{\mathsf{vrf}}}}
\newcommand{\skrvrf}{\ensuremath{\sk^{\mathsf{rvrf}}}}
\newcommand{\pkrvrf}{\ensuremath{\pk^{\mathsf{rvrf}}}}
\newcommand{\sksign}{\ensuremath{\sk^{\mathsf{sign}}}}
\newcommand{\pksign}{\ensuremath{\pk^{\mathsf{sign}}}}
\newcommand{\skksign}{\ensuremath{\sk^{\mathsf{kesign}}}}
\newcommand{\pkksign}{\ensuremath{\pk^{\mathsf{kesign}}}}
\newcommand{\pkssale}{\ensuremath{\pk^{\mathsf{ssale}}}}
\newcommand{\D}{\ensuremath{\Delta}}
\newcommand{\skzkvrf}{\ensuremath{\sk^{\mathsf{zkvrf}}}}
\newcommand{\pkzkvrf}{\ensuremath{\pk^{\mathsf{zkvrf}}}}


\newcommand{\tab}[1]{\hspace{.05\textwidth}\rlap{#1}}
\newcommand{\tabdbl}[1]{\hspace{.1\textwidth}\rlap{#1}}
\newcommand{\tabdbldbl}[1]{\hspace{.15\textwidth}\rlap{#1}}
\newcommand{\tabdbldbldbl}[1]{\hspace{.19\textwidth}\rlap{#1}}



\newcommand{\game}[3][]{\operatorname{#2}^{#1}_{#3}(\secpar)}
\newcommand{\transcript}[1]{\langle #1 \rangle}
\newcommand{\eppt}{\pccomplexitystyle{EPPT}}
\newcommand{\pt}{\pccomplexitystyle{PT}}

% \renewcommand{\pcadvstyle}[1]{\ensuremath{\mathsf{#1}}}
% \newcommand{\zdv}{\pcadvstyle{Z}}

% \newcommand{\msg}[1]{\mathsf{#1}}

\newcommand{\simulator}{\ensuremath{\mathsf{Sim}}}
%\newcommand{\minote}[1]{\todo[color=green!30,inline]{\textbf{Michele says:} #1}}

\newcommand{\fvrf}{\mathcal{F}_{\textsf{vrf}}}
\newcommand{\fgvrf}{\mathcal{F}_{\textsf{rvrf}}}
\newcommand{\fcpke}{\mathcal{F}_{\mathsf{CPKE}}}
\newcommand{\pvrf}{\mathsf{\Pi}_{\textsf{rvrf}}}
\newcommand{\svrf}{\simulator_\mathsf{gvrf}}
\newcommand{\fnizk}{\mathcal{F}_{\textsf{nizk}}}
\newcommand{\fkes}{\mathcal{F}_{\textsf{sgke}}}
\newcommand{\fcom}{\ensuremath{\mathcal{F}_{\mathsf{com}}}}
\newcommand{\fsec}{\ensuremath{\mathcal{F}_\mathsf{ED-SMT}}}
\newcommand{\frsc}{\ensuremath{\mathcal{F}_{\mathsf{rSC}}}}
\newcommand{\fsasle}{\ensuremath{\mathcal{F}_{\mathsf{sle}}}}
\newcommand{\finit}{\ensuremath{\mathcal{F}_{\mathsf{init}}}}
\newcommand{\fsig}{\mathcal{F}_{\mathsf{sig}}}
\newcommand{\fros}{\mathcal{F}_{\mathsf{ros}}}
\newcommand{\fzkvrf}{\mathcal{F}_{\mathsf{zkvrf}}}
\newcommand{\fcommit}{\mathcal{F}_{\mathsf{commit}}}
\newcommand{\gclock}{\mathcal{G}_{\mathsf{clock}}}
\newcommand{\fcrs}{\mathcal{F}_{crs}}
\newcommand{\env}{\mathcal{Z}}
\newcommand{\stake}{\mathsf{st}}
\newcommand{\stakeset}{\setsym{ST}}

\newcommand{\sid}{\textsf{sid}}
\newcommand{\pid}{\textsf{pid}}
\newcommand{\user}{\mathsf{P}}
\newcommand{\defeq}{\coloneqq}


\newcommand{\evaluationslist}{\texttt{evaluations}}
\newcommand{\evaluationsecretlist}{\texttt{secrets}}
\newcommand{\vklist}{\texttt{verification\_keys}}
\newcommand{\siglist}{\texttt{signatures}}
\newcommand{\prooflist}{\texttt{proofs}}
\newcommand{\proofzklist}{\texttt{zkproofs}}
\newcommand{\Linklist}{\texttt{links}}
\newcommand{\emptylist}{\emptyset}
\newcommand{\fail}{\mathbf{fail}}
\newcommand{\R}{\mathsf{R}}
\newcommand{\bool}{\textit{bool}}
\newcommand{\lst}{\setsym{L}}
\newcommand{\distribution}{\setsym{D}}

\newcommand{\weak}{\ensuremath{W}}
\newcommand{\inbox}{\ensuremath{\setsym{I}}}
\newcommand{\dqueue}{\ensuremath{\setsym{Q}^\D}}
\newcommand{\wqueue}{\ensuremath{\setsym{Q}^\weak}}
\newcommand{\weaklist}{\ensuremath{\setsym{\weak}}}
\newcommand{\mID}{\ensuremath{\mathsf{mid}}}
\newcommand{\plist}{\ensuremath{\setsym{P}}}
\newcommand{\timeoutlist}{\ensuremath{\setsym{T}}}
\newcommand{\anony}{\ensuremath{\mathfrak{a}}}
\newcommand{\dleqr}{\R_\textsf{dleq}}
\newcommand{\view}{\mathsf{view}} 
\renewcommand{\adv}{\ensuremath{\mathcal{A}}}
\newcommand{\preoutputlist}{\arraysym{pre\-outputs}}



\def\openpk{\ensuremath{\mathsf{b}}\xspace} % redefinition

\renewcommand{\msg}{\ensuremath{\mathsf{input}}\xspace}
\renewcommand{\aux}{\ensuremath{\mathsf{ass}}\xspace}

\newcommand{\PedVRF}{\primalgo{PedVRF}} 

\newcommand\pp{\ensuremath{\mathit{pp}}\xspace}
\newcommand\ppR{\ensuremath{\mathit{pp}_{\mathcal{R}}}\xspace}

\newcommand{\realaux}{\ensuremath{\mathit{aux}}\xspace}
\newcommand\crs{\ensuremath{\mathit{crs}}\xspace}
\newcommand\crspk{\ensuremath{\mathit{crs}_{\mathit{pk}}}\xspace}
\newcommand\crsvk{\ensuremath{\mathit{crs}_{\mathit{vk}}}\xspace}

\newcommand\crsR{\ensuremath{\mathit{crs}_{\mathcal{R}}}\xspace}
\newcommand\crspkR{\ensuremath{\mathit{crs}_{\mathit{pk},\mathcal{R}}}\xspace}
\newcommand\crsvkR{\ensuremath{\mathit{crs}_{\mathit{vk},{\mathcal{R}}}}\xspace}

\newcommand\crspkone{\ensuremath{\mathit{crs}_{\mathit{pk},{\mathcal{R}_1}}}\xspace}

\newcommand\crstwo{\ensuremath{\mathit{crs}_{\mathcal{R}'_2(\pp)}}\xspace}
\newcommand\crspktwo{\ensuremath{\mathit{crs}_{\mathit{pk},{\mathcal{R}'_2}}}\xspace}
\newcommand\crsvktwo{\ensuremath{\mathit{crs}_{\mathit{vk},{\mathcal{R}'_2}}}\xspace}

\newcommand{\gone}{\ensuremath{\mathsf{g}_1}\xspace}
\newcommand{\gtwo}{\ensuremath{\mathsf{g}_2}\xspace}

\newcommand\Kgamma{\ensuremath{K_{\gamma}}\xspace}
\newcommand\Kdelta{\ensuremath{K_{\delta}}\xspace}
\newcommand\cQ{\ensuremath{\mathcal{Q}}\xspace}
\newcommand\cA{\ensuremath{\mathcal{A}}\xspace}
\newcommand\cB{\ensuremath{\mathcal{B}}\xspace}
\newcommand\cC{\ensuremath{\mathcal{C}}\xspace}


\newcommand\barx{\ensuremath{\bar{x}}\xspace}
\newcommand\bary{\ensuremath{\bar{y}}\xspace}
\newcommand\barz{\ensuremath{\bar{z}}\xspace}
\newcommand\barv{\ensuremath{\bar{v}}\xspace}
\newcommand\barsig{\ensuremath{\bar{\sigma}}\xspace}

\newcommand\tw{\ensuremath{\mathit{td}}\xspace}
\newcommand\twone{\ensuremath{\mathit{td}_{\mathcal{R}_1}}\xspace}
\newcommand\twtwo{\ensuremath{\mathit{td}_{\mathcal{R}'_2(\pp)}}\xspace}

\newcommand\twR{\ensuremath{\mathit{td}_{\mathcal{R}}}\xspace}

\newcommand\baromega{\ensuremath{\bar{w}}\xspace}
\newcommand\baromegap{\ensuremath{\bar{w'}}\xspace}
\newcommand\relone{\ensuremath{\mathcal{R}_1}\xspace}
\newcommand\reltwo{\ensuremath{\mathcal{R}_2}\xspace}
\newcommand\relRQ{\ensuremath{\mathcal{R}_{\mathcal{Q}}}\xspace}

\newcommand\baseL{\mathcal{L}}
\newcommand\Lrvrf{\ensuremath{\baseL_{\mathtt{rvrf}}}\xspace}
\newcommand\Leval{\ensuremath{\baseL_{\mathtt{eval}}}\xspace}
\newcommand\Lring{\ensuremath{\baseL_{\mathtt{ring}}}\xspace}
\newcommand\Lfast{\ensuremath{\baseL_{\mathtt{fast}}}\xspace}

\newcommand\baseR{\mathcal{R}}
\newcommand\Reval{\ensuremath{\baseR_{\mathtt{eval}}}\xspace}
\newcommand\Rring{\ensuremath{\baseR_{\mathtt{ring}}}\xspace}
\newcommand\Rfast{\ensuremath{\baseR_{\mathtt{fast}}}\xspace}

\newcommand\hsis{{h'}}
\newcommand\ecEsis{{\mathbb{G}'}}
\newcommand\grEsis{{\mathbf{G}'}}

\newcommand\Lsk{\ensuremath{\baseL_{\mathtt{sk}}}\xspace}
\newcommand\Lpk{\ensuremath{\baseL_{\mathtt{pk}}}\xspace}

\newcommand\rrSNARK{\primalgo{Groth16}\xspace}
\newcommand\rrSNARKweak{\primalgo{Groth16/KZG}\xspace}

\newcommand\negl{\ensuremath{\mathsf{negl}}\xspace}
\newcommand\pieval{\ensuremath{\pi_{\mathtt{eval}}}\xspace}
\newcommand\piring{\ensuremath{\pi_{\mathtt{ring}}}\xspace}

\newcommand\pifast{\ensuremath{\pi_{\mathtt{fast}}}\xspace}
% \newcommand\pifastdot{\ensuremath{\dot{\pi}_{\mathtt{fast}}}\xspace}
\newcommand\pisk{\ensuremath{\pi_{\mathtt{sk}}}\xspace}
\newcommand\pipk{\ensuremath{\pi_{\mathtt{pk}}}\xspace}


%\newcommand{\PoK}{\ensuremath{\primalgo{PoK}}\xspace}
\newcommand{\ccgroth}{\ensuremath{\primalgo{ccGroth16}}\xspace}
\newcommand{\SpecialG}{\ensuremath{\primalgo{SpecialG}}\xspace}
\newcommand{\ZKCont}{\ensuremath{\primalgo{ZKCont}}\xspace}
\newcommand{\Preprove}{\ensuremath{\primalgo{Preprove}}\xspace}
\newcommand{\Reprove}{\ensuremath{\primalgo{Reprove}}\xspace}
\newcommand{\Setup}{\ensuremath{\primalgo{Setup}}\xspace}
\newcommand{\VerifyCom}{\ensuremath{\primalgo{VerCom}}\xspace}
\newcommand{\Sim}{\ensuremath{\primalgo{Sim}}\xspace}
%\newcommand{\nizkone}{\ensuremath{\primalgo{NIZK_{\mathcal{R}_1}}}\xspace}
\newcommand{\nizktwo}{\ensuremath{\primalgo{NIZK_{\mathcal{R}'_2(\mathit{pp})}}}\xspace}
\newcommand{\nizkR}{\ensuremath{\primalgo{NIZK_{\mathcal{R}}}}\xspace}

%\newcommand{\Gen}{\ensuremath{\primalgo{KeyGen}}\xspace}

\newcommand{\inner}{\mathtt{inner}}

\def\maybestack#1#2{\eprint{ #1, #2 }{
    \begin{aligned}
        &#1, \\
        % \exists \openring \textrm{\ s.t.\ }
        &#2  \\      
    \end{aligned}
}}



\usepackage{tcolorbox}
\tcbset{colback=white}
% \usepackage{todonotes}
\sloppy

\begin{document}

%%
%% The "title" command has an optional parameter,
%% allowing the author to define a "short title" to be used in page headers.

\title{Ethical identity, ring VRFs, and zero-knowledge continuations}

%%
%% The "author" command and its associated commands are used to define
%% the authors and their affiliations.
%% Of note is the shared affiliation of the first two authors, and the
%% "authornote" and "authornotemark" commands
%% used to denote shared contribution to the research.
%\author{Ben Trovato}
%\authornote{Both authors contributed equally to this research.}
%\email{trovato@corporation.com}
%\orcid{1234-5678-9012}
%\author{G.K.M. Tobin}
%\authornotemark[1]
%\email{webmaster@marysville-ohio.com}
%\affiliation{%
%  \institution{Institute for Clarity in Documentation}
%  \streetaddress{P.O. Box 1212}
%  \city{Dublin}
%  \state{Ohio}
%  \country{USA}
%  \postcode{43017-6221}
%}
%
%\author{Lars Th{\o}rv{\"a}ld}
%\affiliation{%
%  \institution{The Th{\o}rv{\"a}ld Group}
%  \streetaddress{1 Th{\o}rv{\"a}ld Circle}
%  \city{Hekla}
%  \country{Iceland}}
%\email{larst@affiliation.org}
%
%%
%% By default, the full list of authors will be used in the page
%% headers. Often, this list is too long, and will overlap
%% other information printed in the page headers. This command allows
%% the author to define a more concise list
%% of authors' names for this purpose.



\begin{abstract}
	
\def\eprintsmallskip{\smallskip}{}%
We introduce a new cryptographic primitive,  named
\emph{ring verifiable random function (ring VRF)}.
% which enables better anonymous credentials...
% Anonymized
%\eprint{Ring VRFs are}{We introduce ring VRFs, which are}
Ring VRF is a ring signature that proves correct evaluation
of some authorized signer's PRF, while hiding the specific signer's
identity within some set of possible signers, known as the ring. We design a ring VRF protocol which has efficient instantiations with our novel {\em zero-knowledge continuation} technique.
% \eprint{We propose ring VRFs as a natural fulcrum around which a diverse array of zkSNARK circuits turn, making them an ideal target for optimization and eventually standards.}{}
We demonstrate a reusable {\em zero-knowledge continuation} technique,
which works by adjusting a Groth16 trusted setup to hide public inputs
when rerandomizing the Groth16.  We then build ring VRFs that amortizes
expensive ring membership proofs across many ring VRF signatures.
%
Our ring VRF needs only eight $\mathcal{G}_1$ and two
$\mathcal{G}_2$ scalar multiplications, making it the only ring signature
with performance competitive with group signatures.

Ring VRFs produce a unique identity for any given context but remain
unlinkable between different contexts.  These unlinkable but unique
pseudonyms provide a far better balance between user privacy and service
provider or social interests than attribute based credentials like IRMA credentials.
Ring VRFs also support anonymously rationing or rate limiting resource
consumption that winds up vastly more flexible and efficient than
purchases via money-like protocols.

We define the security of ring VRFs in the universally composable (UC) model and show that our protocol is UC secure.
\end{abstract}

%%
%% The code below is generated by the tool at http://dl.acm.org/ccs.cfm.
%% Please copy and paste the code instead of the example below.
%%
%\begin{CCSXML}
%<ccs2012>
% <concept>
%  <concept_id>10010520.10010553.10010562</concept_id>
%  <concept_desc>Computer systems organization~Embedded systems</concept_desc>
%  <concept_significance>500</concept_significance>
% </concept>
% <concept>
%  <concept_id>10010520.10010575.10010755</concept_id>
%  <concept_desc>Computer systems organization~Redundancy</concept_desc>
%  <concept_significance>300</concept_significance>
% </concept>
% <concept>
%  <concept_id>10010520.10010553.10010554</concept_id>
%  <concept_desc>Computer systems organization~Robotics</concept_desc>
%  <concept_significance>100</concept_significance>
% </concept>
% <concept>
%  <concept_id>10003033.10003083.10003095</concept_id>
%  <concept_desc>Networks~Network reliability</concept_desc>
%  <concept_significance>100</concept_significance>
% </concept>
%</ccs2012>
%\end{CCSXML}

%\ccsdesc[500]{Computer systems organization~Embedded systems}
%\ccsdesc[300]{Computer systems organization~Redundancy}
%\ccsdesc{Computer systems organization~Robotics}
%\ccsdesc[100]{Networks~Network reliability}

%%
%% Keywords. The author(s) should pick words that accurately describe
%% the work being presented. Separate the keywords with commas.
\keywords{VRF, ring signature, zero-knowledge, anonymous credentials}
%% A "teaser" image appears between the author and affiliation
%% information and the body of the document, and typically spans the
%% page.
%\begin{teaserfigure}
%  \includegraphics[width=\textwidth]{sampleteaser}
%  \caption{Seattle Mariners at Spring Training, 2010.}
%  \Description{Enjoying the baseball game from the third-base
%  seats. Ichiro Suzuki preparing to bat.}
%  \label{fig:teaser}
%\end{teaserfigure}

%\received{20 February 2007}
%\received[revised]{12 March 2009}
%\received[accepted]{5 June 2009}

%%
%% This command processes the author and affiliation and title
%% information and builds the first part of the formatted document.
\maketitle

% \section{Introduction}

\def\qaudbreak{\eprint{\quad}{\\}}

We introduce ring verifiable random functions (ring VRFs) as a natural
fulcrum around which anonymous credentials turn, in formalization,
in optimizations, in the nuances of use-cases, and in miss-use resistance.
%
Along with some formalizations, we explain portions of their unfolding
story which address three questions:
\begin{enumerate} 
\item
What are the cheapest SNARK proofs?  \qaudbreak
Ones users reuse without reproving.
% \item
% How can credentials use be contextual?  \qaudbreak
% Prove evaluation of a secret function.
\item
How can identity be safe for general use?  \qaudbreak
By revealing nothing except users' uniqueness.
\item
How can ration card issuance be transparent?  \qaudbreak
By asking users trust a public list, not certificates.
\end{enumerate}

We model the security of ring VRFs in the universally composable (UC) \cite{canetti1,canetti2} and  prove that our ring VRF protocol is UC secure. Thus, we guarantee the strongest security along with  practicality.
% First
\paragraph{Zero-knowledge continuations:}

Rerandomizable zkSNARKs like Groth16 \cite{Groth16} admit a
transformation of a valid proof into another valid but unlinkable
proof of the exact same statement.  In practice, rerandomization
was never deployed because the public inputs link the usages.

We demonstrate in \S\ref{sec:rvrf_cont} a simple transformation of
any Groth16 zkSNARK into a {\it zero-knowledge continuation} whose
public inputs become opaque Pedersen commitments, with cheaply
rerandomizable blinding factors and proofs.
These zero-knowledge continuations then prove validity of the contents
of Pedersen commitments, but can now be reused arbitrarily many times,
without linking the usages. 

As recursive SNARKs shall remain extremely slow,
we expect zero-knowledge continuations via rerandomization become
essential for zkSNARKs used outside the crypto-currency space.

% \smallskip 
\paragraph{Ring VRFs:}

A {\it ring verifiable random function} (ring VRF) is a ring signature
that proves correct evaluation of some pseudo-random function (PRF)
determined by the actual key pair used in signing. % (see \S\ref{sec:rvrf_games}).
We build extremely efficient and flexible ring VRFs by amortizing a
zero-knowledge continuation that unlinkably proves ring membership
of a secret key, and then cheaply proving individual VRF evaluations.

As the PRF output is uniquely determined by the signed message and
signers actual secret key, we can therefore link signatures by the
same signer if and only if they sign identical messages.
In effect, ring VRFs restrict anonymity similarly to but less than
 linkable ring signatures do, which makes them multi-use and contextual.

% Second
% \smallskip
\paragraph{Identity uses:}

As an identity system, ring VRFs evaluated on a specific context or
domain name output a unique identity for the user at that domain or
context (see \S\ref{sec:app_identity}), which thereby prevents
Sybil behavior and permits banning specific users.
Yet users' activities remain unlinkable across distinct contexts or
domains, which supports diverse ethical identity usages.

We contrast this ethically straightforward ring VRF based identity
with the ethically problematic case of attribute based credential
schemes like IRMA (``I Reveal My Attributes'') credentials \cite{IRMAcredentials},
 which are now marketed as an online privacy solution.
IRMA could improve privacy in narrow situations of course, but
overall attribute based credentials should {\it never} be considered
fit for general purpose usage, like the prevention of Sybil behavior.

Aside from general purpose identity, our existing offline
verification processes often better protect user privacy and human
rights than adopting online processes like IRMA.
%
In particular, there are many proposals by the W3C for attribute based
credential usage in \cite{w3c_vc_use_cases}, but broadly speaking they
all bring matching harmful uses.  % https://www.w3.org/TR/vc-use-cases/
As an example, if users could easily prove their employment online when
applying for a bank account, then job application sites could similarly
demand proof of current employment, a clear injustice.

In general, abuse risks dictate that IRMA verifiers should be tightly
controlled by legislation, which becomes difficult internationally. 
%
Ring VRFs avoid these abuse risks by being truly unlinkable, and thus
yield anonymous credentials which safely avoid legal restrictions.

{\it Any ethical general purpose identity system should be based
upon ring VRFs, not attribute based credentials like IRMA.}

We credit Bryan Ford's work on proof-of-personhood parties \cite{pop2008,pop2017}
% https://bford.info/pub/dec/pop-abs/  https://bford.info/pub/net/sybil-abs/
with first espousing the idea that anonymous credentials should produce
contextual unique identifiers, without leaking other user attributes.

As a rule, there exist simple VRF variants for all anonymous credentials
like IRMA \cite{IRMAcredentials} or group signatures \cite{group_sig_survey}.
We focus exclusively upon ring VRFs for brevity, and because alone
ring VRFs contextual linkability covers more important use cases.

% Third
% \smallskip
\paragraph{Rationing uses:}

Ring VRFs yield rate limiting or rationing systems, which work
similarly to identity applications, except their VRF inputs should also
include an approximate date and a bounded counter, and
 then their outputs should be tracked as nullifiers.
Yet, these nullifiers need only temporarily storage, which improves 
efficiency over anonymous money schemes like ZCash and blind signed tokens.

We expect a degree of fraud whenever deploying purely certificate
based systems, as witnessed by the litany of fraudulent TLS and covid
certificates.  Ring VRFs help mitigate fraudulent certificate concerns
because the ring is a database and can be audited.

We know governments have ultimately little choice but to institute
rationing in response to shortages caused by climate change, ecosystem
collapse, and peak oil.  Ring VRFs could help avoid ration card fraud,
and thereby reduce social unrest, while also protecting essential privacy.

Ring VRFs need heavier verifiers than single-use token credentials
based on OPRFs \cite{PrivacyPass} or blind signatures.
Yet, ring VRFs avoid these schemes separate issuance phase entirely,
and sometimes even their registration phase.  Instead, fresh tokens
merely require adjusting the approximate date in the VRF input.
This reduces complexity, simplifies scaling, and increases flexibility.

In particular, if governments issue ration cards based upon ring VRFs
then these credentials could safely support other use cases, like
free tiers in online services or games, and advertiser promotions,
as well as identity applications like prevention of spam and online abuse.

In this, we need authenticated domain separation of products or identity
consumers in queries to users' ring VRF credentials.  We briefly discuss
some sensible patterns in \S\ref{???} below, but overall authenticated
domain separation resemble TLS certificates except simpler in that
roots of trust can self authenticate if root keys act as domain separators.





\endinput




As a field, anonymous credentials come in myriad flavors,
many of which exist to limits the anonymity provided, ala
 attribute based credentials and group signatures. % \cite{group_sig_survey}.
% aka anonymized signatures
%
Ring VRFs by weakening anonymity only contextually provide a safer,
more private, more flexible, more powerful, and more ethical
choice for all everyday anonymous credential use cases.  % needs:  ???



% 
\section{Identity}

% “We can judge our progress by the courage of our questions and the depth of our answers, our willingness to embrace what is true rather than what feels good.” 
% - Carl Sagan

% https://twitter.com/IdentityZack/status/1480631954689216516

% bryan ford https://twitter.com/brynosaurus/status/1460094634567344133

% answer https://twitter.com/valkenburgh/status/1442894421289103361
% https://twitter.com/harryhalpin/status/1443053685219725315
% https://twitter.com/OR13b/status/1442964741022830594
% https://twitter.com/jeffburdges/status/1443539630033362948
% https://twitter.com/Steve_Lockstep/status/1448653579330342916

% https://github.com/dckc/awesome-ocap/issues/17

% https://twitter.com/smdiehl/status/1459825936757493770

% https://twitter.com/edri/status/1483818492646281225

% Zeroth law:  A robot may not harm humanity, or, by inaction, allow humanity to come to harm.
% First law:  A robot may not injure a human being or, through inaction, allow a human being to come to harm.

An identity system must not harm humanity or its human users, to do otherwise is clearly unethical.  

Identity systems for human users have three participants, an identity provider, an identity consumer, and the user being identified.  There exist two methods by which ethical identity systems avoid harming users, either 
\begin{itemize}
\item special identity systems enforce that identity consumers owe users some legal duty that prevents miss-using the user's details, or else
\item general identity systems merely constrain user activity, often only rate limiting, but avoid providing identity consumers with any user details.
\end{itemize}
In other words, identity consumers should always first prove to the identity provider that they owe the user a legal duty appropriate to the details being revealed by the identity provider.

\subsection{Legal duties}

In this paper, we discuss only cryptographic protocols for general identity systems that avoid legal entanglements by only proving user uniqueness and not providing user details.  We first in this section briefly discuss wider examples that help motivate this problem by clarifying the legal and ethical complexities that arise when revealing user details.

As an unethical example, our largest advertising companies like Google and Facebook track private users using OAuth \cite{oauth}, with the intent to waste users time with increased advertising engagement, manipulate public opinion, ensnare users into unnecessary purchases, often by harming users' psyche, and accumulate personal data users might otherwise wish kept hidden.

As an only moderately harmful example, websites often prevent abuse by demanding commenters identify themselves by email address, which creates moral hazards and should expose the website operators to legal risks.

As beneficial identity examples, financial institutions act as an identity provider for their own identity consumer logic by issuing login credentials, but then owe their customers some fiduciary duty and strongly discourage using the same login credentials elsewhere.  

As a more nuanced example, an employer identifies employees to a personel management service by way of an external OAuth service, but the employer has some legal relationship with the personel management service, the OAuth service, and the employee, so any resulting harms rest upon the employer-employee relationship.  

We think Google Single Sign-on or Facebook Connect cannot play the role of OAuth service even in this employer-employee example, and indeed cannot ever be used ethically, because they aggressively track the employee outside the employer-employee relationship.  At the same time, an employees' Github account might or might not serve this role depending upon the specific employee and how they use Github outside work.  

... passports or medical ...

\subsection{Unlinkable identity}

We now lay aside such identity systems that represent a distinguished purpose tied to onerous three-way legal relationships between the parties.  Instead we turn our attention towards the range of identity systems that avoid providing any user details.  

At present, CAPTCHAs provide a popular defense against automated abuse.  There also exist cryptographic tools that amplify defenses against automated abuse, like blind signatures or verifiable oblivious pseudorandom functions (VOPRFs), as used in Privacy Pass \cite{privacypass}.  These dispence signle-use tokens within some limits imposed by other identity sources, rate limits, payments, or CAPTCHAs.  

We think single-use tools like CAPTCHAs, blind signatures, and VOPRFs adequately deter abuse in most use cases.  Yet, there also exist situations where abusers cannot be dissuaded by solving another CAPTCHAs or spending another token, like when abuse takes a personal character, or due to a larger profit motive.  

In such harder cases, we still need an anonymous credential so that identity consumers and providers cannot collude to track users, but identity consumers banning problematic users seemingly demands that users have different stable identities with each distinct identity consumer.  
To our knowledge, this identity formulation originates with proof-of-personhood parties \cite{pop2008,pop2017}.
% https://bford.info/pub/dec/pop-abs/
% https://bford.info/pub/net/sybil-abs/

We expect stable identities arise from multi-use anonymous credentials, like group signatures or ring signatures.  In group signatures, an identity provider holds a group manger secret key, with which they both issues credentials and deanonymize users.  We only want identity consumers to recognize returning users, making the deanonymization operation unacceptable.  

Ring signatures have classically given signers' control over their anonymity set aka ``ring'', which turns out mostly useless in practice.  Instead, realistic ring signatures like Zcash's circuits \cite{zcash_prorocol} have a shared public commitment to their ``ring'', so then users need only an opening for their own public key's presence in the ring. 



% sharing economy 
% business-to-business 



%   We think identity consumers should avoid imposing unnecessary constraints upon users and that rate limiting tools usually suffice.  Yet, there exist identity consumers who depend upon stronger Sybil defenses or an ability to ban problematic users.   


\section{Ring VRF Overview}
\label{sec:overview}

As a beginning, we introduce the ring VRF interface, give a simple
unamortized non-interactive zero-knowledge (NIZK) protocol that
realizes the ring VRF properties discussed later in our UC model,
and give some intuition for our later amortization trick.
Similar to VRF \cite{vrf_micali}, a ring VRF construction needs: 

\begin{itemize}
\item $\rVRF.\KeyGen $ outputs $ (\sk, \pk)$ algorithm,
 which creates a random secret key \sk and associated public key \pk;

\item $\rVRF.\Eval : (\sk,\msg) \mapsto \Out$ which deterministically computes the VRF output \Out from a secret key \sk and a message \msg.
\end{itemize}
%
% Although many constructions exist,
%Our \rVRF.\KeyGen and \rVRF.\Eval initially resemble EC VRFs like \cite{nsec5,VXEd25519,draft-irtf-cfrg-vrf-10}.
% In other words,
% internally we prove a VUF output $\PreOut = \sk H_{\grE}(\msg)$,
% with a hash-to-curve $H_{\grE}$, so then applying a PRF $\Hout$ yields a
% VRF output $\Out = \Hout(\msg, h \PreOut)$ ala \cite[Prop. 1]{vrf_micali},
% using a key pair like $\pk = \sk \genG$ for a generator $\genG$.

%We demand pseudo-randomness properties from \Eval, which could mirror
%\cite{vrf_micali} if desired.  We provide a UC definition resembling
%\cite{praos,ucvrf} which handles adversarial keys better however. %NOT CLEAR WHAT HANDLING BETTER MEANS

We demand a pseudo-randomness property from \Eval. In our construction in \S\ref{sec:pederson_vrf},  \rVRF.\KeyGen and \rVRF.\Eval resemble EC VRF like \cite{nsec5,VXEd25519,draft-irtf-cfrg-vrf-10}.

% TODO: Should this text be moved elsewhere?
% and prove it corresponds to $\rVRF.\Eval$ for some plausible signer.

% As an instructive but insecure over simplification, 
In contrast to VRF, a ring VRF scheme has the following algorithms operating directly upon
 set of public keys \ring:
\begin{itemize}
\item $\rVRF.\rSign : (\sk,\ring,\msg) \mapsto \sigma$ \,
    returns a ring VRF signature $\sigma$ for an input \msg.
\item $\rVRF.\rVerify : (\ring,\msg,\sigma) \mapsto \Out \, \lor \perp$ \,
    returns either an output $\Out$ or else failure $\perp$.
\end{itemize}

Ring VRFs differ from VRFs in that they do not expose a specific signer,
and instead prove the signer's key lies in  \ring,
much like how ring signatures differ from signatures.
Ring VRFs differ from ring signatures in that the verification process of Ring VRFs outputs the evaluation output \Out of the signer if the signature is verified with $ \ring $. So  the ring signature  actually proves that $ \Out $ is the evaluation output of the signer. 

After successful verification, our verifier should be convinced that $\pk \in \ring$, that
$\Out = \rVRF.\Eval(\sk,\msg)$ for some $(\sk,\pk) \leftarrow \rVRF.\KeyGen$. We demand anonymity meaning that the verifier learns nothing about the signer except that the signer's evaluation value of the signed message $ \msg $ is $ \Out $ and the signer's public key is in $ \ring $.

In other words, this simplified ring VRF could be instantiated by making
\rVRF.\Eval a pseudo-random (hash) function, and using a NIZK for a relation
\vspace{-3mm}
\doublecolumn{
	\begin{scriptsize}
		$$ \rel_{\mathsf{rvrf}} = \Setst{ (\Out, \msg, \ring);(\sk,\pk)}{
		\begin{aligned}
			& (\pk,\sk) \leftarrow \rVRF.\KeyGen,\\
			& \pk \in \ring \\
			& \Out = \rVRF.\Eval(\sk,\msg)
		\end{aligned}
		} $$
	\end{scriptsize}
}{
	$$ \rel_{\mathsf{rvrf}} = \Setst{ (\Out, \msg, \ring);(\sk,\pk)}{
	\begin{aligned}
		& (\pk,\sk) \leftarrow \rVRF.\KeyGen,\\
		& \pk \in \ring \\
		& \Out = \rVRF.\Eval(\sk,\msg)
	\end{aligned}
} $$
}


% TODO:  \PRF vs \rVRF.\Eval here??
% Although convenient for security arguments, % formalization

The zero-knowledge property of the NIZK ensures that our verifier learns nothing about the specific
signer, except that their key is in the ring and maps $\msg$ to $\Out$.
Importantly, pseudo-randomness also says that \Out is an identity
for the specific signer, but only within the context of \msg.

% \smallskip

Aside from proving an evaluation using \rVRF.\Eval, 
we always need \rVRF.\Sign and \rVRF.\Verify to sign some associated data \aux,
as otherwise the ring VRF signature become unmoored and permits replay attacks.
%
As an example, our identity protocol below in \S\ref{sec:app_identity}
yields the same ring VRF outputs each time the same user logs into the
same site, which suffers replay attacks unless \aux binds the
ring VRF signature to the TLS session.

\eprint{Indeed, regular (non-anonymous) VRF uses always encounter similar tension
with VRF inputs \msg being smaller than full message bodies $(\msg,\aux)$.
As an example, Praos \cite{praos} binds their VRF public key together
with a second public key for another (forward secure) signature scheme,
with which they sign their \aux, the block itself.
%
An EC VRF should expose an \aux parameter which it hashes when computing
its challenge hashes.  Aside from saving redundant signatures, exposing
\aux avoids user key handling mistakes that create replay attacks.}{}

Ring VRFs cannot so easily be combined with other signatures, which
makes \aux essential,%
\eprint{\footnote{If ring VRFs authorized creating blocks in an anonymous Praos blockchain then \aux must include the block being created, or else others could steal their block production turn.}}{}
but thankfully our ring VRF construction in \S\ref{sec:pederson_vrf} exposes \aux exactly like EC VRFs should do.%
\eprint{\footnote{We suppress multiple input-output pairs until \S\ref{subsec:multi_io} below, but they work like in \cite{PrivacyPass} too.}}{}

% \smallskip

If one used the $\rVRF$ interface described above, then one needs time
$O(|\ring|)$ in \rVRF.\rSign and \rVRF.\rVerify merely to read their \ring
argument, which severely limits applications.
Instead, ring signatures run asymptotically faster by replacing the \ring
argument with a set commitment to \ring, roughly like what ZCash does \cite{zcash_protocol}. Therefore, we introduce the following algorithms for $ \rVRF $.
\begin{itemize}
% \item $\rVRF.\CheckRing : \ring \mapsto \comring$ takes a set \ring of public keys and returns a public key set commitment \comring.
\item $\rVRF.\CommitRing : (\ring,\pk) \mapsto (\comring,\openring)$ \,
    returns a commitment for a set \ring of public keys, and
    optionally the opening \openring if $\pk \in \ring$ as well.
\item $\rVRF.\OpenRing : (\comring,\openring) \mapsto \pk \, \lor \perp$ \,
    returns a public key \pk, provided \openring correctly opens
    the ring commitment \comring, or failure $\perp$ otherwise.
\end{itemize}

We thus replace the membership condition $\pk \in \ring$ in the above
relation and NIZK by the opening condition
$ \pk = \rVRF.\OpenRing(\comring,\openring) \textrm{\ for some known \ } \openring \mathperiod $
% $\pk = \OpenRing(\comring,\openring)$.
%
% $$ \pi_0 = \NIZK \Setst{ \Out, \msg, \comring }{
%     \begin{aligned}
%         \exists (\pk,\sk) &\leftarrow \KeyGen,  \quad
%           \Out = \PRF(\sk,\msg)  \\
%         \exists \openring \textrm{\ s.t.\ }
%           \pk &= \OpenRing(\comring,\openring)  \\      
%     \end{aligned}
% } $$

% \smallskip

\eprint{Addressing these concerns, our notion should really be named 
 \emph{ring verifiable random function with additional data}
and its basic methods look like
\begin{itemize}
\item $\rVRF.\rSign : (\sk,\openring,\msg,\aux) \mapsto \sigma$, \quad and
\item $\rVRF.\rVerify : (\comring,\msg,\aux,\sigma) \mapsto \Out \,\, \lor \perp$.
\end{itemize}}{}


Although an asymptotic improvement, our opening \rVRF.\OpenRing based condition invariably
still winds up being computationally expensive to prove inside a zkSNARK.
We solve this obstacle in \S\ref{sec:rvrf_cont}  by introducing
{\em zero-knowledge continuations}, a new zkSNARK technique built from
rerandomizable Groth16s \cite{Groth16} and designed for SNARK composition and reuse.

As a step towards this, we split the relation $ \rel_{\mathsf{rvrf}} $ into a relation
for \rVRF evaluation and a relation, which enforces our
computationally expensive condition $\pk = \rVRF.\OpenRing(\comring,\openring)$.
We want to reuse the proof generated for latter across multiple \rVRF signatures, so anonymity
requires we rerandomize a Groth16 SNARK for it
ala \cite[Theorem 3, Appendix C, pp. 31]{RandomizationGroth16}.
%
Yet, we connect together the NIZKs for the two relations.
%demands some hiding commitment \compk to \pk.

%


%\def\tmpAA{\Out = \rVRF.\Eval(\sk,\msg)}%
%\def\tmpBB{\textrm{\compk commits to\ \sk}}%
%$$ \rel_{eval} = \Setst{ (\Out, \msg, \aux, \compk); \sk}{
%	\eprint{
%		\tmpAA, \, \tmpBB
%	}{
%		\begin{aligned}
%			&\tmpAA, \\
%			&\tmpBB \\
%		\end{aligned}
%	}
%} $$
%
%\def\tmpAA{\textrm{\compk commits to $ \sk $ with public key\ }}%
%\def\tmpBB{\rVRF.\OpenRing(\comring,\openring)}%
%$$ \Rring = \Setst{ (\compk, \comring);(\sk,\pk) }{
%	\eprint{
%		\tmpAA \pk = \tmpBB
%	}{
%		\begin{aligned}
%			&\tmpAA \\
%			&\, \pk = \tmpBB \\
%		\end{aligned}
%	}
%} $$

%TODO: WE SHOULD EXPLAIN IT BETTER FOR EPRINT
%We discovered the SNARK for the language \Lring becomes incredibly efficient for the prover if one specializes
%the original Groth16 SNARK construction:  An inner original Groth16 SNARK for $\Lring^\inner$
%handles the secret key \sk directly via its public inputs, but
%\sk and even \pk remain secret by transforming the trusted setup to have
%a rerandomizable Pedersen commitment \compk outside this Groth16 SNARK.
%$$ \Lring^\inner = \Setst{ \sk, \comring}{
%    \eprint{
%    (\pk,\sk) \leftarrow \rVRF.\KeyGen, \, % \textrm{\,and }
%    \pk = \rVRF.\OpenRing(\comring,\openring) 
%    }{
%    \begin{aligned}
%        &(\pk,\sk) \leftarrow \rVRF.\KeyGen, \\
%        % \exists \openring \textrm{\ s.t.\ }
%        &\pk = \rVRF.\OpenRing(\comring,\openring)  \\      
%    \end{aligned}
%    }
%} $$
%
%Our zero-knowledge continuation in \S\ref{sec:rvrf_cont} rerandomizes
%$\compk = \pk + b \, K$ without reproving the Groth16 SNARK for $\Lring^\inner$.
%For this, the secret key \sk must be a public input of $\Lring^\inner$, and
%the Groth16 trusted setup must be expanded by a secret multiple of
% the otherwise independent point $K$.
%
%In \S\ref{sec:pederson_vrf}, we introduce an extremely efficient NIZK
%for $ \rel_{eval} $, which also provides an essential proof-of-knowledge for \compk.


\endinput


% We define ring VRFs in \S\ref{sec:rvrf_games} and \S\ref{sec:rvrf_uc_fun} below, but
Ring VRFs are firstly ring signatures broadly interpreted, in that they
prove an involved public key lies inside some commitment \comring to
the plausible signer set, known as the ring.
Anyone could compute \comring from this set of public keys.
%
At the same time, ring VRFs prove correct output of a PRF keyed by
the signer's actual secret key, and evaluated on a supplied message \msg,
which then links ring VRF signatures on the same \msg.

\smallskip


\section{Background}
\label{sec:background}

\def\secparam{\ensuremath{\lambda}\xspace}

\def\ecE{{\mathbb{E}}}
\def\grE{{\mathbf{E}}}
\def\genE{E}
\def\genG{G}
\def\genB{K} %{\genE_{\mathrm{bind}}}

\def\ecJ{{\mathbb{J}}}
\def\grJ{{\mathbf{J}}}
\def\genJ{J}

% As our ring VRF is built by composing them, 
We briefly recall the primitives and security assumptions underlying
both Chaum-Pedersen DLEQ proofs and pairing based zkSNARKs. 


\subsection{Elliptic curves}

We obey mathematical and cryptographic implementation convention by using additive notation for elliptic curve and multipicative notation for eliptic curve scalar multiplications. 

We always implicitly have a paramater generation procedure $\mathtt{Params}$ that takes a security level $\secparam \in \N$ and returns elliptic curve paramaters including some prime numbers and efficient algorithms for computing elliptic curve operations.  As customary, we treat $\secparam$ and the output of $\mathtt{Params}$ as fixed paramaters, which makes sense because humans run $\mathtt{Params}$ manually in practice. 

As implicit outputs of $\mathtt{Params}$, we work with an elliptic curve $\ecE[\F]$ over some base field $\F$ of (prime) characteristic $q_{\grE}$, along with a distinguished subgroup $\grE \le \ecE[\F]$ of prime order $p_{\grE} \approx 2^{2\secparam}$.  As $\grE$ distinguishes $\ecE[\F]$, we let $h_{\grE}$ denote the cofactor of $\grE$ in $\ecE[\F]$, meaning $\ecE[\F]$ has $h_{\grE} p_{\grE}$ points.
% but abbreviate $h = h_{\grE}$, $p = p_{\grE}$, and $q = q_{\grE}$ when $\grE$ is clear from context.
We write $\grE$ without subscript, and abbreviate $h = h_{\grE}$, $p = p_{\grE}$, and $q = q_{\grE}$, when $\ecE$ is either our uinque pairing friendly curve or else the only curve in view.

We let $H_p : \{0,1\}^* \to \F_p$ or $H_q : \{0,1\}^* \to \F_q$ denote random oracles (RO) with a range $\F_p$ or $\F_q$.  We let $H_\ecE : \{0,1\}^* \to \ecE$ or $H_\grE : \{0,1\}^* \to \grE$ denote a hash-to-curve for $\ecE$ or hash-to-group for $\grE$, which we model as a random oracles too.  We note $H_\grE(x) = h H_\ecE(x)$ always works, although more efficent exist.

\smallskip

Almost all SNARKs like \cite{Groth16} or \cite{plonk} employ a pairing friendly elliptic curve $\ecE$ over a field $\F_q$ of characteristic $q \approx 2^{2\secparam}$, which comes equipped with a type III pairing on subgroups of prime order $p \approx 2^{2\secparam}$:  We let $q_1,q_2,q_T$ denote small powers of $q$, and let $\grE_1 \le \ecE[\F_{q_1}]$ and $\grE_2 \le \ecE[\F_{q_2}]$ and $\grE_T \le \F_{q_T}^\times$ denote subgroups all of prime order $p$.  We also let $e : \grE_1 \times \grE_2 \to \grE_T$ denote a type III pairing, meaning a computable bilinear map without known efficiently computable maps between $\grE_1$ and $\grE_2$.  Also $q_i = q_{\grE_i}$ for $i=1,2$ in our above notation.  

Any pairing friendly elliptic curve $\ecE$ provides solutions to the decisional Diffie-Hellman problem (DDH).  We do however assume the computational Diffie-Hellman problem (CDH) remains hard in $\ecE$.  We remark that $H_\grE$ being a random oracle plus CDH hardness prevents computable relationships between $H_\grE$ outputs.

% TODO: Pairing assumptions required by Groth16

\smallskip

% We shall require ZCash Sapling style ``Jubjub'' Edwards curves, whose base field characteristic divides of the order of a pairing friendly elliptic curve, thereby making Jubjub base field arithmetic SNARK friendly, and hence Jubjub elliptic curve operations as well \cite{}.

We let $\ecJ$ denote a ZCash Sapling style ``JubJub'' Edwards curve associated to the pairing friendly elliptic curve $\ecE$, meaning $\ecJ$ has base field $\F_p$ whose characteristic $q_{\grJ} = p$ matches the group order $p$ of $\grE_1 \cong \grE_2 \cong \grE_T$.  As in ZCash Sapling, we now prove statements about elliptic curve operations inside $\ecJ$ by proving base field arithmetic in $\F_p$, which our $q_{\grJ} = p$ makes relatively inexpensive inside SNARKs on $\ecE$.

As above, $\grJ \le \ecJ[\F_p]$ has large prime order $p_{\grJ}$ and a small cofactor $h_{\grJ}$.  We always support $4 p_{\grJ} < p$ because if $\ecJ$ is an Edwards curve then $h_{\grJ} \ge 4$ which imposes this by the Hasse bound.

\smallskip

We ask that deserialization prove that putative elements of $\grE$ lie in
$\ecE[\F]$ by verifying curve equations, perhaps including twist checks.

Anytime $\ecE$ represents a pairing friendly curve then we ask that
deserialization prove elements of $\grE_1$, $\grE_2$, and $\grE_T$
lie inside the correct subgroup of order $p$,
 which typically requires checking whether $|\grE| X = 1$ or similar.
As our SNARKs works with points in $\ecJ$ directly, we conversely
prefer writing $\grJ$ equations in $\ecJ[\F_p]$ and explicitly describe
where one clears the cofactor $h_{\grJ}$.  We handled $\grE$ withr
$\ecE$ not necessarily pairing friendly similarly to $\ecJ$.
We scrape by with only CDH hardness for $\grJ$ for convenience,
although DDH winds up hard in $\grJ$.


\subsection{Zero-knowledge proofs}

\newcommand\rel{\ensuremath{\mathcal{R}}\xspace}
\newcommand\lang{\ensuremath{\mathcal{L}}\xspace}

% refs.
% https://people.csail.mit.edu/silvio/Selected%20Scientific%20Papers/Zero%20Knowledge/Noninteractive_Zero-Knowkedge.pdf
%   Alright but kinda poorly phrases
% https://inst.eecs.berkeley.edu/~cs276/fa20/notes/Multiple%20NIZK%20from%20general%20assumptions.pdf
%   Addresses the ZK definitions better
% 

We let \rel denote a polynomial time decidable relation, so the language
 $\lang = \{x \mid \exists \omega (\omega,x) \in \rel \}$ lies in NP.
All non-interactive zero-knowledge proof systems have some setup procedure $\mathtt{Setup}$ that takes our parameters generated by $\mathtt{Params}$ and some ``circuit'' description of \rel, and produces a structured reference string (SRS).

A non-interactive proof system for $\rel$ consists of \Prove and \Verify PPT algorithms
\begin{itemize}
%\item $\NIZK.\setup(\rel) \rightarrow (crs, \tau)$ ---- Given the relation $ \rel $ it outputs a common reference string $ crs $ and a trapdoor $ \tau $ for $ \rel $.
\item $\NIZK_\rel.\Prove(\omega, x) \mapsto \pi$ creates a proof $\pi$ for a witness and statement pair $(\omega,x) \in \rel$.
\item $\NIZK_\rel.\Verify(x, \pi)$ returns either true of false, depending upon whether $\pi$  proves $x$.
\end{itemize}	
which satisfy the following completeness, zero-knowledge, and knowledge soundness definitions.

\begin{definition}\label{def:nizk_completeness}
We say $\NIZK_\rel$ is {\em complete} if $\Verify(x, \Prove(\omega,x)$ succeeds for all $(\omega,x) \in \rel$.  % with high probability
\end{definition}

\def\advV{\ensuremath{V^*}\xspace} % Why not use \adv here?

\begin{definition}\label{def:nizk_zero_knowledge}
We say $\NIZK_\rel$ is {\em zero-knowledge} if
there exists a PPT simulator $\NIZK_\rel.\Simulate(x) \mapsto \pi$
that outputs proofs for statement $x \in L$ alone, which are
computationally indistinguishable from legitimate proofs by \Prove,
i.e.\ any non-uniform PPT adversary \advV cannot distinguish pairs $(x,\pi)$
generated by \Simulate or by \Prove except with odds negligible in \secparam
(see \cite[Def. 9, \S A, pap. 29]{RandomizationGroth16}). %  or ...
\end{definition}

\def\advP{\ensuremath{P^*}\xspace} % Why not use \adv here?

\begin{definition}\label{def:nizk_knowledge_sound}
We say $\NIZK_\rel$ is {\em (white-box) knowledge sound} if
for any non-uniform PPT adversary \adv who outputs a statement $x \in \lang$ and proof $\pi$
there exists a PPT extractor algorithm $\Extract$ that white-box observes $\advP$ and
if $\Verify(x,\pi)$ holds then $\Extract$ returns an $\omega$ for which $(\omega,x) \in \rel$
(see \cite[Def. 7, \S A, pap. 29]{RandomizationGroth16}).
\end{definition}

Our zero-knowledge continuations in \S\ref{sec:rvrf_cont} demand
rerandomizing existing zkSNARKs, which only Groth16 supports \cite{Groth16}.
We therefore introduce some details of Groth16 \cite{Groth16} there,
when we tamper with Groth16's SRS and $\mathtt{Setup}$ to create zero-knowledge continuations. 
% TODO: Do we describe Groth16 \cite{Groth16} enough?

% In this, we exploit several arguments given by \cite{RandomizationGroth16},
% but for now we recall that \cite{RandomizationGroth16} proves that Groth16
% satisfies: % white-box weak simulation-extractablity .
%
% \begin{definition}\label{def:nizk_weak_simulation_extractable}
% We say $\NIZK_\rel$ is {\em white-box weak simulation-extractable} if
% for any non-uniform PPT adversary \advP with oracle access to \Simulate
% who outputs a statement $x \in \lang$ and proof $\pi$,
% there exists a PPT extractor algorithm $\Extract$ that white-box observes $\advP$ and
% if \advP never queried $x$ and $\Verify(x,\pi)$ holds
% then $\Extract$ returns an $\omega$ for which $(\omega,x) \in \rel$
% (see \cite[Def. 7, \S 2.3, pap. 29]{RandomizationGroth16}).
% \end{definition}

TODO: AGM and Groth16 here?


\subsection{Universal Composable (UC) Model}

TODO: Chat on why UC is here?

A protocol $ \phi $ in the UC model is an execution between distributed interactive Turing machines (ITM). Each ITM has a storage to collect the incoming messages from other ITMs, adversary \adv or the environment $ \env $. $ \env $ is an entity to represent the external world outside of the protocol execution.  The environment $ \env $ initiates ITM instances (ITIs) and the adversary \adv with arbitrary inputs and then terminates them to collect the outputs.
% An ITM that is initiated by $ \env $ is called ITM instance (ITI). 
We identify an ITI with its session identity $ \sid $ and its ITM's identifier $ \pid $. In this paper, when we call an entity as a party in the UC model we mean an ITI with the identifier $ (\sid, \pid) $.

We define the ideal world where there exists an ideal functionality $ \mathcal{F} $ and the real world where a protocol $ \phi $ is run as follows:

\paragraph{Real world:} $ \env $ initiates ITMs and \adv to run the protocol instance with some input $ z \in \{0,1\}^* $  and a security parameter $ \secparam $. After $ \env $ terminates the protocol instance, we denote the output of the real world by the random variable $ \mathsf{EXEC}(\secparam, z)_{\phi, \adv, \env} \in \{0,1\} $. Let $ \mathsf{EXEC}_{\phi, \adv, \env} $ denote the ensemble $ \{\mathsf{EXEC}(\secparam, z)_{\phi, \adv, \env} \}_{z \in \{0,1\}^*} $.

\paragraph{Ideal world:} $ \env $ initiates ITMs and a simulator $ \sim $ to contact with the ideal functionality $ \mathcal{F} $ with some input $ z \in \{0,1\}^* $  and a security parameter $ \secparam $. $ \mathcal{F} $ is trusted meaning that it cannot be corrupted.
$ \sim $ forwards all messages forwarded by $ \env $ to $ \mathcal{F} $. The output of execution with $ \mathcal{F} $ is denoted by a random variable $ \mathsf{EXEC}(\secparam, z)_{\mathcal{F},\sim, \env} \in \{0,1\}$.  Let $ \mathsf{EXEC}_{\mathcal{F},\sim, \env} $ denote the ensemble $ \{\mathsf{EXEC}(\secparam, z)_{\mathcal{F}, \sim, \env} \}_{z \in \{0,1\}^*} $.

TODO: \secparam should likely be implicit, especially since it appears in both worlds.

\begin{definition}[UC-Security of $ \phi $] \label{def:uc}
Given a real world protocol $ \phi $ and an ideal functionality $ \mathcal{F} $ for the protocol $ \phi $, we call that $ \phi $ is UC-secure if $ \phi $ UC-realizes $ \mathcal{F} $ if for all PPT adversaries \adv, there exists a simulator $ \sim  $ such that for any environment $ \env $,
 $\mathsf{EXEC}_{\phi, \adv, \env}$ indistinguishable from $\mathsf{EXEC}_{\mathcal{F},\sim, \env}$
\end{definition}

TODO: if ... if makes no sense.  These definitions need much clearer explanation, or more likely citations to places with clear explanations. 

\begin{definition}[UC-Security of $ \phi $ in the hybrid world]
Given a real world protocol $ \phi $ which runs some (polynomially many) functionalities $ \{\mathcal{F}_1, \mathcal{F}_2, \ldots, \mathcal{F}_k\} $ in the ideal world and an ideal functionality $ \mathcal{F} $ for the protocol $ \phi $, we call that $ \phi $ is UC-secure in the hybrid model $ \{\mathcal{F}_1, \mathcal{F}_2, \ldots, \mathcal{F}_k\} $ if $ \phi $ UC-realizes $ \mathcal{F} $ if for all PPT adversaries \adv, there exists a simulator $ \sim  $ such that for any environment $ \env $,
 $\mathsf{EXEC}_{\phi, \adv, \env}$ is indistinguishable from $\mathsf{EXEC}_{\mathcal{F},\sim, \env}$.
\end{definition}

% REMARKS:  Removed excessive notation $\approx$.














\endinput



BROKEN BOLOW THIS




We fix $J \in \ecJ$ as a generator for public keys.  Any $\KeyGen$ algorithm randomly samples a secret keys $\sk \in \F_q$ and then computes its associate public keys $\pk = \sk J$.  We shall not discuss infrastructure that authorizes public keys.  Yet although our results do not require proof-of-knowledge on $\pk$ per se, we still strongly recommend that back certifications accompany any certificates that authorize $\pk$.

\smallskip





%
\section{VRF-AD security}
\label{sec:games}

We say a VRF-AD-KC denoted \VRF is {\em secure} if it satisfies
 correctness, uniqueness, and pseudo-randomness as defined below,
 as well as being existentially unforgeable as a signature on $(\msg,\aux)$.
%
We caution that VRF security remain subtle, in part due to
signer and forger each being adversarial in some security properties.
%
% At a high level however VRF security assumptions boil down to translating the PRF definition into the public key setting.
% TODO: What of the above two lines?  Merge?

% We follow \cite{agg_dkg} by distinguishing an algorithm $\VRF.\Eval$,
%  instead of defining it by the equality in correctness,
% which simplifies requiring that verifying honest signatures gives a well-defined function.
% $\VRF.\Eval$ always has more optimized instantiations anyways.

We demand unforgability on $(\msg,\aux)$ because alone
the usual VRF conditions only yield unforgeability for \msg.

\begin{definition}\label{def:vrf_sign_oracle}
We let \ora{Sign} denote a CMA oracle, which creates and stores
a key pair $(\pk,\sk) \leftarrow \KeyGen$, returning \pk, and
thereafter answers oracle calls $\ora{Sign}(\msg,\aux)$ by 
logging $(\msg,\aux)$ and returning $\Sign(\sk,\msg,\aux)$.
\end{definition}

\begin{definition}
We say a VRF-AD satisfies {\em existential unforgeability (EUF-CMA-KC)} if
any PPT adversary \adv has only a negligible advantage in $\secparam$
in the usual chosen-message game adapted to key commitments:
\begin{itemize}
  \item \adv receives $\pk$ from \ora{Sign}, % of Definition \ref{def:vrf_sign_oracle}
  repeatedly queries \ora{Sign},
  and finally produces $\pk,\msg,\aux,\sigma$.
  \item \adv wins if $\Verify(\pk,\msg,\aux,\sigma)$ succeeds, and
  \adv never queried $\ora{Sign}(\msg,\aux)$.
\end{itemize}
\end{definition}

% TODO: Any chat here?

\begin{definition}
We say a VRF-AD satisfies {\em VRF correctness} if
 $\Out = \Verify(\pk,\msg,\aux,\Sign(\sk,\msg,\aux))$ succeeds
whenever $(\pk,\sk) \leftarrow \KeyGen$, and
$\Eval : (\sk,\msg) \mapsto \out$ is a well-defined function.
\end{definition}
% TODO: Is the second condition supurfluous?

We recast the uniqueness as VRFs being well-defined functions of
their public key too, at least up to cryptographic assumptions,
but our definition is clearly equivalent to uniqueness given in
\cite[Def. 2 \S3.2, pp. 4]{vrf_micali} or \cite[Def. 3, pp. 8]{agg_dgk}.

\begin{definition}
We say a VRF-AD satisfies {\em uniqueness} if
if anytime some PPT adversary \adv produces $\msg$, $\pk$, and $\aux_i$, $\sigma_i$ for $i=1,2$, then
$\Verify(\pk,\msg,\aux_1,\sigma_1) = \Verify(\pk,\msg,\aux_2,\sigma_2)$
unless either $\Verify$ returns failure, except with odds negligible in $\secparam$.
\end{definition}

\begin{definition}
We say a VRF-AD satisfies {\em strong uniqueness} if
there exists a (not efficiently computable) function
 $F : (\msg,\pk) \mapsto \Out$ such that
anytime some PPT adversary \adv produces $\msg$, $\pk$, $\aux$, and $\sigma$
then $\Verify(\pk,\msg,\aux,\sigma) \in \{ F(\msg,\pk), \perp \}$
except with odds negligible in $\secparam$.
\end{definition}
% TODO: Keep?

We say VRFs are public key analogs of PRFs, but actually this PRF analogy
fails in the ``residual pseudo-randomness'' definitions by
Micali, et al. \cite[Def. VRF (3) \S3.2, pp. 4]{vrf_micali},
 which employs \ora{Sign} in EUF-CMA-like games,
 but says nothing for adversarially generated keys.

\begin{definition}
We say a VRF-AD-KC satisfies {\em public keyed} or {\em residual pseudo-randomness} if 
any PPT adversary \adv has only a negligible advantage in $\secparam$
in this chosen-message game:
\begin{itemize}
	\item[]
	\adv receives $\pk$ from \ora{Sign} of Definition \ref{def:vrf_sign_oracle},
	repeatedly queries \ora{Sign}, and produces $\msg,\aux$.
	If \adv never queried $\ora{Sign}(\msg,\cdot)$ then
	\adv wins by distinguishing $\msg \mapsto \Eval(\sk,\msg)$ from a random.
\end{itemize}
\end{definition}

In \cite{praos}, there exists a UC functionality which captures the
desired PRF analogy, but brings unnecessary restrictions.

We know a function family $\{ F_\msg : \pk \mapsto F(\msg,\pk) \}$ exists
by strong uniqueness, although not efficiently computable, so intuitively
our VRF-AD is {\em pseudo-random} if an adversary cannot distinguish
$F_\msg$ from a random function.
% TODO: Keep?

\bigskip

MISTAKES BELOW THIS POINT

\bigskip 

As a formalization, we provide a black-box game-based definition which
treats \msg as the PRF key, and handles adversarially supplied keys as
PRF inputs by not necessarily terminating.

\begin{definition}
We say a VRF-AD-KC satisfies {\em message keyed pseudo-randomness} if 
any PPT adversary \adv for whom the following black-box game always
terminates has only a negligible advantage in $\secparam$ of winning.
\begin{itemize}
	\item[]
	Sample a random \msg, a random function $\rho$ with the same range as \Eval, and a bit $b$.
	\adv queries \ora{Verify} by providing both a public key \pk and
	a PPT (black-box) secret key algorithm $f_\sk : () \mapsto (\aux,\sigma)$
	such that repeatedly trying $\Out \leftarrow \Verify(\pk,\msg,f_\sk(\msg))$
	eventually succeeds.
	\ora{Verify} always returns \Out and $\rho(\pk)$ but with their order depending upon $b$.
	\adv wins by guessing $b$, aka by distinguish \Verify from $\rho$.
\end{itemize}
\end{definition}

There are also verifiable unpredictable function (VUF), which replace
pseudo-randomness by the weaker {\em unpredictability} definition from
\cite[Def. VUF (3) \S3.2, pp. 5]{vrf_micali} or \cite[Def. 4, pp. 8]{agg_dgk}.
Interestingly VUFs often suffice threshold security assumptions \cite{agg_dkg}.

\begin{definition}
We say a VRF-AD-KC satisfies {\em residual unpredictability} if 
any PPT adversary \adv has only a negligible advantage in $\secparam$
in this chosen-message game:
\begin{itemize}
	\item[]
	\adv receives $\pk$ from \ora{Sign} of Definition \ref{def:vrf_sign_oracle},
    repeatedly queries \ora{Sign}, and produces $\msg,\aux$.
    If \adv never queried $\ora{Sign}(\msg,\cdot)$ then
    \adv wins by guessing $\Eval(\sk,\msg)$ for an unqueried \msg.
\end{itemize}
\end{definition}

Also, if $H'(\cdot,k)$ is a PRF then \cite[Proposition 1]{vrf_micali}
shows computing $\Out = H'(\Verify(\cdots), \msg)$ transforms
 residual unpredictability into a residual pseudo-randomness.
As $H'$ is cheap, we conclude implementers should prefer VRFs over more subtle VUFs.

\begin{definition}
We say a VRF-AD-KC satisfies {\em message keyed unpredictability} if 
any PPT adversary \adv for whom the following black-box game always
terminates has only a negligible advantage in $\secparam$ of winning.
\begin{itemize}
	\item[]
	Sample a random \msg.
	\adv queries \ora{Verify} by providing both a public key \pk and
	a PPT (black-box) secret key algorithm $f_\sk : () \mapsto (\aux,\sigma)$ such that
	repeatedly trying $\Out \leftarrow \Verify(\pk,\msg,f_\sk(\msg))$ eventually succeeds.
	\ora{Verify} always returns \Out.
	\adv wins by correctly guessing $\Out = F(\msg,\pk)$ for an unqueried \pk. 
\end{itemize}
\end{definition}

TODO: Justify?

TODO: Relationships?  


\subsection{Confusion}
% \smallskip

Although \cite[\S3.2 $\fvrf$]{praos} handles pseudo-randomness better than \cite{vrf_micali},
they formalize VRFs with detached outputs via the two algorithms:
% \begin{itemize}
% \item
$\VRF.\primalgo{EvalProve}(\sk,\msg,\aux) \mapsto (\Out,\sigma)$, in which $\sigma = \VRF.\Sign(\sk,\msg,\aux)$ and $\Out = \VRF.\Eval(\sk,\msg)$, and
% \item
$\VRF.\primalgo{VerifyProof}(\pk,\msg,\aux,\Out,\sigma)$ which returns true only if $\VRF.\Verify(\pk,\msg,\aux,\sigma)$ returns $\Out$.
% \end{itemize}
We strongly prefer the \Sign and \Verify formulation from \cite{agg_dkg}
primarily because the \primalgo{EvalProve}, and \primalgo{VerifyProof}
formulation causes implementation and deployment mistakes:

EC VRF signatures have the form $\sigma = (\PreOut,\pi)$ in which some
inner proof $\pi$ proves correctness of some associated VUF output $\PreOut$. % aka ``pre-output''.  % ``pre-pseudo-random''
It follows $\VRF.\Eval$ never corresponds to $\PreOut$, but if one describes
protocols with an \primalgo{EvalProve} formulation then exposing $\PreOut$
invariably confuses developers into miss-using $\PreOut$ as the output.
% In other words, actual code never corresponds to an \primalgo{EvalProve} and \primalgo{VerifyProof} formulation.

The ``pre-output'' $\PreOut$ preserves algebraic relationships between
secret keys, so protocols described by the \primalgo{EvalProve} formulation
have implementations with broken pseudo-randomness, and perhaps
 related key vulnerabilities and mishandled cofactors.
% We need $\PreOut$ to be exposed by implementations so ...
We avoided the VUF formalism taken by \cite{agg_dkg} in part because
 VUFs obfuscate this difficulty to developers.

As a caveat, there exist UC formalisms that appear simpler with
the \primalgo{EvalProve} and \primalgo{VerifyProof} formulation, like in \cite{praos}.
We therefore propose that VRFs and protocols using VRFs should always be
described using the the \Sign and \Verify formulation, which provides
implementers with a sensible description, but then if needed adopt
 \primalgo{EvalProve} and \primalgo{VerifyProof} only inside the UC formulation itself.
We feel imposing this mental translation upon paper authors and reviewers
 beats imposing the reverse upon developers with less cryptographic knowledge.



\endinput 



\smallskip

There exist VUFs like RSA-FDH or BLS signatures that lack auxiliary data
% There even exist bespoke VRFs that relax correctness to some non-trivial
% relation on the space of secret keys and messages,
%  seemingly including some Rabin variants. 
Yet, these all suffer from either large signature sizes (RSA) or
 slow verification (BLS).
%  VRFs like single-layer XMSS, .

Instead, one prefers instantiating VRFs similarly to
 \cite{nsec5} or \cite{VXEd25519} using Chaum-Pedersen DLEQ proofs \cite{CP92} % Or should it be CP93 ??
 because they provide both small signatures and fast verification.
In these, our auxiliary data \aux can be verified for free,
by binding \aux into the challenge hash, like a Schnorr signature.
VRF protocols could often reduce bandwidth and verifier time this way,
 but some like Sassafras depend upon \aux. 





\endinput % no UC VRF discussion here




 


\subsection{Ring VRF in the UC Model}

In Figure \ref{f:gvrf}, we give a UC functionality $\fgvrf$ for ring VRFs,
which we ourselves shall use in other works.  In $\fgvrf$, we suppress
auxiliary data and ring commitment details to make our UC functionality
more accessible, meaning our ring commitment is simply the full ring both
here and in Appendix \ref{???}.

We give several important remarks that help elucidate $\fgvrf$:

First, Each party is distinguished by unique verification key which is given by the simulator. Verification keys have the identifier role of  the signatures and outputs rather than  influencing the value of them. Therefore, there exists no secret key as in the real world protocol.

Second, the verification algorithm of a ring VRF outputs the corresponding evaluation value of the verified signature. Therefore, $ \fgvrf $  outputs the corresponding output during the signature verification if the signature is verified. However, it achieves this together with the anonymous key which is not defined in the ring VRF in the real world.  If $ \fgvrf $ did not define an anonymous key of each signature, then there would be no way that $ \fgvrf $ determines the actual verification key of the signature $ \sigma $ and outputs the evaluation value because $ \sigma $ does not need to be associated with the signer's key or unique. Therefore, $ \fgvrf $ associates an anonymous key independent from the signer's key $ \pkrvrf $ for each $ m $ and $ \pkrvrf $ so that this key behaves as if it is the verification key of the signature during the verification.
	
Third, $ \fgvrf $ does not have a separate signing protocol for malicious parties as parties because they can generate it as they like. If they generate a signature, it is added to the $ \fgvrf $'s records as valid or invalid when a party comes for the verification of it.  Its validity depends on $ \simulator $ as it can be seen in \ref{cond:malicioussignature}. 
	
Fourth, once $ \simulator $ obtains an anonymous key of a message $ m $ generated for an honest party, we let $ \simulator $ learn the  evaluation of  $ m $ and  $ W $. $ \simulator $ can do this in $ \fgvrf $ in three ways: The first way is via malicious ring VRF evaluation i.e., send the message $ (\msg{eval}, \sid, \pkrvrf_i,W,m) $. Here, if $ W $ is an anonymous key of $ m, \pkrvrf  $, $ \fgvrf $ returns $ \evaluationslist[m, W] $ even if $ \pkrvrf \neq \pkvrf_i $. $ \fgvrf $ returns it independent from which verification key given in $ \simulator $'s message. The second way is via malicious requests of signatures and outputs.  Here, $ \simulator $ also learns  all honest signatures of $ m $ generated for a given ring and anonymous key. The last way is via verification.  In a nutshell, evaluation of a message for an honest party is secret to $ \simulator $ till $ \simulator $ obtains the corresponding anonymous key.
	
% The functionality lets parties generate a key (Key Generation), evaluate a message with the party's key (Ring VRF Evaluation), sign a message by one of the keys (Ring VRF signature) and verify the signature and obtain the evaluation output without knowing the key used for the signature and evaluation (Ring VRF Verification). 
%
%%We also define linking procedures in $ \fgvrf $ to link a signature with its associated key. So, if a party wants to reveal its identity at some point, it can use the linking process to show that the evaluation is executed with its key (Linking Signature). Later on, anyone can verify the linking signature (Linking Verification).
%
%In a nutshell, the functionality $\fgvrf$, when given as input a message $m$ and a key set $\pkring$ (that we call a ring) of party, allows to create $n$ possible different outputs pseudo-random that appear independent from the inputs. The output can be verified to have been computed correctly by one of the participants in $\pkring$ without revealing who they are. At a later stage, the author of the ring VRF output can prove that the output was generated by him and no other participant could have done so.

We designed $ \fgvrf $ so that it achieves the following properties:

\paragraph{Randomness:}  The evaluation of $ (m, \pkrvrf_i) $,  which is $ \evaluationslist[m,W] $ where $ \anonymouskeymap[m,W] = \pkrvrf_i $, is random.

%Given that the evaluation of $  m, \pkrvrf  $ for any verification key $ \pkrvrf $ and for any message $ m $ has never been given to $ \simulator $, the probability that $ \simulator $ guesses the evaluation of $ m, \pkrvrf_i $ is $ \frac{1}{2^{\ell_\rvrf}} $, given that $  $

Each element in $ \out $ are selected randomly by $ \fgvrf $. Therefore, $ \fgvrf $ satisfies randomness.  Evaluation of $ m, \pkrvrf_i $ where $ \pkrvrf_i $ is an honest key is generated by first assigning a random anonymous key $ W $ to it and then assigning a random evaluation value $ y $ to $ m, W $. Therefore, honest evaluations are always random. Malicious evaluation of $ m, \pkrvrf_i $, where $ \pkrvrf_i $ is not an honest verification key, is generated by first assigning an anonymous key $ W $ given by $ \simulator $ to it and then assigning a random value $ y $ to $ m, W $. Since $ \fgvrf $ checks whether $ W $ is an anonymous key of another verification key, evaluation of $ m, \pkrvrf_i $ is always random. If $ \fgvrf $ did not check this, then the evaluation of $ m, \pkrvrf_i $ would be the same as the evaluation of $ m, \pkrvrf_j  \neq \pkrvrf_i$ whose anonymous key is $ W $.


\paragraph{Determinism:} Once evaluation of $ (m, \pkrvrf_i) $, which is $ \evaluationslist[m,W] $ where $ \anonymouskeymap[m,W] = \pkrvrf_i $, is set, it cannot be changed. 

$ \fgvrf $ satisfies determinism because it checks whether $ (m, \pkrvrf_i) $ is evaluated before any time that it needs it. The only way for $ \simulator $ to change the evaluation of $ (m, \pkrvrf_i) $ is by changing the anonymous key of $ (m, \pkrvrf_i)  $ but the anonymous key cannot change once it is set i.e., $ \fgvrf $ always checks whether there exists $ W $ where $ \anonymouskeymap[m,W]  = \pkrvrf_i$ whenever it needs the evaluation value.

\paragraph{Anonymity:} An honest signature $ \sigma $ of a message $ m $ verified by a ring $ \pkring $ does not give any information about its signer except that its key is in $ \pkrvrf $.

$ \fgvrf $ satisfies anonymity. $ \simulator $ learns the signature of an honest party with an anonymous key $ W $ via malicious requests of signatures and outputs. Remark that $ W $ is sampled randomly and the signature generation algorithm $ \gen_{sign} $ generates a signature independent from the public key and the message. Therefore, the signature does not give any information about its signer except that its key is one of the honest keys in  $ \pkring  $.

\paragraph{Unforgeability:}  If an honest party with a public key $ \pkrvrf $ never signs a message $ m $ for a ring $ \pkring $, then no party is able to generate a valid signature of $ m $ for $ \pkring $ signed by $ \pkrvrf $. 

We need verify that $ \sim $ cannot generate a signature $ \sigma $ that signs a message $ m $ for a ring $ \pkring $ by an honest party's key $ \pkrvrf $. In other words, we need to verify that if $ \fgvrf $ never received a message $ (\msg{sign}, \sid,\pkring,\pkrvrf,m) $ from an honest party $ \user $ with the key $ \pkrvrf $, $ \fgvrf $ cannot have a record $ [m, W, \pkring, \sigma, 1] $ such that $ \anonymouskeymap[m,W]  = \pkrvrf$ (meaning that $ \sigma $ is a valid signature generated by the honest key $ \pkrvrf $). 
$ \sim $ cannot execute the forgery attack by sending a message $ (\msg{sign}, \sid,\pkring,\pkrvrf,m) $ to $ \fgvrf $ because $ \fgvrf $ checks whether the sender of this message is associated with the key $ \pkrvrf $ to generate a signature. Another way for $ \sim $ to create a forgery is by sending an honest key $ \pkrvrf_\simulator $  in \ref{cond:malicioussignature}. However, it is not allowed by $ \fgvrf $ in \ref{cond:forgery}.


%
%to make $ \fgvrf $ set $ \anonymouskeymap[m,W] $ with an honest key for any $ m,W $.  Let's see whether this is possible. The places that $ \fgvrf $ can set $ \anonymouskeymap[m,W] = \pkrvrf $  is during ``malicious ring VRF evaluation", ``honest ring VRF signature and evaluation" and ``verification". Clearly, this event cannot happen via malicious ring VRF evaluation  because $ \pkvrf $ is an honest key. It cannot happen via  ``honest ring VRF signature and evaluation''  because we assume that the party $ \user $ with the key $ \pkrvrf $ never asked for it. It cannot happen via ``verification'' because \ref{cond:forgery} prevents $ \simulator $ to generate an anonymous key for $ \pkrvrf $.      This shows us that $ \simulator $ cannot generate a signature.

%
% to generate a valid signature is via verification i.e., when a party sends a message  $ (\msg{verify}, \sid,\pkring,W,m, \sigma) $ to $ \fgvrf $.  During the verification, if $ \fgvrf $ is in \ref{cond:differentsignature} and \ref{cond:simulatorbit} then the validity of the signature is decided by $ \simulator $. If $ \fgvrf $ is in \ref{cond:simulatorbit}, it means that there exists no $ \anonymouskeymap[m,W] = \pkrvrf \in \mathcal{P}_H $ because if it existed, $ \fgvrf $ would be in \ref{cond:differentsignature}. Therefore, the signature verified in \ref{cond:simulatorbit} cannot be a signature of an honest party's key.  This means that $ \simulator $ cannot generate a forgery via \ref{cond:simulatorbit}. So, the only left way for $ \simulator $ to generate a forgery is via \ref{cond:differentsignature}.
%If $ \fgvrf $ is in \ref{cond:differentsignature}, then $ \anonymouskeymap[m,W] $ belongs to an honest party and another signature $ \sigma' \neq \sigma $ has already been stored as valid for $ W, \pkring, m $.  If $ \sigma' $ is not generated by this honest party, then it means that $ \simulator $ forges. Let's see whether this is possible. If there exists a record $ [m, W, \pkring, \sigma',1] $ , it means that $ \anonymouskeymap[m,W] = \pkrvrf \in \mathcal{P}_H$ exists. The places that $ \fgvrf $ can set $ \anonymouskeymap[m,W] = \pkrvrf $  is during ``malicious ring VRF evaluation", ``honest ring VRF signature and evaluation" and ``verification". Clearly, this event cannot happen via malicious ring VRF evaluation  because $ \pkvrf $ is an honest key. It cannot happen via  ``honest ring VRF signature and evaluation''  because we assume that the party $ \user $ with the key $ \pkrvrf $ never asked for it. It cannot happen via ``verification'' because \ref{cond:forgery} prevents $ \simulator $ to generate an anonymous key for $ \pkrvrf $.      This shows us that $ \simulator $ cannot generate a signature via \ref{cond:differentsignature}.

\paragraph{Uniqueness:} The number of verified outputs via signatures for a message and a ring $ \pkring $ is not more than $ |\pkring| $.
 
 We need to verify that number of outputs for a message $ m $ that are verified by the ring $ \pkring $ is not greater than $ |\pkring| $.
Assume that there exist $ t + 1 $ verified signatures $ \setsym{S} = \{\sigma_1, \sigma_2, \ldots, \sigma_t\} $ of a message $ m $ for a ring $ \pkring $ where $ |\pkring| = t $ and $ \fgvrf $ outputs $ 1, y_i $ for each query $ (\msg{verify}, \sid, \pkring, W_i, m, \sigma_i) $ where $ y_i \neq y_j $ for all $ i \neq j $. For each $ \sigma_i \in \setsym{S} $, $ \anonymouskeymap[m,W_i] = \pkrvrf_i $ and $ \out[m,W_i] = y_i $. Clearly, in this case, $ W_i \neq W_j $ for all $ i \neq j $ since $ y_i \neq y_j $ by our assumption. $ \fgvrf $ generates a signature for an honest party if the honest key is in the ring. Therefore, the number of anonymous keys (so that evaluation outputs) for the honest signatures is at most number of honest keys in $ \pkring $. Since we know that once anonymous key $ W_i $ is set for $ m, \pkvrf_i $, it cannot be changed. This means that there exist at least $ |\pkring_{mal}| + 1 $ signatures in $ \setsym{S} $. When $ \fgvrf $ verifies a malicious signature, it checks in \ref{cond:uniqueness} whether the number of malicious anonymous keys for $ m $ are greater than the number of malicious keys in the corresponding ring. Therefore, the simulator can generate at most $ |\pkring_m| $ anonymous keys generate for a ring $ \pkring $. This implies that the number of verified outputs of malicious parties   is $ |\pkring|_m $. 

\paragraph{Robustness:} $ \simulator $ cannot prevent an honest party to evaluate, sign or verify.

The only place that $ \fgvrf $ does not respond any query is when it aborts. This happens only when it selects an honest anonymous key which already existed. Since this happens in negligible probability,  $ \fgvrf $ is robust.

%Now, we verify that $ \fgvrf $ satisfies these properties. During our analysis, when we say that a message $ m $ signed by an anonymous key $ W $ we mean that $ [m,W,.,.,1] $ is recorded. We say that the signature is honest if $ \anonymouskeymap[m,W] = \pkrvrf $ is an honest party's key.
%
%
%
%\paragraph{\textbf{Uniqueness:}}
%
%\paragraph{\textbf{Robustness:}} We check whether $ \simulator$ can prevent an honest party signing and verifying. $ \fgvrf $ does not abort during the verification so an honest party can verify all signatures. $ \fgvrf $ aborts during the honest signing process if $ \gen_{sign}(m, \pkring) $ generates a signature which was invalidated before i.e., there exists a record $ [m, W, \pkring, \sigma,0] $.
%
%
%
%
%\begin{enumerate}[label={{R-} }{{\arabic*}}, start = 1]
%
%%\item The ring VRF signature does not need to be random but it must be \emph{unique}  for its ring and the message. The reason of it to have a mapping from a ring VRF signature to its evaluation output. The map is necessary for $ \fgvrf $ to output the corresponding evaluation value for the signature during the verification process i.e, $ [m, \pkring, \sigma] \rightarrow \pkrvrf, \evaluationsecretlist[m, \pkring][\pkvrf] \rightarrow y $.
%\item In classical VRF, a VRF $ F $ is a deterministic function which maps a message and a public key to a random output. While in ring VRF, a message, a public key and a ring map to a random value, the verification algorithm of a ring VRF does not take the key as an input because it should be hidden. Therefore, the verification should be executed without the public key.  So, the functionality $ \fgvrf $ needs to find a way to verify the ring VRF output of a message, a public key and a ring map without knowing the public key. Because of this, $ \fgvrf $ generates an anonymized key $ W $ for each evaluation so that a message $ m $  and $ W $ maps to the random output. One can imagine this  as if a VRF output is generated with the input message $ m $ and the key $ W $ as in classical VRF i.e.,  $ F(m, W) $. 
%
%\item  If an honest party signs a message for a ring and obtains a signature, $ \fgvrf $ allows the simulator to generate another signature in \ref{cond:differentsignature} if the simulator wants. We remark that this is not a security issue because an honest party has already committed to sign the message.  A similar condition  exists in the EUF-CMA secure signature functionality $ \fsig $ \cite{canettiFsig}.
%
%\item \ref{cond:simulatorbit} of the ring VRF verification process covers the case where the adversary decides whether accepting the signature generated for its key if  it could be a valid signature for the ring i.e., the malicious key is in the ring and the anonymous key in the verification request is unique.
%
%\item The linking signature and the linking verification works similar to the EUF-CMA secure signature functionality $ \fsig $ \cite{canettiFsig}.


%\end{enumerate}


%\begin{definition}[Anonymous $ \fgvrf $]\label{def:anonymity}
%	We call that $ \fgvrf $ is anonymous if the outputs of $ \gen_{sign} $ and $ \gen_W $ are pseudo-random.
%	%TODO define this more formally.
%\end{definition}


%
%Below, we define the real-world execution of a ring VRF.
%\begin{definition}[Ring-VRF (rVRF)]\label{def:ringvrf}
%	%TODO ADD anonymous key here
%	A ring VRF is a VRF with a  function $ F(.):\{0,1\}^* \rightarrow \{0,1\}^{\ell_\rvrf} $ and with the following algorithms:
%	
%	\begin{itemize}
%		\item $ \rvrf.\keygen(1^\kappa) \rightarrow (\skrvrf,\pkrvrf)$ where $ \kappa $ is the security parameter,
%	\end{itemize}
%	Given list of public keys $ \pkring = \set{\pkrvrf_1, \pkrvrf_2, \ldots, \pkrvrf_n}$, a message $ m \in \{0,1\}^{\ell_m} $
%	\begin{itemize}
%		\item $ \rvrf.\eval(\skrvrf_i, \pkring, m) \rightarrow y$
%		\item $ \rvrf.\sign(\skrvrf_i, \pkring, m)\rightarrow (\sigma,W) $ where  $\sigma $ is a signature of the message $ m $ signed by $ \skrvrf_i, \pkring $ and $ W $ is an anonymous key.
%		\item $ \rvrf.\verify(\pkring,W, m,\sigma) \rightarrow  (b, y)$ where $ b \in \{0,1\} $ and $ y \in \{0,1\}^{\ell_\rvrf}\cup \{\perp\} $. $ b =1 $ means verified and $ b = 0 $ means not verified.
%%		\item $ \rvrf.\link(\skrvrf_i, \pkring,W,m, \sigma) \rightarrow \hat\sigma $ where  $ \hat\sigma $ is a signature that links signer of the ring signature $ \sigma $. 
%%		\item $ \rvrf.\link\verify( \pkrvrf_i,\pkring,W, m, \sigma, \hat\sigma)\rightarrow b$ where $ b \in \{0,1\} $. $ b =1 $ means verified and $ b = 0 $ means not verified.
%	\end{itemize}
%	
%\end{definition}

%
\section{Pedersen VRF and rVRF}
\label{sec:pederson_vrf}

% We adopt standard notaton for pairing friendly curves, so 

An EC VRF like \cite{nsec5,VXEd25519,draft-irtf-cfrg-vrf-10} consists
of a Chaum-Pedersen DLEQ proof between the signer's public key
$\pk = \sk \genG$ and a VUF output $\PreOut = \sk H_{\grE}(\msg)$,
so applying a PRF yields a VRF output
 $\Out = H'(\msg, h \PreOut)$ ala \cite[Proposition 1]{vrf_micali}.
%
Our {\em Pedersen VRF} \PedVRF alters the EC VRF by repalcing the
public key by a Pedersen commitment $\pk + \openpk \, \genB$, which
instantiates the $\Reval$ NIZK from \S\ref{sec:overview} efficently.
\footnote{As Groth16 dominates ring VRF verification costs,
we describe only the non-batchable variant analogous to
\cite{nsec5,VXEd25519,draft-irtf-cfrg-vrf-10}, but
 batch verifiable flavors exist.}

% We define security for only our ring VRF constructions, but clearly
%  \PedVRF consists of algorithms having superficially similar signatures.
% \footnote{We do not define security for \PedVRF because pseudo-randomness becomes too interesting}

We select a base point $\genG$ for our public key arbitrarily, % by any desired method,
but then fix a second generator $\genB$ of $\grE$ independent from $\genG$.
%
We define \KeyGen exactly like EC VRF, but
 \Eval differs by not injecting \pk into \msg:
\begin{itemize}
\item $\PedVRF.\KeyGen$ \quad returns $\sk \leftsample \F_p$ and $\pk = \sk \, \genG$.
\item $\PedVRF.\Eval : (\sk,\msg) \mapsto H'(\msg, h \, \sk \, H_{\grE}(\msg))$
\end{itemize}
% \item $\PedVRF.\KeyGen$ selects a secret key \sk uniformly at random from $\F_p$ and computes the public key $\pk = \sk \, \genG$. 
% \item $\PedVRF.\Eval(\sk,\msg)$ takes a secret key \sk and an input $\msg$, and
%  then returns a VRF output $H'(\msg, h \, \sk \, H_{\grE}(\msg))$.

\noindent We form Pedersen-like commitments to this public key \pk:
\begin{itemize}
\item $\PedVRF.\CommitKey$ \,
returns a blinding factor $\openpk \leftsample \F_p$
and a commitment $\compk = \pk + \openpk \, \genB$.
\item $\PedVRF.\OpenKey$ \,
returns $\pk = \compk - \openpk \, \genB$.
\end{itemize}
% \item $\PedVRF.\CommitKey$ selects a blinding factor $\openpk$ uniformly
%  at random from $\F_p$ and computes the commitment $\compk = \pk + \openpk \, \genB$.
% \item $\PedVRF.\OpenKey$ just returns $\pk = \compk - \openpk \, \genB$.
Alone these hide \pk, but they only provide a binding commitment
provided that $\PedVRF.\Verify$ below succeeds too.

\begin{itemize}
\item $\PedVRF.\Sign : (\sk,\openpk,\msg,\aux) \mapsto \sigma$ \,
    % takes a secret key \sk and blinding factor \openpk, an input $\msg$, and auxiliary data \aux, and then performs
    first computes $\In := H_{\grE}(\msg)$ and $\PreOut := \sk \, \In$,
    samples $r_1,r_2 \leftsample \F_p$,
    computes $R = r_1 \genG + r_2 \genB$, and $R_\msg = r_1 \In$, and
    finally $c = H_p(\aux,\msg,\compk,\PreOut,R,R_m)$,
     along with $s_1 = r_1 + c \, \sk$ and $s_2 = r_2 + c \, \openpk$.
    and finally returns the signature $\sigma = (\PreOut,c,s_1,s_2)$.
% \begin{enumerate}
%    \item compute the VRF input point $\In := H_{\grE}(\msg)$ and pre-output $\PreOut := \sk \, \In$,
%    \item Sample random $r_1,r_2 \leftarrow \F_p$ and compute $R = r_1 \genG + r_2 \genB$ and $R_\msg = r_1 \In$.
%    \item Compute the challenge $c = H_p(\aux,\msg,\compk,\PreOut,R,R_m)$,
%     along with $s_1 = r_1 + c \sk$ and $s_2 = r_2 + c \, \openpk$.
%    \item Return the signature $\sigma = (\PreOut,c,s_1,s_2)$.
% \end{enumerate}
\item $\PedVRF.\Verify : (\compk,\msg,\aux,\sigma) \mapsto \Out \,\, \lor \perp$ \,
    parses $\sigma = (\PreOut,c,s_1,s_2)$, 
    recomputes $\In := H_{\grE}(\msg)$, as well as
    $R = s_1 \genG + s_2 \genB - c \, \compk$, and
    $R_m = s_1 \In - c \PreOut$, and finally
    if $c = H_p(\aux,\msg,\compk,\PreOut,R,R_\msg)$ then it return $H'(\msg, h \, \PreOut)$, 
         or failure $\perp$ otherwise.
% \begin{enumerate}
%    \item recompute the VRF input point $\In := H_{\grE}(\msg)$,
%    \item computes $R = s_1 \genG + s_2 \genB - c \, \compk$ and $R_m = s_1 \In - c \PreOut$, and
%    \item returns $H'(\msg, h \, \PreOut)$ if $c = H_p(\aux,\msg,\compk,\PreOut,R,R_\msg)$ or failure otherwise.
% \end{enumerate}
\end{itemize}

\noindent We obtain EC VRF if we choose $\openpk = 0 = r_2$ in \Sign and demand $s_2 = 0$ in \Verify.
%
\eprint{We define security only for our ring VRF constructions, but clearly
 \PedVRF consists of algorithms having superficially similar signatures.}{}
% \footnote{We do not define security for \PedVRF because pseudo-randomness becomes too interesting}

\smallskip
% \subsection{Pedersen rVRF-AD}

As described in \S\ref{sec:overview},
we instantiate a rVRF-AD from \PedVRF plus a ring commitment scheme
 $\rVRF.\{ \CommitRing, \OpenRing \}$.
\rVRF inherits $\rVRF.\KeyGen = \PedVRF.\KeyGen$ and
 $\rVRF.\Eval = \PedVRF.\Eval$ of course, but requires.
We need zero-knowledge ring membership proof for the relation \Rring
which handles both $\PedVRF.\OpenKey$ and $\rVRF.\OpenKey$ efficently.
% \vspace{-0.1in}
$$ \Rring = \Setst{ \compk, \comring }{
    \eprint{ \exists \openpk,\openring \textrm{\ s.t.\ } }{}
    \genfrac{}{}{0pt}{}{\PedVRF.\OpenKey(\compk,\openpk) \quad}{\,\, = \rVRF.\OpenKey(\comring,\openring)}
} \mathperiod $$

\begin{itemize}
\item $\rVRF.\rSign : (\sk,\openring,\msg,\aux) \mapsto \rho$ \,
 % takes a secret key \sk, a ring opening \openring, a message \msg, and \aux, and then % auxiliary data
 samples $\openpk \leftsample \F_p$ and
 % computes a proof $\piring$, a signature $\sigma$, and
 returns $\rho = (\compk,\piring,\sigma)$ where      % a ring VRF signature
 $$ \piring = \NIZK.\Prove(\Rring,\compk,\comring,\openpk,\openring) \quad\textrm{and} $$
 $$ \sigma = \PedVRF.\Sign(\sk,\openpk,\msg,\aux \doubleplus \compk \doubleplus \piring \doubleplus \comring) \mathperiod $$ % finally
\item $\rVRF.\rVerify : (\comring,\msg,\aux,\rho) \mapsto \Out \,\, \lor \perp$ \,
 parses $\rho$ as $(\compk,\piring,\sigma,)$ and returns
 $$ \PedVRF.\Verify(\compk,\msg,\aux \doubleplus \compk \doubleplus \piring \doubleplus \comring,\sigma) $$
 if $\NIZK.\Verify(\piring,\compk,\comring)$ succeeds too.
\end{itemize}

\begin{proposition}\label{prop:pedersen_rvrf}
$\rVRF$ satisfies ring uniqueness, ring unforgeability, and ring anonymity.
\end{proposition}






\endinput





TODO:  Eprint form?

\begin{itemize}
\item $\rVRF.\rSign : (\sk,\openring,\msg,\aux) \mapsto \sigma$ takes
 a secret key \sk, a ring opening \openring, a message \msg, and \aux, and then % auxiliary data
 % \begin{enumerate}
 % \item
 generates \openpk, computes a ring membership proof $\piring$
  $$ \piring = \NIZK \Setst{ \compk, \comring }{
  \exists \openpk,\openring \textrm{\ s.t.\ } 
  \genfrac{}{}{0pt}{}{\PedVRF.\OpenKey(\compk,\openpk) \quad}{\,\, = \rVRF.\OpenKey(\comring,\openring)}
  } $$
 % \item
 computes the signature
  $$ \sigma = \PedVRF.\Sign(\sk,\openpk,\msg,\aux \doubleplus \compk \doubleplus \piring \doubleplus \comring), \quad\textrm{and} $$ % finally
 % \item
 returns the ring VRF signature $\rho = (\compk,\piring,\sigma)$.
 % \end{enumerate}
\item $\rVRF.\rVerify$ takes $(\comring,\msg,\aux,\rho)$,
 parses $\rho$ as $(\compk,\piring,\sigma,)$,  and then returns
 $$ \PedVRF.\Verify(\compk,\msg,\aux \doubleplus \compk \doubleplus \piring \doubleplus \comring,\sigma) $$
 iff $\NIZK.\Verify(\piring,\compk,\comring)$ succeeds. 
\end{itemize}























\begin{lemma}\label{prop:pedersen_vrf_hiding}
$\PedVRF$ is a correct key commitment and key hiding.
\end{lemma}

Although Pedersen commitments are perfectly hiding, our $\R_\msg$ makes $\sigma$ only computationally hiding.

\begin{proposition}\label{prop:pedersen_vrf}
Assuming AGM in $\grE$, % $\ecE$ modulo $h$,
our $\PedVRF$ satisfies VRF correctness, key binding, uniqueness,
pseudo-randomness, and unforgeability. % (EUF-CMA-KC) on $(\msg,\aux)$.
\end{proposition}

We need this second verification equation in \PedVRF, but not in \ThinVRF,
because otherwise our $s_2 \genB$ term provides enough freedom to tamper
with the pre-ouputs.  

We could however generalize \PedVRF to $k$ messages $\msg_1,\ldots,\msg_k$
similarly to \ThinVRF in \S\ref{subsec:vrf_thin}:  We compute for
$j=1,\ldots,k$ the $k$ distinct
points $\In_j := H_{\grE}(\msg_j)$, pre-outputs $\PreOut := \sk \, \In$,
delinearization challenges
 $c_j = H_p(\aux,\msg_1,\ldots,\msg_k,\compk,\Out_{0,1},\ldots,\Out_{0,k},j)$,
and then use the \PedVRF proof for
 $\In = \sum_j c_j \In_j$ and $\Out = \sum_j c_j \Out_j$.

% TODO: Proof correct?  Use same citations as schnorrkel.








\section{Zero-knowledge Continuations}
\label{sec:rvrf_cont}

\noindent In the following, we describe a NIZK for a relation $\rel$ where
$$\rel = \{(\bary, \barz; \barx, \baromega, \baromegap):  (\bary, \barx; \baromega) \in \relone, (\barz, \barx; \baromegap) \in \reltwo \},$$
and $\relone$, $\reltwo$ are some NP relations. Our NIZK is designed to efficiently re-prove membership for relation $\relone$
via a new technique which we call \emph{zero-knowledge continuation}. In practice, using a NIZK that is a zero-knowledge continuation 
ensures one essentially needs to create only once an otherwise expensive proof for $\relone$ which can later be 
re-used multiple times (just after inexpensive re-randomisations) while preserving knowledge soundness and zero-knowledge. 
Below, we formally define zero-knowledge continuation. In section~\ref{sec:rvrf_groth16} we instantiate it via a \emph{special(ized) 
Groth16} or \SpecialG, and finally, in section~\ref{subsec:rvrf_faster} we use it to build a ring VRF with fast amortised prover time. \\

\noindent In addition, the anonymity property of our ring VRF demands we not only finalise multiple times a component of the zero-knowledge 
continuation and but also each time the result remains unlinkable to previous finalisations, meaning our ring VRF stays zero-knowledge 
even with a continuation component being reused. We formalise such a more general zero-knowledge property in 
section~\ref{sec:rvrf_groth16} and give an instantiation of our NIZK fulfilling such a property in section~\ref{subsec:rvrf_faster}. 
%Moreover, the anonymity property of a ring VRF demands we finalise the amortized ``continuation'' multiple
%times, with each time being unlinkable to the others, meaning our rVRF
%stays zero-knowledge even with the continuation being reused.


%\begin{definition}[ZK Continuations] A zero-knowledge continuation $\SpecialG_\rel$ consists of four algorithms 
%($\SpecialG_\rel.\Preprove$, $\SpecialG_\rel.\Reprove$, $\ldots$, $\SpecialG_\rel.\Verify$) such that:
%\begin{itemize}
%\item $\SpecialG_\rel.\Preprove : (\bar{y}, \bar{x}; \bar{\omega}) \mapsto (X,\pi)$ \,
%constructs a commitment $X$ and a proof $\pi$ for relation $\rel$ from a vector 
%of inputs $\bar{y}$ (called \em{transparent}), a vector of inputs $\bar{x}$ (called \em{opaque}), and witnesses $\bar{\omega}$.
%\item $\SpecialG_\rel.\Reprove : (X,\pi) \mapsto ((X',\pi'); b)$ \,
%finalises the commitment $X'$ and proof $\pi'$ and returns an opening $b$ for the commitment. 
%\item $\SpecialG_\rel.\Verify(\bar{y}; (X',\pi') )$ \, 
%verifies the 
%\end{itemize}
%%TO DO: add an algorithm to the $\SpecialG_\rel$ such that: Our \Verify needs a separate proof-of-knowledge for $X'$, 
%%the production of which requires knowledge of $\bar{x}$, and occurs in parallel to \Reprove.

%We define (white-box) knowledge soundness for zero-knowledge continuations
%exactly like for zero-knowledge proofs, but with the composition 
%$\Prove : (\bar{y}, \bar{x}; \bar{\omega}) \mapsto \Reprove(\Preprove(\bar{y}, \bar{x}; \bar{\omega}))[0]$
%as well as this additional proof-of-knowledge.
%Zero-knowledge however should hold even if \Reprove gets invoked multiple
%times upon the same \Preprove results, again even with the additional proof-of-knowledge.
%\end{definition} 

\begin{definition}[ZK Continuations]
\label{def:zk_cont}
 A zero-knowledge continuation $\ZKCont$ for a relation $\relone$ with 
input $(\bary, \barx)$ and witness $\baromega$ is a tuple of efficient algorithms 
($\ZKCont.\Setup$, $\ZKCont.\Gen$, $\ZKCont.\Preprove$, $\ZKCont.\Reprove$, $\ZKCont.\VerifyCom$, $\ZKCont.\Verify$, $\ZKCont.\Sim$) 
such that for implicit security parameter $\lambda$,
\begin{itemize}

\item $\ZKCont.\Setup: (1^{\lambda}) \mapsto (\crs, \pp, \tw)$ a setup algorithm that on input the security parameter 
outputs a common reference string $\crs$ and a list $\pp$ of public parameters,

\item $\ZKCont.\Gen: (\crs, \relone) \mapsto (\crspk, \crsvk)$ \, 
outputs a pair of proving key $\crspk$ and verification key $\crsvk$, 

\item $\ZKCont.\Preprove: (\crspk, \bar{y}, \bar{x}, \bar{\omega}, \relone) \mapsto (X, \pi, b)$ \,
constructs commitment $X$ from a vector of inputs $\bar{x}$ (called \emph{opaque}) and 
constructs proof $\pi$ from vector of inputs 
$\bar{y}$ (called \emph{transparent}), from $\bar{x}$ and vector of witnesses $\bar{\omega}$, and 
also outputs $b$ as the opening for $X$,

\item $\ZKCont.\Reprove: (\crspk, X, \pi, b, \relone) \mapsto (X', \pi', b')$ \,
finalises commitment $X'$ and proof $\pi'$ and returns an opening $b'$ for the commitment, 

\item $\ZKCont.\VerifyCom: (\pp, X, \barx, b) \mapsto 0/1$ \, 
verifies that indeed $X$ is a commitment to $\barx$ with opening (e.g., randomness) $b$ and 
outputs 1 if indeed that is the case and 0 otherwise,
 
\item $\ZKCont.\Verify: (\crsvk, \bar{y}, X', \pi', \relone) \mapsto 0/1$ \, outputs $1$ in case it accepts and $0$ otherwise,

\item $\ZKCont.\Sim: (\tw, \bary, \relone) \mapsto \pi$ takes as input a simulation trapdoor $\tw$ and statement $(\bary, \barx)$ and returns an argument $\pi$,
\end{itemize}
and satisfies perfect completeness for $\Preprove$ and for $\Reprove$,  knowledge soundness and zero-knowledge as defined below:\\
%TO DO: Re-write this: We define (white-box) knowledge soundness for zero-knowledge continuations
%exactly like for zero-knowledge proofs, but with the composition 
%$\Prove : (\bar{y}, \bar{x}; \bar{\omega}) \mapsto \Reprove(\Preprove(\bar{y}, \bar{x}; \bar{\omega}))[0]$
%as well as this additional proof-of-knowledge.
%Zero-knowledge however should hold even if \Reprove gets invoked multiple
%times upon the same \Preprove results,
%again even with the additional proof-of-knowledge.
\noindent \textbf{Perfect Completeness for $\Preprove$} For every $(\bary, \barx; \baromega) \in \relone$ it holds:
\begin{align*}
\mathit{Pr} (& \ZKCont.\Verify(\crsvk, \bar{y}, X, \pi, \relone) = 1 \ \wedge \ \ZKCont.\VerifyCom (\pp, X, \barx, b) = 1\  | \ \\ 
                   & (\crs, \pp) \leftarrow \ZKCont.\Setup (1^{\lambda}), (\crspk, \crsvk) \leftarrow \ZKCont.\Gen(\crs, \relone), \\ 
                   & (X, \pi, b) \leftarrow \ZKCont.\Preprove(\crspk, \bar{y}, \bar{x}, \bar{\omega}, \relone)) = 1
\end{align*}

\noindent \textbf{Perfect Completeness for $\Reprove$} For every efficient adversary $A$ it holds: 
\begin{align*}
\mathit{Pr} (& (\ZKCont.\Verify(\crsvk, \bar{y}, X, \pi, \relone) = 1  = >  \ZKCont.\Verify(\crsvk, \bar{y}, X', \pi', \relone) = 1)  \ \wedge \  \\
                   & \wedge \ (\ZKCont.\VerifyCom (\pp, X, \barx, b) = 1 => \ZKCont.\VerifyCom (\pp, X', \barx, b') = 1) \ | \\
                   & (\crs, \pp) \leftarrow \ZKCont.\Setup (1^{\lambda}), (\crspk, \crsvk) \leftarrow \ZKCont.\Gen(\crs, \relone), \\ 
                   & (\bary, \barx, X, \pi, b) \leftarrow A(\crs,\pp, \relone), (X', \pi', b') \leftarrow \ZKCont.\Reprove(\crspk, X, \pi, b, \relone)) = 1
\end{align*}
\noindent \textbf{Knowledge Soundness} For every efficient adversary $A$ there exists an efficient extractor $E$ such that 
\begin{align*}
\mathit{Pr} (& (\ZKCont.\Verify(\crsvk, \bar{y}, X, \pi, \relone) = 1) \ \wedge \ (\ZKCont.\VerifyCom(\pp, X, \bar{x}, b) = 1) \ \wedge\ \\
                   & \wedge \ ( (\bary, \barx; \baromega) \notin \relone) \ | \ (\crs, \pp) \leftarrow \ZKCont.\Setup (1^{\lambda}), (\bary, \barx, X, \pi, b) \leftarrow A(\crs, \pp, \relone), \\ 
& \baromega \leftarrow E^{A}(\crs, \relone)) = \negl (\lambda),
\end{align*}
\noindent %where $\ZKCont.\Preprove_{|X}(\crspk, \bar{y}, \bar{x}, \bar{\omega}, \relone; b)$ means running the part of algorithm 
%$\ZKCont.\Preprove$ that computes and outputs $X$ with its regular inputs and using $b$ when randomness is required; 
where by $ E^{A}$ we denote the extractor $E$ that has access to all of adversary's $A$ messages during the protocol with the honest verifier. \\

\noindent \textbf{Perfect Zero-knowledge w.r.t. $\relone$} For all $\lambda \in \mathbb{N}$, for all  $(\bary, \barx; \baromega) \in \relone$, for all $X$, for all $b$, 
for every adversary $A$ it holds:
\begin{align*}
\mathit{Pr}(&(\crs, \pp) \leftarrow \ZKCont.\Setup (1^{\lambda}), (\crspk, \crsvk) \leftarrow \ZKCont.\Gen(\crs, \relone), \\ 
                  & (\pi', X', \_) \leftarrow \ZKCont.\Reprove (\crspk, X, \pi, b, \relone) \ | \\ 
                  & A(\crs, \pp, \pi', X', \relone)= 1 \ \wedge \ \ZKCont.\Verify(\crsvk, \bary, X, \pi, \relone) = 1) =  \\
= \mathit{Pr}(&(\crs, \pp) \leftarrow \ZKCont.\Setup (1^{\lambda}), (\crspk, \crsvk) \leftarrow \ZKCont.\Gen(\crs, \relone), \\
                     & (\pi', X') \leftarrow \ZKCont.\Sim(\tw, \bary, \relone) \ | \\ 
                     & A(\crs, \pp, \pi', X', \relone) = 1 \ \wedge \ \ZKCont.\Verify(\crsvk, \bary, X, \pi, \relone) = 1)
\end{align*}
 
\end{definition} 

% $$ \Lring = \Setst{ \compk, \comring }{
%  \exists \openpk,\openring \textrm{\ s.t.\ } 
%  \genfrac{}{}{0pt}{}{\PedVRF.\OpenKey(\compk,\openpk) \quad}{\,\, = \rVRF.\OpenRing(\comring,\openring)}
% } \mathperiod $$

% \smallskip
\subsection{Specialised Groth16}
\label{sec:rvrf_groth16}

Below we instantiate our zero-knowledge continuation notion with a scheme based on Groth16~\cite{Groth16} SNARK;
hence, we call our instantiation \emph{specialised Groth16} or \emph{special G}. In order to do that, we need a 
reminder of the definition of Quadratic Arithmetic Program (QAP).

\begin{definition}[QAP] A Quadratic Arithmetic Program (QAP) $\cQ = (\cA, \cB, \cC, t(X))$ of size $m$ 
and degree $d$ over a finite field $\F_q$ is defined by three sets of polynomials $\cA = \{a_i(X)\}_{i=0}^m$, 
$\cB = \{b_i(X)\}_{i=0}^m$, $\cC = \{c_i(X)\}_{i=0}^m$ of degree less than $d-1$ and a target degree $d$ polynomial $t(X)$. Given 
$\cQ$ we define $\relRQ$ over pairs $(\barx, \baromega) \in \F_q^{n} \times \F_q^{m-n}$ that holds iff there exist a polynomial $h(X)$ of degree 
at most $d-2$ such that:
$$(\sum_{k=0}^m y_k \cdot a_k(X)) \cdot (\sum_{k=0}^m y_k \cdot b_k(X)) = (\sum_{k=0}^m y_k \cdot c_k(X)) + h(X)t(X) \ \ (\ast)$$ 
where $y_0 =1$, $y_k = x_k$ for all $k \in [n]$ and $y_k = w_{k-n}$ for $k  \in [m] \setminus [n]$ and $\barx = (x_1, \ldots, x_n)$ and 
$\baromega = (w_1, \ldots, w_{m-n})$. 
\end{definition}


%Let $\mathbb{F}_q$ be a prime field, 
%let $G_1$, $G_2$, $G_T$ be as defined in section~\ref{??}, let $e$, $g$, $h$ be defined as $\ldots$. Let $t(X)$ and
%$\{u_i(X),v_i(X),w_i(X)\}_{i=0}^m$ be polynomials in $\F_q[X]$, let $\ldots$ be $\ldots$ such that there exists $h(X) \in \F_q[X]$ with
% $$ \sum_{i=0}^m a_iu_i(X) \cdot  \sum_{i=0}^m a_iv_i(X)  = \sum_{i=0}^m a_iw_i(X) + h(X)t(X)  \ (\ast)$$
%Then let $\relone = \{ (;) \ | \ (;)  (\ast) \}$

\begin{definition}[Specialised Groth16]
\label{insta:sg16} 
Specialised Groth16 is the following instantiation of our zero-knowledge continuation notion in Definition~\ref{def:zk_cont}:
\begin{itemize}
\item $\ZKCont.\Setup: (1^{\lambda}) \mapsto (\crs, \pp, \tw)$. \\ 
\noindent Pick $\alpha, \beta, \gamma, \delta, \tau  \xleftarrow{\$} \F_q^{*}$.  \\
Let $\tw = (\alpha, \beta, \gamma, \delta, \tau, \eta)$. \\ 
Let $\crs = ([\barsig_1]_1, [\barsig_2]_2)$ where \\ 
$\barsig_1$ = ($1$, $\alpha$, $\beta$, $\delta$, $\{\tau_i\}_{i=1}^{d-1}$, $\{\frac{\beta a_i(\tau)+ \alpha b_i(\tau)+ c_i(\tau)}{\gamma}\}_{i=1}^n$,  
$\frac{\eta}{\gamma}$, $\{\frac{\beta a_i(\tau)+ \alpha b_i(\tau)+ c_i(\tau)}{\delta} \}_{i=n+1}^m$, $\{\frac{1}{\delta}\sigma^it(\sigma)\}_{i=0}^{d-2}$, $\frac{\eta}{\delta}$) \\ 
and $\barsig_2 = (1, \beta, \gamma, \delta, \{\tau^i\}_{i=0}^{d-1})$. \\
Let $\pp = ({\color{red}\ldots})$. 
\end{itemize} 
\end{definition}

\begin{comment}
Zero-knowledge invariably comes from random blinding factors.
Zero-knowledge continuations need rerandomizable zkSNARKs,
meaning Groth16 \cite{Groth16}, but beyond rerandomization their
unlinkability demands hiding public inputs.
In our case, we ``specialize'' Groth16 to permit alteration of \openpk
in the $\PedVRF.\OpenKey$ invocation without reproving our heavy
$\rVRF.\OpenRing$ invocation.

In Groth16 \cite{Groth16}, we have an SRS $S$ consisting of curve
points in $\grE_1$ and $\grE_2$ that encode the circuit being proven.
We follow \cite{Groth16} in discussing the SRS $S$ in terms of
its ``toxic waste''
 $(\alpha,\beta,\delta,\gamma,\tau^1,\tau^2,\ldots) \in \F_q^*$.
In other words, we could write say $[ f(\tau)/\delta ]_1$ or $[\delta]_2$
to denote an element of our SRS $S$ in $\grE_1$ or $\grE_2$ respectively,
computed by scalar multiplication of the Groth16 generators from
the toxic waste $\tau$ and $\delta$,
 but for which nobody knows the underlying $\tau$ or $\delta$ anymore.

In the SRS $S$, we distinguish the verifiers' string/key of elements
 $\chi_1,\ldots,\chi_k, [\alpha]_1 \in \grE_1$ and
 $[\beta]_2, [\gamma]_2, [\delta]_2 \in \grE_2$.
% as separate from the provers' much longer string of elements in $\grE_1$ and $\grE_2$.
A Groth16 \cite{Groth16} proof takes the form 
 $\pi = (A,B,C) \in \grE_1 \times \grE_2 \times \grE_1$.
A verifier then produces a $X = \sum_i^k x_i \chi_i \in \grE_1$ from
 the instance's public inputs $x_i \in \F_p$ and then checks 
$$ e(A,B) = e([\alpha]_1, [\beta]_2) \cdot
 e(X, [\gamma]_2) \cdot e(C, [\delta]_2) \mathperiod $$

We need the rerandomization algorithm from \cite[Fig.~1]{RandomizationGroth16}:
% to build a zero-knowledge continuation:
% https://eprint.iacr.org/2020/811
% https://github.com/arkworks-rs/groth16/pull/16/files
% \algo{rerandomize}
An existing SNARK $(A,B,C)$ is transformed into a fresh
SNARK $(A',B',C')$ by sampling random $r_1,r_2 \in \F_p$ and computing
% $$
% A' = {1 \over r_1} A, \qquad
% B' = r_1 B + r_1 r_2 [\delta]_2, \qquad
% C' = C + r_2 A \mathperiod
% $$
$$ \begin{aligned}
A' &= {1 \over r_1} A \\
B' &= r_1 B + r_1 r_2 [\delta]_2 \\
C' &= C + r_2 A \mathperiod \\
\end{aligned} $$
At this point, our $x_i$ remain identical after rerandomization,
so $X$ links $(A,B,C)$ to $(A',B',C')$.
Alone rerandomization cannot alter public inputs $x_i$, so
we instead need an opaque public input point $X$, which then becomes
part of our proof and incurs its own separate proof of correctness.

We build {\em special Groth16} aka \SpecialG by adding one fresh
basepoint $\genB_\gamma$ independent from all others,
 including the $H_{\grE}(\msg)$ in \PedVRF.%
\footnote{Apply the underlying $H_\grE$ to an input outside the \msg domain for example.}
In the trusted setup, we build one additional prover SRS element
$$ \genB_\delta := {\gamma\over\delta} \genB_\gamma \mathperiod $$
% Although $\genB_\gamma$ is independent,  we create $\genB_\delta$ during the trusted setup,  so the toxic waste $\gamma$ and $\delta$ remain secret.
After $\genB_\delta$ is created, our toxic waste $\gamma$ and $\delta$
disappear and subversion resistance could be checked
 like in \cite{cryptoeprint:2019/1162} plus also checking
$$ e(\genB_\gamma, [\gamma]_2) = e(\genB_\delta, [\delta]_2) \mathperiod $$

We now have a zero-knowledge continuation $\pi = (X,A,B,C)$ from which
our algorithm $\SpecialG.\Reprove : (X,A,B,C) \mapsto ((X',A',B',C'); b)$ produces an
unlinkable instance $\pi' = (X',A',B',C')$ by
 first sampling random $b,r_1,r_2 \in \F_p$ and then computing
$$ \begin{aligned}
X' &= X + b \genB_\gamma \\
A' &= {1 \over r_1} A \\
B' &= r_1 B + r_1 r_2 [\delta]_2 \\
C' &= C + r_2 A + b \genB_\delta \mathperiod \\
\end{aligned} $$
As our two $b$ terms cancel in the pairings, our special Groth16
rerandomization reduces to the standard Groth16 rerandomization
construction above,
 except with $X$ now an opaque Pedersen commitment.

% TODO:  Should we be saying opaque less and Pedersen more below?

Along side opaque inputs in $X = \sum_i^k x_i \chi_i$,
our verifier should typically enforce specific values by assembling
a few {\em transparent} inputs $Y = \sum_i^l y_i \Upsilon_i$ themselves.
In particular, our ring VRF verifiers should enforce the commitment
\comring for $\ring$, even if they outsource computing \comring.
We thus write $\SpecialG.\Preprove : (\bar{y}, \bar{x}; \bar{\omega}) \mapsto (X,A,B,C)$
where $(A,B,C) \leftarrow \primalgo{Groth16}.\Prove(\bar{y}, \bar{x}; \bar{\omega})$,
so a full \Prove algorithm works by composing \Preprove and \Reprove.

At this point $\SpecialG.\Verify(\bar{y}; (X',A',B',C') )$
 computes $X' + Y = X' + \sum_i^l y_i \Upsilon_i$ and checks
 the tuple $(X' + Y,A',B',C')$ like Groth16 does,
$$ e(A',B') = e([\alpha]_1, [\beta]_2) \cdot
 e(X' + Y, [\gamma]_2) \cdot e(C', [\delta]_2) \mathperiod $$
As our verifier does not build $X'$ themselves, we prove nothing
with this pairing equation unless the verifier separately checks
 a proof-of-knowledge that $X' = b \genB_\gamma + \sum_i^k x_i \chi_i$
 for some unknown $b,\bar{x}$.

\begin{lemma}\label{lem:unlinkable}
Our rerandomization procedure % $(X,A,B,C) \mapsto (X',A',B',C')$
transforms honestly generated \SpecialG zero-knowledge continuation $(X,A,B,C)$
into identically distributed \SpecialG proof $(X',A',B',C')$,
with identical opaque inputs $x_1,\ldots,x_k$ and
 identical transparent inputs $y_1,\ldots,y_l$.
\end{lemma}

\begin{proof}[Proof idea.]
Adapt the proof of Theorem 3 in \cite[Appendix C, pp. 31]{RandomizationGroth16}.
\end{proof}

% \begin{corollary}\label{cor:unlinkable}
%	If $\sigma'$ and $\sigma''$ are \PedVRF{}s then ???
% \end{corollary}

All told, our opaque rerandomization trick converts any conventional
Groth16 zkSNARK $\pi$ for $\Lring^\inner$ into a zkSNARK $\pi'$ for $\Lring$
with inputs split into a transparent part $\bar{y}$ vs opaque unlinkable part $X$.
% We explore two concrete $\pi$ proposals below.

Importantly, rerandomization requires only
 four scalar multiplications on $\ecE_1$ and
 two scalar multiplications on $\ecE_2$,
which  BLS12 curves make roughly equivalent to
 eight scalar multiplications on $\ecE_1$.

\begin{lemma}\label{lem:knowledge_soundness}
Assuming AGM plus the $(2n-1,n-1)$-DLOG assumption, and
 circuit size less than $n$,
then our zero-knowledge continuation \SpecialG satisfies knowledge soundness.
\end{lemma}

\begin{proof}[Proof idea.]
As our \Prove is composed from \Preprove and \Reprove, our challenger
learns the actual public input wire values and blinding factors.
Adapt the proof of Theorem 2 in \cite[\S3, pp. 9]{RandomizationGroth16},
observing that $K_\gamma$ and $K_\delta$ never interact with other elements. 
%TODO: Alistair or Oana, Do we even need the first sentense here?  nything more to say about the second?
\end{proof}

In fact, one could prove zero-knowledge continuations satisfy
weak white-box simulation extractability,  % under similar restrictions,
much like Theorem 1 in \cite[\S3, pp. 8 \& 11]{RandomizationGroth16}.
%TODO:  Alistair or Oana, what the hell did I mean by this?  -Jeff
We depend upon the specific simulator though, which itself increases
our dependence upon the usage of the zero knowledge continuation.
\end{comment}

\subsection{Continuation}
\label{subsec:rvrf_faster}

% TODO \PedVRF.\OpenKey(\compk,\openpk)

\def\longeq{=\mathrel{\mkern-10mu}=}% {=\joinrel=} % https://tex.stackexchange.com/questions/35404/is-there-a-wider-equal-sign
We describe a much faster choice \pifast for \piring with
opaque inputs $x_1 \longeq \sk$ and transparent inputs $y_1 \longeq \comring$
 so that taking
 $\genG \longeq \chi_1$, $\genB \longeq \genB_\gamma$, and $\openpk \longeq b$
in \PedVRF yields an incredibly fast amortized ring VRF prover.
Also \PedVRF itself proves knowledge of $X' =  \sk\, \chi_1 + b \genB_\gamma $,
 as required by $\SpecialG.\Verify$.
% $$ X' + Y = \comring\, \Upsilon_1 + \sk\, \chi_1 + b \genB_\gamma $$

A priori, we do not know $\chi_1$ during the trusted setup for $\pifast$,
which prevents computing $\pk = \sk\, \chi_1$ inside $\pifast$.
Instead, we propose $\ring$ contain commitments to $\sk$ over
some Jubjub curve $\ecJ$, while $\sk \in \F_p$ remains a scale for $\grJ$.

We know the large subgroup $\grJ$ of $\ecJ$ typically has smaller prime
order $p_\grJ$ than $\grE$, itself due to $\ecJ$ being an Edwards curve.
%
We thus choose $\sk_0,\sk_1 < p_\grJ$ with at least $\lambda$ bits
so that
 $\PedVRF.\sk = \sk_0 + \sk_1 \, 2^{\lambda} \mod p_\grE$
becomes our secret key.
\footnote{If $\lambda \approx 128$ then $p, p_\grJ > 2^{2\lambda-3}$.}
Our $\rVRF.\KeyGen$ \eprint{returns}{shall now return}
a secret key of the form $\rVRF.\sk = (\sk_0,\sk_1,d)$
 with $d \leftsample \F_{p_\grJ}$ and
a public key of the form
 $\rVRF.\pk = \sk_0\, \genJ_0 + \sk_1\, \genJ_1 + d \genJ_2$,
for some independent $\genJ_0,\genJ_1,\genJ_2 \in \grJ$. % (see \S\ref{subsec:AML_KYC}).
\footnote{Interestingly we avoid range proofs for $\sk_1$ and $\sk_2$ by this independence.}
We thus have a fairly efficient instantiation for $\Lring^\inner$ give by

$$ \Lfast^\inner = \Setst{ \sk_0 + 2^{128} \sk_1, \comring }{
 \eprint{ \exists d,\openring \textrm{\ s.t.\ } }{}
 % 0 < \sk_0,\sk_1 < 2^{128} \textrm{\ and\ } 
 \genfrac{}{}{0pt}{}{ \eprint{\rVRF.}{}\OpenRing(\comring,\openring) }{ \,\, = \sk_0 \genJ_0 + \sk_1 \genJ_1 + d \genJ_2 }
} \mathperiod $$

Applying our rerandomization \Reprove to $\pifast^\inner$ with opaque input
yields a zkSNARK $\pifast$ with the extra $\PedVRF.\OpenKey$ arithmetic to
have exactly the form $\piring$.

We explain later in \S\ref{sec:ring_hiding} how one could
choose $\chi_1$ independent before doing the trusted setup,
 and then wire $\chi_1$ into $\pifast$ inside $C$.
We could then prove $\pk = \sk\, \chi_1$ directly inside $\pifast^\inner$,
but doing so here requires slow non-native field arithmetic.

At this point, $\PedVRF.\Sign$ requires two scalar multiplications on $\ecE_1$
 and two on the somewhat faster $\ecE'$,
so together with rerandomization costing four scalar multiplications
on $\ecE_1$ and two on $\ecE_2$, our amortized prover time
 runs faster than 12 scalar multiplications on typical $\ecE_1$ curves. 
We expect the three pairings dominate verifier time, but
 verifiers also need five scalar multiplications on $\ecE_1$.

As an aside, one could construct a second faster curve with the same
group order as $\grE$, which speeds up two scalar multiplications
 in both the prover and verifier. 

Importantly, our fast ring VRF' amortized prover time now rivals
group signature schemes' performance \cite{group_sig_survey,}.
We hope this ends the temptation to deploy group signature like
 constructions where the deanonymization vectors matter.

% BEGIN TODO: Oana

\begin{theorem}\label{thm:knowledge_soundness}
\rVRF instantiated with \pifast and \PedVRF satisfies knowledge soundness.
\end{theorem}

\begin{proof}[Proof stetch.]
An extractor for \PedVRF reveals the opening of $X$ for us,
so our result follows from Lemma \ref{lem:knowledge_soundness}.
\end{proof}

% \begin{corollary}\label{cor:???}
% Our Pedersen ring VRF instantiated with \pifast satisfies ring unforgability and uniqueness.
% \end{corollary}

% \begin{theorem}\label{thm:pifast_anonymity}
% \rVRF instantiated with \pifast and \PedVRF satisfies zero-knowledge.
% \end{theorem}
%
% \begin{proof}[Proof stetch.]
% Assuming the same \comring, we know the zero-knowledge continuations
% are identically distributed by Lemma \ref{lem:unlinkable},
% even when reusing a zero-knowledge continuation $(X,A,B,C)$.
% It follows the typical simulator for \PedVRF ... WHAT???
% \end{proof}

% \begin{corollary}\label{cor:???}
% Our Pedersen ring VRF instantiated with \pifast satisfies ring anonymity.
% \end{corollary}

% END TODO: Oana


%
%%\section{Ring updates}
\label{sec:ring_updates}

We discuss \pifast representing public keys in $\grE$ in $\grJ$ already,
% along with circuit implementation details of $\PedVRF.\{ \CommitKey, \OpenKey \}$,
but otherwise mostly treated the ring commitment scheme
$\rVRF.\{ \CommitRing, \CommitKey, \OpenKey \}$ like a black box.

Although our $\rVRF.\rSign$ run fast, all users must update their
stored zkSNARK \pifast every time the ring $\ctx$ changes.
Almost any circuit works for \pifast though,
 which permits diverse optimizations depending upon usecase.


\subsection{Merkle trees} % {Poseidon}

In either \pifast and \pisafe configurations, 
our $\rVRF.\{ \CommitRing, \CommitKey, \OpenKey \}$ could implement a
Merkle tree using zkSNARK friendly hash functions like Poseidon \cite{poseidon}.
%
All users need $O(\log |\ctx|)$ data with every update, which sounds
reasonable but not free.  There is a fast moving literature on securing
and optimizing zkSNARK friendly hash functions, with different techniques
being better suited to different zkSNARKs or even curves.

TODO: Arity 9 for 300 constraints?   % \cite{Groth16} vs plookup \cite{plookup}.

We leave deeper discussion of zkSNARK friendly Merkle to the literature.
Instead we spend this section focusing upon the diversity of circuit
designs that fit our \pifast and \pisafe framework.


\subsection{Vector commitments}

Instead of Merkle trees, our zkSNARK $\pi$ could use polynomial based
vector commitments \cite{KZG} or so called ``Verkle trees'' \cite{??Verkle??} too.
%
Among these, aggregatable subvector commitments \cite{aSVC} permit
one server to compute a KZG commitment \comring together with all users'
ring opening \openring,  and then send each user their \openring encrypted.
Instead of Groth16 \cite{Groth16}, our \pisafe then consists
of additively blinding this KZG commitment and then open it in zero-knowledge.

TODO: Explain zero-knowledge KZG opening positions?!?

A priori, we cannot construct KZG commitments to secret keys,
so \pifast cannot be replaced by a KZG commitment so simply.
Yet, we could split \pifast into two parts similar to \pisafe and $\pisafedot$.

replace $\pisafedot$ with a zero-knowledge continuation
that swaps the KZG blinding for the secret key blinding used by \pifast.

We represent public keys in $\grE_1$ over $\ecJ$ like \pifast does,
so $J = \sk_0\, \genJ_0 + \sk_1\, \genJ_1 + d \genJ_2$.
After placing these into a KZG commitment over $\ecE$, a transperent
openning yields $J.x Y_2 + J.y Y_3$, so our blinded opening could
then yield $J.x Y_2 + J.y Y_3 + b \genB_\gamma$.
At this point, we need a zkSNARK \pifastdot to translate $J$ into $\sk$, so
$$ \pifastdot = \rrSNARK \Setst{ \sk_0 + \sk_1 2^{128}, \pk }{ 
 \exists d \textrm{\ s.t.\ }
 % 0 < \sk_0,\sk_1 < 2^{128} \textrm{\ and\ } 
 \pk = \sk_0 \genJ_0 + \sk_1 \genJ_1 + d \genJ_2
} \mathperiod $$
We compute $\pifastdot$ only once ever, unlike $\pisafedot$.

We must reblind the KZG commitment with each VRF signature, and
tweak $\PedVRF$ to prove knowledge for $Y_2$ and $Y_3$, so 
our marginal prover time winds up higher than naieve \pifast here.
We impose additional pairings upon verifiers too, likely suboptimal.


\subsection{Certificates} % \& revokation}

If an authority grants ring membership, then ring membership proofs
could simply verify some certificate by the authority, likely using
a signature on JubJub.

In this, we prefer a SNARK friendly random oracle,
because conventional random oracles cost like 30k constraints.
We also need a variable base scalar multiplication, which costs like
4k constraints, as well as a couple fixed base scalar multiplication.
A priori, these fixed base scalar multiplications cost roughly 700
constraints each, but ocasionally they cost only half this.   

We conjecture one fixed based scalar multiplication could be replaced
by adapting implicit certificate scheme technqiues,
 instead of simply a signature on a user provided key.

We typically need expiration dates in certificates, likely demanding
a range proof and maybe requiring that \pifast be recomputed more often.

% \subsection{Revokation}

As a rule, one needs some revokation path for certificates,
despite the underlying signature not being revokable. 
%
We suggest maitaining a seperate revokation list and then inside
\pifast prove non-membership in the revokation list.
% perhaps via \cite{???}.
In this way, we update \pifast only when the revokation list updates.
We expect this represents a significatn savings because the revokation
list could update far less often than the full ring \ctx itself.
% perhaps corresponding with expiration checks
% especially since ring membership cannot be traced across site so easily.

We already trust an authority with issuing certificates, so we trust
them with managing therevokation list too.  As such, our revokation list
non-membership proofs merely requires proving adjacency of the revoked
public keys lexicographically before and after our own public key.
If the revokation list requires secrecy, then VRFs could hide its ordering,
similarly to NSEC5 \cite{nsec5}.


\subsection{Append only logs}

If an append only log grants ring membership, then a recursive SNARK
could validate ring membership with each recursive addition being
relatively inexpensive.

In this, we need a $\pifast^0$ similar to \pifast as well as a
$\pifast^n$ that proves some $\comring_{n-1}$ to be the ancestor of
its own $\comring_n$ and recursively proves some $\pifast^{n-1}$ with
its own $\comring_{n-1}$ and the same $\sk$.
We expect half pairing cycles fit this usage nicely, although they complicate provers.

Append only logs still depand \pifast or \pisafe be reproven whenever
\ctx updates however, so they only reduce bandwidth they not CPU usage.
A priori, we expect prover complexity plus update fequency makes
append only logs suboptimal, but
 they remain an interesting corner of the design space.

%%\section{Ring hiding}% {Hiding rings} % ring membership circuits}
\label{sec:ring_hiding}

At first, one imagines sites would accept few rings because each ring
gives some users multiple ``Sybil'' identities within the site.
In practice however, we think many sites benefit from accepting
multiple overlapping rings for convenience and/or reach, but then
tolerate the resulting few ``Sybil'' users.

As sites accept more rings, we increase risks that each user's ring
\ctx reveals private user attributes, especially if
 users join many rings, sites accept many rings, and
 user agents manage the association poorly.
As a solution, we suggest tweaking \pifast to prove the ring itself
lies in some permitted set of rings, but hide the specific ring used.

We could achieve this using recursion inside \pifast of course,
but doing so lies out of scope.  We instead discuss using other
zero-knowledge continuation techniques or similar.


\subsection{Unique circuit}

As a first step, if all rings use the same circuit, then we hide the
ring among several rings using a second zero-knowledge continuation.
As this closely resembles \S\ref{subsec:rvrf_side_channel}, we prefer
a blind opening of a polynomial commitment \cite{KZG} to \comring choices,
accomplished with Caulk+ \cite{caulk+} or Caulk \cite{caulk} or similar.

As a special case, if users cannot change their keys too quickly, then
one could reduce the frequency with which users reprove their original
zero-knowledge by using multiple \comring choices across the history
of the same evolving ring database.

In this, we initially reserve space in for future \comring by padding
the polynomial commitment with say the base point, and then later
append new \comring using \cite{aSVC}.


\subsection{Multi-circuit}

We handle \comring in the multi-circuit case somewhat like the
unique circuit case.  We caution however that circuits should domain
separate their enforce \comring suitably.

\smallskip

A priori, \pifast chooses $\genG = \Chi_1$, which reveals the circuit,
due to its dependence upon the SRS like
$$ \Chi_1 = \left[ {\beta u_1(\tau) + \alpha v_1(\tau) + w_1(\tau) \over \gamma} \right]_1 \mathperiod $$

Instead, we propose to stabilize the public input SRS elements across circuits:
We choose $\Chi_{1,\gamma}$ independent before selecting the circuit
or running its trusted setup.
We then merely add an SRS element $\Chi_{1,\delta}$, for usage in $C$, that binds
our independent $\Chi_{1,\gamma}$ to the desired definition, so
$$ \Chi_{1,\delta} := \left[ {\beta u_1(\tau) + \alpha v_1(\tau) + w_1(\tau) - \gamma \Chi_{1,\gamma} \over \delta} \right]_1 \mathperiod $$
At this point, we replace $\Chi_1$ by $\Chi_{1,\gamma}$ everywhere and
include $\comring \, \Chi_{1,\delta}$ inside $C$.

In this way, all ring membership circuits could share identical
public input SRS points $\Chi_{1,\gamma}$, and similarly $\Chi_0$ if desired.

\smallskip

In their trusted setups, all Groth16 circuits wind up with unique
toxic waste $\alpha,\beta,\delta$ and hence unique SRS elements
$[\alpha]_1 \in \ecE_1$ and $[\beta]_2, [\delta]_2, [\gamma]_2 \in \grE_2$,
and a unique $\grE_T$ element $e([\alpha]_1, [\beta]_2)$.
We hence encounter the open questions:
How should we optimize blinding the verifier key elements derived from toxic waste?
In particular, could we choose some toxic waste elements identically across multiple trusted setups?

We could fix $\gamma=1$ across circuits for example. but in general
this question depends upon other factors, especially if the trusted setups
run concurrently.
If for example, $\alpha$ and/or $\beta$ could be identical across
concurrent trusted setups, then we avoid extensive complexity in
 handling $([\alpha]_1, [\beta]_2) \in grE_T$ terms.

We expect the $[\delta]_2$ might differ between different circuits.
As a nice solution, our Groth16 trusted setup could construct a
KZG polynomial commitment $\rho$ along with openings to the
various $\delta$ as $[\delta]_2$.  At this point, our signer could
blind open $\rho$ to the curve point $[\delta]_2$.
We think concurrent trusted setups could skip much complexity of
Caulk+ \cite{caulk+} or Caulk \cite{caulk} because actually the
 trusted setup erases all knowledge of all openings of $\rho$.

\endinput



\subsection{Post-quantum}





\subsection{SnarkPack}

TODO: Handle $\pi$ hashes?



%% Handan writes: We cannot prove the UC security of our protocol with these options. Also, they are not formally described for a conference submission. We  should just say merkle tree or ring in the protocol description section
%%\section{Application: Identity}
\label{sec:app_identity}

Anonymous VRFs yield anonymous identity systems:
After a user and service establish a secure channel and
the server authenticates itself with certificates, then
the user authenticates themselves by providing an anonymous
VRF signature with input \msg being the server's identity,
thus creating an anonymous or pseudo-nonymous identified session.

We expand this identified session workflow with an extra
update operation suitable for our ring VRF's amortized prover.
We discuss only \pifast here but all techniques apply to \pisk and \pipk similarly. 

\begin{itemize}
\item {\em Register} --
 Adds users' public key commitments into some public ring \ctx,
 after verifying the user does not currently exist in \ctx.
\item {\em Update} --
 User agents regenerate their stored SNARK \pifast every time \ctx changes,
 likely receiving \comring and \openring from some ring management service.
\item {\em Identify} --
 Our user agent first opens a standard TLS connection to a server \msg,
 both checking the server's name is \msg and checking certificate
 transparency logs, and then computes the shared session id \aux.
 Our user agent computes the user's identity
  $\mathtt{id} = \PedVRF.\Eval(\sk,\msg)$ on the server \msg,
 % Our user agent next rerandomizes \pifast, \compk, and \openpk, computes
 %  $\sigma = \PedVRF.\Sign(\sk,\openpk,\msg,\aux \doubleplus \compk \doubleplus \pifast)$
 and finally sends the server their ring VRF signature
 $\rVRF.\rSign(\sk,\openring,\msg,\aux)$ % $ = (\compk,\pifast,\sigma)$.
\item {\em Verify} -- 
 After receiving $(\compk,\pifast,\sigma)$ in channel \aux,
 the server named \msg checks \pifast on the input $\compk + \comring$,
 checks the VRF signature and obtains the user's identity $\mathtt{id}$.
 $$ \mathtt{id} = \PedVRF.\Verify(\compk,\msg,\aux \doubleplus \compk \doubleplus \pifast,\sigma) $$
\end{itemize}


\subsection{Browsers}

We must not link users' identities at different web sites, so user agents
must disable all cross site resource loading, referrer information, etc.
Yet, user agents could still load purely static resources, without metadata
like cookies or referrer information, especially purely content addressable
resources.

In other words, web browsers mostly fail these baseline privacy requirements.
We expect Tor browser and Brave both behave correctly however.
Apple's Safari trends towards preventing invasive cross site resources too.  
% There also do exist decentralized web aka web 3.0 projects whose stated aims
% include more privacy.
In any case, one could always specify rules against cross site privacy invasions
whenever writing ring VRF browser specifications.


\subsection{AML/KYC}\label{subsec:AML_KYC}

We shall not discuss AML/KYC in detail, because the entire field lacks
clear goals, and thus winds up being ineffective
 \cite{doi:10.1080/25741292.2020.1725366}.
% https://www.tandfonline.com/doi/full/10.1080/25741292.2020.1725366
% via https://twitter.com/ronaldpol/status/1491548352189587460
We do however observe that AML/KYC conflicts with security and privacy
laws like GDPR.  As a compromise between these regulations,
one needs a compliance party who know users' identities,
 while another separate service party knows the users' activities.
We propose this more efficient solution:

Instead our compliance party becomes an identity issuer who maintains
a ring \ctx consisting of one unique public key for each user.
Arbitrarily many service providers could ring VRF based identity proofs.
If later asked or subpoenaed, users could prove their relevant identities
to investigators, or maybe prove which services they use and do not use. 

Interestingly, \PedVRF could be run ``backwards'' to prove a specific
ring VRF output does not belong to the user, without revealing the users'
identities to investigators. 

Although our applications mostly ignore key multiplicity. 
AML/KYC demands suspects prove non-involvement using ring VRFs.

\begin{definition}\label{def:rvrf_exculpability}
We say \rVRF is {\em exculpatory} if we have an efficient algorithm
for equivalence of public keys, but a PPT adversary \adv cannot
find non-equivalent public keys $\pk_0,\pk_1$ with colliding VRF outputs.
% (perfectly or computationally)
% (either ever or with advantage negligible advantage in $\secparam$)
\end{definition}

We ad hoc rings make little sense for AML/KYC, so these ring VRFs become
exculpatory if one merely dedupliates keys during registration , like via
$\rVRF.\rVerify(\CommitRing(\{\pk\}),\mathsf{ring_name},\mathtt{""},\sigma)$.


\subsection{Moderation}
\label{subsec:moderation}

All discussion or collaboration sites have behavioral guidelines and
moderation rules that deeply impact their culture and collective values.

Our ring VRFs enables a simple blacklisting operation:
If a user misbehaves, then sites could blacklist or otherwise penalizes
their site local identity $\mathtt{id}$.
As $\mathtt{id}$ remains unlinked from other sites, we avoid thorny
questions about how such penalties impact the user elsewhere, and thus
can assess and dispense justice more precisely. 

At the same time, there exist sites who must forget users' histories
eventually, such as when users invoke GDPR compliance or to give children
room to make social mistakes.  In these cases, we suggest injecting
approximate date information into \msg along with the site name,
so \msg becomes site name along with the current year plus month or week.
In this way, users have only one stable $\mathtt{id}$ within the
approximate date range, but they obtain fresh $\mathtt{id}$s merely
by waiting until the next month or week.

As in \cite{PrivacyPass}, we could adjust \PedVRF to simultaneously
prove multiple VRF input-output pairs $(\msg_j,\mathtt{id}_j)$.
As doing so links these pairs together, sites could make users link
pseudo-nym creation date and current date, so users could have multiple
active pseudo-nyms, but only one active pseudo-nym per time period,
which prevents spam.
If instead we link only adjacent dates, then pseudo-nyms could
be abandoned and replaced, but abandoned pseudo-nyms cannot then
be reclaimed without linking to intervening dates.

In these ways, sites encode important aspects of their moderation rules
into the ring VRF inputs requested.  
% We expect this makes sites' values and culture more uniform, predictable, and transparent.


\subsection{Reduced pairings}
\label{sec:reduced_pairings}

At a high level, we distinguish moderation-like applications discussed
above, which resemble classic identity applications like AML/KYC, from
rate limiting applications discussed in the next section. 
%
In moderation-like applications, ring VRF outputs become long-term
stable identities, so users typically reidentify themselves many times
to the same sites.

As an optimization, our zero-knowledge continuation
should deterministically choose the coefficients $r_1,r_2,b$ used for
rerandomization in \S\ref{sec:rvrf_cont},
 seeded by \msg and \sk, meaning $r_1,r_2,b \leftarrow H(\sk,\msg)$. 
%
In this way, each $\mathtt{id}$ comes packages with the same unique % Groth16 SNARK
\pifast, so the verifier could cache valid pairs
$(\mathtt{id},H(\pifast),\mathtt{diffdate})$, and reaccept \pifast
without checking the Groth16 pairing equation whenever found cached.
%
We spend most verifier time checking the Groth16 pairing equation, so
this saves considerable CPU time. % assuming our cache wind up fast enough.

We still risk denial-of-service attacks by users who vary $r_1,r_2,b$ 
randomly however.  We therefore set $\mathtt{diffdate}$ to be the date
when our server last saw a different $H(\pifast)$ associated to
$\mathtt{id}$, or empty if $\mathtt{id}$ always used the same $H(\pifast)$.
We rate limit and verify more lazily if $\mathtt{diffdate}$ is non-empty,
and optionally verify somewhat lazily even if no cache entry exists.


%%\section{Application: Rate limiting}
\label{sec:app_rate_limits}

We showed in \S\ref{sec:app_identity} how ring VRFs give users only
one unique identity for each input \msg.  
We explained in \S\ref{subsec:moderation} that choosing \msg to be
the concatenation of a base domain and a date gives users a stream of changing identities.

We next discuss giving users exactly $n > 1$ ring VRF outputs aka
``identities'' per date, as opposed to the unique identity 


% \subsection{Implementation}

As a trivial implementation, we could include a counter $k = 1 \ldots n$
in \msg, so $\msg = \mathtt{domain} \doubleplus \mathtt{date} \doubleplus k$.


\subsection{Avoiding linkage}

Our trivial implementation leaks information about ring VRF outputs'
 ownership by revealing $k$:
%
An adversary Eve observes two ring VRF signatures with the same
$\mathtt{domain}$ and $\mathtt{date}$ so
$\msg_i = \mathtt{domain} \doubleplus \mathtt{date} \doubleplus k_i$
for $i=1,2$, but with different outputs $\Out_1$ and $\Out_2$.
If $k_1 \ne k_2$ then Eve learns nothing, but if $k_1 = k_2$ then
 Eve learns that $sk_1 \ne \sk_2$, maybe representing different users. 

We do not necessarily always care if Eve learns this much information,
but scenarios exist in which one cares.  We therefore briefly describe
several mitigations:

If $n$ remains fixed forever, then we could simply let all users
register $n$ ring VRF public keys in \ctx.
If $n$ fluctuates under an upper bound $N$, then we could create $N$
rings $\ctx_i$ for $i = 1 \ldots N$, and
 then blind \comring in \pifast similarly to \S\ref{sec:ring_hiding}.

Although simple, these two approaches require users construct $n$ or $N$
different $\pipk$ proofs every time the ring \ctx updates.

Instead of proving ring membership of one public key, $\pipk$ could
prove ring membership of a Merkle commitment to multiple keys, so
users have $\pisk^1,\ldots,\pisk^N$ for each of their multiple keys.

As a more flexible approach,
we could compute the hash-to-curve $\In := H_{\grE}(\msg)$ inside an
unamortized SNARK $\pi_{\mathtt{in}}$ and reveal only a Pedersen-like commitment
to $\In + \openpk^{-1} \genB$.  We then adjust \PedVRF to yield
a proof-of-knowledge of $\Out/\In$ subject to soundness of this
SNARK $\pi_{\mathtt{in}}$.

TOTO: Explain better!!

In all cases, we incur costs by hiding part of the input \msg, so
deployment should seriously consider if leaking $k$ suffices.


\subsection{Ration cards}

As a species, we expect $+3^{\circ}$ C or more likely $+4^{\circ}$ C
over the pre-industrial climate by 2100 \cite{IPCC}, which shall
reduce the Earth's carrying capasity below 1 billion people \cite{carrying_capasity}.
In the shorter term, we expect shortages of resources, energy, goods,
water, and food beginning during the next several decades, due to
climate change, ecosystem damage or collapse, and resource exhaustion
ala peak oil.  Invariably, nations manage shortages through rationing,
like during WWI, WWII, and the oil shocks.  

Ring VRFs support anonymous rationing:
Instead of treating ring VRF outputs like identities,
we treat them like nullifiers which could each be spent exactly once.

\def\expiry{e}
We fix a set $U$ of limited resource types, and dynamically define
an expiry date $\expiry_{u,d_0}$ and an availability $n_{u,d_0}$, 
both dependent upon the resource $u \in U$ and current date $d_0$.
We typically want a randomness beacon $r_d$ too, which prevents
anyone learning $r_d$ much before date $d$. 
% Among other usages, this reduces damages from key compromises.
As ring VRF inputs, we choose
 $\msg = u \doubleplus r_d \doubleplus d \doubleplus k$
where $u \in U$ denotes a limited resource,
 $d$ denotes an non-expired date meaning $\expiry_{u,d_0} < d \le d_0$,
 and $1 \le k \le n_{u,d_0}$.
In this way, our rationing system controls both daily consumption
via $n_{u,d_0}$ and time shifted demand via expiry time $\expiry_{u,d_0}$.

Importantly, our rationing system retains ring VRF outputs as nullifiers,
filed under their associated date $d$ and resource $u$, so nullifiers
expire once $d \le \expiry_{u,d_0}$ which permits purging old data rapidly.

We remark that fully transferable assets could have constrained lifetimes
too, which similarly eases nullifier management when implements using
blind signatures, zcash sapling, etc.  Yet, all these tokens require
an explicit issuance stage, while ring VRFs self-issue.

Among the political hurdles to rationing, we know certificates have
a considerable forgery problem, as witnessed by the long history of
fraudulent covid and TLS certificates.  It follows citizens would
justifiably protest to ration carts that operate by simple certificates.
Ring VRFs avoid this political unrest by proving membership in a public list.


\subsection{Multi-constraint rationing}

We could proving multiple \ThinVRF outputs with one signatures
in \S\ref{sec:vrf_thin}.  We needed \PedVRF to isolate the blinding 
factor when using a Pedersen commitment instead of a public key, but
exactly the same technique works for proving multiple \PedVRF outputs
in one ring VRF signature.

We could therefore impose simultanious rationing constraints for multiple
resources $u_1,\ldots,u_k$ by producing one ring VRF signature in which
\PedVRF proves correctness of pre-ouptuts for multiple messages 
 $\msg_j = u_j \doubleplus r_d \doubleplus d \doubleplus k$ for $j=1 \ldots k$.

As an example, purchasing some prepared food product could require espending
rations for multiple base food sources, like making a cake from wheat, butter,
eggs, and sugar.  


\subsection{Decommodification}

There exist many reasons to decommodify important services,
like energy, water, or internet,
 beyond rationing real physical shortage.
Ring VRFs fit these cases using similar \msg formulations.

As an example, a municipal ISP allocates some limited bandwidth capacity
among all residents.  It allocates bandwidth fairly by verifying ring VRFs
signatures on hourly \msg and then tracking nullifiers until expiry.

Aside from essential government services, commercial service providers
typically offers some free service tier, usually because doing so
familiarizes users with their intimidating technical product.

Some free and paid tier examples include DuoLingo's heats on mobile, 
continuous integration testing services, and many dating sites.

A priori, rate limiting cases benefit from unlinkability among individual
usages, not merely at some site boundary like moderation requires.
We thus use each ring VRF output only once, which prevents our cashing
trick of \S\ref{sec:reduced_pairings} from reducing verifier pairings.

Although rationing sounds valuable enough, we foresee services like ISP,
VPNs, or mixnets having many low value transactions.
In such cases, ring VRFs could authorize issuing a limited number of
fast simple single-use blind issued credentials, like blind signatures
ala GNU Taler \cite{taler} or PrivacyPass OPRF tokens \cite{PrivacyPass},
 which both solve the leakage of $k$ above too.

In principle, commercial service providers could sell the same tokens,
which avoids leaking whether the user uses the free or commercial tier.


\subsection{Delegation}

We sometimes want to delegate spending ring VRF outputs, without
creating a fully transferable asset.  In particular, parents might
delegate limited internet or streaming service access to a child,
but without making the token full transferable.

Among blind issued credential many support this, both
GNU Taler \cite{taler} and PrivacyPass \cite{PrivacyPass} could
be redeemed by a delegatee family member who trusts the original
delegator not to double spend, but transactions with untrusted spenders
risks double spending.  

Ring VRFs usage typically demands spender authenticate the specific
spending operations inside the the associated data \aux, but adjusting
\aux requires knowing \sk, perhaps unacceptable to the delegator.

We could however achieve delegation by treating the ring VRF like a
certificate that authenticates another public key held by the delegatee.
GNU Taler achieves delegation and other features like this.

We could similarly treat the ring VRF like an adaptor certificate aka
implicit certificate.  In other words, the delegatee learns the full
ring VRF signatures, but then the delegatee hides $s_1$ from downstream
recipients, and instead merely prove knowledge of $s_1$, usually via
a key exchange or another Schnorr signature.


%
\section{Applications}
\label{sec:app_short}

We briefly outline how ring VRFs could be used for
 identity, moderation, rationing, and games.


\eprint{\subsection{Identity}
\label{subsec:app_identity}}{\noindent{\textbf{Identity:}}}
Ring VRF outputs  provide users with stable identities across
arbitrarily many services given several conditions:
First, our ring VRF input should be stable for a given services,
 like by using services' urls.
Second, we demand an encrypted connection between the user agent and
the service, in which the service authenticates itself first,
 like by verifying TLS certificates.
Third, the user agent avoids identity leakage between different services,
 like by denying cross site resources.
Fourth, the server trusts the ring membership, like by trusting
 a third party who enforces a ring registration procedure.
Also, this third party updates users as the ring membership evolves.

An HTTPS workflow satisfying these conditions resembles:
\renewcommand{\pifast}{\ensuremath{\pi_{\ring}}}
\begin{itemize}
\item {\em Register} --
	Adds users' public key commitments into some \ring,
	after verifying the user does not currently exist in \ring.
\item {\em Update} --
	User agents regenerate their stored signature using
	$\SpecialG.\Preprove$
	each time \ring changes\eprint{, perhaps even receiving \comring and \openring
	from a ring management service.}{.}
\item {\em Identify} --
	Our user agent first opens a standard TLS connection to a server \msg,
	both checking the server's url is \msg and checking certificate
	transparency logs, and then computes the shared session id \aux.
	Our user agent computes the user's identity
	$\mathtt{id} = \PedVRF.\Eval(\sk,\msg)$ on the server id \msg,\eprint{
	Our user agent next rerandomizes \pifast, \compk, and \openpk using
	$\SpecialG.\Reprove( \pk, \pifast^\inner )$, computes
	$\sigma = \PedVRF.\Sign(\sk,\openpk,\msg,\aux \doubleplus \compk \doubleplus \pifast)$,
	and finally sends the server their ring VRF signature $(\compk,\pifast,\sigma)$}{ and generates ring VRF signature.}
	\eprint{Our user agent rejects identity requests from resources besides
	top/outer most frame.}{}
\item {\em Verify} -- 
	\eprint{After receiving $(\compk,\pifast,\sigma)$ in channel \aux,
	the server \msg checks $\SpecialG.\Verify( \comring, (\compk,\pifast) )$,
	checks the VRF signature, and obtains the user's identity $\mathtt{id}$, ala \\
	$\mathtt{id} = \PedVRF.\Verify(\compk,\msg,\aux \doubleplus \compk \doubleplus \pifast,\sigma)$.}{It verifies the ring VRF signature. If it verifies, it checks whether $ \mathtt{id}$ equals to the evaluation value generated from the verification process.}
\end{itemize}

\eprint{Anonymity depends largely upon certificate authentication, including
certificate transparency logs, in that users could otherwise login to
a site with fraudulent credentials.
%  We think cross site restrictions
%and \aux being the channel limit this attack vector somewhat though.
If stronger defences are desired then instead of \msg being the url,
\msg could be an air gapped public ``root'' key for the site or CA, which
then also certifies its TLS certificate.  }{}

\eprint{As an optimization, \Reprove could rerandomize deterministically based
upon $H(\sk,\msg)$, so servers could then cache $\pifast$ verification.
}{}

\eprint{}{\begin{comment}}
\subsection{AML/KYC}
\label{subsec:AML_KYC}

We shall not discuss AML/KYC in detail, because the entire field lacks
clear goals, and thus winds up being ineffective
\cite{doi:10.1080/25741292.2020.1725366}.
% https://www.tandfonline.com/doi/full/10.1080/25741292.2020.1725366
% via https://twitter.com/ronaldpol/status/1491548352189587460
We do however observe that AML/KYC typically conflicts with security
and privacy laws like GDPR.  As a compromise between these regulations,
one needs a compliance party who know users' identities,
while another separate service party knows the users' activities.
We propose a safer and more efficient solution:

Instead our compliance party becomes an identity issuer who maintains
a public \ring, and privately knows the users behind each public key.
As above, identity systems could employ \ring freely for diverse purposes.
If later asked or subpoenaed, users could prove their relevant identities
to investigators, or maybe prove which services they use and do not use. 

Interestingly \PedVRF could run ``backwards'' like
$H_{\grE'}(\msg) \ne \sk^{-1} \, \PreOut$
to show a ring VRF output associated to $\PreOut$
does not belong to the user, without revealing the users'
identity $\Hout(\msg, \sk \, H_{\grE'}(\msg))$ to investigators. 

Our applications mostly ignore key multiplicity. 
AML/KYC demands suspects prove non-involvement using ring VRFs.

\begin{definition}\label{def:rvrf_exculpability}
	We say \rVRF is {\em exculpatory} if we have an efficient algorithm
	for equivalence of public keys, but a PPT adversary \adv cannot
	find non-equivalent public keys $\pk_0,\pk_1$ with colliding VRF outputs.
	% (perfectly or computationally)
	% (either ever or with advantage negligible advantage in $\secparam$)
\end{definition}

A priori, our JubJub representations $\sk_0 \genJ_0 + \sk_1 \genJ_1$
used in \S\ref{subsec:rvrf_faster} and \S\ref{subsec:rvrf_side_channel}
costs us exculpability from Definition \ref{def:rvrf_exculpability}.
% Ad hoc rings ...
% Rings used for AML/KYC would be maintained by an authority and require
% some registration procedure, using government issued identity documents.

There is however a natural {\em exculpable public key} flavor $(\pk,\sigma)$,
in which
$\sigma = \Sign(\sk, \CommitRing(\{ \pk \},\pk).\openring, \mathtt{ring\_name}, \mathtt{""})$.
The singleton ring $\{ \pk \}$ ensure that 
$\rVerify(\CommitRing(\{\pk\}), \mathtt{ring\_name}, \mathtt{""}, \sigma)$
uniquely determines the secret key, so exculpability holds
if joining the ring requires $(\pk,\sigma)$.

% \begin{proposition}
% \end{proposition}
\eprint{}{\end{comment}}

%TODO:we don't prove anything related to multiple VRF input-output. We should make sure its security before adding it.
\eprint{\subsection{Moderation}
\label{subsec:moderation}}{\noindent\textbf{Moderation:}}
\eprint{%
All discussion or collaboration sites have behavioral guidelines and
moderation rules that deeply impact their culture and collective values.%
}{}
Our ring VRFs enables a simple blacklisting operation:
If a user misbehaves, then sites could blacklist or otherwise penalizes
their site local identity $\mathtt{id}$.
As $\mathtt{id}$ remains unlinked from other sites, we avoid thorny
questions about how such penalties impact the user elsewhere, and thus
can assess and dispense justice more precisely. 
At the same time, there exist sites who must forget users' histories
eventually, like under a ``right to be forgotten'' principle ala GDPR.
% or an ethical principles of social mistakes being ephemeral.

As users have distinct $\mathtt{id}$ for each \msg,
we obtain ephemeral identities if \msg consists of the url plus
the current week and month, or some other approximate date.
At this point, users have only one stable $\mathtt{id}$ within each
approximate date range, but they obtain fresh $\mathtt{id}$s merely
by waiting until the next week or month.

We then adjust \PedVRF to simultaneously prove multiple VRF input-output
pairs $(\msg_j,\mathtt{id}_j)$.  As in \cite{PrivacyPass}, we merely
delinearize $ \In  = H_\grE(\msg)$ and \PreOut in \rSign and \rVerify like:
\eprint{\begin{align*}
	x &= H(\msg_j,\mathtt{id}_j,\ldots,\msg_j,\mathtt{id}_j) \\
	\In &= \sum_j H_p(x,j) \, \In_j \\
	\PreOut &= \sum_j H_p(x,j) \, \PreOut_j \\
\end{align*}}{$ H(\msg_j,\mathtt{id}_j,\ldots,\msg_j,\mathtt{id}_j),
\In = \sum_j H_p(x,j),
\PreOut = \sum_j H_p(x,j) \, \PreOut_j $.}
In this way, \PedVRF proves the same secret key controls two or more
ephemeral identities, thereby constructing a stable identity from the
ephemeral identities.

At login, our site demands linked two input-output pairs given by
$\msg_1 = \mathtt{site\_name} \doubleplus \mathtt{current\_month}$ and
$\msg_2 = \mathtt{site\_name} \doubleplus \mathtt{registration\_month}$,
so users could have multiple active pseudo-nyms given by $\mathtt{id}_2$,
but only one active pseudo-nym per week, enforced by deduplicating
$\mathtt{id}_1$, which still prevents spam and abuse.
\eprint{Alternatively, we could associate users pseudo-nyms with their recently
seen $\mathtt{id}_1$ but link adjacent months.  In other words, we define
$\msg_j$ by the $j$th previous month, until reaching a previously used
$\mathtt{id}_1$.  In this model, pseudo-nyms could be abandoned, but
abandoned pseudo-nyms cannot then be reclaimed without linking intervening ones.}{}
% Although more costly, sites could permanently bans a few problematic
% users via the inequality proofs described in \S\ref{subsec:AML_KYC} too.
In these ways, sites encode important aspects of their moderation rules
into the ring VRF inputs they demand.  
% % We expect this makes sites' values and culture more uniform, predictable, and transparent.


\eprint{\section{Rate limiting}
\label{sec:app_rate_limits}}{\noindent\textbf{Rate limiting:}}
As a rate limiting device, we repeat this approximate date trick from
moderation, but also include a counter $k = 1 \ldots n$ in \msg, so
 $\msg = \mathtt{domain} \doubleplus \mathtt{date} \doubleplus k$.
Instead of treating ring VRF outputs like identities,
we now treat them like nullifiers which could each be spent exactly once,
 similarly to the nullifiers in ZCash or ecash systems.
We do leak information about nullifiers' ownership by revealing $k$ here:
An adversary Eve observes two ring VRF signatures with the same
$\mathtt{domain}$ and $\mathtt{date}$ so
$\msg_i = \mathtt{domain} \doubleplus \mathtt{date} \doubleplus k_i$
for $i=1,2$, but with different outputs $\Out_1$ and $\Out_2$.
If $k_1 \ne k_2$ then Eve learns nothing, but if $k_1 = k_2$ then
Eve learns that $sk_1 \ne \sk_2$, representing different users. 
We do not always care if Eve learns this much information, but users'
threat models should be clearly understood before making this choice.
In principle, we could hide $k$ if we replace \PedVRF by a flavor of
\Reval implemented using Groth16, 
 but which still fits our formulation in \S\ref{sec:overview}.
Indeed these \Reval choices could provide post-quantum anonymity, 
without expensive post-quantum soundness, perhaps interesting if leaking $k$ matters.


\eprint{\subsection{Ration cards}
\label{subsec:app_ration_carts}}{\noindent\textbf{Ration Cards:}}
\eprint{As a species, we expect $+3^{\circ}$C over the pre-industrial climate
by 2100 \cite{IPCC2022}, or more likely above $+4^{\circ}$C given
tipping points \cite{tipping2022}.  % https://www.youtube.com/watch?v=LxoyaCSWFGs
At these levels, we experience devastating famines as the Earth's
carrying capacity drops below one billion people \cite{carrying_capacity}.
In the near term, our shortages of resources, energy, goods, water,
and food shall steadily worsen over the next several decades, due to
climate change, ecosystem damage or collapse, and resource exhaustion
ala peak oil.  We expect synchronous crop failures around the 2040s
in particular \cite{climaterisk2021}. % https://nitter.it/ThierryAaron/status/1442442451541807109#m
% off topic: https://12ft.io/proxy?q=https%3A%2F%2Fwww.bbc.com%2Ftravel%2Farticle%2F20220816-why-theres-no-dijon-in-dijon-mustard
Invariably, nations manage shortages through rationing, like during WWI, WWII, and the oil shocks.  
}{}
Anonymous rationing works much like rate limiting, except with
 multiple resources, an issuing authority, and limited time shifting:
%
\def\expiry{e}
We fix a set $U$ of limited resource types, overseen by
an authority who certifies verifiers from a key $\mathtt{root}$.
We dynamically define an expiry date $\expiry_{u,d}$ and an availability $n_{u,d}$,
both dependent upon the resource $u \in U$ and date $d$.
% We typically want a randomness beacon $r_d$ too, which prevents
% anyone learning $r_d$ much before date $d$. 
As ring VRF inputs for the spend operation, we choose
$\msg = \mathtt{root} \doubleplus u \doubleplus d \doubleplus k$
where $u \in U$ denotes a limited resource,
the expiry check $d < \mathtt{today} \le \expiry_{u,d}$ passes,
and $1 \le k \le n_{u,d}$.
We also choose \aux to be a preliminary receipt signed by the merchant.
%
At this point, our merchant sends the ring VRF signature to the authority,
who enforces that each nullifier by spent at most once.
Our authority stores the nullifiers until expiry aka $d \le \expiry_{u,d_0}$.

% We do not discuss ring updates here, 

% We remark that fully transferable assets could have constrained lifetimes
% too, which similarly eases nullifier management when implements using
% blind signatures, ZCash sapling, etc.  Yet, all these tokens require
% an explicit issuance stage, while ring VRFs self-issue.

Among the political hurdles to rationing, certificates have
a \eprint{considerable}{} forgery problem, as witnessed by the \eprint{long}{} history of
fraudulent covid and TLS certificates.  It follows citizens would
justifiably protest to ration carts that operate by simple certificates.
Ring VRFs avoid this political unrest by proving membership in a public list.

\eprint{}{\begin{comment}}
\subsection{Multi-constraint rationing}
\label{subsec:multi_io}

% \cite{PrivacyPass}
As in \S\ref{subsec:moderation}, we could impose simultaneous rationing
constraints for multiple resources $u_1,\ldots,u_k$ by producing one
ring VRF signature in which \PedVRF proves correctness of pre-outputs
for multiple messages 
$\msg_j = \mathtt{root} \doubleplus u_j \doubleplus r_d \doubleplus d \doubleplus k$ for $j=1 \ldots k$.

As an example, purchasing some prepared food product could require spending
rations for multiple base food sources, like making a cake from wheat, butter,
eggs, and sugar.  

\subsection{Decommodification}

There exist many reasons to decommodify important services, like
energy, water, or internet, beyond rationing real physical shortages.
Ring VRFs fit these cases using similar \msg formulations.

As an example, a municipal ISP allocates some limited bandwidth capacity
among all residents.  It allocates bandwidth fairly by verifying ring VRFs
signatures on hourly \msg and then tracking nullifiers until expiry.

Aside from essential government services, commercial service providers
typically offers some free service tier, usually because doing so
familiarizes users with their intimidating technical product.

Some free and paid tier examples include DuoLingo's hearts on mobile, 
continuous integration testing services, and many dating sites.

A priori, rate limiting cases benefit from unlinkability among individual
usages, not merely at some site boundary like moderation requires.
We thus use each ring VRF output only once, which prevents our cashing
trick of \S\ref{sec:reduced_pairings} from reducing verifier pairings.

Although rationing sounds valuable enough, we foresee services like ISP,
VPNs, or mixnets having many low value transactions.
In such cases, ring VRFs could authorize issuing a limited number of
fast simple single-use blind issued credentials, like blind signatures
ala GNU Taler \cite{taler} or PrivacyPass OPRF tokens \cite{PrivacyPass},
which both solve the leakage of $k$ above too.
%
In principle, commercial service providers could sell the same tokens,
which avoids leaking whether the user uses the free or commercial tier.
\eprint{}{\end{comment}}



%
%% The next two lines define the bibliography style to be used, and
%% the bibliography file
\bibliographystyle{ACM-Reference-Format}
\bibliography{../climate,../commit,../anoncred,../sassafras,../identity,../vrf,../zkp}


%%
%% If your work has an appendix, this is the place to put it.
\appendix

\newcommand{\Gen}{\ensuremath{\mathsf{Gen}}}

\newcommand{\anonymouskeymap}{\ensuremath{\mathtt{anonymous\_key\_map}}}
\newcommand{\anonymouskeylist}{\mathcal{W}}
\renewcommand{\sim}{\simulator}

\begin{figure}
\eprint{\footnotesize}{\scriptsize} 
\begin{tcolorbox}[left=2pt,right=2pt]
	{  $ \fgvrf $ runs a PPT algorithms  $\Gen_{sign} $ during the execution and is parametrized with  sets $ \setsym{S}_{eval} $ and $ \setsym{S}_W $ where $ \setsym{S}_{eval} $ and $ \setsym{S}_W $ generated by a set up function $ \mathsf{Setup}(1^\secparam) $.
	%We need to select W randomly because we need it to be unique
		
				
			\textbf{[Key Generation.]} upon receiving a message $(\oramsg{keygen}, \sid)$ from a party $\user_i$, send $(\oramsg{keygen}, \sid, \user_i)$ to the simulator $\simulator$.
			Upon receiving a message $(\oramsg{verificationkey}, \sid, X,\pk)$ from $\simulator$, verify that $X,\pk$ has not been recorded before for $ \sid $ i.e., there exists no $ (X', \pk') $ in $ \vklist $ such that $ X' = X $ or $ \pk' = \pk $. If it is the case, store in the table $\vklist$, under $\user_i$, the value $X,\pk$ and return $(\oramsg{verificationkey}, \sid, \pk)$ to $ \user_i$.
				
			%\textbf{[Malicious Key Generation.]} upon receiving a message $(\oramsg{keygen}, \sid, \pk)$ from $\simulator$, verify that $\pk$ was not yet recorded, and if so record in the table $\vklist$ the value $\pk$ under $\simulator$. Else, ignore the message.
				
			%\item[Honest Ring VRF Evaluation.] upon receiving a message $(\oramsg{eval}, \sid, \ring, \pk_i, m)$ from $\user_i$, verify that 
			%$\pk_i \in \ring$ 
			%and  
			%there exists $ \pk_i $ in $\vklist $ associated with $ \user_i $. If that was not the case, just ignore the request.
			%If there exists no $ W $ such that $ \anonymouskeymap[W] = (m, \ring, \pk_i) $, let $ W \leftsample \bin^\secparam $ and  $y \leftsample \setsym{S}_{eval}$. Then, set $ \evaluationslist[m, W] = y$ and $ \anonymouskeymap[W] = (m, \ring,\pk_i) $.
			%Return $(\oramsg{evaluated}, \sid, \ring, m, W, y)$ to $ \user_i $.
			%The functionality does not check whether the evaluater's public key is in the ring because here we consider m, \ring as an input of the evaluation which is evaluated by a party who is not neccesarily in the ring. 
			\textbf{[Corruption:] } 
			upon receiving $ (\oramsg{corrupt}, \sid, \user_i) $ from $ \simulator $, remove $ \pk_i $ from $ \vklist[\user_i] $ and store $ \pk_i $ to $ \vklist $ under $ \sim $. Return $ (\oramsg{corrupted}, \sid,\user_i) $.
			
			\textbf{[Malicious Ring VRF Evaluation.]} upon receiving a message $(\oramsg{eval}, \sid, \pk_i, W, m)$ from $\sim$, if $ \pk_i $ is recorded under an honest party's identity or if there exists $ W'\neq W $ where $ \anonymouskeymap[\msg,W'] = \pk_i $, ignore the request.
			Otherwise, record in the table $\vklist$ the value $\pk_i$ under $\simulator$ if $ \pk_i $ is not in $ \vklist $.
			
			 If  $\anonymouskeymap[\msg,W]  $ is not defined before, set $ \anonymouskeymap[\msg,W] = \pk_i $ and let   $y \leftsample \setsym{S}_{eval}$ and set $ \evaluationslist[\msg, W] = y$.
			
			In any case (except ignoring), obtain $ y = \evaluationslist[\msg, W] $ and return $(\oramsg{evaluated}, \sid,  \msg, \pk_i,W, y)$ to $ \user_i $.

			%upon receiving a message $(\oramsg{eval}, \sid, \pk_i, W, m)$ from $\sim$, if $ \pk_i $ is recorded under an honest party's identity or if there exists $ \anonymouskeymap, \pk_i] \neq W $ or if there exists a record for a key $ \pk \neq \pk_{i}$ such that $ \anonymouskeymap[m, \pk] = W $, ignore the request. Otherwise, record in the table $\vklist$ the value $\pk_i$ under $\simulator$ if $ \pk_i $ is not in $ \vklist $. If $ \anonymouskeymap[m,\pk_i]  $ is not defined, set $ \anonymouskeymap[m,\pk_i] = W $ and let   $y \leftsample \setsym{S}_{eval}$ and set $ \evaluationslist[m, W] = y$.
			%Return $(\oramsg{evaluated}, \sid,  m, W, \evaluationslist[m, W])$ to $ \user_i $.
				
			\textbf{[Honest Ring VRF Signature and Evaluation.]} upon receiving a message $(\oramsg{sign}, \sid, \ring, \pk_i,\aux, \msg)$ from $\user_i$, verify that $\pk_i \in \ring$ and that there exists a public key $\pk_i$ associated to $\user_i$ in $ \vklist $. If it is not the case, just ignore the request. 	
			If there exists no $ W' $ such that $ \anonymouskeymap[\msg,W'] =  \pk_i $, let $ W \leftsample \setsym{S}_W $ and let $y \leftsample \setsym{S}_{eval}$. If there exists $ W $ where $ \anonymouskeymap[\msg,W] $ is defined, then abort. Otherwise, set $ \anonymouskeymap[\msg,W] = \pk_i $ and set $ \evaluationslist[\msg, W] = y$.
			
			In any case (except ignoring and aborting), obtain $ W, y $ where $ \anonymouskeymap[\msg,W] =\pk_i $ and $ \evaluationslist[\msg, W] = y$  and run  $ \Gen_{sign}(\ring, W,\pk,\aux,\msg) \rightarrow \sigma $. 
			%Verify that $ [\msg,\aux, W,\ring, \sigma, 0] $ is not recorded. If it is recorded, abort. Otherwise,
			Record $ [\msg,\aux, W, \ring,\sigma, 1] $. Return $(\oramsg{signature}, \sid, \ring,W,\aux,\msg, y, \sigma)$ to $\user_i$.
			
			%\item[Malicious VRF evaluation.] upon receiving a message $(\oramsg{evalprove}, \sid, \ring, m)$ from $\simulator$, check that $\vklist$ has a public key associated to $\simulator$. If not, ignore the request. If $\evaluationslist[\ring, m][\simulator]$ is not set, sample $y \leftsample \bin^{\ell(\secparam)}$ and set $\evaluationslist[\ring, m][\simulator] \defeq y$ (and $\signaturelist[\ring,m]$ to $\emptyset$). If $\signaturelist[\ring, m]$ contains a proof (i.e., if $\signaturelist[\ring, m]$ is not empty), return $(\oramsg{evaluated}, \sid, y)$ to $\simulator$. Else, ignore the request.
			
			%\item[Verification.] upon receiving a message $(\oramsg{verify}, \sid, \ring, m, y, \sigma)$, from any party forward the message to the simulator. If there exists a $\pk_i$ among the values of \texttt{verification\_keys}, and there exists $\sigma \in \signaturelist[\ring, m]$, set $b = 1$. Else, set $b =0$. Finally, output $(\oramsg{verified}, \sid, \ring, m, y, \sigma, b)$.
			\textbf{[Malicious Requests of  Signatures.]} upon receiving a message $ (\oramsg{request}, \sid, \ring, W, \aux,\msg) $ from $ \simulator $, obtain all existing valid signatures $ \sigma $ such that $ [\msg, \aux,W,\ring,\sigma, 1] $ is recorded and add them in a list $ \lst_{\sigma} $. 	Return $ (\oramsg{requests}, \sid, \ring, W,\aux,\msg, \lst_{\sigma})  $ to $ \simulator $.
			
			
			\textbf{[Ring VRF Verification.]} upon receiving a message $(\oramsg{verify}, \sid, \ring,W, \aux, \msg, \sigma)$ from a party, do the following: 
    		% \begin{list}[label={{C}}{{\arabic*}}, start = 1]
			% https://texblog.net/help/latex/ltx-260.html
			\newcounter{FunCond}
			\begin{list}{\hspace*{1pt} C\arabic{FunCond}}{\usecounter{FunCond}\setlength\leftmargin{0.15in}}
				\item If there exits a record $ [\msg,\aux,W,\ring,\sigma, b'] $, set $ b = b' $. (This condition guarantees the completeness and consistency.)
				%					\item Else if $ \pk  $ is an honest verification key where $ \anonymouskeymap[W] = (.,., \pk) $ and there exists no record $ [m, \ring, W, \sigma', 1] $ for any $ \sigma' $, then let $ b= 0  $.
				%					(This condition guarantees unforgeability meaning that if an honest party never signs a message $ m $ for a ring $ \ring $, then the verification fails.)\label{cond:forgery}
				
				%\item Else if there exists a record  such as $ [m,W,\ring,\sigma, b'] $, set $ b = b' $. (This condition guarantees consistency meaning that all identical verification requests will output the same $ b $) 
				\label{cond:consistency}
				\item Else if $ \anonymouskeymap[\msg,W]  $ is an honest verification key and  there exists a record $ [\msg,\aux, W,\ring, \sigma', 1] $ for any $ \sigma' $, then let $ b=1 $ and record $ [\msg,\aux, W,\ring,\sigma, 1] $. (This condition guarantees that if $ \msg $ is signed by an honest party for the ring $ \ring $ at some point, then the signature is $ \sigma' \neq \sigma $ which is generated by the adversary is valid) \label{cond:differentsignature}
				
				\item \label{cond:malicioussignature}Else relay the message $(\oramsg{verify}, \sid, \ring,W,\aux, \msg, \sigma)$ to $ \simulator $ and receive back the message $(\oramsg{verified}, \sid, \ring,W,\aux, \msg, \sigma, b_{\simulator}, \pk_\simulator)$.  Then check the following:

				\begin{enumerate}
					\item If $ W \notin \anonymouskeylist[\msg,\ring] $ and $ |\anonymouskeylist[\msg, \ring]| > |\ring_{mal}| $ where $ \ring_{mal} $ is a set of malicious keys in $ \ring $, set $ b = 0 $.
					(This condition guarantees  uniqueness meaning that the number of verifying outputs that $ \sim $ can generate for $ \msg, \ring $ is at most the  number of malicious keys in $ \ring $.)\label{cond:uniqueness}.
					
					\item Else if $ \pk_\simulator $ is an honest verification key, set $ b = 0 $. (This condition guarantees unforgeability meaning that if an honest party never signs a message $ \msg$ for a ring $ \ring $)\label{cond:forgery}
					%\item \label{cond:forgerymalicious}Else if there exists $ \anonymouskeymap[W] = (m', \ring',.)  $ where $ (m', \ring') \neq (m, \ring) $ or $ \counter[m, \ring] > |\ring_m| $ where $ \ring_m $ is a set of keys in $ \ring $ which are not honest or $ b_{\simulator} = 0 $ or $ \pk_\simulator $ belongs to an honest party, set $ b = 0 $ and record $ [m, \ring,W,\sigma, 0] $. (This condition guarantees that if $ W $ is an anonymous key of a different message and ring or the number of anonymous keys of malicious parties in $ \ring $ is more than their number or     $ \simulator $ does not verify $ \sigma $, then the verification fails.)
					
					\item Else if there exists $ W' \neq W $ where  $ \anonymouskeymap[\msg,W'] = \pk_\simulator $, set $ b = 0 $. \label{cond:differentWforsamepk} (This condition guarantees that there exists a unique anonymous key for each $ (\msg, \pk_\simulator) $)
					\item Else set $ b = b_\sim$. \label{cond:simulatorbit}
				\end{enumerate}		

			\end{list}
			In the end,  record $ [\msg,\aux,W,\ring,\sigma, 0] $ if it is not stored. If $ b = 0 $, let $y = \perp $. Otherwise,   do the following:
			\begin{itemize}
				\item if $ W \notin \anonymouskeylist[\msg,\ring] $, add $ W $ to $ \anonymouskeylist[\msg,\ring]  $.
				%\item if $ \pk_\simulator $ is not recorded, record it in $ \vklist $ under $ \simulator $.
				\item if $ \evaluationslist[\msg,W] $ is not defined, set $ \evaluationslist[\msg, W]\leftsample \setsym{S}_{eval}$, $ \anonymouskeymap[\msg,W]  = \pk_\simulator$.  Set $ y= \evaluationslist[\msg, W]$.
				\item otherwise, set $ y = \evaluationslist[\msg, W]$. 	
			\end{itemize}
			Finally, output $(\oramsg{verified}, \sid, \ring,W, \aux,\msg, \sigma, y, b)$ to the party.
			
	

	}
\end{tcolorbox}
\caption{Functionality $\fgvrf$.\label{f:gvrf}}
\end{figure}




\newcommand{\name}{rVRF}
\section{Security of Our Ring VRF Construction} 
\label{ap:ucproof}
\label{sec:ringvrfconstrnoPK}
\newcommand{\GG}{\grE}
\newcommand{\FF}{\F}
\newcommand{\hash}{H}
\newcommand{\hashG}{\hash_\grE}
\newcommand{\gen}{\mathsf{Gen}}
\newcommand{\hkeys}{\mathtt{honest\_keys}}
\newcommand{\malkeys}{\mathtt{malicious\_keys}}
\newcommand{\rcom}{R_{\mathsf{dleq}}}
\newcommand{\rsnark}{\Rring}
\newcommand{\counter}{\mathsf{counter}}
\newcommand{\bdv}{\mathcal{B}}
\newcommand{\abort}{\textsc{Abort}}

%Before giving the security proof of our protocol, we give the protocol in Section \ref{sec:pederson_vrf} without the abstraction from $ \PedVRF $ for the sake of  clarity of the security proof.
%
%We instantiate parameter generation by constructing a group $\GG$ of order $ p $ and two generators $ \genG, \genB \in  \GG$.  We consider three hash functions: $ \hash, \hash': \{0,1\}^* \rightarrow \FF_p $ and a hash-to-group function $\hashG : \{0,1\}^* \rightarrow \GG$ and . \name \ works as follows:
%
%\begin{itemize}
%	\item $ \rVRF.\KeyGen(1^\kappa):  $ It selects randomly a secret key $ x \in \FF_p$ and computes the public key $ X = xG $. In the end, it outputs $ \sk = x $ and $ \pk = X $.
%	
%	%It also generates  PoK for the discrete logarithm of $ X $ for the relation $ \R_{dl} $, $ \NIZK.\Prove(\rdl, (x, (X, G, \GG))) \rightarrow \pi_{dl} $.
%	
%	%\begin{equation}
%	%	\rdl = \{(x,(X,G,\GG)): X,G \in \GG, x \in \FF_p, x = xG\}
%	%\end{equation}
%	
%	%For this, it does the following: $ a \leftsample \FF_p $, $ c = \hash'(a\genG, X) $, $ s = a + cx $. 
%	
%	%	\item $ \rVRF.\eval(\sk, \ring, m) $: It lets $ P = \hashG(m, \ring) $ and computes $ W = xP  $. Then, it outputs $ y = \hash(m, \ring, W) $. So, the deterministic function $ F $ in our rVRF protocol is $ F(\sk, \ring, m) = H(m, \ring, x\hashG(m,\ring)) $.
%	%	
%	\item $ \rVRF.\Sign(\sk, \ring, m):$ It lets $ \In = \hashG(m) $ and computes the pre-output $ \PreOut= x\In$. The signing algorithm works as follows: 
%
%	\begin{itemize}
%		
%		\item It first commits to its secret key $
%		x$ i.e., $ \compk = X + \openpk \, \genB $ where $ \openpk \leftsample \FF_p $.
%		\item It generates a Chaum-Pedersen DLEQ proof $ \pi_{dleq} $ showing the following relation by running the algorithm $ \NIZK_{\rcom}.\Prove(((\genG, \genB,\GG,\compk,\PreOut,\In); (x, \openpk))) $ which outputs $ \rightarrow \pi_{dleq}$
%		\eprint{\begin{align}
%				\rcom= \{((x, \openpk), (\genG, \genB,\GG,\compk,\PreOut,\In)): 
%				\compk = x\genG + \openpk\, \genB, \PreOut = x \,\In \} \label{rel:commit} 
%		\end{align}}{
%		\begin{align}
%			\rcom= \{((x, \openpk), (\genG, \genB,\GG,\compk,\PreOut,\In)): \\
%			\compk = x\genG + \openpk\, \genB, \PreOut = x \,\In \} \label{rel:commit} \nonumber
%		\end{align}}
%		Here $ \Prove $ algorithm runs a non-interactive Chaum-Pedersen DLEQ proof with the Fiat-Shamir transform:  Sample random $r_1, r_2 \leftarrow \F_p$.
%		Let $R = r_1 \genG + r_2 K$, $R_m = r_1 \In$, and
%		$c = \hash'(\ring, m, \PreOut,\compk,R,R_m)$.
%		Set $\pi_{dleq} = (c,s_1,s_2)$ where $s_1 = r_1 + c x$ and $s_2= r_2 + c \, \openpk$.
%		\item %It obtains $ crs $ from $ \gcrs $ for the second proof by sending the message $ (\oramsg{learncrs}, \sid) $ to $ \gcrs $. Then,
%		%It constructs a Merkle tree $ \mathsf{MT} $ with the nodes $ X_i $ where  $ X_i \in \pk_i $ and $ \pk_i \in  \ring $. We denote its  root by $ \mathsf{root} $. In the end, 
%		It generates the second proof $ \pi_{ring} $ for the following relation with  the witness $ (\ring, x, \openpk) $. 		
%		
%		\eprint{\begin{equation}
%				%\rsnark = \{((\mathsf{copath}, X, \openpk),(G,\genB\GG,\mathsf{root}, \compk)): C-\openpk K = X, \mathsf{MT}.\Verify(\mathsf{copath}, X, \mathsf{root} ) \rightarrow 1\} \label{rel:snark}
%				\rsnark = \{(X, \openpk),(\genG,\genB,\GG,\ring, \compk)): \compk-\openpk \, \genB = X \in \ring\} \label{rel:snark}
%			\end{equation}
%		}{\begin{align}
%			\rsnark = \{(X, \openpk),(\genG,\genB,\GG,\ring, \compk)): \\\compk-\openpk \, \genB = X \in \ring\} \label{rel:snark} \nonumber
%		\end{align}
%	}
%		
%		%Here, $ \mathsf{copath} $ is a copath of the Merkle tree $ \mathsf{MT} $. $ \mathsf{MT}.\Verify(\mathsf{copath}, X, \mathsf{root} ) $ is a verification algorithm of the Merkle tree which verifies whether $ X $ is the one of the leaves of $ \mathsf{MT} $ i.e., compute a root $ \mathsf{root}' $ with $ X $ and $ \mathsf{copath} $ and output 1 if $ \mathsf{root} = \mathsf{root}' $.
%		
%		The second proof $ \pi_{ring} $ is generated by running 
%		$ \NIZK_{\rsnark}.\Prove(((\genG,\genB,\GG,\ring, \compk); (X, \openpk))) $ 
%	\end{itemize}
%	In the end, $ \rVRF.\Sign $ outputs $\sigma = (\pi_{dleq}, \pi_{ring}, \compk, \PreOut) $.
%	
%	\item $ \rVRF.\Verify(\ring,\PreOut, m, \sigma) $: Given $  \sigma = (\pi_{dleq}, \pi_{ring},\compk)  $ and $ \ring, \PreOut $,
%	% it first runs $ \NIZK.\Verify(\rdl,(X_i,\genG,\GG), \pi_{dl_i}) $ for each $ \pk_i= (X_i, \pi_{dl_i})  \in \ring $. If each of key in $ \ring $ verifies,
%	it runs $ \NIZK_{\rcom}.\Verify((\genG, \genB,\GG,\compk,\PreOut,\In), \pi_{dleq} ) $ where $ P = \hashG(m) $. $ \NIZK_{\rcom}.\Verify $ works as follows: $ \pi_{dleq} = (c,s_1, s_2) $, it lets $R' = s_1 \genG + s_2 \, \genB - c \,\compk$ and $R'_m = s_1 \hashG(m) - c \, \PreOut$. It
%	returns true if $c = \hash'(\ring,m,\PreOut,\compk,R',R'_m)$. If  $ \NIZK_{\rcom}.\Verify((\genG, \genB,\GG,\compk,\PreOut,\In), \pi_{dleq} ) $ outputs 1, it runs $ \NIZK_{\rsnark}.\Verify((\genG,\genB,\GG,\ring, \compk), \pi_{ring}) $. 
%	If all verification algorithms verify, it outputs $ 1 $ and the evaluation value $ y =  \hash(m,\PreOut)  $. Otherwise, it outputs $( 0, \perp) $.
%	
%\end{itemize}
%
%\subsection{Security Analysis}

Before we start to analyse our protocol, we should define the algorithm $ \gen_{sign} $  for $ \fgvrf $ and show that $ \fvrf $ with $ \gen_{sign} $ satisfies the anonymity defined in Definition \ref{def:anonymity}. $ \fgvrf $ that \name \ realizes runs  Algorithm \ref{alg:gensign} to generate honest signatures.



%\begin{algorithm}
%	\caption{$\gen_{W}(\ring,\pk,m)$}
%	\label{alg:genW}	 	
%	\begin{algorithmic}[1]
	%		\State$ W \leftsample\GG $
	%		%		\State \textbf{get} $ X \in \pk $
	%		%		\If{$\mathtt{DB}[m, \ring] = \perp  $}		
	%		%		\State{$ a \leftsample \FF_p $}		
	%		%		\State{$\mathtt{DB}[m, \ring] := a$}
	%		%		\EndIf
	%		%		\State$ a \leftarrow \mathtt{DB}[m, \ring] $
	%		%		
	%		%		\State \textbf{return} $ aX $
	%		\State \textbf{return} $ W $
	%	\end{algorithmic}
%	
%\end{algorithm}

\begin{algorithm}
	\caption{$\gen_{sign}(\ring,W,\pk,\aux,\msg)$}
	\label{alg:gensign}	 	
	\begin{algorithmic}[1]
		\State $ c,s_1, s_2 \leftsample \FF_p $
		\State $ \pi_{dleq}  \leftarrow (c,s_1, s_2)$
		\State $ \openpk \leftsample \FF_p $
		\State $ \compk =  \pk + \openpk \, K$
		%\State $ \pi_{dleq} \leftarrow \NIZK.\mathsf{Simulate}(\rcom, (G, \genB,\GG,\compk,W,\In)) $
		%\State \textbf{send} $(\oramsg{learn\_\tau},\sid)  $ to $ \gcrs $
		%\State \textbf{receive} $(\oramsg{trapdoor},\sid, \tau,crs)  $ from $ \gcrs $
		\State $ \comring, \openring \leftarrow \rVRF.\CommitRing(\ring, \pk) $
		\State $ \pi_{ring} \leftarrow \NIZK_{\rsnark}.\Prove(((\genG,\genB,\GG,\comring, \compk); (\pk, \openpk, \openring))) $ 
		\State\Return$ \sigma = (\pi_{dleq},\pi_{ring},\compk,\comring,W) $
	\end{algorithmic}
	
\end{algorithm}


\begin{lemma} \label{lem:anonymity} $ \fgvrf $ running Algorithm \ref{alg:gensign} satisfies anonymity defined in Definition \ref{def:anonymity} assuming that $ \NIZK_{\rsnark} $ is a zero-knowledge and Pedersen commitment is perfectly hiding.
\end{lemma}

\begin{proof}
	We simulate $ \fgvrf $ with Algorithm \ref{alg:gensign} against $ \mathcal{D} $. Assume that the advantage of $ \mathcal{D} $ is $ \epsilon $. Now, we reduce the anonymity game to the following game where we change the simulation of $ \fgvrf $ by changing the Algorithm \ref{alg:gensign}. In our change, we let $ \pi_{ring} \leftarrow \NIZK_{\rsnark}.\mathsf{Simulate}(\genG,\genB,\GG,\comring, \compk) $. Since $ \NIZK_{\rsnark} $ is zero knowledge, there exists an algorithm  $ \NIZK_{\rsnark}.\mathsf{Simulate} $ which generates a proof which is indistinguishable from the proof generated from $ \NIZK_{\rsnark}.\Prove $. Therefore, our reduced game is indistinguishable from the anonymity game. Since in this game, no public key is used while generating the proof and $ W $ and $ \compk $ is perfectly hiding, the probability that  $ \mathcal{D} $ wins the game is $ \frac{1}{2} $. This means that $ \epsilon $ is negligible.		
\end{proof}

We next show that \name \ realizes $ \fgvrf $  in the random oracle model under the assumption of the hardness of the decisional Diffie Hellman (DDH).

%The GDH problem is solving the computational DH problem by accessing the Diffie-Hellman oracle ($ \mathsf{DH}(.,.,.) $) which tells that given triple $ X,Y,Z $ is a DH-triple i.e., $ Z = xyG $ where $ X = xG $ and $ Y = yG $.

%\begin{definition}[$ n $-One-More Gap Diffie-Hellman (OM-GDH) problem]
%	Given   $ p $-order group $ \GG $ generated by $ G $, the challenges $ G, X = xG, P_1, P_2, \ldots, P_{n+1} $ and access to the DH oracle $ \mathsf{DH}(.,.,.) $ and the oracle $ \mathcal{O}_x(.) $ which returns $ x\In$ given input $ \In$, if a PPT adversary $ \mathcal{A} $ outputs $ xP_1, xP_2, \ldots, xP_{n+1} $ with the access of at most $ n $-times to the oracle $ \mathcal{O}_x $, then $ \mathcal{A}  $ solves the $ n $-OM-GDH problem. We say that $ n $-OM-GDH problem is hard in $ \GG $, if for all PPT adversaries, the probability of solving the $ n $-OM-GDH problem is negligible in terms of the security parameter.
%\end{definition}

\begin{theorem}
	Assuming that $ \hashG, \hash,\hash', \hash_\ring $ are random oracles,  the DDH problem is hard in the group structure $ (\GG, \genG,\genB, p) $ and NIZK algorithms are zero-knowledge and knowledge sound, \name \ UC-realizes $\fgvrf$ running Algorithm \ref{alg:gensign} according to Definition \ref{def:uc}.
\end{theorem}

\begin{proof}
	We construct a simulator $ \simulator $ that simulates the honest parties in the execution of \name \ and simulates the adversary in $ \fgvrf $. 
	\begin{itemize}
		%\item \textbf{[Simulation of $ \gcrs $:] }When simulating $ \gcrs $, it runs $ \mathsf{SNARK}.\mathsf{SetUp}(\rsnark) $ which outputs a trapdoor $ \tau $ and $ crs $ instead of picking $ crs $ randomly from the distribution $ \distribution $. Whenever a party comes to learn the $ crs $, $ \simulator $ gives $ crs $ as  $ \gcrs $.
		
		\item \textbf{[Simulation of $ \oramsg{keygen} $:]} Upon receiving $(\oramsg{keygen}, \sid, \user_i)$ from $\fgvrf$, $ \simulator $ samples $x \leftsample \FF_p$ and obtains the key $X = xG$. It adds $ xG $ to lists $ \hkeys $ and $ \vklist $ as a key of $ \user_i $. 
		In the end, $ \simulator $ returns $(\oramsg{verificationkey}, \sid, X)$ to $\fgvrf$. %Whenever the honest party $ \user $ is corrupted by $ \env, $ $ \simulator $ moves the key of $ \user $ to $ \malkeys $ from $ \hkeys $.
		
		\item \textbf{[Simulation of corruption:]} Upon receiving a message $ (\oramsg{corrupted}, \sid, \user_i) $ from $ \fgvrf $, $ \simulator $ removes the public key $ X $ from $ \hkeys $ which is stored as a key of $ \user_i $ and adds $ X $ to $ \malkeys $.
		
		\item\textbf{[Simulation of the random oracles:]} We  describe how $ \simulator $ simulates the random oracles $ \hashG, \hash, \hash' $ against the real world adversaries. 	
		
		$ \simulator $ simulates the random oracle $ \hashG $ as described in Figure \ref{oracle:HgnoPK}. It selects a random element  $ h $ from $ \FF_p $ for each new input and outputs $ hG $ as an output of the random oracle $ \hashG $. Thus, $ \simulator $ knows \emph{the discrete logarithm of each random oracle output of $\hashG  $}. 


		 The simulation of the random oracle $ \hash $ is less straightforward (See Figure \ref{oracle:HnoPK}).
		The value $ W $ can be a pre-output generated by $ \rVRF.\Eval $ or can be an anonymous key of  $ m $ generated by $ \fgvrf $ for an honest party. $ \simulator $ does not need to know about this at this point but $ \hash $ should output $ \evaluationslist[m,W] $ in both cases.	 
		%If $ W $ is a pre-output, $ \simulator $ needs to find corresponding malicious public key in the real world. If it is the case, $ W $ should be equal to $ x\hashG(m, \ring)= xhG $  where $ xG $ is a public key. 
		$ \simulator $ pretends $ W $ as if it is a pre-output. So, $ \simulator $ first obtains the discrete logarithm $ h $ of $ \hashG(m) $ from the $ \hashG $'s database and finds out a public key $ X^* = h^{-1}W $.    
		%If $ X^* $ has not been registered as a malicious key, it registers it to $ \fgvrf $. Thus, $ \simulator $ has a right to ask the output of the message $ m, \ring $ to $ \fgvrf $. 
		If $ X^*$ is not an honest key generated by $ \simulator $, $ \simulator $ can obtain $ \evaluationslist[m,W] $ by sending a message $ (\oramsg{eval}, \sid,X^*,W,m) $.
		Otherwise, it replies by a randomly selected value from $ \FF_p $. Remark that
		if $ W $ is a pre-output generated by $ \adv $, then $ \fgvrf $ matches it with $ m, X^* $ and registers $ X^* $ as a malicious key. If $ W $ is an anonymous key of an honest party in the ideal world, $ \fgvrf $ does not care $ X^* $ and returns an honest evaluation value $ \evaluationslist[m,W] $.
		% Remember that $ \fgvrf $ only replies to the evaluation message of $ \simulator $ if $ W $ is not mapped to another message, ring and public key $ (m', \ring', X')   $. $ W $ cannot be map to $ (m', \ring', X')   \neq  (m, \ring, X*)   $ because it would be aborted during the simulation $ \hashG $ if they were mapped to $ W $.
		During the simulation of $ \hash $, if $ \fgvrf $ aborts, it means that there exists $ W' \neq W $ such that $ \anonymouskeymap[m,W'] = X^* $. This means that $ hX^* = W' \neq W  $, but it is impossible because $ W = hX^* $. Therefore, $ \simulator $ never aborts during the simulation of $ \hash $.
		
		We note that $ \fgvrf $ never generates an anonymous key with an honest verification key. Therefore, if $ X^* = h^{-1}W $ is an honest verification key, $ \simulator $ returns a random value because  $ \evaluationslist[m,W] $ is not defined or will not be defined in $ \fgvrf $ in this case except with a negligible probability. If it ever happens i.e., if $ \fgvrf $ selects randomly $ W = hX^* $, $ \env $ distinguishes the simulation via honest signature verification in the real world. So, this case is covered in our simulation in Figure \ref{oracle:H'}.
		
		\begin{figure}
			\begin{minipage}{4cm}
			\centering
			\noindent\fbox{%
				\parbox{4cm}{%
					\underline{\textbf{Oracle $ \hashG $}} \\
					\textbf{Input:} $ m $ \\
					\textbf{if} $\mathtt{oracle\_queries\_gg}[m] = \perp  $
					
					\tab{$ h \leftsample \FF_p $}
					
					%					\tab{\textbf{for all} $ X \in \ring $}
					%					
					%					\tab{$ W =  hX $}
					%					
					%					\tabdbl{\textbf{if} $ W \in \anonymouskeylist $: \textsc{Abort}}
					%					
					%					\tabdbl{\textbf{else:} \textbf{add} $ W $ \textbf{to} $ \anonymouskeylist $}
					
					\tab{$ P \leftarrow hG $} 
					
					\tab{$\mathtt{oracle\_queries\_gg}[m] := h$}
					
					\textbf{else}:
					
					\tab{$ h \leftarrow \mathtt{oracle\_queries\_gg}[m] $}
					
					\tab{$ P \leftarrow hG$}
					
					\textbf{return $ \In$}
					
			}}	
			\caption{The random oracle $ \hashG $}
			\label{oracle:HgnoPK}
		
	\end{minipage}
\hfill
	\begin{minipage}{7cm}
			\centering
			
			\noindent\fbox{%
				\parbox{7cm}{%
					\underline{\textbf{Oracle $ \hash$}} \\
					\textbf{Input:} $ m,W $ 
					
					\textbf{if} $ \mathtt{oracle\_queries\_h}[m, W] \neq \perp $
					
					\tab{\textbf{return $  \mathtt{oracle\_queries\_h}[m,  W] $}}
					
					%					\textbf{send} $ (\oramsg{request}, \sid, \emptyset,W, m) $ \textbf{to} $ \fgvrf $
					%					
					%					\textbf{receive} $ (\oramsg{requests}, \sid, \emptyset, W, m, \setsym{L}_\sigma, y) $ \textbf{from} $ \fgvrf $
					
					%					\textbf{if} $ y = \perp $
					
					{$ P \leftarrow \hashG(m) $}
					
					{$ h \leftarrow \mathtt{oracle\_queries\_gg}[m] $}
					
					{$ X^* := h^{-1}W $ // candidate verification key} 
					
					{{\textbf{if} $ X^* \notin \hkeys$ }} 
					
					\tab{\textbf{send} $ (\oramsg{eval}, \sid, W, X^*, m) $ \textbf{to} $ \fgvrf $}
					
					\tab{\textbf{if} $ \fgvrf $ ignores: \abort}
					
					\tab{\textbf{receive} $ (\oramsg{evaluated}, \sid, W, m, y) $ \textbf{from} $ \fgvrf $}
					
					\tab{$  \mathtt{oracle\_queries\_h}[m, W]:=y $}
					
					{\textbf{else:} }
					
					\tab{$ y \leftsample \FF_p $}
					
					\tab{$  \mathtt{oracle\_queries\_h}[m,  W]:=y $}
					%					{\textbf{else:} $ \mathtt{oracle\_queries\_h}[m, \ring, W]  = \perp$}
					%					
					%					%\tab{\textbf{return} \textsc{Abort}}
					%					\tab{$ y \leftsample \bin^\lambda $}
					%					
					%					\tab{$\mathtt{oracle\_queries\_h}[m, \ring, W] := y $}
					
					%	\textbf{else:} $  \mathtt{oracle\_queries\_h}[m,  W]:=y $
					
					\textbf{return $  \mathtt{oracle\_queries\_h}[m,  W] $}
					
			}}	
			\caption{The random oracle $ \hash $}
			\label{oracle:HnoPK}
		\end{minipage}
		\end{figure}
		
		The simulation of the random oracle $ \hash' $ (See Figure \ref{oracle:H'}) checks whether the random oracle query $ (\ring,m,W,\compk,R,R_m) $ is an $ \rcom $ verification query before answering the oracle call. For this, it checks whether $ \fgvrf $ has a recorded valid signature for the message $ m $ and the ring $ \ring $ with the anonymous key $ W $. If there exists such valid signature where $ \compk $ is part of it, $ \simulator $ checks whether the first proof of the signature $ (c,s_1, s_2) $ generates $ R, R_m $ as in $ \rVRF.\Verify $ in order to make sure that it is a $ \rcom $ verification query. If it is the case, it assigns $ c $ as an answer of $ \hash' (\ring,m,W,\compk,R,R_m) $ so that $ \rcom $ verifies. However, if this input has already been set to another value which is not equal to $ c $ or $ W $ is a pre-output of an honest key, then $ \simulator $ aborts because the output of the real world for this signature and the ideal world will be different.
		We remind that if an anonymous key $ W $ of an honest party  for a message $ m $ sampled by $ \fgvrf $ equals to a pre-output generated by $ \rVRF.\Sign $  for the same honest party's key and the message $ m $, then $ \env $ can distinguish the ideal and real world outputs because the evaluation value in the ideal world and real world for $ m,W $ will be different because of the simulation of the random oracle $ \hash $ i.e., $ \mathtt{oracle\_queries\_h}[m,W] \neq \evaluationslist[m,W] $.  Therefore, $ \simulator $ aborts if it is ever happen.
		
		\begin{figure}
			\centering
			
			\noindent\fbox{%
				\parbox{\columnwidth}{%
					\underline{\textbf{Oracle $ \hash' $}} \\
					\textbf{Input:} $ (\aux,\comring, \msg,W,\compk,R,R_m) $ \\					
					
					\textbf{send} $ (\oramsg{request\_signatures},\sid, \aux, W,\msg) $
					
					\textbf{receive} $ (\oramsg{signatures},\sid,\msg, \setsym{L}_\sigma) $
					
					\textbf{if} $ \exists \sigma \in \setsym{L}_\sigma $ where $ \compk,\comring   \in \sigma $
					
					\tab{\textbf{get} $ \pi_1 = (c,s_1, s_2) \in \sigma $} 
					
					\tab{\textbf{if} $ R = s_1\genG + s_2 \genB -c\compk, R_m = s_1 \hashG(m) - c W $}  
					
					\tabdbl{$ h := \mathtt{oracle\_queries\_gg}[m,W] $ }
					
					\tabdbl{\textbf{if} $ \mathtt{oracle\_queries\_h\_CP}[\comring, m,W,\compk,R,R_m]  = \perp $}
					
					\tabdbldbl{$ \mathtt{oracle\_queries\_h\_CP}[\comring, m,W,\compk,R,R_m]  := c$}
					
					\tabdbl{\textbf{else if} $( \mathtt{oracle\_queries\_h\_CP}[\comring, m,W,\compk,R,R_m]  \neq c $ }
					
					\tabdbl{\textbf{or} $ X^* = h^{-1}W \in \hkeys) $: \abort}
					
					
					\textbf{if} $ \mathtt{oracle\_queries\_h\_CP}[\comring, m,W,\compk,R,R_m]  = \perp $
					
					\tab{$ c \leftsample \FF_p $}
					
					\tab{$ \mathtt{oracle\_queries\_h\_CP}[\comring, m,W,\compk,R,R_m]  := c$}
					
					{\textbf{return} $ \mathtt{oracle\_queries\_h\_CP}[\comring, m,W,\compk,R,R_m] $}
					
			}}
			\caption{The random oracle $ \hash' $}
			\label{oracle:H'}
		\end{figure}
		
		
		%		\item \textbf{[Simulation of $ \oramsg{sign} $]} 
		%		The simulator has a table  $\preoutputlist $ to keep the pre-outputs that it selects for each input and the ring of public keys. 
		%		Upon receiving $(\oramsg{sign}, \sid, \ring, m, y)$  from the functionality $\fgvrf$, $ \simulator $ generates the signature $ \sigma $ as follows:
		%		
		%		For the first proof, it samples $ c, s_1, s_2 \in \FF_p $ and $ \compk, W \in \GG$. Then, it lets the first proof be $\pi_1 =  (c, s_1, s_2) $. 
		%		In addition, it sets $ R = sG+ \delta K+ c\compk $ and $ R_m = s \hashG(m, \ring)+ cW $ and maps the input $ \ring,m, W,\compk, R, R_m$ to $ c $ in the table of the random oracle $ \hash' $ so that $ \pi_1 $ verifies in the real-world execution.  
		%		It adds $ W $ to the list $ \preoutputlist[m, \ring] $.
		%		
		%		$ \simulator $ gets the trapdoor $ \tau $ that it generated during the simulation of $ \gcrs $ to simulate the second proof. Then, it runs $ \mathsf{SNARK}.\mathsf{Simulate}(\rsnark,\tau, crs) $ and obtains $ \pi_2 $.
		%		
		%		In the end, $ \simulator $  responds by sending the message $(\oramsg{signature}, \sid, \ring, m, \sigma = (\pi_1, \pi_2, \compk, W))$ to the $ \fgvrf $.  It also lets $ \mathtt{oracle\_queries\_h}[m, \ring, W] $ be $ y $, if it is not defined yet. If it is defined with another value $ y' \neq y $, then it aborts.
		%TODO: Talk about this abort case happens with a negl probability. 
		
		
		\item \textbf{[Simulation of $ \oramsg{verify} $]} Upon receiving  $(\oramsg{verify}, \sid, \ring,W, \aux,\msg, \sigma)$ from the functionality $\fgvrf$, $ \simulator $ runs the two NIZK verification algorithms run for $ \rcom, \rsnark $ with the input $ \comring, \msg, \sigma, W $ described in $ \rVRF.\Verify $ algorithm of \name if $ \sigma $ can be parsed as $ (\pi_1,\pi_2, \compk, \comring) $. If  all verify, it sets $ b_{\simulator} =1 $. Otherwise it sets $ b_{\simulator} =0  $.
		
		\begin{itemize}
			\item 		If $ b_\simulator = 1 $, it sets $ X = h^{-1} W$ where $ h = \mathtt{oracle\_queries\_gg}[m] $ and sends  $ (\oramsg{verified}, \sid, \ring, W,\aux, m, \sigma, b_\simulator, X) $ to $ \fgvrf $ and receives back $ (\oramsg{verified}, \sid, \ring, W, \aux, m, \sigma, y, b) $. 
			\begin{itemize}
				\item If $ b \neq b_\simulator $, it means that the signature is not a valid signature in the ideal world, while it is in the real world. So, $ \simulator $ aborts in this case.
				
				If $ \fgvrf $ does not verify a ring signature even if  it is verified in the real world, $ \fgvrf $ is in either C3-\ref{cond:uniqueness}, \ref{cond:forgery} or C3-\ref{cond:differentWforsamepk}.
				If $ \fgvrf $ is in C3-\ref{cond:uniqueness}, it means that $ \counter[m,\ring] > |\ring_m| $. If $ \fgvrf $ is in C3-\ref{cond:forgery}, it means that $ X $ belongs to an honest party but this honest party never signs $ m $ for  $ \ring $. So, $ \sigma $ is a forgery.	 If $ \fgvrf $ is in C3- \ref{cond:differentWforsamepk}, it means that there exists $ W' \neq W $ where $ \anonymouskeymap[m,W'] = X $. If $ [m,W'] $ is stored before, it means that $ \simulator $ obtained $ W' = hX $ where $ h = \mathtt{oracle\_queries\_h}[m] $ but it is impossible to happen since $ W = hX $.
				\item If $ b = b_\simulator $, it sets $ \mathtt{oracle\_queries\_h}[m,W] = y $, if it is not defined before.
				% In short, if $ \simulator $ aborts because $ b\neq b_\simulator $ it means either $ W $ of an honest party is not unique and $ \adv $ in the real world generates a forgery signature of $ (m, \ring, \sigma) $ with $ W $ or the adversary in the real world generates anonymous keys for $ (m, \ring) $ more than the number of adversarial keys in $ \ring $.
				%				 
				
				%	\item If $ b = b_\simulator $, set $ \mathtt{oracle\_queries\_h}[m, W] = y $. Here, if $ \sigma $ is a signature of an honest party, $ \simulator $ sets its output with respect to the output selected by $ \fgvrf $. 
				%    Remark that we do not need to set $ \mathtt{oracle\_queries\_h\_CP} $ because it already verifies in the real world.
			\end{itemize}
			\item If $ b_\simulator = 0 $, it sets $ X = \perp $ and sends  $ (\oramsg{verified}, \sid, \ring, W,\aux, m, \sigma, b_\simulator, X) $ to $ \fgvrf $. Then, $ \simulator $ receives back $ (\oramsg{verified}, \sid, \ring, W, \aux,m, \sigma, \perp, 0) $. 
			%			\begin{itemize}
				%				\item If $ b \neq b_\simulator $, it means that it was a signature of an honest party and $ \NIZK.\Verify $ for $ \rcom $ does not validate in the real world. So, $ \simulator $ sets $ \mathtt{oracle\_queries\_h}[m, \ring,W] = y $ and $ \mathtt{oracle\_queries\_h\_CP}[\ring, m, W, \compk, R', R_m'] = c $ where $ R' = s\genG + \delta K+ c\compk  $, $ R_m = s \hashG(m,\ring) + cW$. 
				%				Now, the signature verifies in the real world as well.
				%				\item If $ b = b_\simulator $, $ \simulator $ doesn't need to do anything.
				%			\end{itemize}
			
		\end{itemize}
		
		
		
		
		
		
	\end{itemize}

	Now, we need to show that the outputs of honest parties in the ideal world are indistinguishable from the honest parties in the real world. 
	
	\begin{lemma}\label{lem:honestoutput}
			Assuming that DDH problem is hard on the group structure $ (\GG, \genG,\genB) $, the outputs of honest parties in the real protocol \name\ are indistinguishable from the output of the honest parties in $ \fgvrf $.
	\end{lemma}
		
		\begin{proof}
			Clearly, the evaluation outputs of the ring signatures in the ideal world identical to the real world protocol because  the outputs are randomly selected by $ \fgvrf $ as the random oracle $ \hash $ in the real protocol. The only difference is the ring signatures of honest parties (See Algorithm \ref{alg:gensign}) since the pre-output $ W $ and $ \pi_1 $ is generated differently in Algorithm \ref{alg:gensign} than $ \rVRF.\Sign $. The distribution of $ \pi_{dleq} = (c,s_1, s_2) $ and $ \compk $ generated by Algorithm \ref{alg:gensign} and the distribution of $ \pi_{dleq} = (c,s_1, s_2) $ and $ \compk $ generated by $ \rVRF.\Sign $ are from uniform distribution so they are indistinguishable. 
			
			Now, we show that the anonymous key $ W $ selected randomly fro $ \GG $ and pre-output $	 W $ generated by $ \rVRF.\Sign $ are indistinguishable. For this,  we need show that selecting $ W $ randomly from $ \GG $ and computing $ W $ as $x \hashG(m, \ring) $ are indistinguishable.
			We  show this under the assumption that the DDH problem  is hard.  In other words, we show that if there exists an adversary $ \adv' $ that distinguishes anonymous keys of honest parties in the ideal world and anonymous key of the honest parties in the real protocol then we construct another adversary $ \bdv $ which breaks the DDH problem. 
			We use the hybrid argument to show this.
			We define hybrid simulations $ H_{i} $ where  the anonymous keys of first $ i $ honest parties are computed as described in $ \rVRF.\Sign $ and the rest are computed by selecting them randomly. Without loss of generality, $ \user_1, \user_2, \ldots, \user_{n_h} $ are the honest parties. Thus, $ H_0 $ is equivalent to the anonymous keys of the ideal protocol  and $ H_{n_h}  $ is equivalent to the anonymous keys of honest parties in the real world.  We construct an adversary $ \bdv $ that breaks the DDH problem given that there exists an adversary $ \adv' $ that distinguishes hybrid games $ H_i $ and $ H_{i + 1} $ for $ 0 \leq i < n_h $. $\bdv $ receives the DDH challenges $ X,Y, Z \in \GG $ from the DDH game and simulates the game against $ \adv' $ as follows: $\bdv $ generates the public key of all  honest parties' key as usual by running $ \rVRF.\KeyGen$ except party $ \user_{i+1} $. It lets $ \user_{i+1} $'s public key be $ X $. $ \bdv $ gives $ \GG, \genG = Y, \genB $ as parameters of \name. 
			
			%		\begin{figure}
				%		\centering
				%		
				%		\noindent\fbox{%
					%			\parbox{8cm}{%
						%				\underline{\textbf{Oracle $ \hashG $ in \ref{game:DDH} by the DDH adversary $\simulator $}} \\
						%				\textbf{Input:} $ m, \ring $ \\
						%				\textbf{if} $\mathtt{oracle\_queries\_gg}[m, \ring] = \perp  $
						%				
						%				\tab{$ h \leftsample \FF_p $}
						%				
						%				\tab{\fbox{$ P \leftarrow hY $}}
						%				
						%				\tab{$\mathtt{oracle\_queries\_gg}[m, \ring] := h$}
						%				
						%				\textbf{else}:
						%				
						%				\tab{h $\leftarrow \mathtt{oracle\_queries\_gg}[m, \ring]$}
						%				
						%				\tab{\fbox{$ P \leftarrow hY $}}
						%				
						%				\textbf{return $ \In$}
						%				
						%		}}	
				%		\caption{The simulation of the random oracle $ \hashG $ by $\simulator $. The different steps than Figure \ref{oracle:Hg} are in the box.}
				%		\label{oracle:HgbyB}
				%		\end{figure}
			
			$\bdv $ simulates the ring signatures of first $ i $ parties as in the real protocol and the parties $ \user_{i+2}, \ldots, \user_{n_h} $ as follows: it generates a ring signature and its anonymous key by running Algorithm \ref{alg:gensign} and selecting randomly, respectively. The simulation of $ \user_{i + 1} $ is different. It lets the public key of $ \user_{i + 1} $ be $ X$. Whenever $ \user_{i+1} $ needs to sign an input $ m$ for a ring $ \ring $ where $ \pk_{i+1} \in \ring$, it obtains $ P = \hashG(m) = hY $ from $ \mathtt{oracle\_queries\_gg} $ and lets $ W = hZ $. Remark that if $ (X,Y,Z)$ is a DH triple (i.e., $  \mathsf{DH}(X,Y,Z) \rightarrow 1 $), $ \user_{i+1} $ is simulated as in \name \ because $ W = x\In$ in this case. Otherwise, $ \user_{i+1} $ is simulated as in the ideal world because $ W $ is random. So, if $  \mathsf{DH}(X,Y,Z)  \rightarrow 1$, $\simulator $ simulates $ H_{i+1} $. Otherwise, it simulates $ H_{i} $. In the end of the simulation, if $ \adv $ outputs $ i $, $\simulator $ outputs $ 0 $ meaning $  \mathsf{DH}(X,Y,Z) \rightarrow 0$. Otherwise, it outputs $ i + 1 $. The success probability of $\simulator $ is equal to the success probability of $ \adv' $ which distinguishes $ H_i $ and $ H_{i +1} $. Since DDH problem is hard, $\simulator $ has negligible advantage in the DDH game. So, $ \adv' $ has a negligible advantage too. Hence, from the hybrid argument, we can conclude that $ H_0    $ which corresponds the output of honest parties in  \name\ and $ H_q  $ which corresponds to  the output of honest parties in ideal world are indistinguishable.
			
			This concludes the proof of showing the output of honest parties in the ideal world are indistinguishable from the output of the honest parties in the real protocol.
		\end{proof}	
		
		Next we show that the simulation executed by $ \simulator $ against $ \adv $ is indistinguishable from the real protocol execution.
		
		\begin{lemma} \label{lem:simulation-ind}
			The view of $ \adv $ in its interaction with the simulator $ \simulator $ is indistinguishable from the view of $ \adv $ in its interaction with real honest parties assuming that CDH is hard in $ \GG $, $\hashG, \hash, \hash', \hash_\ring $ are random oracles and $ \NIZK_{\rcom}, \NIZK_{\rsnark} $ are knowledge sound.
		\end{lemma}
		
		
		\begin{proof}
			The  simulation against the real world adversary $ \adv $ is identical to the real protocol except the output of the honest parties and cases where $ \simulator $ aborts. We have already shown in Lemma \ref{lem:honestoutput} that the output of honest parties are indistinguishable from the real protocol. Next, we show that the abort cases happen with a negligible probability during the simulation. $ \simulator $ aborts during the simulation of random oracles $ \hash $ and $ \hash' $ and during the simulation of verification. We have already explained that the abort case during the simulation of $ \hash $ cannot happen. The abort case happens in the simulation of $ \hash' $ if $ W = hX $ where $ X $ is an honest verification key or if $ \mathtt{oracle\_queries\_h\_CP}[\comring,m,W,\compk,R,R_m] $ has already been defined by a value which is different than $ c $. The first case happens in $ \hash' $ if $ \fgvrf $ selects a random $ W \in \GG$ for an anonymous key of $ m, \pk = X $ and the random oracle $ \hashG $ selects a random $ h \in \FF_p  $ where $ \hashG(m) = hG $ where $ W = hX $. Clearly, this can happen with a negligible probability in $ \secparam $. The 
			second case happens in $ \hash' $ if $ \adv $ queries with the input $ (\comring,m,W,\compk,R,R_m) $ before $ (\pi_1,\pi_2,\compk,W) $ generated by $ \gen_{sign} $. Since $ \compk $ is random, the probability that $ \adv $ guesses even $ \compk $ before it is generated is negligible.
			Now, we are left with the abort case during the verification.
			For this, we show that if there exists an adversary $ \adv $ which makes $ \simulator $ abort during the simulation, then we construct another adversary $ \bdv $ which breaks the CDH problem. 
			
			Consider a CDH game in a prime $ p $-order group  $ \GG $ with the challenges $ \genG,U, V \in \GG$. The CDH challenges are given to the simulator $ \bdv $. Then $ \bdv $ runs a simulated copy of $ \env $ and starts to simulate $ \fgvrf $ and $ \simulator $ for $ \env $. For this, it first runs the simulated copy of $ \adv $ as $ \simulator $ does. $ \bdv $ provides $ (\GG, p, \genG , \genB) $ as a public parameter of the ring VRF protocol to $ \adv $.
			
			Whenever $ \bdv $ needs to generate a ring signature for $ m $ on behalf of an honest party with a public key $ X $, it behaves exactly as $ \fgvrf $ except that it runs   Algorithm \ref{alg:gensignbdv} to generate the signature. 
			%			
			%			\begin{algorithm}
				%				\caption{$\gen_{W}(X, m)$}
				%				\label{alg:genWbdv}	 	
				%				\begin{algorithmic}[1]
					%					\If{$ DB_W[m, X] = \perp $}
					%					\State $ W \leftsample \GG$
					%					\State $ DB_W[m, X] := W $
					%				%	\State \textbf{add} $ W $ to list $ \anonymouskeylist[m,\ring] $
					%					\EndIf
					%					\State \textbf{return} $ DB_W[m, X] $
					%				\end{algorithmic}
				%			\end{algorithm}
			
			
			\begin{algorithm}
				\caption{$\gen_{sign}(\ring,W,X,m)$}
				\label{alg:gensignbdv}	 	
				\begin{algorithmic}[1]
					\State $ c,s_1, s_2 \leftsample \FF_p $
					\State $ \pi_{dleq}  \leftarrow (c,s_1, s_2)$
					\State $ \openpk \leftsample \FF_p $
					\State $ \compk =  \pk + \openpk \, K$
						\State $ R' = s\genG +\delta K + c\compk$
					\State $ R_m = s\hashG(m, \ring) + c W $
					\State $ \comring, \openring \leftarrow \rVRF.\CommitRing(\ring) $
					\State $ \mathtt{oracle\_queries\_h\_CP}[\comring,m, W, \compk,R',R'_m] = c$						
					%\State $ \pi_{dleq} \leftarrow \NIZK.\mathsf{Simulate}(\rcom, (G, \genB,\GG,\compk,W,\In)) $
					%\State \textbf{send} $(\oramsg{learn\_\tau},\sid)  $ to $ \gcrs $
					%\State \textbf{receive} $(\oramsg{trapdoor},\sid, \tau,crs)  $ from $ \gcrs $
					\State $ \pi_{ring} \leftarrow \NIZK_{\rsnark}.\Prove(((\genG,\genB,\GG,\comring, \compk); (\pk, \openpk, \openring))) $ 
					\State\Return$ \sigma = (\pi_{dleq},\pi_{ring},\compk,\comring,W) $
				\end{algorithmic}
				
			\end{algorithm}
			
			
			Clearly the ring signature of an honest party outputted by $ \simulator $ (remember $ \fgvrf$ generates it by Algorithm \ref{alg:gensign}) and the ring signature generated by $ \bdv $ are the same except that it sets up the random oracle $ \hash' $ so that $ \pi_1 $ verifies for $ \rcom $. Therefore, the simulation of $ \hash' $ is simulated as a usual random oracle by $ \bdv $.
			
			In order to generate the public keys of honest parties, $ \bdv $ picks a random $ r_x\in \FF_p $ and generates the public key of each honest party as $ r_xV$.
			Remark that $ \bdv$  never needs to know the secret key of honest parties to simulate them since $ \bdv $ selects anonymous keys randomly  and generates the ring signatures  without the secret keys. Therefore, generating the honest public keys in this way is indistinguishable. 			
			\begin{figure}
				\centering
				
				\noindent\fbox{%
					\parbox{\columnwidth}{%
						\underline{\textbf{Oracle $ \hash$}} \\
						\textbf{Input:} $ m,W $ 
						
						\textbf{if} $ \mathtt{oracle\_queries\_h}[m,  W] = \perp $
						
						\tab{$ y \leftsample \{0,1\}^{\ell_\rVRF} $}
						
						\tab{$  \mathtt{oracle\_queries\_h}[m,  W]:=y $}
						
						
						\textbf{return $  \mathtt{oracle\_queries\_h}[m, W] $}
						
				}}	
				\caption{The random oracle $ \hash $}
				\label{oracle:HbyB}
			\end{figure}
			
			\begin{figure}
				\centering
				
				\noindent\fbox{%
					\parbox{\columnwidth}{%
						\underline{\textbf{Oracle $ \hashG $}} \\
						\textbf{Input:} $ m$ \\
						\textbf{if} $\mathtt{oracle\_queries\_gg}[m] = \perp  $
						
						\tab{$ h \leftsample \FF_p $}
						
						
						\tab{$ P \leftarrow hU $} 
						
						\tab{$\mathtt{oracle\_queries\_gg}[m] := h$}
						
						\textbf{else}:
						
						\tab{$ h \leftarrow \mathtt{oracle\_queries\_gg}[m] $}
						
						\tab{$ P \leftarrow hU$}
						
						\textbf{return $ \In$}
						
				}}	
				\caption{The random oracle $ \hashG $}
				\label{oracle:HgbyB}
			\end{figure}
			
			Simulation of $ \hashG $ is as described in Figure \ref{oracle:HgbyB} i.e., it returns $ hU $ instead of $ hG $. The simulation of $ \hashG $ is indistinguishable from the simulation of $ \hashG $ in Figure \ref{oracle:HgnoPK}. 
			$ \bdv $ simulates the random oracle $ \hash $ in Figure \ref{oracle:HbyB} a usual random oracle. The only difference from the simulation of $ \hash $ by $ \simulator $ is that $ \bdv $ does not ask for the output of $ \hash(m,W) $ to $ \fgvrf $. This difference is indistinguishable from the simulation of $ \hash $ by $ \simulator $ because $ \simulator $ gets it from $ \fgvrf $ which selects it randomly as $ \bdv  $ does. Remark that since $ \hashG $ is not simulated as in Figure \ref{oracle:HgnoPK}, $ \bdv $ cannot check whether $ W $ is an anonymous key generated by an honest key or not.  However, it does not need this information because $ \hash $ is simulated as a usual random oracle. $ \bdv $ also simulates $ \hash_\ring $ for the ring commitments as a usual random oracle.
			
			During the simulation, when $ \adv $ outputs a signature $ \sigma = (\pi_{dleq},\pi_{ring},\compk,\comring,W) $ of message $ m $ with $ \aux $, $ \bdv $ runs $ \rVRF.\Verify(\comring,m, \aux, \sigma) $. If it verifies, it finds the corresponding ring $ \ring $ of $ \comring $ by checking the random oracle $ \hash_\ring $'s database. Remark that there exists $ \ring $ where Merkle tree root of $ \ring $ is $ \comring $ because if it was not the case $ \sigma $ would not verify which also checks  $ \pi_{ring} $. Then, if  $ W \notin \anonymouskeylist[m, \ring] $, $ \simulator $ increments  $ \counter[m,\ring] $ and adds $ W $ to $ \anonymouskeylist[m,\ring] $.
			Then it runs the extractor algorithm of $ \NIZK_{\rsnark} $ i.e., $ \mathsf{Ext}(\rsnark,\pi_{\pk_j},(\genG,\genB,\GG,\comring, \compk) ) \rightarrow b, \openring$. Then it obtains a public key $ X $ by running $ \rVRF.\OpenRing(\comring,\openring) $ where $ X\in \ring $ and $ \compk = X + \openpk \, \genB $. It also runs 	the extractor algorithm of $ \NIZK_{\rcom} $ i.e., $ \mathsf{Ext}(\rcom,\pi_{dleq},(\genG,\genB,\GG, \compk,W,\hashG(m))) \rightarrow (\hat{x},\hat{\openpk} )$ such that $ \compk = \hat{x}\genG + \hat{\openpk} \, \genB $ and $ W = \hat{x} \hashG(m) $. 
			
			If $ X  $ is an honest public key and $ X = \hat{x}G $, $ \bdv $ solves the CDH problem as follows: $ W = \hat{x} h U $ where $ h = \mathtt{oracle\_queries\_gg}[m] $. Since $ X = r V $, $ W = \hat{x}huG =rhuV $. So, $ \bdv $ outputs $ r^{-1}h^{-1}W $ as a CDH solution and simulation ends. Remark that this case happens when $ \simulator $ aborts because of \ref{cond:forgery}.
			
			If $  \counter[m,\ring] = t \geq |\ring_{mal}| $, $ \bdv $ obtains all the signatures $ \{\sigma_i\}_{i =1}^t $ that make $ \bdv $ increment $ \counter[m,\ring] $ and solves the CDH problem as follows: Remark that this case happens when $ \simulator $ aborts because of \ref{cond:uniqueness}.
			
			For all $ \sigma_j = (\pi_{com_j},\pi_{\pk_j},\compk_j,W_j) \in \{\sigma_i\}_{i =1}^t $, $ \bdv $ runs $ \mathsf{Ext}(\rsnark,\pi_{\pk_j},(\genG,\genB,\GG,\ring, \compk_j) ) \rightarrow X_j, \openpk_j$ and adds $ X_j $ to a list $ \setsym{X}  $ where $ X_j\in \ring $ and $ \compk_j = X_j + \openpk_j \,\genB $. One of the following cases happens:
			
			\begin{itemize}
				\item All $ X_j$ in $ \setsym{X} $ are different: If $ \bdv $ is in this case, it means that there exists one public key $ X_a \in \setsym{X} $ which is honest. Then $ \bdv $ runs the extractor algorithm of $ \NIZK_{\rcom} $ i.e., $ \mathsf{Ext}(\rcom,\pi_{com_a},(\genG,\genB,\GG,\ring, \compk_a,W_a,\hashG(m))) \rightarrow (\hat{x}_a,\hat{\openpk}_a )$ such that $ \compk_a = \hat{x}_a\genG + \hat{\openpk}_a \,\genB $ and $ W_a = \hat{x}_a \hashG(m) $.  If $ \bdv $ is in this case, $ \hat{x}_a\genG\neq X_a $ because otherwise it would solve the CDH as described before. Therefore, $ \openpk_a \neq \hat{\openpk}_a $. Since $ X_a + \openpk_a \, \genB = \hat{x}_a\genG + \hat{\openpk}_a \,\genB  $ and $ X_a = r_aV $ where $ r_a $ is generated by $ \bdv $ during the key generation process, $ \bdv $ obtains a representation of $ V = \gamma \genG + \delta \genB $ where $ \gamma = \hat{x}_ar^{-1}_a  $ and $ \delta = (\hat{\openpk}_a -\openpk)\,r_a^{-1} $. Then $ \bdv $ stores $ (\gamma, \delta) $ to a list $ \mathsf{rep} $. If $ \mathsf{rep} $ does not include another element $ (\gamma', \delta')  \neq (\gamma, \delta) $, $ \bdv $ rewinds $ \adv $ to the beginning with a new random coin.  Otherwise, it obtains $ (\gamma', \delta') $ which is another representation of $ V $ i.e., $ V = \gamma' \genG + \delta' \genB $. Thus, $ \bdv $ can find discrete logarithm of $ V $ on base $ G $ which is $ v = \gamma + \delta \theta $ where $ \theta = (\gamma - \gamma')(\delta' - \delta)^{-1} $. $ \bdv $ outputs $ vU $ as a CDH solution and the simulation ends.
				
				
				\item There exists at least two $ X_a,X_b \in \setsym{X} $ where $ X_a = X_b $. $ \bdv $ runs the extractor algorithm of $ \NIZK_{\rcom} $ i.e., $ \mathsf{Ext}(\rcom,\pi_{com_a},(\genG,\genB,\GG,\ring, \compk_a,W_a,\hashG(m))) \rightarrow (\hat{x}_a,\hat{\openpk}_a )$ and $ \mathsf{Ext}(\rcom,\pi_{com_b},(\genG,\genB,\GG,\ring, \compk_b,W_b,\hashG(m))) \rightarrow (\hat{x}_b,\hat{\openpk}_b )$ such that $ \compk_a = \hat{x}_a\genG + \hat{\openpk}_a \, \genB \compk_b = \hat{x}_b\genG + \hat{\openpk}_b \, \genB $ and $ W_a = \hat{x}_a \hashG(m), W_b = \hat{x}_b \hashG(m) $. Since $ W_a \neq W_b $, $ \hat{x}_a \neq \hat{x}_b $.  Therefore, $ \bdv $ can obtain  two different and non trivial representation of $ X_a = X_b $ i.e., $ X_a = X_b = \hat{x}_a\genG + (\hat{\openpk}_a - \openpk_a) \, \genB = \hat{x}_b\genG + (\hat{\openpk}_b - \openpk_b) \, \genB  $. Thus, $ \bdv $ finds the discrete logarithm of $ K = U $ in base $ G $ which is $ u = \frac{\hat{x}_a - \hat{x}_b}{\hat{\openpk}_a -\openpk_a -\hat{\openpk}_b + \openpk_b} $. $ \bdv $ outputs $ uV $ as a CDH solution.
			\end{itemize}
			
			
			
			
			
			
			
			
			
			
			
			
			%	
			%		
			%		
			%			$ \bdv $ solves CDH if $ \bdv $ is in the abort case of simulation of $ \hash$ in Figure \ref{oracle:H} by outputting $ r^{-1}h^{-1}W $ is the CDH solution of $ U,V $. $ r^{-1}h^{-1}W $ is the CDH solution because $ vU $ is a solution of $ CDH $ where $ V =  $
			%		
			%		is in this case $ \bdv $ outputs  $ r^{-1}h^{-1}W $  where $ X^* = rV $ and $ h = \mathtt{oracle\_queries\_gg}[m,\ring] $ and simulation ends. Remark that if $ \bdv $ aborts during the simulation of $ \hash $ it means that $ X^* $ belongs to an honest party and $ X^* =  h^{-1}W = rV = rvG$.  Therefore, $ r^{-1}h^{-1}W $ is the CDH solution of $ U,V $.
			%				  
			%		During the simulation if $ \bdv $ sees a valid forgery ring signature  $ m, \ring, \sigma = (\pi_{dleq}, \pi_{ring}, C, W) $ where $ W $ is an anonymous key generated by $ \bdv $ for $ (m',\ring') \neq (m, \ring) $, $ \bdv $ aborts. $ \Pr[x\hashG(m',\ring') = W; xG \in \ring'| \ring',W] $ is negligible because $ \hashG $ is a random oracle.
			%		%TODO exact probability
			%				  
			
			%		During the simulation if $ \bdv $ sees a forgery ring signature  $ m, \ring, \sigma = (\pi_{dleq}, \pi_{ring}, C, W) $ where $ X = h^{-1}W $ is an honest key, then $ \bdv $ does the following: It runs the extractor algorithms on $ \pi_{ring} $ i.e., $ \ext(\rsnark,..) $ and obtains $ X' \in \ring $ and $ \openpk' $ where $ C = X' + \openpk' K $ and $ \pi_{dleq} $ i.e., $ \ext(\rcom,..)  \rightarrow x, \openpk$ where $ C = x\genG + \openpk K$ and $ W = x\hashG(m, \ring)= xhV$.  Then, $ \bdv $ outputs the CDH of $ U, V $ which is $r^{-1}xU  $. This is correct CDH solution because $ X= rV = xG $, $ V= r^{-1}xG $.
			%		We remark  a forgery signature corresponds to the abort case of $ \simulator $ during the verification because $ \fgvrf $ is in \ref{cond:forgerymalicious}, $ \pk_\simulator  $ is an honest party's key. 
			
			%		During the simulation if $ \bdv $ sees $  k > |\ring_m| $-valid and malicious ring signatures $ \{\sigma_1, \sigma_2, \ldots,\sigma_k\} $ of the message $ m$ signed by $\ring $ whose anonymous keys are $ \{W_1, W_2, \ldots, W_k\} $, respectively , it runs $ \ext(\rsnark,..) $ for each valid malicious signatures $ \sigma_i $(signatures that are not generated by $ \bdv $) and obtains $ \openpk'_i, X'_i \in \ring $. In this case, the one of following two cases must happen:
			%		
			%		\begin{itemize}
				%			\item There exists $ X'\in \ring$ which is an honest key. In this case, $ \bdv  $ runs $ \pi_{dleq} $ i.e., $ \ext(\rcom,..)  \rightarrow x, \openpk$ and stores $ r V = x^*\genG + (\openpk - \openpk')K = x^*\genG + b K$ to $DB $. If $ DB $ is empty, rewind $ \adv $ to the beginning of the simulation. If it is not empty i.e., there exists $ \hat{r}X = \hat{x} \genG + \hat{b}K $, then $ \bdv $ first checks whether $ r = \hat{r} $. If it is the case, it aborts. If it is not the case, it finds the discrete logarithm of $ G=U $ on base $ K $ which is $ t = \frac{\hat{r}^{-1}\hat{b}^*-{r}^{-1}b}{{r}^{-1}\hat{x}-\hat{r}^{-1}\hat{x}} $.  Then, it outputs the CDH of $ U,V $ which is $ tV $. 
				%			We remark that $ \bdv $ aborts after rewinding with a negligible probability because it selects $ r $ randomly.
				%			%TODO exact probability
				%			\item  There exists $ X' \in \ring$ which is the  output of two different signatures.
				%		\end{itemize}
			%		
			%		
			%		During the simulation if $ \bdv $ sees a valid ring signature  $ m, \ring, \sigma = (\pi_{dleq}, \pi_{ring}, C, W) $ where $ X = h^{-1}W \notin \ring$, then $ \bdv $ does the following: It runs the extractor algorithms on $ \pi_{ring} $ i.e., $ \ext(\rsnark,..) $ and obtains $ X' \in \ring $ and $ \openpk' $ where $ C = X' + \openpk' K $ and $ \pi_{dleq} $ i.e., $ \ext(\rcom,..)  \rightarrow x, \openpk$ where $ C = x\genG + \openpk K$ and $ W = x\hashG(m, \ring)= xhV$. In this case, $ X \neq X' $, so $ \openpk \neq \openpk' $. 
			%				  
			%				  
			%		\begin{itemize}
				%			\item If $ X' $ is honest, then store $ r V = x^*\genG + (\openpk - \openpk')K = x^*\genG + b K$ to $DB $. If $ DB $ is empty, rewind $ \adv $ to the beginning of the simulation. If it is not empty i.e., there exists $ \hat{r}X = \hat{x} \genG + \hat{b}K $, then $ \bdv $ first checks whether $ r = \hat{r} $. If it is the case, it aborts. If it is not the case, it finds the discrete logarithm of $ G=U $ on base $ K $ which is $ t = \frac{\hat{r}^{-1}\hat{b}^*-{r}^{-1}b}{{r}^{-1}\hat{x}-\hat{r}^{-1}\hat{x}} $.  Then, it outputs the CDH of $ U,V $ which is $ tV $. 
				%			We remark that $ \bdv $ aborts after rewinding with a negligible probability because it selects $ r $ randomly.
				%			%TODO exact probability
				%				  	
				%			\item If $ X' $ is  a malicious key, $ \bdv $ runs the extractor algorithm on PoK proof $ \pi_{dl} $ of $ X' $ i.e., $ \ext(\rdl,..,) $ which outputs $ x' $ where $ X' = x'G $. Since $ x' \neq x^* $, $ \bdv  $ has a Pedersen commitment $ C $ with two openings so it can find the discrete logarithm of $ K$ on base $ G $ which is $ t = \frac{x^* - x'}{\openpk' - \openpk^*} $.  In the end, it outputs the CDH of $ X,Y $ which is $ tU $. 
				%
				%		 \end{itemize}
			%		
			%	
			So, the probability of $ \bdv $ solves the CDH problem is equal to the probability of $ \adv $ breaks the forgery or uniqueness in the real protocol. Therefore,  if there exists $ \adv $ that makes $ \simulator$ aborts during the verification, then we can construct an adversary $ \bdv $ that solves the CDH problem except with a negligible probability.
			
			
				  
		\end{proof}
		This completes the security proof of our ring VRF protocol.
	\end{proof}


\eprint{\section{Ring VRF Variations}
\label{sec:morefuncs}
In this section, we give ring VRF functionalities which give more security properties than the basic ring VRF functionality $ \fgvrf $ that we define in Figure \ref{f:gvrf}.

\newcommand{\faux}{\fgvrf^{\mathsf{aux}}}
\subsection{Ring VRF with Auxiliary DATA}
We define a variation of ring VRF which signs also an auxiliary data (aux) along with a message. It is very similar to $ \fgvrf $. It additionally requires unforgeability notion for $ \aux $ as well. See Figure \ref{f:aux} for the details.
\begin{figure}
\begin{tcolorbox}
	{  $ \faux $ runs two PPT algorithms $\gen_{sign} $ during the execution.
		
		\begin{description}
			
			\item[Key Generation.] Same as in $ \fgvrf $.
			
			\item[Malicious Ring VRF Evaluation.] Same as in $ \fgvrf $.
			
			\item[Honest Ring VRF Signature.] upon receiving a message $(\oramsg{sign}, \sid, \ring, \pk_i, m, \underline{aux})$ from $\user_i$, verify that $\pk_i \in \ring$ and that there exists a public key $\pk_i$ associated to $\user_i$ in the table $ \vklist $. If that wasn't the case, just ignore the request. 	
			If there exists no $ W' $ such that $ \anonymouskeymap[m,W'] =  \pk_i $, let $ W \leftarrow \setsym{S}_W$. Then, let $y \leftsample \setsym{S}_{eval}$ and set $ \anonymouskeymap[W] = (m,\pk_i) $ and set $ \evaluationslist[m, W] = y$.
			%					\begin{itemize}
				%						%\item If there exists $ W \in  \anonymouskeymap  $, abort.
				%						\item Else 
				%						%TODO define what \in \anonymouskeymap mean
				%					\end{itemize}
			%			    \end{itemize}
		Obtain $ W, y $ where  $ \evaluationslist[m, W] = y$, $ \anonymouskeymap[m,W] = \pk_i $ and run  $ \gen_{sign}(\ring, W, m,\underline{aux}) \rightarrow \sigma $. Verify that $ [m, \underline{aux},W, \sigma, 0] $ is not recorded. If not, abort. Otherwise, record $ [m,\underline{aux}, W, \sigma, 1] $. Return $(\oramsg{signature}, \sid, \ring,W,m,\underline{aux}, y, \sigma)$ to $\user_i$.
		\item[Ring VRF Verification.] Same as in $ \fgvrf $ except that $ \faux $ checks records $  [m,\underline{aux},W,\ring,\sigma, b]   $ in the places where $ \fgvrf $ checks $ [m,W,\ring,\sigma, b] $.
	\end{description}
	
}
\end{tcolorbox}
\caption{Functionality $\faux$.\label{f:aux}}
\end{figure}



%
%\begin{theorem}
%\name \ with AD over the group structure $ (\GG,p,\genG,\genB) $ realizes $ \faux $ in Figure \ref{f:aux} in the random oracle model assuming that NIZK is zero-knowledge and knowledge extractable, the decisional Diffie-Hellman (DDH) problem are hard in $ (\GG,p,\genG,\genB)  $. 
%\end{theorem}
%
%\begin{proof}
%The proof is similar to the proof of Theorem \ref{thm:rvrf}. $ \gen_{sign} $ as well as $ \mathtt{oracle\_queries\_h\_schnor} $ simulated by $ \bdv $ takes the input $ aux $ in Algorithm \ref{alg:gensignbdv}.
%\end{proof}




\newcommand{\frvrfsec}{\fgvrf^s}
\subsection{Secret Ring VRF}
We also define another version of $ \fgvrf $ that we call $ \frvrfsec $. $ \frvrfsec $ operates as $ \fgvrf $. In addition, it also lets a party generate a secret  element to check whether it satisfies a certain relation i.e., $ ((m,y), (\eta, \pk_i)) \in \rel $ where $ \eta $ is the secret random element. If it satisfies the relation, then $ \frvrfsec $ generates a proof. Proving works as $ \mathcal{F}_{zk} $ \cite{zkfunc} except that a part of the witness ($ \eta $) is generated randomly by the functionality. $ \frvrfsec $ is useful in applications where a party wants to show that the random output $ y $ satisfies a certain relation without revealing his identity.

\begin{figure}
	%	\sassafras{\scriptsize}{\scriptsize}
	\begin{tcolorbox}
		{
			%\par\hrulefill\\
			$ \frvrfsec$ for a relation $ \mathcal{R} $ behaves exactly as $ \fgvrf $. Differently, it has an algorithm $ \gen_{\pi} $ and  it additionally does the following:
			\begin{description}
				\item [Secret Element Generation of Malicious Parties.]upon receiving a message $(\oramsg{secret\_rand}, \sid, \ring,\pk,W, m)$ from $\simulator$, verify that $ \anonymouskeymap[m,W] =  \pk_i $. If that was not the case, just ignore the request. If $ \evaluationsecretlist[m,W] $ is not defined, obtain $ y = \evaluationslist[m, W] $. Then, run $ \gen_{\eta}(m,\pk_i, y) \rightarrow \eta  $ and store $ \evaluationsecretlist[m,W] = \eta $. Obtain $ \eta = \evaluationsecretlist[m,W] $  and return $(\oramsg{secret\_rand}, \sid, \ring, W, \eta)$ to $ \user_i $.
				
				\item[Secret Random Element Proof.] upon receiving a message $(\oramsg{secret\_rand}, \sid, \pk, W, m)$ from $\user_i$, verify that $ \anonymouskeymap[m,W] =  \pk_i $. If that was not the case, just ignore the request. If $ \evaluationsecretlist[m,W] $ is not defined, run  $ \gen_{\eta}(m,\pk_i, y) \rightarrow \eta  $ and store $ \evaluationsecretlist[m,W] = \eta $. Obtain $ \eta \leftarrow\evaluationsecretlist[m,W] $ and $ y \leftarrow \evaluationslist[m,W] $. If $ ((m, y),(\eta,\pk_i)) \in \mathcal{R} $,  run  $ \gen_\pi(W, m) \rightarrow \pi $ and add $ \pi $ to a list $ \proofzklist[m, W] $. Else, let $ \pi  $ be $ \perp $. Return $(\oramsg{secret\_rand}, \sid, W, \eta, \pi)$ to $ \user_i $.
				
				\item[Secret Verification.] upon receiving a message $(\oramsg{secret\_verify}, \sid, W, m, \pi)$, relay the message to $ \simulator $ and receive $(\oramsg{secret\_verify}, \sid, W, m, \pi, \pk,\eta)$. Then,
				
				\begin{itemize}
					\item if $ \pi \in \proofzklist[m,W,\ring] $, set $ b = 1 $.
					\item else if $ \evaluationsecretlist[W,m] = \eta$ and $ ((m, y, \ring),(\eta,\pk_i)) \in \mathcal{R} $, set $ b = 1 $ and add to the list $ \proofzklist[m,W,\ring] $.
					\item else set $ b = 0 $.
				\end{itemize}
				Send $(\oramsg{verification}, \sid, \ring, W, m, \pi, b)$ to $ \user_i $.
			\end{description}
		}
	\end{tcolorbox}
	\caption{Functionality  $ \frvrfsec $.\label{f:gvrfzk}}
\end{figure}
}{}

\end{document}
\endinput
%%
%% End of file `sample-sigconf.tex'.
