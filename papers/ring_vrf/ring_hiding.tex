\section{Ring hiding}% {Hiding rings} % ring membership circuits}
\label{sec:ring_hiding}

At first, one imagines sites would accept few rings because each ring
gives some users multiple ``Sybil'' identities within the site.
In practice however, we think many sites benefit from accepting
multiple overlapping rings for convenience, reach, etc., but then
tollerate the resulting few ``Sybil'' users.

As sites accept more rings, we increase risks that each user's ring
\ctx reveals private user attributes, especially if
 users join many rings, sites accept many rings, and
 user agents manage the association poorly.
As a solution, we suggest tweaking \pifast to prove the ring itself
lies in some permitted set of rings, but hide the specific ring used.

We could achieve this using recursion inside \pifast of course,
but doing so lies out of scope.  We instead discuss using other
zk continuation techniques or similar.

\subsection{Unique circuit}

As a first step, if all rings use the same circuit, then we hide the
ring through openning a blinded polynomial commitment \cite{KZG}: 
In \S\ref{subsec:rvrf_faster}, our \pifast takes public input
 $X = \comring\, Y_0 + \compk$ where $\compk = \sk\, Y_1 + b \genB_\gamma$.
Instead of revealing \comring, we prove correctness of \comring in
 $X'' = \comring\, Y_0 + d'' \genB_\gamma$.
We prove $X''$ has this structure by opening a polynomial commitment
\cite{KZG}, but over a larger slower recursive elliptic curve
 like BW6 \cite{BW6}.

TODO: More details?

If using $\pisafe$ anyways then we could prove correctness for \comring
using $\pisafe'$ too, which saves pairings over adding KZK.

\subsection{Multi circuit}

Although $\gamma=1$ remains viable, all circuits wind up with
unique $\delta$ and hence unique $\ecE_2$ SRS element $[\delta]_2$.
We hide $[\delta]_2$ so we suggest proving correctness of $[\delta]_2$
using a blinded polynomial commitment \cite{KZG} over BW6 \cite{BW6},
except this time a multipicative blinding works better.

TODO: More details?

At this point, we have blinded and proven correct both the
ring commitment \comring and the circuit commitment $[\gamma]_2$.
A priori, \pifast chooses $\genG = Y_1$, which reveals the circuit too, like
$$ Y_1 = [{1\over\gamma} (\beta u_1(\tau) + \alpha v_1(\tau) + w_1(\tau))]_1 \mathperiod $$

Instead, we propose to stabalize the public input SRS elements:
We choose $Y_{1,\gamma}$ independent before selecting the circuit
 or running its trusted setup.
We then merely add an SRS element $Y_{1,\delta}$, for usage in $C$, that binds
 our independent $Y_{1,\gamma}$ to the desired definition, so
$$ Y_{1,\delta} := [{1\over\delta} (\beta u_1(\tau) + \alpha v_1(\tau) + w_1(\tau) - \gamma Y_{1,\gamma})]_1 \mathperiod $$
At this point, we replace $Y_1$ by $Y_{1,\gamma}$ everywhere and
 include $\comring \, Y_{1,\delta}$ inside $C$.

In this way, all ring membership circuits could share identical
public input SRS points $Y_{1,\gamma}$, and similarly $Y_0$ if desired.

% Interestingly the SRS ceremony could safely output points for both forms

\subsection{SnarkPack}

TODO: Handle $\pi$ hashes?


