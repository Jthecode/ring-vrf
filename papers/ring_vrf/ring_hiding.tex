\section{Anonymized ring unions}% Ring hiding}% {Hiding rings}
\label{sec:ring_hiding}

We briefly discuss ring VRFs whose \ring consists of the union of several
smaller rings, but which hide to which ring the user belongs.  In this,
we bring out one interesting zero-knowledge continuation technique.


\subsection{Identical circuit}

As a first step, if all rings use the same circuit, then we hide the
ring among several rings using a second zero-knowledge continuation,
not unlike \S\ref{subsec:rvrf_side_channel}.
We could then blind open a polynomial commitment \cite{KZG} to our
\comring choices, Caulk+ \cite{caulk+} or Caulk \cite{caulk} or similar
as in \S\ref{subsec:rvrf_caulk}.

As a special case, if users cannot change their keys too quickly, then
one could reduce the frequency with which users reprove their original
zero-knowledge by using multiple \comring choices across the history
of the same evolving ring database.

% In this, we initially reserve space in for future \comring by padding
% the polynomial commitment with say the base point, and then later
% append new \comring using \cite{aSVC}.


\subsection{Multi-circuit}

We need a new trick if the $\chi_i$ come from different circuit's
trusted setups.  A priori, our zero-knowledge continuation \pifast
fixes some $\genG = \chi_1$, which reveals the circuit,
due to its dependence upon the SRS like
$$ \chi_1 = \frac{\beta u_1(\tau) + \alpha v_1(\tau) + w_1(\tau)}{\gamma} \cdot \gone \ \mathperiod $$

Instead, we propose to stabilize the public input SRS elements across circuits:
We choose $\chi_{1,\gamma}$ independently before selecting the circuit
or running its trusted setup.
We then merely add an SRS element $\chi_{1,\delta}$, for usage in $C$, that binds
our independent $\chi_{1,\gamma}$ to the desired definition, so
$$ \chi_{1,\delta} := \frac{\beta u_1(\tau) + \alpha v_1(\tau) + w_1(\tau) - \gamma \chi_{1,\gamma}}{\delta} \cdot \gone \mathperiod $$
At this point, we replace $\chi_1$ by $\chi_{1,\gamma}$ everywhere and
our proofs add $\comring \, \chi_{1,\delta}$ to $C$.

In this way, all ring membership circuits could share identical
public input SRS points $\chi_{1,\gamma}$, and similarly $\chi_0$ if desired.


\smallskip

At this point, one still needs to hide the SRS elements
 $\delta \cdot \gtwo, \gamma \cdot \gtwo \in \grE_2$ and
 $e(\alpha \cdot \gone, \beta \cdot \gtwo) \in \grE_T$.
We leave this as an exercise to the reader.




\endinput



In their trusted setups, all these Groth16 circuits wind up with unique
toxic waste $\alpha,\beta,\delta$ and hence unique SRS elements
 $\alpha \cdot \gone \in \ecE_1$ and $\beta \cdot \gtwo, \delta \cdot \gtwo, \gamma \cdot \gtwo \in \grE_2$,
and a unique $\grE_T$ element $e(\alpha \cdot \gone, \beta \cdot \gtwo)$.
We hence encounter the open questions:
How should we optimize blinding the verifier key elements derived from toxic waste?
In particular, could we choose some toxic waste elements identically across multiple trusted setups?

We could fix $\gamma=1$ across circuits for example. but in general
this question depends upon other factors, especially if the trusted setups
run concurrently.
If for example, $\alpha$ and/or $\beta$ could be identical across
concurrent trusted setups, then we avoid extensive complexity in
 handling $(\alpha \cdot \gone, \beta \cdot \gtwo) \in \grE_T$ terms.

We expect the $\delta \cdot \gtwo$ might differ between different circuits.
As a nice solution, our Groth16 trusted setup could construct a
KZG polynomial commitment $\rho$ along with openings to the
various $\delta$ as $\delta \cdot \gtwo$.  At this point, our signer could
blind open $\rho$ to the curve point $\delta \cdot \gtwo$.
We suspect concurrent trusted setups could skip much complexity of
Caulk+ \cite{caulk+} or Caulk \cite{caulk} because actually the
 trusted setup erases all knowledge of all openings of $\rho$.



\endinput



\subsection{Post-quantum}





\subsection{SnarkPack}

TODO: Handle $\pi$ hashes?



\endinput



% We cannot really use this text before the applications sections, but
% really why bother when the interesting part here is the zk continuation tool


Identity application:  

At first, one imagines sites would accept few rings because each ring
gives some users multiple ``Sybil'' identities within the site.
In practice however, we think many sites benefit from accepting
multiple overlapping rings for convenience and/or reach, but then
tolerate the resulting few ``Sybil'' users.

As sites accept more rings, we increase risks that each user's \ring
reveals private user attributes, especially if
users join many rings, sites accept many rings, and
user agents manage the association poorly.
As a solution, we suggest tweaking \pifast to prove the ring itself
lies in some permitted set of rings, but hide the specific ring used.

% We could achieve this using recursion inside \pifast of course,
% but doing so lies out of scope.  We instead discuss using other
% zero-knowledge continuation techniques or similar.


