\section{Ring hiding}% {Hiding rings} % ring membership circuits}
\label{sec:ring_hiding}

At first, one imagines sites would accept few rings because each ring
gives some users multiple ``Sybil'' identities within the site.
In practice however, we think many sites benefit from accepting
multiple overlapping rings for convenience, reach, etc., but then
tollerate the resulting few ``Sybil'' users.

As sites accept more rings, we increase risks that each user's ring
\ctx reveals private user attributes, especially if
 users join many rings, sites accept many rings, and
 user agents manage the association poorly.
As a solution, we suggest tweaking \pifast to prove the ring itself
lies in some permitted set of rings, but hide the specific ring used.

We could achieve this using recursion inside \pifast of course,
but doing so lies out of scope.  We instead discuss using other
zero-knowledge continuation techniques or similar.

\subsection{Unique circuit}

As a first step, if all rings use the same circuit, then we hide the
ring through openning a blinded polynomial commitment \cite{KZG} as follows. 

In \S\ref{subsec:rvrf_faster}, our \pifast takes public input
 $X = \comring\, Y_0 + \compk$ where $\compk = \sk\, Y_1 + b \genB_\gamma$.
Instead of revealing \comring, we prove correctness of \comring in
 $X'' = \comring\, Y_0 + d'' \genB_\gamma$.

As an initial condition, we build two polynomial commitments \cite{KZG}
on $\ecE$ with similarly indexed opennings:
\begin{enumerate}
\item to all our \comring choices over the basepoint $Y_0$, and
\item to $1$ over the basepoint $\genB_\gamma$ at numerous points.
\end{enumerate}
We could construct or update the \comring commitment using \cite{aSVC} if desired.

We now construct a commitment opening to $X''$ as desired, and
open at a hidden index. ...

TODO: More details?

If using $\pisafe$ anyways then we could prove correctness for \comring
using $\pisafedot$ too, which saves pairings over adding KZK.

\subsection{Multi circuit}

We handle \comring in the multi-circuit case almost like in the
unique circuit case, except that circuit should enforce \comring
suitability. 

All circuits wind up with unique $\alpha,\beta,\delta$ and
hence unique SRS elements $[\alpha]_1 \in \ecE_1$ and
 $[\beta]_2, [\delta]_2, [\gamma]_2 \in \grE_2$,
and perhaps a unique $\grE_T$ element $e([\alpha]_1, [\beta]_2)$.
We could fix $\gamma=1$ across circuits, but do not require this.

We want polynomial commitment \cite{KZG} that prove correctness
of these values, and which support appropriate blinding operations. 
We prefer polynomial commitments in $\ecE$ because then
we face a straightforward task of multipicatively blinding
all terms identically using Chaum-Pedersen DLEQ proofs,
and openning them all at identical hidden indices (TODO???).

TODO: More details?  Actually sane?!?  $([\alpha]_1, [\beta]_2) \in grE_T$?!?  etc.

Initially, we reserve space in the polynomial commitment for future
circuits, by including commitments to the basepoint.
As we later add circuits, we then using \cite{aSVC} construct opennings to
$[\alpha]_1 \in \ecE_1$ and $[\beta]_2, [\delta]_2, [\gamma]_2 \in \grE_2$,
 during each circuits' trusted setup.
We now prove the multiplicative blindings correct using Chau-Pedersen
DLEQ proofs, and then open all these blinded polynomial commitments
to the desired points
 ${t \over s} [\alpha]_1, t s [\beta]_2, t [\delta]_2, t [\gamma]_2$.

% TODO:  Anything about using a blinded polynomial commitment \cite{KZG} over $\mathtt{BW6}$ \cite{BW6}, provided $\ecE = \mathtt{BLS12-377}$.

At this point, we have blinded and proven correct both the
ring commitment \comring and the circuit commitments $[\gamma]_2$,
$[\delta]_2$, and $e([\alpha]_1, [\beta]_2)$.
A priori, \pifast chooses $\genG = Y_1$, which reveals the circuit too,
due to depending upon the SRS like
$$ Y_1 = \left[ {\beta u_1(\tau) + \alpha v_1(\tau) + w_1(\tau) \over \gamma} \right]_1 \mathperiod $$

Instead, we propose to stabalize the public input SRS elements:
We choose $Y_{1,\gamma}$ independent before selecting the circuit
 or running its trusted setup.
We then merely add an SRS element $Y_{1,\delta}$, for usage in $C$, that binds
 our independent $Y_{1,\gamma}$ to the desired definition, so
$$ Y_{1,\delta} := \left[ {\beta u_1(\tau) + \alpha v_1(\tau) + w_1(\tau) - \gamma Y_{1,\gamma} \over \delta} \right]_1 \mathperiod $$
At this point, we replace $Y_1$ by $Y_{1,\gamma}$ everywhere and
 include $\comring \, Y_{1,\delta}$ inside $C$.

In this way, all ring membership circuits could share identical
public input SRS points $Y_{1,\gamma}$, and similarly $Y_0$ if desired.

% Interestingly the SRS ceremony could safely output points for both forms

\subsection{SnarkPack}

TODO: Handle $\pi$ hashes?


