
\subsection{VRF-AD-KC security}
\label{subsec:vrf_def}


Any signature scheme requires a \KeyGen algorithm of course, but we also
support hiding public keys inside a commitment scheme \CommitKey and \OpenKey.
Zero-knowledge continuations then work by running \OpenKey inside yet
another zero-knowledge proof. 

\begin{definition}
A {\em verifiable random function with auxiliary data and key commitments} (VRF-AD-KC) consists of several algorithms:
\begin{itemize}
\item $\VRF.\KeyGen$ and returns a public key \pk and a secret key \sk, which one typically instantiates via come commitment scheme. 
%
\item $\VRF.\CommitKey : (\pk,\ctx) \mapsto (\compk,\openpk)$ takes a public key \pk and a commitment context \ctx, and returns a public key commitment \compk to \sk and secret opening data \openpk.
\item $\VRF.\OpenKey : (\compk,\openpk) \mapsto \pk$ opens a public key commitment \compk given the specified opening data \openpk.
%
\item $\VRF.\Eval : (\sk,\msg) \mapsto \Out$ takes a secret key \sk and an input $\msg$, and then returns a VRF output $\Out$.
\item $\VRF.\Sign : (\sk,\openpk,\msg,\aux) \mapsto \sigma$ takes a secret key \sk, a public key opening \openpk, an input \msg, and auxiliary data \aux, and then returns a VRF signature $\sigma$.
\item $\VRF.\Verify$ takes $(\compk,\msg,\aux,\sigma)$ for a public key commitment \compk, an input \msg, and auxiliary data \aux, and then returns either an output $\Out$ or else failure $\perp$.
\end{itemize}
\end{definition}
% PPA vs DPA ?

% We typically define VRFs for secret keys 
% We say secret keys are equivalent whenever their evaluation map $F_\sk$ given by
%  $\msg \apsto \Eval([\sk],\msg)$ defines the same function.
% We also define an equivalence relation upon secret keys with classes denoted $[\sk]$
% because secret keys could contain a public key opening data with only limited impact upon the VRF output. 

% \subsection{VRF-AD-KC security}

We say a VRF-AD-KC denoted \VRF is {\em secure} if it satisfies
 correctness, uniqueness, and pseudo-randomness as defined below,
 as well as being existentially unforgeable as a signature on $(\msg,\aux)$
 and being binding in one of the senses discussed blow.
We caution that VRF security remains complex, in part due to
signer and forger each being adversarial in some security properties,
and that ring VRFs make this worse by verifiers being adversarial.

We follow \cite{agg_dkg} by distinguishing an algorithm $\VRF.\Eval$,
 instead of defining it by the equality in correctness,
which simplifies requiring that verifying honest signatures gives a well-defined function.
$\VRF.\Eval$ always has more optimized instantiations anyways.
% We merge correctness of commitment and VRF here because
% our VRF correctness invokes $\CommitKey$ by necessity.

% \subsection{VRF-AD-KC security}

\begin{definition}
We say a VRF-AD-KC satisfies {\em commitment correctness} if
 $\OpenKey \circ \CommitKey$ returns the same public key \pk.
\end{definition}

\begin{definition}
We say a VRF-AD-KC satisfies {\em VRF correctness} if
$(\pk,\sk) \leftarrow \KeyGen$ and $(\compk,\openpk) \leftarrow \CommitKey(\pk,\ctx)$
imply
$\Verify(\compk,\msg,\aux,\Sign(\sk,\openpk,\msg,\aux)) = \Eval(\sk,\msg)$.
% perhaps except with odds negligible in $\secparam$.
\end{definition}

We demand unforgability on $(\msg,\aux)$ because alone
 the usual VRF conditions only yield unforgability for \msg.
We need the usual EUF-CMA game here, except attackers access
\CommitKey in our EUF-CMA-KC game, so if desired
 they could obtain multiple signatures under the same \compk,
 and our signing oracle \ora{Sign} enforces commitments.

\begin{definition}\label{def:vrf_sign_oracle}
We let \ora{Sign} denote a CMA oracle, which creates and stores
a key pair $(\pk,\sk) \leftarrow \KeyGen$ and thereafter
answers oracle calls $\ora{Sign}(\compk,\openpk,\msg,\aux)$ by 
logging $(\msg,\aux)$ and returning $\Sign(\sk,\openpk,\msg,\aux)$,
provided $\pk = \OpenKey(\compk,\openpk)$, or aborting otherwise.
\end{definition}

% TODO \eprint
\begin{definition}
We say a VRF-AD-KC satisfies {\em existential unforgability (EUF-CMA-KC)} if
any PPT adversary \adv has only a negligible advantage in $\secparam$
in the usual chosen-message game adapted to key commitments:
\begin{itemize}
 \item \adv receives $\pk$ from \ora{Sign}, % of Definition \ref{def:vrf_sign_oracle}
 repeatedly queries \ora{Sign},
  and finally produces $\compk,\msg,\aux,\sigma,\openpk$.
 \item \adv wins if $\Verify(\compk,\msg,\aux,\sigma)$ succeeds,
  $\OpenKey(\compk,\openpk) = \pk$, and
   \adv never queried $\ora{Sign}(\cdot,\cdot,\msg,\aux)$.
\end{itemize}
\end{definition}

We do not demand the commitment scheme $\CommitKey$ and $\OpenKey$
be hiding in the definition of VRF-AD-KC security.
Yet, we do internally employ the usual hiding definition from
\cite[pp.8]{cryptoeprint:2019:1185} for commitment schemes however.
% We could employ a weaker chosen-message-like hiding property, but
% this full strength versions suffices.

\begin{definition}
We say a VRF-AD-KC is {\em key hiding} if any PPT adversary \adv
who creates a pair of public keys $\pk_1,\pk_2$
has only negligible advantage for identifying which lies behind a commitment
 $\compk \leftarrow \CommitKey(\pk_i,\ctx)$.
\end{definition}

We want a commitment binding property with unique openings,
analogous to \cite[pp.9]{cryptoeprint:2019:1185}.
% except weakened to require the signature verify too.

\begin{definition}\label{def:vrf_key_binding}
We say a VRF-AD-KC is {\em key binding} if no PPT adversary \adv
produces \compk, \msg, \aux and $\sk_i,\openpk_i$ for $i=1,2$
so that
 % $\CommitKey(\pk_1,\openpk_1) = \compk = \CommitKey(\pk_2,\openpk_2)$ and
 $\OpenKey(\compk,\openpk_1) \ne \OpenKey(\compk,\openpk_2)$ and
 $\Verify(\compk,\msg,\aux,\Sign(\sk_i,\openpk_i,\msg,\aux))$
both succeed for $i=1,2$, except with odds negligible in $\secparam$.
\end{definition}

We weaken binding like this because, from \S\ref{subsec:vrf_pederson}
onward, verification provides a proof-of-knowledge that turns
Pedersen commitments to secret keys into commitments to public keys.
% We always take $\ctx = \emptyset$ when using this ``unique'' key binding
% condition, like in the subsequent two sections, but
%  \ctx becomes important later in analogous ring properties.

\begin{definition}
We say a VRF-AD-KC satisfies {\em uniqueness} if anytime some PPT adversary \adv
produces $\msg$ and $\compk_i,\openpk_i,\aux_i,\sigma_i$ for $i=1,2$
 with $\OpenKey(\compk_1,\openpk_1) = \OpenKey(\compk_2,\openpk_2)$, then also
$\Verify(\compk_1,\msg,\aux_1,\sigma_1) = \Verify(\compk_2,\msg,\aux_2,\sigma_2)$
unless either $\Verify$ returns failure,
except with odds negligible in $\secparam$.
\end{definition}

If desired, one easily simplifies VRF-AD-KC to a VRF-AD by
 taking $\compk = \pk$ and fixing $\openpk = \mathtt{""}$,
 which makes $\VRF.\CommitKey$ and $\VRF.\OpenKey$ trivial.

As in \cite{vrf_micali} and \cite{agg_dkg}, VRF pseudo-randomness
resembles the definition of a pseudo-random function family, in that
adversaries cannot distinguish a random function from the evaluation
map $F_\sk : \msg \mapsto \Eval(\sk,\msg)$,
except now the adversary knows \pk and gets chosen-message queries to \Sign.

\begin{definition}
We say a VRF-AD-KC satisfies {\em pseudo-randomness} if 
any PPT adversary \adv has only a negligible advantage in $\secparam$
in this chosen-message game:
\begin{itemize}
 \item \adv receives $\pk$ from \ora{Sign} of Definition \ref{def:vrf_sign_oracle},
  repeatedly queries \ora{Sign}, and produces $\compk,\openpk,\msg,\aux$. 
 \item If \adv never queried $\ora{Sign}(\cdot,\cdot,\msg,\cdot)$ then
  \adv wins by distinguishing $\Eval(\sk,\msg)$ from random.
\end{itemize}
\end{definition}

We remark that uniqueness plus pseudo-randomness together make VRFs 
act like signatures on \msg, but say nothing about \aux.
Also, pseudo-randomness in \cite{vrf_micali} merges \Eval and \Sign.

We handle cofactors explicitly in this work.  In particular, we impose
a one-to-one map from secret keys \sk to PRFs $F_\sk$, thanks to
 pseudo-randomness, but doing so imposes some subtleties and maybe overkill.

TODO: Handle later
 % TODO: Why?  Explain better.  Also h \PreOut ??

% \smallskip

As in \cite{agg_dkg}, a {\em verifiable unpredictable function} (VUF)
means \adv cannot guess $F_\sk$ in the same game.

\begin{definition}
We say a VRF-AD-KC satisfies {\em unpredictability} if 
any PPT adversary \adv has only a negligible advantage in $\secparam$
in this chosen-message game:
\begin{itemize}
 \item \adv receives $\pk$ from \ora{Sign} of Definition \ref{def:vrf_sign_oracle},
  repeatedly queries \ora{Sign}, and produces $\PreOut,\compk,\openpk,\msg,\aux$.
 \item If \adv never queried $\ora{Sign}(\cdot,\cdot,\msg,\cdot)$ then
  \adv wins if $\PreOut = \Eval(\sk,\msg)$.
\end{itemize}
\end{definition}

We focus upon VRFs because
if $H(k,\cdot)$ is a PRF then computing $\Out = H'(\PreOut, \msg)$
typically transforms a VUF output $\PreOut$, associated to an input $\msg$,
into a VRF output $\Out$, like \cite[Proposition 1]{vrf_micali} does.
As hashes are cheap, implementers should prefer VRFs over more subtle VUFs.
% TODO:  What did I mean here?!?  -JEff
% but some caveats remain to be discussed in \S\ref{sec:IO}.

% \smallskip

% There exist works like \cite[\S3.2 $\fvrf$]{praos} that formalizes 

% In UC works like \cite[\S3.2 $\fvrf$]{praos}, VRFs wind up formalized by
% In \cite[\S3.2 $\fvrf$]{praos}, VRFs are formalized by algorithms
There exist works like \cite[\S3.2 $\fvrf$]{praos} that formalizes VRFs with detached ouputs via the two algorithms
% \begin{itemize}
% \item
$\VRF.\primalgo{EvalProve}(\sk,\msg,\aux) \mapsto (\Out,\sigma)$, in which $\sigma = \VRF.\Sign(\sk,\msg,\aux)$ and $\Out = \VRF.\Eval(\sk,\msg)$, and
% \item
$\VRF.\primalgo{VerifyProof}(\pk,\msg,\aux,\Out,\sigma)$ which returns true only if $\VRF.\Verify(\pk,\msg,\aux,\sigma)$ returns $\Out$.
% \end{itemize}
We strongly prefer the \Sign and \Verify formulation from \cite{agg_dkg}
primarily because the \primalgo{EvalProve}, and \primalgo{VerifyProof}
formulation causes implementation and deployment mistakes:

EC VRF signatures have the form $\sigma = (\PreOut,\pi)$ in which some
inner proof $\pi$ proves correctness of some associated VUF output $\PreOut$. % aka ``pre-output''.  % ``pre-pseudo-random''
It follows $\VRF.\Eval$ never corresponds to $\PreOut$, but if one describes
protocols with an \primalgo{EvalProve} formulation then exposing $\PreOut$
invariably confuses developers into miss-using $\PreOut$ as the output.
% In other words, actual code never corresponds to an \primalgo{EvalProve} and \primalgo{VerifyProof} formulation.

The ``pre-output'' $\PreOut$ preserves algebraic relationships between
secret keys, so protocols described by the \primalgo{EvalProve} formulation
have implementations with broken pseudo-randomness, and perhaps
 related key vulnerabilities and mishandled cofactors.
% We need $\PreOut$ to be exposed by implementations so ...
We avoided the VUF formalism taken by \cite{agg_dkg} in part because
 VUFs obfuscate this difficulty to developers.

As a caveat, there exist UC formalisms that appear simpler with
the \primalgo{EvalProve} and \primalgo{VerifyProof} formulation, like in \cite{praos}.
We therefore propose that VRFs and protocols using VRFs should always be
described using the the \Sign and \Verify formulation, which provides
implementers with a sensible description, but then if needed adopt
 \primalgo{EvalProve} and \primalgo{VerifyProof} only inside the UC formulation itself.
We feel imposing this mental translation upon paper authors and reviewers
 beats imposing the reverse upon developers with less cryptographic knowledge.

\smallskip

There exist VUFs like RSA-FDH or BLS signatures that lack auxiliary data
% There even exist bespoke VRFs that relax correctness to some non-trivial
% relation on the space of secret keys and messages,
%  seemingly including some Rabin variants. 
Yet, these all suffer from either large signature sizes (RSA) or
 slow verification (BLS).
%  VRFs like single-layer XMSS, .

Instead, one prefers instantiating VRFs similarly to
 \cite{nsec5} or \cite{VXEd25519} using Chaum-Pedersen DLEQ proofs \cite{CP92} % Or should it be CP93 ??
 because they provide both small signatures and fast verification.
In these, our auxiliary data \aux can be verified for free,
by binding \aux into the challenge hash, like a Schnorr signature.
VRF protocols could often reduce bandwidth and verifier time this way,
 but some like Sassafras depend upon \aux. 





\endinput % no UC VRF discussion here






























\subsection{UC}

TODO:  Should we give a relatively simple non-harmful UC functionality here?

TODO:  Can we prove this simpler UC functionality from the game?  Can our proof be close to the Praos proof?  If not then why not?


















\endinput 



\def\sid{\ensuremath{\mathtt{sid}}}

\calF_\VRF interacts with signers $U_1,\ldots,U_n$ as follows:
\begin{itemize}
\item  Key Generation --- Upon receiving a message $(\mathtt{``KeyGen''},\sid)$ from a signer $U_i$, hand
$(\mathtt{``KeyGen''}, \sid, U_i)$ to the adversary. Upon receiving (\mathtt{``VerificationKey''}, \sid, U_i, v) from the adversary, if $U_i$ is honest, verify that $v$ is unique, record the pair $(U_i, v)$ and return $(\mathtt{``VerificationKey''}, \sid, v)$ to $U_i$. Initialize the table $T(v,\cdot)$ to empty.
\item  Malicious Key Generation --- Upon receiving a message $(\mathtt{``KeyGen''},\sid,v)$ from $S$, verify that $v$ has not being recorded before; in this case initialize table $T(v,\cdot)$ to empty and record the pair $(S,v)$.

\item  VRF Sign --- Upon receiving a message $(\mathtt{``VRFSign''},\sid,m)$ from $U_i$, verify that some pair $(U_i, v)$ is recorded. If not, then ignore the request. Else, send $(\mathtt{``VRFSign''}, \sid, U_i, m)$ to the adversary. Upon receiving (Eval, \sid, m, \pi) from the adversary, if value $T(v, m)$ is undefined, verify that $\pi$ is unique, pick a random value y from ${\0, \1}^{\ell_\VRF}$ and set $T(v, m) = (y, \{\pi\})$. Else, if T(v,m) = (y,S), set T(v,m) = (y,S ∪ \{\pi\}). In any case, output (Evaluated,\sid,y,\pi) to P.

\item  Malicious VRF Evaluation --- Upon receiving a message (Eval, \sid, v, m) from S for some v, do the following. First, if (S,v) is recorded and T(v,m) is undefined, then choose a random value y from {0,1}lVRF and set T(v,m) = (y,∅). Then, if T(v,m) = (y,S) for some S ̸= ∅, output
(Evaluated, \sid, y) to S, else ignore the request.

\item VRF Verify --- Upon receiving a message (Verify,\sid,m,y,\pi,v′) from some party P, send
(Verify, \sid, m, y, \pi, v′) to the adversary. Upon receiving (Verified, \sid, m, y, \pi, v′) from the adversary do:
\begin{enumerate}
1. If v′ = v for some (Ui,v) and the entry T(Ui,m) equals (y,S) with \pi ∈ S, then set f = 1.
2. Else, if v′ = v for some (Ui,v), but no entry T(Ui,m) of the form (y,{...,\pi,...}) is recorded,
then set f = 0.
3. Else, initialize the table T(v′,·) to empty, and set f = 0.
Output (Verified, \sid, m, y, \pi, f ) to P .
\end{enumerate}
\end{itemize}





\item  VRF Evaluation --- Upon receiving a message (Eval, \sid, m) from $U_i$, verify that some pair $(U_i, v)$
is recorded. If not, then ignore the request. Then, if the value $T(v,m)$ is undefined, pick a random value y from {0, 1}lVRF and set T (v, m) = (y, ∅). Then output (Evaluated, \sid, y) to P , where y is such that T(v,m) = (y,S) for some S.



