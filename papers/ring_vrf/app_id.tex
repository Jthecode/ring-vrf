\section{Application: Identity}
\label{sec:app_identity}

Anonymous VRFs yield anonymous identity systems:
After a user and service establish a secure channel and
the server authenticates itself with certificates, then
the user authenticates themselves by providing an anonymous
VRF signature with input \msg being the server's identity,
thus creating an anonymous or pseudo-nonymous identified session.

We expand this identified session workflow with an extra
update operation suitable for our ring VRF's amortized prover.
We discuss only \pifast here but adapting our techniques to \pisafe appears similar. 

\begin{itemize}
\item {\em Register} --
 Adds users' public key commitments into some public ring \ctx,
 after verifying the user does not currently exist in \ctx.
\item {\em Update} --
 User agents regenerate their stored SNARK \pifast every time \ctx changes,
 likely receiving \comring and \openring from some ring management service.
\item {\em Identify} --
 Our user agent first opens a standard TLS connection to a server \msg,
 both checking the server's name is \msg and checking certificate
 transparency logs, and then computes the shared session id \aux.
 Our user agent computes the user's identity
  $\mathtt{id} = \PedVRF.\Eval(\sk,\msg)$ on the server \msg,
 % Our user agent next rerandomizes \pifast, \compk, and \openpk, computes
 %  $\sigma = \PedVRF.\Sign(\sk,\openpk,\msg,\aux \doubleplus \compk \doubleplus \pifast)$
 and finally sends the server their ring VRF signature
 $\rVRF.\rSign(\sk,\openring,\msg,\aux)$ % $ = (\compk,\pifast,\sigma)$.
\item {\em Verify} -- 
 After receiving $(\compk,\pifast,\sigma)$ in channel \aux,
 the server named \msg checks \pifast on the input $\compk + \comring$,
 checks the VRF signature
 $\mathtt{id} = \PedVRF.\Verify(\compk,\msg,\aux \doubleplus \compk \doubleplus \pifast,\sigma)$,
 and obtains the user's identity $\mathtt{id}$
\end{itemize}


\subsection{Browsers}

We must not link users' identities at different web sites, so user agents
must disable all cross site resource loading, referrer information, etc.
Yet, user agents could still load purely static resources, without metadata
like cookies or referrer information, especially purely content addressable
resources.

In other words, web browsers mostly fail these baseline privacy requirements.
We expect Tor browser and Brave both behave correctly however.
Apple's Safari trends towards preventing invasive cross site resources too.  
% There also do exist decentralized web aka web 3.0 projects whose stated aims
% include more privacy.
In any case, one could always specify rules against cross site privacy invasions
whenever writing ring VRF browser specifications.


\subsection{AML/KYC}

We shall not discuss AML/KYC in detail, because the entire field lacks
clear goals, and thus winds up being ineffective
 \cite{doi:10.1080/25741292.2020.1725366}.
% https://www.tandfonline.com/doi/full/10.1080/25741292.2020.1725366
% via https://twitter.com/ronaldpol/status/1491548352189587460
We do however observe that AML/KYC conflicts with security and privacy
laws like GDPR.  As a compromise between these regulations,
one needs a compliance party who know users' identities,
 while another seperate service party knows the users' activities.
We propose this more efficent solution:

Instead our compliance party becomes an identity issuer who maintains
a ring \ctx consisting of one unique public key for each user.
Arbitrarily many service providers could ring VRF based identity proofs.
If later asked or subpoenaed, users could prove their relevant identities
to investigators, or maybe prove which services they use and do not use. 

Interestingly, \PedVRF could be run ``backwards'' to prove a specific
ring VRF output does not belong to the user, without revealing the users'
identities to investigators. 


\subsection{Moderation}
\label{subsec:moderation}

All discussion or collaboration sites have behavioral guidelines and
moderation rules that deeply impact their culture and collective values.

Our ring VRFs enables a simple blacklisting operation:
If a user misbehaves, then sites could blacklist or otherwise penalizes
their site local identity $\mathtt{id}$.
As $\mathtt{id}$ remains unlinked from other sites, we avoid thorny
questions about how such penalties impact the user elsewhere, and thus
can assess and dispense justice more precisely. 

At the same time, there exist sites who must forget users' histories
eventually, such as when users invoke GDPR compliance or to give children
room to make social mistakes.  In these cases, we suggest injecting
approximate date information into \msg along with the site name,
so \msg becomes site name along with the current year plus month or week.
In this way, users have only one stable $\mathtt{id}$ within the
approximate date range, but they obtain fresh $\mathtt{id}$s merely
by waiting until the next month or week.

We discussed in \S\ref{sec:vrf_pederson} how one signle \PedVRF
signature $\sigma$ could simultaneously prove multiple input-output
pairs $(\msg_j,\mathtt{id}_j)$.  As doing so links these pairs
together, sites could make users link pseudo-nym creation date and
current date, so users could have multiple active pseudo-nyms,
but only one active pseudo-nym per time period, which prevents spam.
If instead we link only adjacent dates, then pseudo-nyms could
be abandoned and replaced, but abandoned pseudo-nyms cannot then
be reclaimed without linking to intervening dates.

In these ways, sites encode important aspects of their moderation rules
into the ring VRF inputs requested.  
% We expect this makes sites' values and culture more uniform, predictable, and transparent.


\subsection{Reduced pairings}
\label{sec:reduced_pairings}

At a high level, we distinguish moderation-like applications discussed
above, which resemble classic identity applications like AML/KYC, from
rate limiting applications discussed in the next section. 
%
In moderation-like applications, ring VRF outputs become long-term
stable identities, so users typically reidentify themselves many times
to the same sites.

As an optimization, our zero-knowledge continuation
should deterministically choose the coefficients $r_1,r_2,b$ used for
rerandomization in \S\ref{sec:rvrf_cont},
 seeded by \msg and \sk, meaning $r_1,r_2,b \leftarrow H(\sk,\msg)$. 
%
In this way, each $\mathtt{id}$ comes packages with the same unique
Groth16 SNARK \pifast, so the verifier could cache valid pairs
$(\mathtt{id},H(\pifast),\mathtt{diffdate})$, and reaccept \pifast
without checking the Groth16 pairing equation whenever found cached.
%
We spend most verifier time checking the Groth16 pairing equation, so
this saves considerable CPU time. % assuming our cache wind up fast enough.

We still risk denial-of-service attacks by users who vary $r_1,r_2,b$ 
randomly however.  We therefore set $\mathtt{diffdate}$ to be the date
when our server last saw a different $H(\pifast)$ associated to
$\mathtt{id}$, or empty if $\mathtt{id}$ always used the same $H(\pifast)$.
We rate limit and verify more lazily if $\mathtt{diffdate}$ is non-empty,
and optionally verify somewhat lazily even if no cache entry exists.

