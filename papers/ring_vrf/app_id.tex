\section{Application: Identity}
\label{sec:app_identity}

Anonymous VRFs yield anonymous identity systems:
After a user and service establish a secure channel and
the server authenticates itself with certificates, then
the user authenticates themselves by providing an anonymous
VRF signature with input \msg being the service's identity,
thus creating an pseudo-nonymous identified session with
a pseudonym unlinkable from other contexts.

We expand this identified session workflow with an extra
update operation suitable for our ring VRF's amortized prover.
We discuss only \pifast here but all techniques apply to \pisk and \pipk similarly. 

\begin{itemize}
\item {\em Register} --
 Adds users' public key commitments into some \ring,
 after verifying the user does not currently exist in \ring.
\item {\em Update} --
 User agents regenerate their stored SNARK \pifast every time \ring changes,
 likely receiving \comring and \openring from some ring management service.
\item {\em Identify} --
 Our user agent first opens a standard TLS connection to a server \msg,
 both checking the server's name is \msg and checking certificate
 transparency logs, and then computes the shared session id \aux.
 Our user agent computes the user's identity
  $\mathtt{id} = \PedVRF.\Eval(\sk,\msg)$ on the server \msg,
 % Our user agent next rerandomizes \pifast, \compk, and \openpk, computes
 %  $\sigma = \PedVRF.\Sign(\sk,\openpk,\msg,\aux \doubleplus \compk \doubleplus \pifast)$
 and finally sends the server their ring VRF signature
 $\rVRF.\rSign(\sk,\openring,\msg,\aux)$ % $ = (\compk,\pifast,\sigma)$.
\item {\em Verify} -- 
 After receiving $(\compk,\pifast,\sigma)$ in channel \aux,
 the server named \msg checks \pifast on the input $\compk + \comring$,
 checks the VRF signature and obtains the user's identity $\mathtt{id}$, ala \\
 $\mathtt{id} = \PedVRF.\Verify(\compk,\msg,\aux \doubleplus \compk \doubleplus \pifast,\sigma)$.
\end{itemize}


\subsection{Browsers}

We must not link users' identities at different web sites, so user agents
should carefully limit cross site resource loading, referrer information, etc.
User agents could always load purely static resources, without metadata
like cookies or referrer information.
% especially purely content addressable resources.
At least Tor browser already takes cross site resource concerns seriously,
while Safari and Brave may limit invasive cross site resources too.
% In any case, one could always specify rules against cross site privacy invasions
% whenever writing ring VRF browser specifications.

We somewhat trust the CAs and CT log system with users' identities in
the above protocol, in that users could login to a site with fraudulent
credentials.  We think cross site restrictions limit this attack vector.
If stronger defenses are desired then instead of \msg being the site name,
\msg could be a public ``root'' key for the specific site, which then
also certifies its TLS certificate.  Ideally its secret key remains air gaped.


\subsection{AML/KYC}\label{subsec:AML_KYC}

We shall not discuss AML/KYC in detail, because the entire field lacks
clear goals, and thus winds up being ineffective
 \cite{doi:10.1080/25741292.2020.1725366}.
% https://www.tandfonline.com/doi/full/10.1080/25741292.2020.1725366
% via https://twitter.com/ronaldpol/status/1491548352189587460
We do however observe that AML/KYC typically conflicts with security
and privacy laws like GDPR.  As a compromise between these regulations,
one needs a compliance party who know users' identities,
 while another separate service party knows the users' activities.
We propose a safer and more efficient solution:

Instead our compliance party becomes an identity issuer who maintains
a public \ring, and privately tracks each users' public key.
Arbitrarily many service providers could employ ring VRF based identity
proofs for diverse purposes.
If later asked or subpoenaed, users could prove their relevant identities
to investigators, or maybe prove which services they use and do not use. 

Interestingly, \PedVRF could be run ``backwards'' to prove a specific
ring VRF output does not belong to the user, without revealing the users'
identities to investigators. 

Our applications mostly ignore key multiplicity. 
AML/KYC demands suspects prove non-involvement using ring VRFs.

\begin{definition}\label{def:rvrf_exculpability}
We say \rVRF is {\em exculpatory} if we have an efficient algorithm
for equivalence of public keys, but a PPT adversary \adv cannot
find non-equivalent public keys $\pk_0,\pk_1$ with colliding VRF outputs.
% (perfectly or computationally)
% (either ever or with advantage negligible advantage in $\secparam$)
\end{definition}

A priori, our JubJub representations $\sk_0 \genJ_0 + \sk_1 \genJ_1$
used in \S\ref{subsec:rvrf_faster} and \S\ref{subsec:rvrf_side_channel}
costs us exculpability from Definition \ref{def:rvrf_exculpability}.
Ad hoc rings make little sense for AML/KYC though, meaning they already
require some ring registration procedure.
%
If desired, exculpability could easily be repaired if joining the ring
requires some secret key dedupliation proof, like 
$\rVRF.\rVerify(\CommitRing(\{\pk\}),\mathsf{ring_name},\mathtt{""},\sigma)$.

% \begin{proposition}
% \end{proposition}


\subsection{Moderation}
\label{subsec:moderation}

All discussion or collaboration sites have behavioral guidelines and
moderation rules that deeply impact their culture and collective values.

Our ring VRFs enables a simple blacklisting operation:
If a user misbehaves, then sites could blacklist or otherwise penalizes
their site local identity $\mathtt{id}$.
As $\mathtt{id}$ remains unlinked from other sites, we avoid thorny
questions about how such penalties impact the user elsewhere, and thus
can assess and dispense justice more precisely. 

At the same time, there exist sites who must forget users' histories
eventually, such as when users invoke GDPR compliance or to give children
room to make social mistakes.  In these cases, we suggest injecting
approximate date information into \msg along with the site name,
so \msg becomes site name along with the current year plus month or week.
In this way, users have only one stable $\mathtt{id}$ within the
approximate date range, but they obtain fresh $\mathtt{id}$s merely
by waiting until the next month or week.

As in \cite{PrivacyPass}, we could adjust \PedVRF to simultaneously
prove multiple VRF input-output pairs $(\msg_j,\mathtt{id}_j)$.
As doing so links these pairs together, sites could make users link
pseudo-nym creation date and current date, so users could have multiple
active pseudo-nyms, but only one active pseudo-nym per time period,
which prevents spam.
If instead we link only adjacent dates, then pseudo-nyms could
be abandoned and replaced, but abandoned pseudo-nyms cannot then
be reclaimed without linking to intervening dates.

In these ways, sites encode important aspects of their moderation rules
into the ring VRF inputs requested.  
% We expect this makes sites' values and culture more uniform, predictable, and transparent.


\subsection{Reduced pairings}
\label{sec:reduced_pairings}

At a high level, we distinguish moderation-like applications discussed
above, which resemble classic identity applications like AML/KYC, from
rate limiting applications discussed in the next section. 
%
In moderation-like applications, ring VRF outputs become long-term
stable identities, so users typically reidentify themselves many times
to the same sites.

As an optimization, our zero-knowledge continuation
should deterministically choose the coefficients $r_1,r_2,b$ used for
rerandomization in \S\ref{sec:rvrf_cont},
 seeded by \msg and \sk, meaning $r_1,r_2,b \leftarrow H(\sk,\msg)$. 
%
In this way, each $\mathtt{id}$ comes packages with the same unique % Groth16 SNARK
\pifast, so the verifier could cache valid pairs
$(\mathtt{id},H(\compk \doubleplus \pifast),\mathtt{diffdate})$, and reaccept \pifast
without checking the Groth16 pairing equation whenever found cached.
%
We spend most verifier time checking the Groth16 pairing equation, so
this saves considerable CPU time. % assuming our cache wind up fast enough.

We still risk denial-of-service attacks by users who vary $r_1,r_2,b$ 
randomly however.  We therefore set $\mathtt{diffdate}$ to be the date
when our server last saw a different $H(\pifast)$ associated to
$\mathtt{id}$, or empty if $\mathtt{id}$ always used the same $H(\pifast)$.
We rate limit and verify more lazily if $\mathtt{diffdate}$ is non-empty,
and optionally verify somewhat lazily even if no cache entry exists.

