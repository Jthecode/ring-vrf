\section{Application: Identity}
\label{sec:app_identity}

Anonymous VRFs yield anonymous identity systems:
After a user and service establish a secure channel and
the server authenticates itself with certificates, then
the user authenticates themselves by providing an anonymous
VRF signature with input \msg being the server's identity,
thus creating an anonymous or pseudo-nonymous identified session.

We expand this identified session workflow with an extra
update operation suitable for our ring VRF's amortized prover.
We discuss only \pifast here but adapting our techniques to \pisafe appears similar. 

\begin{itemize}
\item {\em Register} --
 Adds users' public key commitments into some public ring \ctx,
 after verifying the user does not currently exist in \ctx.
\item {\em Update} --
 User agents regenerate their stored SNARK \pifast everytime \ctx changes,
 likely recieving \comring and \openring from some ring management service.
\item {\em Identify} --
 Our user agent first opens a standard TLS connection to a server \msg,
 both checking the server's name is \msg, and checking certificate
 transperency logs, and computes the shared session id \aux.
 Our user agent computets the user's identity
  $\mathtt{id} = \PedVRF.\Eval(\sk,\msg)$ on the server \msg.
 Our user agent next rerandomizes \pifast, \compk, and \openpk, computes
 $\sigma = \PedVRF.\Sign(\compk,\msg,\aux \doubleplus \compk \doubleplus \pifast,\sigma)$
 and finally sends the server their ring VRF signature:
 $\rVRF.\rSign(\sk,\openring,\msg,\aux) = (\compk,\pifast,\sigma)$.
\item {\em Verify} -- 
 After recieving $(\compk,\pifast,\sigma)$ in channel \aux,
 the server named \msg checks \pifast on the input $\compk + \comring$,
 checks the VRF signature
 $\mathtt{id} = \PedVRF.\Verify(\compk,\msg,\aux \doubleplus \compk \doubleplus \pifast,\sigma)$,
 and obtains the user's identity $\mathtt{id}$
\end{itemize}


\subsection{Browsers}

We must not link users' identiteis at different web sites, so user agents
must disable all cross site resource loading, referer information, etc.
Yet, user agents could still load purely static resources, without metadata
like cookies or referer information, especially purely content addressible
resoruces.

In other words, web browsers mostly fail these baseline privacy requirements.
We expect Tor browser and Brave both behave correctly however.
Apple's Safari trends towards preventing invasive cross site resoruces too.  
There also do exist decentralized web aka web 3.0 projects whose stated aims
include more privacy.

In any case, we could always specify rules against cross site privacy invasions
whenever writing ring VRF browser specifications.


\subsection{AML/KYC}

We shall not discuss AML/KYC in detail, because the entire field lacks
clear goals, and thus winds up being ineffective
 \cite{doi:10.1080/25741292.2020.1725366}.
% https://www.tandfonline.com/doi/full/10.1080/25741292.2020.1725366
% via https://twitter.com/ronaldpol/status/1491548352189587460
We do however observe that AML/KYC conflicts with security and privacy
laws like GDPR typically require that one compliance party know users'
identities, while another service party knows the users' activities.

In principle, our compliance party could maintained the ring \ctx,
which then gives users unique identities with service parties.
If later required, users could prove their identities to investigators,
or merely prove which services they use and do not use. 


\subsection{Moderation}
\label{subsec:moderation}

All discussion or collaboration sites have behavioral guidlines and
moderation rules that deeply impact their culture and collective values.

Our ring VRFs enables a simple blacklisting operation:
If a user missbehaves, then sites could blacklist or otherwise penalizes
their site local identity $\mathtt{id}$.
As $\mathtt{id}$ remains unlinked from other sites, we avoid thorny
questions about how such pentalties impact the user elsewhere, and thus
can assess and despence justice more precisely. 

At the same time, there exist sites who must forget users' histories
eventually, such as when users invoke GDPR compliance or to give children
room to make social mistakes.  In these cases, we suggest injecting
approximate date information into \msg along with the site name,
so \msg becomes site name along with the current year plus month or week.
In this way, users have only one stable $\mathtt{id}$ within the
approximate date range, but they obtain fresh $\mathtt{id}$s merely
by waiting until the next month or week.

If desired, users still have the choice to prove multiple $(\msg,\mathtt{id})$
pairs simultaniously in the same $\sigma$, which links these pairs
together so sites could still give pseudo-nym stability.
If a site thus links pseudo-nym creation date and current date, then
users have multiple active pseudo-nyms, but only one active pseudo-nym
per time period, which prevents spam.  If instead we link only adjuacent
dates, then pseudo-nyms could be abandonded and repalced, but abandonded
pseudo-nyms cannot then be reclaimed without linking to intervening dates.

In these ways, sites encode important aspects of their moderation rules
into the ring VRF inputs requested.  We expect this makes sites' values
and culture more uniform, predictable, and transperent.


% \subsection{Rate limits}


