\section{Applications}
\label{sec:app_short}

We briefly outline how ring VRFs could be used for
 identity, moderation, rationing, and games.


\subsection{Identity}
\label{subsec:app_identity}

Ring VRF outputs could provide users with stable identities across
arbitrarily many services given several conditions:
First, our ring VRF input should be stable for a given services,
 like by using services' urls.
Second, we demand an encrypted connection between the user agent and
the service, in which the service authenticates itself first,
 like by verifying TLS certificates.
Third, the user agent avoids identity leakage between different services,
 like by denying cross site resources.
Fourth, the server trusts the ring membership, like by trusting
 a third party who enforces a ring registration procedure.
Also, this third party updates users as the ring membership evolves.

An HTTPS workflow satisfying these conditions resembles:

\begin{itemize}
\item {\em Register} --
	Adds users' public key commitments into some \ring,
	after verifying the user does not currently exist in \ring.
\item {\em Update} --
	User agents regenerate their stored SNARK $(\pk,\pifast^\inner)$ using
	$\SpecialG.\Preprove( (\sk_1,\sk_2,\openring); (\sk,\comring) )$
	each time \ring changes, perhaps even receiving \comring and \openring
	from some ring management service.
\item {\em Identify} --
	Our user agent first opens a standard TLS connection to a server \msg,
	both checking the server's url is \msg and checking certificate
	transparency logs, and then computes the shared session id \aux.
	Our user agent computes the user's identity
	$\mathtt{id} = \PedVRF.\Eval(\sk,\msg)$ on the server id \msg,
	Our user agent next rerandomizes \pifast, \compk, and \openpk using
	$\SpecialG.\Reprove( \pk, \pifast^\inner )$, computes
	$\sigma = \PedVRF.\Sign(\sk,\openpk,\msg,\aux \doubleplus \compk \doubleplus \pifast)$,
	and finally sends the server their ring VRF signature $(\compk,\pifast,\sigma)$
	% $\rVRF.\rSign(\sk,\openring,\msg,\aux)$ % $ = (\compk,\pifast,\sigma)$.
	Our user agent rejects identity requests from resources besides
	top/outer most frame.
\item {\em Verify} -- 
	After receiving $(\compk,\pifast,\sigma)$ in channel \aux,
	the server \msg checks $\SpecialG.\Verify( \comring, (\compk,\pifast) )$,
	checks the VRF signature, and obtains the user's identity $\mathtt{id}$, ala \\
	$\mathtt{id} = \PedVRF.\Verify(\compk,\msg,\aux \doubleplus \compk \doubleplus \pifast,\sigma)$.
\end{itemize}

Anonymity depends largely upon certificate authentication, including
certificate transparency logs, in that users could otherwise login to
a site with fraudulent credentials.  We think cross site restrictions
 and \aux being the channel limit this attack vector somewhat though.
If stronger defenses are desired then instead of \msg being the url,
\msg could be an air gapped public ``root'' key for the site or CA, which
then also certifies its TLS certificate.  

As an optimization, \Reprove could rerandomize deterministically based
upon $H(\sk,\msg)$, so servers could then cache $\pifast$ verification.


\eprint{}{\begin{comment}}
\subsection{AML/KYC}
\label{subsec:AML_KYC}

We shall not discuss AML/KYC in detail, because the entire field lacks
clear goals, and thus winds up being ineffective
\cite{doi:10.1080/25741292.2020.1725366}.
% https://www.tandfonline.com/doi/full/10.1080/25741292.2020.1725366
% via https://twitter.com/ronaldpol/status/1491548352189587460
We do however observe that AML/KYC typically conflicts with security
and privacy laws like GDPR.  As a compromise between these regulations,
one needs a compliance party who know users' identities,
while another separate service party knows the users' activities.
We propose a safer and more efficient solution:

Instead our compliance party becomes an identity issuer who maintains
a public \ring, and privately knows the users behind each public key.
As above, identity systems could employ \ring freely for diverse purposes.
If later asked or subpoenaed, users could prove their relevant identities
to investigators, or maybe prove which services they use and do not use. 

Interestingly \PedVRF could run ``backwards'' like
$H_{\grE'}(\msg) \ne \sk^{-1} \, \PreOut$
to show a ring VRF output associated to $\PreOut$
does not belong to the user, without revealing the users'
identity $\Hout(\msg, \sk \, H_{\grE'}(\msg))$ to investigators. 

Our applications mostly ignore key multiplicity. 
AML/KYC demands suspects prove non-involvement using ring VRFs.

\begin{definition}\label{def:rvrf_exculpability}
	We say \rVRF is {\em exculpatory} if we have an efficient algorithm
	for equivalence of public keys, but a PPT adversary \adv cannot
	find non-equivalent public keys $\pk_0,\pk_1$ with colliding VRF outputs.
	% (perfectly or computationally)
	% (either ever or with advantage negligible advantage in $\secparam$)
\end{definition}

A priori, our JubJub representations $\sk_0 \genJ_0 + \sk_1 \genJ_1$
used in \S\ref{subsec:rvrf_faster} and \S\ref{subsec:rvrf_side_channel}
costs us exculpability from Definition \ref{def:rvrf_exculpability}.
% Ad hoc rings ...
% Rings used for AML/KYC would be maintained by an authority and require
% some registration procedure, using government issued identity documents.

There is however a natural {\em exculpable public key} flavor $(\pk,\sigma)$,
in which
$\sigma = \Sign(\sk, \CommitRing(\{ \pk \},\pk).\openring, \mathtt{ring\_name}, \mathtt{""})$.
The singleton ring $\{ \pk \}$ ensure that 
$\rVerify(\CommitRing(\{\pk\}), \mathtt{ring\_name}, \mathtt{""}, \sigma)$
uniquely determines the secret key, so exculpability holds
if joining the ring requires $(\pk,\sigma)$.

% \begin{proposition}
% \end{proposition}
\eprint{}{\end{comment}}


\subsection{Moderation}
\label{subsec:moderation}

\eprint{%
All discussion or collaboration sites have behavioral guidelines and
moderation rules that deeply impact their culture and collective values.%
}{}

Our ring VRFs enables a simple blacklisting operation:
If a user misbehaves, then sites could blacklist or otherwise penalizes
their site local identity $\mathtt{id}$.
As $\mathtt{id}$ remains unlinked from other sites, we avoid thorny
questions about how such penalties impact the user elsewhere, and thus
can assess and dispense justice more precisely. 

At the same time, there exist sites who must forget users' histories
eventually, like under a ``right to be forgotten'' principle ala GDPR.
% or an ethical principles of social mistakes being ephemeral.

As users have distinct $\mathtt{id}$ for each \msg,
we obtain ephemeral identities if \msg consists of the url plus
the current week and month, or some other approximate date.
At this point, users have only one stable $\mathtt{id}$ within each
approximate date range, but they obtain fresh $\mathtt{id}$s merely
by waiting until the next week or month.

We then adjust \PedVRF to simultaneously prove multiple VRF input-output
pairs $(\msg_j,\mathtt{id}_j)$.  As in \cite{PrivacyPass}, we merely
delinearize \In and \PreOut in \rSign and \rVerify like:
\begin{align*}
	x &= H(\msg_j,\mathtt{id}_j,\ldots,\msg_j,\mathtt{id}_j) \\
	\In &= \sum_j H_p(x,j) \, \In_j \\
	\PreOut &= \sum_j H_p(x,j) \, \PreOut_j \\
\end{align*}
In this way, \PedVRF proves the same secret key controls two or more
ephemeral identities, thereby constructing a stable identity from the
ephemeral identities.

At login, our site demands linked two input-output pairs given by
$\msg_1 = \mathtt{site\_name} \doubleplus \mathtt{current\_month}$ and
$\msg_2 = \mathtt{site\_name} \doubleplus \mathtt{registration\_month}$,
so users could have multiple active pseudo-nyms given by $\mathtt{id}_2$,
but only one active pseudo-nym per week, enforced by deduplicating
$\mathtt{id}_1$, which still prevents spam and abuse.

Alternatively, we could associate users pseudo-nyms with their recently
seen $\mathtt{id}_1$ but link adjacent months.  In other words, we define
$\msg_j$ by the $j$th previous month, until reaching a previously used
$\mathtt{id}_1$.  In this model, pseudo-nyms could be abandoned, but
abandoned pseudo-nyms cannot then be reclaimed without linking intervening ones.
% Although more costly, sites could permanently bans a few problematic
% users via the inequality proofs described in \S\ref{subsec:AML_KYC} too.

In these ways, sites encode important aspects of their moderation rules
into the ring VRF inputs they demand.  
% % We expect this makes sites' values and culture more uniform, predictable, and transparent.


\section{Rate limiting}
\label{sec:app_rate_limits}

As a rate limiting device, we repeat this approximate date trick from
moderation, but also include a counter $k = 1 \ldots n$ in \msg, so
 $\msg = \mathtt{domain} \doubleplus \mathtt{date} \doubleplus k$.
Instead of treating ring VRF outputs like identities,
we now treat them like nullifiers which could each be spent exactly once,
 similarly to the nullifiers in ZCash or ecash systems.

We do leak information about nullifiers' ownership by revealing $k$ here:
%
An adversary Eve observes two ring VRF signatures with the same
$\mathtt{domain}$ and $\mathtt{date}$ so
$\msg_i = \mathtt{domain} \doubleplus \mathtt{date} \doubleplus k_i$
for $i=1,2$, but with different outputs $\Out_1$ and $\Out_2$.
If $k_1 \ne k_2$ then Eve learns nothing, but if $k_1 = k_2$ then
Eve learns that $sk_1 \ne \sk_2$, representing different users. 

We do not always care if Eve learns this much information, but users'
threat models should be clearly understood before making this choice.
In principle, we could hide $k$ if we replace \PedVRF by a flavor of
\Reval implemented using Groth16, 
 but which still fits our formulation in \S\ref{sec:overview}.
Indeed these \Reval choices could provide post-quantum anonymity, 
without expensive post-quantum soundness, perhaps interesting if leaking $k$ matters.


\subsection{Ration cards}
\label{subsec:app_ration_carts}

As a species, we expect $+3^{\circ}$C over the pre-industrial climate
by 2100 \cite{IPCC2022}, or more likely above $+4^{\circ}$C given
tipping points \cite{tipping2022}.  % https://www.youtube.com/watch?v=LxoyaCSWFGs
At these levels, we experience devastating famines as the Earth's
carrying capacity drops below one billion people \cite{carrying_capacity}.
In the near term, our shortages of resources, energy, goods, water,
and food shall steadily worsen over the next several decades, due to
climate change, ecosystem damage or collapse, and resource exhaustion
ala peak oil.  We expect synchronous crop failures around the 2040s
in particular \cite{climaterisk2021}. % https://nitter.it/ThierryAaron/status/1442442451541807109#m
% off topic: https://12ft.io/proxy?q=https%3A%2F%2Fwww.bbc.com%2Ftravel%2Farticle%2F20220816-why-theres-no-dijon-in-dijon-mustard
Invariably, nations manage shortages through rationing, like during WWI, WWII, and the oil shocks.  

Anonymous rationing works much like rate limiting, except with
 multiple resources, an issuing authority, and limited time shifting:
%
\def\expiry{e}
We fix a set $U$ of limited resource types, overseen by
an authority who certifies verifiers from a key $\mathtt{root}$.
We dynamically define an expiry date $\expiry_{u,d}$ and an availability $n_{u,d}$,
both dependent upon the resource $u \in U$ and date $d$.
% We typically want a randomness beacon $r_d$ too, which prevents
% anyone learning $r_d$ much before date $d$. 
As ring VRF inputs for the spend operation, we choose
$\msg = \mathtt{root} \doubleplus u \doubleplus d \doubleplus k$
where $u \in U$ denotes a limited resource,
the expiry check $d < \mathtt{today} \le \expiry_{u,d}$ passes,
and $1 \le k \le n_{u,d}$.
We also choose \aux to be a preliminary receipt signed by the merchant.
%
At this point, our merchant sends the ring VRF signature to the authority,
who enforces that each nullifier by spent at most once.
Our authority stores the nullifiers until expiry aka $d \le \expiry_{u,d_0}$.

% We do not discuss ring updates here, 

% We remark that fully transferable assets could have constrained lifetimes
% too, which similarly eases nullifier management when implements using
% blind signatures, ZCash sapling, etc.  Yet, all these tokens require
% an explicit issuance stage, while ring VRFs self-issue.

Among the political hurdles to rationing, we know certificates have
a considerable forgery problem, as witnessed by the long history of
fraudulent covid and TLS certificates.  It follows citizens would
justifiably protest to ration carts that operate by simple certificates.
Ring VRFs avoid this political unrest by proving membership in a public list.

\eprint{}{\begin{comment}}
\subsection{Multi-constraint rationing}
\label{subsec:multi_io}

% \cite{PrivacyPass}
As in \S\ref{subsec:moderation}, we could impose simultaneous rationing
constraints for multiple resources $u_1,\ldots,u_k$ by producing one
ring VRF signature in which \PedVRF proves correctness of pre-outputs
for multiple messages 
$\msg_j = \mathtt{root} \doubleplus u_j \doubleplus r_d \doubleplus d \doubleplus k$ for $j=1 \ldots k$.

As an example, purchasing some prepared food product could require spending
rations for multiple base food sources, like making a cake from wheat, butter,
eggs, and sugar.  

\subsection{Decommodification}

There exist many reasons to decommodify important services, like
energy, water, or internet, beyond rationing real physical shortages.
Ring VRFs fit these cases using similar \msg formulations.

As an example, a municipal ISP allocates some limited bandwidth capacity
among all residents.  It allocates bandwidth fairly by verifying ring VRFs
signatures on hourly \msg and then tracking nullifiers until expiry.

Aside from essential government services, commercial service providers
typically offers some free service tier, usually because doing so
familiarizes users with their intimidating technical product.

Some free and paid tier examples include DuoLingo's hearts on mobile, 
continuous integration testing services, and many dating sites.

A priori, rate limiting cases benefit from unlinkability among individual
usages, not merely at some site boundary like moderation requires.
We thus use each ring VRF output only once, which prevents our cashing
trick of \S\ref{sec:reduced_pairings} from reducing verifier pairings.

Although rationing sounds valuable enough, we foresee services like ISP,
VPNs, or mixnets having many low value transactions.
In such cases, ring VRFs could authorize issuing a limited number of
fast simple single-use blind issued credentials, like blind signatures
ala GNU Taler \cite{taler} or PrivacyPass OPRF tokens \cite{PrivacyPass},
which both solve the leakage of $k$ above too.
%
In principle, commercial service providers could sell the same tokens,
which avoids leaking whether the user uses the free or commercial tier.
\eprint{}{\end{comment}}


