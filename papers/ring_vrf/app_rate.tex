\section{Application: Rate limiting}
\label{sec:app_rate_limits}

We showed in \S\ref{sec:app_identity} how ring VRFs give users only
one unique identity for each input \msg.  
We explained in \S\ref{subsec:moderation} that choosing \msg to be
the concatenation of a base domain and a date gives users a stream of changing identities.
%
We next discuss giving users exactly $n > 1$ ring VRF outputs aka
``identities'' per date, as opposed to one unique identity 


% \subsection{Implementation}

As a trivial implementation, we could include a counter $k = 1 \ldots n$
in \msg, so $\msg = \mathtt{domain} \doubleplus \mathtt{date} \doubleplus k$.


\subsection{Avoiding linkage}

Our trivial implementation leaks information about ring VRF outputs'
 ownership by revealing $k$:
%
An adversary Eve observes two ring VRF signatures with the same
$\mathtt{domain}$ and $\mathtt{date}$ so
$\msg_i = \mathtt{domain} \doubleplus \mathtt{date} \doubleplus k_i$
for $i=1,2$, but with different outputs $\Out_1$ and $\Out_2$.
If $k_1 \ne k_2$ then Eve learns nothing, but if $k_1 = k_2$ then
 Eve learns that $sk_1 \ne \sk_2$, maybe representing different users. 

We do not necessarily always care if Eve learns this much information,
but scenarios exist in which one cares.  We therefore briefly describe
several mitigation:

If $n$ remains fixed forever, then we could simply let all users
register $n$ ring VRF public keys in \ring.
If $n$ fluctuates under an upper bound $N$, then we could create $N$
rings $\ring_i$ for $i = 1 \ldots N$, and
 then blind \comring in \pifast similarly to \S\ref{sec:ring_hiding}.

Although simple, these two approaches require users construct $n$ or $N$
different $\pipk$ proofs every time \ring updates.

Instead of proving ring membership of one public key, $\pipk$ could
prove ring membership of a Merkle commitment to multiple keys, so
users have $\pisk^1,\ldots,\pisk^N$ for each of their multiple keys.

% \smallskip

In principle, there exists ring VRFs that hide parts of their input
\msg, but still fit our abstract formulation in \S\ref{sec:overview}.
Although interesting, we caution these bring performance concerns not
discussed here, so deployments should consider if leaking $k$ suffices.


\subsection{Ration cards}

As a species, we expect $+3^{\circ}$C over the pre-industrial climate
by 2100 \cite{IPCC2022}, or more likely above $+4^{\circ}$C given
tipping points \cite{tipping2022}.  % https://www.youtube.com/watch?v=LxoyaCSWFGs
At these levels, we experience devastating famines as the Earth's
carrying capacity drops below one billion people \cite{carrying_capacity}.
In the near term, our shortages of resources, energy, goods, water,
and food shall steadily worsen over the next several decades, due to
climate change, ecosystem damage or collapse, and resource exhaustion
ala peak oil.  We expect synchronous crop failures around the 2040s
in particular \cite{climaterisk2021}. % https://nitter.it/ThierryAaron/status/1442442451541807109#m
% off topic: https://12ft.io/proxy?q=https%3A%2F%2Fwww.bbc.com%2Ftravel%2Farticle%2F20220816-why-theres-no-dijon-in-dijon-mustard
Invariably, nations manage shortages through rationing, like during WWI, WWII, and the oil shocks.  

Ring VRFs support anonymous rationing:
Instead of treating ring VRF outputs like identities,
we treat them like nullifiers which could each be spent exactly once.

\def\expiry{e}
We fix a set $U$ of limited resource types, overseen by
 an authority who certifies verifiers from a key $\mathtt{root}$.
We dynamically define an expiry date $\expiry_{u,d_0}$ and an availability $n_{u,d_0}$,
both dependent upon the resource $u \in U$ and current date $d_0$.
We typically want a randomness beacon $r_d$ too, which prevents
anyone learning $r_d$ much before date $d$. 
% Among other usages, this reduces damages from key compromises.
As ring VRF inputs, we choose
 $\msg = \mathtt{root} \doubleplus u \doubleplus r_d \doubleplus d \doubleplus k$
where $u \in U$ denotes a limited resource,
 $d$ denotes an non-expired date meaning $\expiry_{u,d_0} < d \le d_0$,
 and $1 \le k \le n_{u,d_0}$.
In this way, our rationing system controls both daily consumption
via $n_{u,d_0}$ and time shifted demand via expiry time $\expiry_{u,d_0}$.

Importantly, our rationing system retains ring VRF outputs as nullifiers,
filed under their associated date $d$ and resource $u$, so nullifiers
expire once $d \le \expiry_{u,d_0}$ which permits purging old data rapidly.

We remark that fully transferable assets could have constrained lifetimes
too, which similarly eases nullifier management when implements using
blind signatures, ZCash sapling, etc.  Yet, all these tokens require
an explicit issuance stage, while ring VRFs self-issue.

Among the political hurdles to rationing, we know certificates have
a considerable forgery problem, as witnessed by the long history of
fraudulent covid and TLS certificates.  It follows citizens would
justifiably protest to ration carts that operate by simple certificates.
Ring VRFs avoid this political unrest by proving membership in a public list.


\subsection{Multi-constraint rationing}
\label{subsec:multi_io}

% \cite{PrivacyPass}
As in \S\ref{subsec:moderation}, we could impose simultaneous rationing
constraints for multiple resources $u_1,\ldots,u_k$ by producing one
ring VRF signature in which \PedVRF proves correctness of pre-outputs
for multiple messages 
 $\msg_j = \mathtt{root} \doubleplus u_j \doubleplus r_d \doubleplus d \doubleplus k$ for $j=1 \ldots k$.

As an example, purchasing some prepared food product could require spending
rations for multiple base food sources, like making a cake from wheat, butter,
eggs, and sugar.  


\subsection{Decommodification}

There exist many reasons to decommodify important services, like
energy, water, or internet, beyond rationing real physical shortages.
Ring VRFs fit these cases using similar \msg formulations.

As an example, a municipal ISP allocates some limited bandwidth capacity
among all residents.  It allocates bandwidth fairly by verifying ring VRFs
signatures on hourly \msg and then tracking nullifiers until expiry.

Aside from essential government services, commercial service providers
typically offers some free service tier, usually because doing so
familiarizes users with their intimidating technical product.

Some free and paid tier examples include DuoLingo's hearts on mobile, 
continuous integration testing services, and many dating sites.

A priori, rate limiting cases benefit from unlinkability among individual
usages, not merely at some site boundary like moderation requires.
We thus use each ring VRF output only once, which prevents our cashing
trick of \S\ref{sec:reduced_pairings} from reducing verifier pairings.

Although rationing sounds valuable enough, we foresee services like ISP,
VPNs, or mixnets having many low value transactions.
In such cases, ring VRFs could authorize issuing a limited number of
fast simple single-use blind issued credentials, like blind signatures
ala GNU Taler \cite{taler} or PrivacyPass OPRF tokens \cite{PrivacyPass},
 which both solve the leakage of $k$ above too.
%
In principle, commercial service providers could sell the same tokens,
which avoids leaking whether the user uses the free or commercial tier.


\subsection{Delegation}

Almost all single-use blind signed tokens have an implicit delegation
protocol, in which token holders transfer token credentials without
sacrificing their own access.
As double spending remains possible, delegatees must trust delegators.
% PrivacyPass \cite{PrivacyPass} only supports this transfer style.
GNU Taler \cite{taler} argues against taxing such trusting transfers,
like when parents give their kids spending money, but enforces taxability
only when also preventing double spending.

In our rationing scheme, spenders authenticate their specific spending
operations inside the associated data \aux in a rVRF-AD signature.
As doing so requires knowing \sk, delegators place enormous trust in
delegatees, which likely precludes say parents delegating to children.

We could however achieve delegation by treating the ring VRF like a
certificate that authenticates another public key held by the delegatee.
In fact, delegators could limit delegatees uses too in this certificate,
like how GNU Taler achieves parental restrictions. % \cite{???}

We remark that \PedVRF has adaptor signatures aka implicit certificate mode:
A delegatee learns the full ring VRF signature, but the delegatee hides
the blinding factor signature $s_1$ in \PedVRF from downstream recipients,
and instead merely prove knowledge of $s_1$, say via
 a key exchange or another Schnorr signature with the base point $K$.
EC VRFs lack this mode.



