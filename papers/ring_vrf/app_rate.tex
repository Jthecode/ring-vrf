\section{Application: Rate limiting}
\label{sec:app_rate_limits}

We showed in \S\ref{sec:app_identity} how ring VRFs give users only
one unique identity for each input \msg.  
We explained in \S\ref{subsec:moderation} that choosing \msg to be
the concatenation of a base domain and a date gives users a stream of changing identities.

We next discuss giving users exactly $n > 1$ ring VRF outputs aka
``identites'' per date, as opposed to the unique identity 


% \subsection{Implementation}

As a trivial implementation, we could include a counter $k = 1 \ldots n$
in \msg, so $\msg = \mathtt{domain} \doubleplus \mathtt{date} \doubleplus k$.


\subsection{Avoiding linkage}

We actually leaks information about ring VRF outputs' ownership
 in this trivial implementation, due to revealing $k$.
%
An adversary Eve observes two ring VRF signatures with different outputs
$\Out_1$ and $\Out_2$, and the same data but with $k_1$ and $k_2$.
If $k_1 \ne k_2$ then Eve learns nothing, but if $k_1 = k_2$ then
 Eve learns that $sk_1 \ne \sk_2$, maybe representing different users. 

We do not necessarily care if Eve learns this much information,
but scenarions exist in which one cares.  We therefore briefly describe
several easy solutions:

If $n$ remains fixed forever, then we could simply let all users
register $n$ ring VRF public keys in \ctx.
If $n$ fluctuates with an upper bound $N$, then we could create $N$
rings $\ctx_i$ for $i = 1 \ldots N$, and
 then blid \comring in \pifast similarly to \S\ref{subsec:hiding_rings}.

Although simple, these two approaches require users construct $N$
different $\pifast$ proofs everytime the ring \ctx updates.
We have additional options if using the \pisafe approach: 

Instead of proving ring membership of one public key,
$\pisafe$ could prove ring membership of a KZG commitment to $n$ keys.
Our unamortized SNARK $\pisafe'$ could then open the specific key in
zero-knowledge, which sounds acceptable once already working in the \pisafe case.

We could compute the hash-to-curve $\In := H_{\grE}(\msg)$ inside the
unamortized SNARK $\pisafe'$ and reveal only a Pedersen-like commitment
to $\In + \openpk^{-1} \genB$.  We then adjust \PedVRF to yield
a proof-of-knowledge of $\Out/\In$ subject to soundness of this
$\pisafe'$.

In all cases, we incur costs by hiding part of the input \msg, so
deployment should seriously consider if leaking $k$ suffices.

% TODO: Games part
% Although our examples fit this poorly, we could choose $N$ large and
% then reject many $\Ou$, so that statistically all users recieved the
% desired number of accepted $\Ou$.
% If so, we decrease the odds that $k_1 = k_2$ by
%  increasing $N$ and decreasing odds of accepting some $\Out$.


\subsection{Ration cards}

As a species, we expect shortages of resources, energy, goods, water,
and food beginning during the next several decades, due to climate change,
ecosystem damage or collapse, and resource exaustion ala peak oil. 
Invariably, we manage shortages through rationing, like during WWI,
WWII, and the oil shocks.  

Ring VRFs support anonymous rationing:
Instead of treating ring VRF outpus like identitys, although
this still holds somewhat.  We treat them like nullifiers which
could each be spent exactly once.

We fix a set $U$ of limited resources, and dynamically define
an expiry date $\bar{d}_{u,d_0}$ an availability $n_{u,d_0}$, 
both dependent upon the resource $u \in U$ and current date $d_0$.
We typically want a randomness beacon $r_d$ too, which prevents
anyone learning $r_d$ much before date $d$. 
As ring VRF inputs, we choose
 $\msg = u \doubleplus r_d \doubleplus d \doubleplus k$
where $u \in U$ denotes a limited resoruce,
 $d$ denotes an non-expired date meaning $\bar{d}_{u,d_0} < d \le d_0$,
 and $1 \le k \le n_{u,d_0}$.
In this way, our rationing system controls both daily consumption
via $n_{u,d_0}$ and time shifted demand via expiry time $\bar{d}_{u,d_0}$.

Importantly, our rationing system retains ring VRF outputs as nullifiers,
filed under their associated date $d$ and resource $u$, so nullifiers
expire once $d \le \bar{d}_{u,d_0}$ which permits perging old data rapidly.


\subsection{Decommodification}

There exist many reasons to decommodify important services,
like energy, water, or internet, beyond real physical shortaged.
Indeed real service providers typical prefer subscription models.
Ring VRFs fit these cases using similar $\msg$ formulations.

As an example, VPN providers or movie streaming services have some
fixed capasity, and hopefully want to know nothing about their
customers internet usage, so users purchase a ring membership, and
then authenticate their bandwidth usages by providing ring VRF signatures.
As above, our service must track nullifiers, but only until expirly.

...


\subsection{Anonymity }



base context concatenated with both
the date and a bounded counter $1..k$, so that users have exactly $k$
identites on a given date.



and especially 

We described encoding site rules into 

moderation rules
into ring VRF usage by appending the year and week to the VRF input \msg.

We need 











\subsection{Delegation}



