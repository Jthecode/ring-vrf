
\def\tmpaux{\aux \doubleplus \piring \doubleplus \comring}
\def\tmpeprintaux{\eprint{\aux'}{\tmpaux}}
\def\tmpindent{\hspace*{5pt}}
\section{The First Ring VRF Construction}% {Ring VRFs from the Pedersen VRF}
\label{sec:pederson_vrf}



$ \rVRF.\Setup(1^\secparam) $ generates the public parameters $ \pprvrf = (\crsrvrf,\crsnark, p,\grE, \genG,\genB, \setsym{S}_{eval}  = \F_p)$. Here, $ p $ is a prime order of the group $ \grE $ with generators $ \genG,\genB $.  $ \crsrvrf, \crsnark $ are generated by $ \NIZK_{\Rring}.\Setup(1^\secparam) $ and $ \NARK_{\relcomring}.\Setup(1^\secparam) $, respectively. Our ring VRF construction deploys random oracles $H_p,H: \{0,1\}^* \to \F_p$, $H_{\grE}: \{0,1\}^* \to \grE$.

%and $ H_{\ring} $ for constructing a Merkle tree.
%In this section we focus upon Pedersen VRF and the relations 
%describing a SNARK for ring membership; we discuss the zero-knowledge continuation
%that makes the overall ring VRF efficient in the next section.

We build our  ring VRF protocol with an efficient evaluation proof, which
we call the Pedersen VRF  denoted \PedVRF.
%It deploys a NIZK protocol for the relation $\Reval$ introduced in a general form 
%in \S\ref{sec:overview}.  We first introduce $ \PedVRF $.



%We refer readers to \S\ref{sec:ec_background} for notation,
% like our curves and hash functions.
%In particular miss-use resistance dictates \PedVRF be instantiated with
%two elliptic curves: $\ecE$ (or $\ecE_1$) handles key commitments
% build with two independent base points $\genG$ and $\genB$,
%We hash-to a ``sister'' Edwards curve $\ecEsis$ with a subgroup $\grEsis$
% of the same order $p$ as $\grE$.
%In practice $\grEsis$ has cofactor $\hsis$ divisible by 4, while
% $\grE$ might have effective cofactor 1 if deserialization enforces subgroup checks.
%Any readers only interested in theoretical security arguments
%should assume $\ecEsis = \ecE$ and $\hsis = 1 = h$,
% while implementers should read more carefully.

\paragraph{Pedersen VRF:} 
We construct \PedVRF following a similar approach as  other VRF constructions
\cite{nsec5,VXEd25519,draft-irtf-cfrg-vrf-10}. The distinctions in $ \PedVRF $ are that it does not expose any public key and the public key in these constructions used for verification is replaced  by a Pedersen commitment to the secret key \sk.

\begin{itemize}
	\item $\PedVRF.\KeyGen$  outputs  $\sk \leftsample \F_p$. % and $\pk = \sk \, \genG$.
	\item $\PedVRF.\Eval(\sk,\msg) \rightarrow \Out$: It outputs the evaluation value of $ \msg $ which is $ \Out = H(\msg, \PreOut) $  where $\PreOut = \sk \, H_{\grE}(\msg)$.

%\noindent We add an algorithm to obtain a Pedersen commitment to the secret key \sk.


\item $\PedVRF.\CommitKey(\sk) \rightarrow (\compk, \openpk) $: It selects randomly a blinding factor $ \openpk \in \F_p $ and outputs the Pedersen commitment $ \compk =   \sk \, \genG + \openpk \, \genB$ and $ \openpk $.  
	% \item $\PedVRF.\OpenKey(\compk,\openpk)$ \,
	% returns $\pk = \compk - \openpk \, \genB$.
	% \item $\PedVRF.\OpenKey(\sk,\openpk)$ \,
	% returns $\compk = \sk \, \genG + \openpk \, \genB$.
\end{itemize}
%We do not expose an opening algorithm here because opening occurs inside
%our zero knowledge continuation,
%as described in $\Rring$ and \S\ref{sec:rvrf_cont} blow.

 \Sign and \Verify algorithms of \PedVRF are directly  aligned with the proving system of Chaum-Pedersen DLEQ for relation $\mathcal{R}_{eval}$ (see below),
instantiated by a Fiat-Shamir transform of a sigma protocol.

\eprint{$$ \mathcal{R}_{eval} = \Setst{
		\begin{aligned}
			& (\compk,\PreOut,\msg) ; \\ 
			& \,\, (\sk, \openpk) 
		\end{aligned}
	}{
		\begin{aligned}
			& \compk = \sk\,\genG + \openpk\,\genB, \\
			& \,\, \PreOut = \sk \, H_{\grE}(\msg) 
		\end{aligned}
	}  \mathperiod \label{rel:commit} 
	$$}{
\begin{footnotesize}
$$ \mathcal{R}_{eval} = \Setst{
	\begin{aligned}
		& (\compk,\PreOut,\msg) ; \\ 
		& \,\, (\sk, \openpk) 
	\end{aligned}
}{
	\begin{aligned}
		& \compk = \sk\,\genG + \openpk\,\genB, \\
		& \,\, \PreOut = \sk \, H_{\grE}(\msg) 
	\end{aligned}
}  \mathperiod \label{rel:commit} 
$$
\end{footnotesize}
}




\begin{itemize}
	\item $\PedVRF.\Sign(\sk,\openpk,\msg,\aux) \rightarrow \sigma$: It receives as an input a secret key $ \sk $, a blinding factor $ b \in \F_p $, an input $ \msg $ to prove its evaluation and an associated data $ \aux $ to sign.
	% takes a secret key \sk and blinding factor \openpk, an input $\msg$, and associated data \aux, and then performs
	% $ \NIZK_{\mathcal{R}_{eval}}.\Prove(((\genG, \genB,\grE,\compk,\PreOut,\In); (\sk, \openpk))) $
	% $ \sigma = (\PreOut,\compk,\pi_{eval}) $
	It first computes $\PreOut := \sk \, H_{\grE}(\msg)$ and $ \compk = \sk \genG + \openpk\genB $. Then, it runs $\NIZK_{\rel_{eval}}.\Prove(\compk, \PreOut, \msg;\sk,\openpk)  $ which generates a Chaum-Pedersen DLEQ proof
	for relation $\mathcal{R}_{eval}$ i.e.,
	let $r_1,r_2 \leftsample \F_p$
	and compute $R = r_1 \genG + r_2 \genB, R_m = r_1 H_{\grE}(\msg) $ and $c = H_p(\aux,\msg,\compk,\PreOut,R,R_m)$, finally compute $s_1 = r_1 + c \, \sk$ and $s_2 = r_2 + c \, \openpk$ and let $ \pi = (c,s_1,s_2) $.
	In the end, it returns the signature $\sigma = (\pi, \PreOut) $.
	% and return the signature $\sigma = (\PreOut,c,s_1,s_2)$.
	
	\item $\PedVRF.\Verify(\compk,\msg,\aux,\sigma) \rightarrow (\Out \ \vee \perp)$: 
	It verifies $ \sigma $ with  $ \compk $ by running  $ \NIZK_{\mathcal{R}_{eval}}.\Verify(\compk,\PreOut,\msg; \pi_{eval} ) $ i.e.,
	parse $\sigma = (\PreOut,c,s_1,s_2) $ and check if  
	$c = H_p(\aux,\msg,\compk,\PreOut,R,R_m)$ where  $ R = s_1 \genG + s_2 \genB - c \, \compk$ 
	and $ R_m = s_1 H_{\grE}(\msg) - c \, \PreOut$.
	% First parse $\sigma = (\PreOut, \pi_{eval} =(c,s_1,s_2))$,
	% recompute $\In := H_{\grEsis}(\msg)$, 
	% $R = s_1 \genG + s_2 \genB - c \, \compk$, and
	% $R_m = s_1 \In - c \, \PreOut$.
	% Finally if $c = H_p(\aux,\msg,\compk,\PreOut,R,R_m)$ holds,
	If this verifies, it outputs $H(\msg,\PreOut)$. Otherwise, it outputs failure $\perp$.
\end{itemize}






% DESCRIPTION WITH COFACTOR
%\begin{itemize}
%    \item $\PedVRF.\KeyGen$ \, returns $\sk \leftsample \F_p$. % and $\pk = \sk \, \genG$.
%    \item $\PedVRF.\Eval : (\sk,\msg) \mapsto H'(\msg, \hsis\,\PreOut)$ where $\PreOut = \sk \, H_{\grEsis}(\msg)$.
%\end{itemize}
%\noindent We instead add an algorithm to obtain a Pedersen commitment to the secret key \sk.
%\begin{itemize}
%    \item $\PedVRF.\CommitKey(\sk)$ \,
%    returns a blinding factor $\openpk \leftsample \F_p$
%    and a commitment $\compk = \sk \, \genG + \openpk \, \genB$.
%    % \item $\PedVRF.\OpenKey(\compk,\openpk)$ \,
%    % returns $\pk = \compk - \openpk \, \genB$.
%    % \item $\PedVRF.\OpenKey(\sk,\openpk)$ \,
%    % returns $\compk = \sk \, \genG + \openpk \, \genB$.
%\end{itemize}
%We do not expose an opening algorithm here because opening occurs inside
%our zero knowledge continuation,
% as described in $\Rring$ and \S\ref{sec:rvrf_cont} below.
%
%Our \Sign and \Verify algorithms of \PedVRF correspond to
%the \Prove and \Verify algorithms of a Chaum-Pedersen DLEQ proof
% for relation $\mathcal{R}_{eval}$,
%instantiated by a Fiat-Shamir transform of a sigma protocol.
%$$ \mathcal{R}_{eval} = \Setst{
	%  \begin{aligned}
		%    & (\compk,\PreOut,\In) ; \\ 
		%    & \,\, (\sk, \openpk) 
		%  \end{aligned}
	%}{
	%  \begin{aligned}
		%    & \compk = \sk\,\genG + \openpk\,\genB, \\
		%    & \,\, \PreOut = \sk\,\In 
		%  \end{aligned}
	%}  \mathperiod \label{rel:commit} 
%$$
%%
%\begin{itemize}
%	\item $\PedVRF.\Sign : (\sk,\openpk,\msg,\aux) \mapsto \sigma$ \,
%	% takes a secret key \sk and blinding factor \openpk, an input $\msg$, and associated data \aux, and then performs
%    % $ \NIZK_{\mathcal{R}_{eval}}.\Prove(((\genG, \genB,\grE,\compk,\PreOut,\In); (\sk, \openpk))) $
%    % $ \sigma = (\PreOut,\compk,\pi_{eval}) $
%	First compute $\In := H_{\grEsis}(\msg)$ and $\PreOut := \sk \, \In$ and \compk.
%	Next sample random $r_1,r_2 \leftsample \F_p$
%	to compute $R = r_1 \genG + r_2 \genB$ and $R_m = r_1 \In$.
%	Compute the challenge $c = H_p(\aux,\msg,\compk,\PreOut,R,R_m)$.
%	Finally compute $s_1 = r_1 + c \, \sk$ and $s_2 = r_2 + c \, \openpk$.
%	and return the signature $\sigma = (\PreOut,R,R_m,s_1,s_2)$.
%	% and return the signature $\sigma = (\PreOut,c,s_1,s_2)$.
%
%	\item $\PedVRF.\Verify : (\compk,\msg,\aux,\sigma) \mapsto \Out \,\, \lor \perp$ \,
%	% $ \NIZK_{\mathcal{R}_{eval}}.\Verify(\genG, \genB,\grE,\compk,\PreOut,\In, \pi_{eval} ) $
%	First parse $\sigma = (\PreOut, \pi_{eval} = (R,R_m,s_1,s_2))$,
%	recomputes $\In := H_{\grEsis}(\msg)$ and 
%	$c = H_p(\aux,\msg,\compk,\PreOut,R,R_m)$.
%    Finally if $h \, R = h \, (s_1 \genG + s_2 \genB - c \, \compk)$ and
%    and $h \, R_m = h \, (s_1 \In - c \, \PreOut)$ both hold,
%	% First parse $\sigma = (\PreOut, \pi_{eval} =(c,s_1,s_2))$,
%	% recompute $\In := H_{\grEsis}(\msg)$, 
%    % $R = s_1 \genG + s_2 \genB - c \, \compk$, and
%    % $R_m = s_1 \In - c \, \PreOut$.
%    % Finally if $c = H_p(\aux,\msg,\compk,\PreOut,R,R_m)$ holds,
%    then return $H(\msg, \hsis\,\PreOut)$, which equals $\PedVRF.\Eval(\sk,\msg)$,
%    or return failure $\perp$ otherwise.
%\end{itemize}



%\noindent We described the deterministically batchable flavor analogous
%to \cite{HdVBatchEd25519} because $s_2$ makes our signature large enough
%that half-aggregation makes sense, unlike EC VRF. %NOT CLEAR SENTENCE
The verifier in $ \PedVRF $ verifies that the secret key computed to generate $ \PreOut $ and the secret key used to generate $ \compk $ are the same. Therefore, $ H(\msg, \PreOut) $ is the correct evaluation value of $ \msg $ with this secret key since $ \PreOut $ is correct. In addition to this, they verify that $ \aux $
is signed by the same key since $ \pi $ functions akin to a  Schnorr-like signature. We note that $ \PedVRF $ is not a VRF due to the absence of a public key but it can be transformed into EC-VRF \cite{nsec5,VXEd25519,draft-irtf-cfrg-vrf-10} if the conditions $\openpk = r_2 = 0$ in \Sign and $ \pk = \sk G $ are imposed.
%but our public key handling in \PedVRF breaks VRF definitions somewhat. %NOT CLEAR SENTENCE

Now, we are ready to describe our first ring VRF construction.
\paragraph{The Ring VRF Construction:} The main building blocks of our construction are $ \PedVRF $, $ \NIZK_{\Rring}, \NARK_{\relcomring} $ (relations defined below) and two commitment schemes $ \com $ and $ \com^* $ which is a deterministic commitment scheme. 

\begin{itemize}
	\item $\rVRF.\KeyGen(\pprvrf)\rightarrow (\sk,\pk)$: It outputs as secret key $ \sk = (x,r) $ where $x,r \leftsample \F_p$ and $ \pk $ as public key where $ \pk = \mathsf{Com}.\mathsf{Commit}(x,r)  $. We deploy this key generation algorithm based on a commitment scheme to be consistent with the key generation algorithm of our second construction in  \S\ref{subsec:rvrf_faster} which necessitates commitment to $ \sk $ to run securely. However, we note that an alternative definition for $ \pk $ is possible, where $ \pk = \sk \genG $ and $ \sk = x $.
	
	\item $\rVRF.\Eval(\sk, \msg) $ runs $\PedVRF.\Eval(x,\msg)$. 
	\eprint{We remark that the evaluation value is generated with \emph{only} the first part of the secret key which is $ x $.}{}
	
	\item $ \rVRF.\CommitRing(\ring,\pk) \rightarrow (\comring,\openring)$: It runs $ \com^*.\commit(\ring) $  and obtains $ \comring $ as a deterministic commitment to $ \ring $. Then, it runs $ \NARK_{\relcomring}.\Prove(\comring, \pk; \ring) $ which outputs $ \picomring $. In the end, it outputs $ \openring = (\pk, \picomring) $.
	$$\rel_{\comring}= \{(\comring,\pk;\ring): \com^*.\commit(\ring) \rightarrow \comring \wedge \pk \in \ring\}$$
	%It computes a Merkle tree root $\comring  $ using $ H_\ring $ considering the elements of $ \ring $ as leaves and generates a  Merkle tree path $ \openring $ that verifies $ \pk \in \ring $.
	
	
	
	\item $ \rVRF.\OpenRing(\comring, \openring) \rightarrow (\pk \ \vee \perp) $: It runs  $ \NARK_{\relcomring}.\Verify(\comring, \openring) $ where $ \openring = (\pk,\picomring) $. If it verifies it outputs $ \pk $. Otherwise, it outputs $ \perp $. 
	
	Here, we deploy $ \NARK $ to show that committed ring contains  a given public key. 
	This enables us to instantiate the signing and verification algorithms of our  protocols without requiring full knowledge of the  ring. It is particularly crucial for applications that involve large-scale rings with millions of users.
%	We choose the ring commitment scheme so the $\rVRF.\OpenRing$ invocation
%	is relatively SNARK friendly in our ring membership relation $ \Rring $. We note that an alternative ring commitment scheme may be used 
%	where $ \comring = \ring $ and $ \openring = \pk $.
	%, however as we will see below, it incurs a high cost of $O(|\ring|)$ for the ring VRF verifier.
		
\eprint{\end{itemize}}{}

\eprint{\noindent The \Sign and \Verify for  our \rVRF are a combination of \Sign and \Verify from \PedVRF and
\Prove and \Verify from $\NIZK_{\Rring}$, as follows:}{}
\eprint{\begin{itemize}}{}
	
	\item $\rVRF.\rSign(\sk,\comring, \openring,\msg,\aux) \rightarrow \sigma$:
	It returns a ring VRF signature $\sigma = (\compk,\piring, \comring,\sigma')$.
  For this, it obtains $(\openpk,\compk) $ by running $ \PedVRF.\CommitKey(\sk)$ and runs  $ \NIZK_{\Rring}.\Prove(\compk,\comring; \openpk,\openring, \pk,\sk) \rightarrow \piring$, then obtains the Pedersen VRF signature  $\sigma' $ by running $ \PedVRF.\Sign(x,\openpk,\msg, \aux')$  where $\aux' \leftarrow \tmpaux$. Here, 
\eprint{$$ \Rring = \Setst{ (\compk,\comring ;\openpk,\openring,\sk = (x,r)) }{
		\begin{aligned}
			&	\pk = \OpenRing(\comring,\openring), \\
			& 	x = \mathsf{Com}.\mathsf{Open}(\pk,r), \\
			& 	\compk = x \genG + \openpk \genB
		\end{aligned}	
	}$$}{
	\begin{footnotesize}
	$$ \Rring = \Setst{ (\compk,\comring ;\openpk,\openring,\sk = (x,r)) }{
		\begin{aligned}
			&	\pk = \OpenRing(\comring,\openring), \\
			& 	x = \mathsf{Com}.\mathsf{Open}(\pk,r), \\
			& 	\compk = x \genG + \openpk \genB
		\end{aligned}	
	}$$
	\end{footnotesize}}

We note that if $ \pk = \sk\genG $ then $ \Rring $ does not need $ \sk $ as a part of its witness. In this case, we need to replace the last two conditions by $ \compk = \pk + \openpk \genB $.

	\item $\rVRF.\rVerify(\comring,\msg,\aux,\sigma) \rightarrow (1,\Out) \,\, \lor (0,\perp)$: It
	parses $\sigma$ as $(\compk,\piring, \comring, \sigma')$,  sets $\aux' \leftarrow \tmpaux$ and runs $\NIZK_{\Rring}.\Verify((\compk,\comring); \piring)$.
	If it fails, returns $ (0,\perp) $. Otherwise, returns $\PedVRF.\Verify(\compk,\msg, \aux', \sigma)$.
\end{itemize}


% BEGIN TODO: Oana

% % Although \PedVRF itself exhibits surprising properties, our gestalt 
% \rVRF satisfies sensible security definitions:
% Pseudo-randomness holds by reduction to singleton rings.
% Ring uniqueness, ring unforgeability, and ring anonymity resemble security
% arguments for other ring signatures built from SNARKs.

% \begin{proposition}\label{prop:rvrf_games}
	% $\rVRF$ satisfies ring uniqueness, ring unforgeability, and ring anonymity.
	% \end{proposition}

% END TODO: Oana

\eprint{We prove in Theorem \ref{thm:firstprotocol} that our first ring VRF construction realizes $ \fgvrf $ in Figure \ref{f:gvrf} but we want to give an intuition first why our scheme is secure. Intuitively, the randomness and the determinism of $ \rVRF.\Eval $ come from the random oracles $ H $ and $ H_{\grE} $.  The anonymity of our ring VRF signature $ (\sigma = (\compk, \pi_{ring}, \PreOut, \pi_{eval}, \comring)) $ comes from the perfect hiding property of Pedersen commitment i.e., $ \compk $ is independent from the signer's key, the zero-knowledge property of $ \NIZK_{\mathcal{R}_{ring}} $ and $ \NIZK_{\mathcal{R}_{eval}} $ and the difficulty of DDH in  $ \grE $ (Lemma \ref{lem:honestoutput}) so that $ \PreOut $ is indistinguishable from a random element in $ \grE $. The unforgeability and uniqueness come from the fact that CDH is hard in $ \grE $ (Lemma \ref{lem:simulation-ind}).}{}

\begin{algorithm}
	\eprint{}{\scriptsize}
	\caption{$\Gen_{sign}(\ring,\sk=(x,r),\pk,\aux,\msg)$}
	\label{alg:gensign}	 	
	\begin{algorithmic}[1]
			\State $ c,s_1, s_2 \leftsample \F_p $
			\State $ \pi_{eval}  \leftarrow (c,s_1, s_2)$
			\State $ \openpk \leftsample \F_p $
			\State $ \compk =  x \genG + \openpk \, K$
			%\State $ \pi_{eval} \leftarrow \NIZK.\mathsf{Simulate}(\rcom, (G, \genB,\GG,\compk,W,\In)) $
			%\State \textbf{send} $(\oramsg{learn\_\tau},\sid)  $ to $ \gcrs $
			%\State \textbf{receive} $(\oramsg{trapdoor},\sid, \tau,crs)  $ from $ \gcrs $
			\State $ \comring, \openring \leftarrow \rVRF.\CommitRing(\ring, \pk) $ \label{line:comring}
			\State $ \pi_{ring} \leftarrow \NIZK_{\Rring}.\Prove(\crs_{\Rring}, \pp_{\Rring}, \comring, \compk; \openpk, \openring, \sk) $ \label{line:piring}
			\State\Return$ \sigma = (\pi_{ring},\compk,\comring,\pi_{eval}) $
		\end{algorithmic}
	
\end{algorithm}
%IF BELOW WILL BE ADDED, REMOVE UC-SIMULATION TO THE APPENDIX
%\begin{theorem}\label{thm:rvrfmain}
%	$ \rVRF $  over  $ pp_{rvrf} $ realizes  $ \fgvrf $ running $ \Gen_{sign} $ in Algorithm \ref{alg:gensign} \cite{canetti1,canetti2} in the random oracle model assuming that $ \NIZK_{\mathcal{R}_{eval}} $ and $ \NIZK_{\mathcal{R}_{ring}}$ are zero-knowledge and knowledge sound, the decisional Diffie-Hellman (DDH) problem are hard in $ \grE  $ \eprint{(so the CDH problem is hard as well) }{} and the commitment scheme $ \mathsf{Com} $ is binding and perfectly hiding. 
%\end{theorem}
%
%
%\begin{algorithm}
%	\eprint{}{\scriptsize}
%	\caption{$\Gen_{sign}(\ring,W,x,\pk,\aux,\msg)$}
%	\label{alg:gensign}	 	
%	\begin{algorithmic}[1]
%		\State $ c,s_1, s_2 \leftsample \F_p $
%		\State $ \pi_{eval}  \leftarrow (c,s_1, s_2)$
%		\State $ \openpk \leftsample \F_p $
%		\State $ \compk =  x \genG + \openpk \, K$
%		%\State $ \pi_{eval} \leftarrow \NIZK.\mathsf{Simulate}(\rcom, (G, \genB,\GG,\compk,W,\In)) $
%		%\State \textbf{send} $(\oramsg{learn\_\tau},\sid)  $ to $ \gcrs $
%		%\State \textbf{receive} $(\oramsg{trapdoor},\sid, \tau,crs)  $ from $ \gcrs $
%		\State $ \comring, \openring \leftarrow \rVRF.\CommitRing(\ring, \pk) $
%		\State $ \pi_{ring} \leftarrow \NIZK_{\Rring}.\Prove(crs, \comring, \compk\; \openpk, \openring, x) $ 
%		\State\Return$ \sigma = (\pi_{eval},\pi_{ring},\compk,\comring,W) $
%	\end{algorithmic}
%	
%\end{algorithm}
%We have the security proof of Theorem \ref{thm:rvrfmain} in Appendix \ref{ap:ucproof}. We give a proof sketch here.
%
%\noindent \textit{Proof Sketch:} We first verify that $ \fgvrf $ running $ \Gen_{sign} $ gives the anonymity defined in Definition \ref{def:anonymity} (See Lemma \ref{lem:anonymity}). \eprint{We can show it by reducing the anonymity game in another game where we  replace   $\NIZK_{\Rring}.\Prove(\comring, \compk\; \openpk, \openring, x) $
%by $ \NIZK_{\Rring}.\Simulate(\comring,\compk) $ in $ \Gen_{sign} $. Since the signature generated in this reduction is independent from honest keys thanks to perfectly hiding $ \compk $, the signatures generated by $ \Gen_{sign} $ are indistinguishable.}{Intuitively, it does not reveal the signer because $ \NIZK_{\Rring} $ is ZK and Pedersen commitment is perfectly hiding.}
%
%After making sure that $ \fgvrf $ running $ \Gen_{sign} $ satisfies all security properties, we show that our protocol $ \rVRF $ realizes $ \fgvrf $. So, we construct a simulator $ \simulator $ that simulates the honest parties in the real protocol $ \rVRF $ and simulates the adversary in the ideal world with $ \fgvrf $. $ \simulator $ simulates the random oracles against adversaries in the real protocol. It simulates $H_\grE $ so that it knows the discrete logarithm of each $ H_\grE $ output. This trapdoor helps $ \simulator $ to learn the public key of the adversary if the adversary ever comes to the random oracle $ H $ to compute the evaluation value of $ \msg $. When it happens, $ \simulator $ asks $ \fgvrf $ to evaluate $ \msg $ for this public key and replies the evaluation value provided by $ \fgvrf $ as a random oracle $ H $. Thus, $ \simulator $ makes the evaluation values of real world malicious parties be consistent with the evaluation values of ideal world malicious parties. Whenever $ \fgvrf $ sends a message $ (\oramsg{verify}, \sid, \ring, W, \aux,\msg, \sigma) $ to $ \simulator $, $ \simulator $ runs $ \rVRF.\Verify(\comring, \msg,\aux,\sigma) \rightarrow (b_\simulator,.)$ and sends a response including $ b_\simulator $. If $ \fgvrf $ replies with a decision $ b \neq b_\simulator $, $ \simulator $ aborts because of the inconsistency. Then, we  make sure that our simulation is indistinguishable. We show this in Lemma \ref{lem:honestoutput} and \ref{lem:simulation-ind} in Appendix \ref{ap:ucproof}:
%We  show that the indistinguishability of outputs in Lemma \ref{lem:honestoutput}. Differently than ideal honest parties, the honest parties in our  protocol have $ \PreOut $ as a part of their signature. The ideal world honest parties has $ W $ instead of it which is sampled randomly by $ \fgvrf $ instead of it. We can verify by the DDH assumption that $ W $ and $ \PreOut $ are indistinguishable.
%We also show that $ \simulator $ does not abort during the simulation except with negligible probability in Lemma \ref{lem:simulation-ind} in Appendix \ref{ap:ucproof} under the assumption of the hardness of the CDH problem. 
%Indeed, we show that if there exists an adversary $ \adv $ which makes $ \simulator $ abort during the simulation, then we construct another adversary $ \mathcal{B} $ which breaks either the CDH problem or the binding property of $ \mathsf{Com}$. 


 

% NOTE:  Is this redundant after the above paragraph?
% The security proof of Theorem \ref{thm:rvrfmain} is in Appendix \ref{ap:ucproof}.



\endinput
