\section{Ring VRFs from the Pedersen VRF}
\label{sec:pederson_vrf}

An EC VRF like \cite{nsec5,VXEd25519,draft-irtf-cfrg-vrf-10} consists of
a verifiable unique function (VUF) given by a Chaum-Pedersen DLEQ proof
between the signer's public key $\pk = \sk \, \genG$ and
 the VUF output $\PreOut = \sk H_{\grE}(\msg)$,
after which a PRF \eprint{evaluation}{} yields a VRF output
 $\Out = H'(\msg, h \, \PreOut)$ ala \cite[Proposition 1]{vrf_micali}.

We build efficient ring VRFs from a construction we call the
{\em Pedersen VRF} \PedVRF, which alters the EC VRF by replacing the
public key by a Pedersen commitment $\pk + \openpk \, \genB$,
We claim \PedVRF instantiates the $\Reval$ NIZK from \S\ref{sec:overview}.
\footnote{As Groth16 dominates ring VRF verification costs,
we describe only the non-batchable variant analogous to
\cite{nsec5,VXEd25519,draft-irtf-cfrg-vrf-10}, but
 batch verifiable flavors exist.}

% We define security for only our ring VRF constructions, but clearly
%  \PedVRF consists of algorithms having superficially similar signatures.
% \footnote{We do not define security for \PedVRF because pseudo-randomness becomes too interesting}

We select a base point $\genG$ for our public key by any desired method,
but then fix a second generator $\genB$ of $\grE$ independent from $\genG$.
%
We define \KeyGen exactly like EC VRF, but
 \Eval differs by not injecting \pk into \msg:
\begin{itemize}
\item $\PedVRF.\KeyGen$ \quad returns $\sk \leftsample \F_p$ and $\pk = \sk \, \genG$.
\item $\PedVRF.\Eval : (\sk,\msg) \mapsto H'(\msg, h \, \sk \, H_{\grE}(\msg))$
\end{itemize}
% \item $\PedVRF.\KeyGen$ selects a secret key \sk uniformly at random from $\F_p$ and computes the public key $\pk = \sk \, \genG$. 
% \item $\PedVRF.\Eval(\sk,\msg)$ takes a secret key \sk and an input $\msg$, and
%  then returns a VRF output $H'(\msg, h \, \sk \, H_{\grE}(\msg))$.

\noindent We form Pedersen-like commitments to this public key \pk:
\begin{itemize}
\item $\PedVRF.\CommitKey$ \,
returns a blinding factor $\openpk \leftsample \F_p$
and a commitment $\compk = \pk + \openpk \, \genB$.
\item $\PedVRF.\OpenKey$ \,
returns $\pk = \compk - \openpk \, \genB$.
\end{itemize}
% \item $\PedVRF.\CommitKey$ selects a blinding factor $\openpk$ uniformly
%  at random from $\F_p$ and computes the commitment $\compk = \pk + \openpk \, \genB$.
% \item $\PedVRF.\OpenKey$ just returns $\pk = \compk - \openpk \, \genB$.
Alone these hide \pk, but they only provide a binding commitment
provided that $\PedVRF.\Verify$ below succeeds too.

\begin{itemize}
\item $\PedVRF.\Sign : (\sk,\openpk,\msg,\aux) \mapsto \sigma$ \,
    % takes a secret key \sk and blinding factor \openpk, an input $\msg$, and auxiliary data \aux, and then performs
    computes $\In := H_{\grE}(\msg)$ and $\PreOut := \sk \, \In$,
    samples $r_1,r_2 \leftsample \F_p$,
    computes $R = r_1 \genG + r_2 \genB$, and $R_\msg = r_1 \In$, and
    finally $c = H_p(\aux,\msg,\compk,\PreOut,R,R_m)$,
     along with $s_1 = r_1 + c \, \sk$ and $s_2 = r_2 + c \, \openpk$.
    and finally returns the signature $\sigma = (\PreOut,c,s_1,s_2)$.
% \begin{enumerate}
%    \item compute the VRF input point $\In := H_{\grE}(\msg)$ and pre-output $\PreOut := \sk \, \In$,
%    \item Sample random $r_1,r_2 \leftarrow \F_p$ and compute $R = r_1 \genG + r_2 \genB$ and $R_\msg = r_1 \In$.
%    \item Compute the challenge $c = H_p(\aux,\msg,\compk,\PreOut,R,R_m)$,
%     along with $s_1 = r_1 + c \sk$ and $s_2 = r_2 + c \, \openpk$.
%    \item Return the signature $\sigma = (\PreOut,c,s_1,s_2)$.
% \end{enumerate}
\item $\PedVRF.\Verify : (\compk,\msg,\aux,\sigma) \mapsto \Out \,\, \lor \perp$ \,
    parses $\sigma = (\PreOut,c,s_1,s_2)$, 
    recomputes $\In := H_{\grE}(\msg)$, as well as
    $R = s_1 \genG + s_2 \genB - c \, \compk$, and
    $R_m = s_1 \In - c \PreOut$, and finally
    if $c = H_p(\aux,\msg,\compk,\PreOut,R,R_\msg)$ then it return $H'(\msg, h \, \PreOut)$, 
         or failure $\perp$ otherwise.
% \begin{enumerate}
%    \item recompute the VRF input point $\In := H_{\grE}(\msg)$,
%    \item computes $R = s_1 \genG + s_2 \genB - c \, \compk$ and $R_m = s_1 \In - c \PreOut$, and
%    \item returns $H'(\msg, h \, \PreOut)$ if $c = H_p(\aux,\msg,\compk,\PreOut,R,R_\msg)$ or failure otherwise.
% \end{enumerate}
\end{itemize}

\noindent We obtain EC VRF if we choose $\openpk = 0 = r_2$ in \Sign and demand $s_2 = 0$ in \Verify.
%
\eprint{We define security only for our ring VRF constructions, but clearly
 \PedVRF consists of algorithms having superficially similar signatures.}{}
% \footnote{We do not define security for \PedVRF because pseudo-randomness becomes too interesting}

\smallskip
% \subsection{Pedersen rVRF-AD}

As described in \S\ref{sec:overview},
we instantiate a rVRF-AD from \PedVRF plus a ring commitment scheme
 $\rVRF.\{ \CommitRing, \OpenRing \}$.
\rVRF inherits $\rVRF.\KeyGen = \PedVRF.\KeyGen$ and
 $\rVRF.\Eval = \PedVRF.\Eval$ of course, but requires.
We need zero-knowledge ring membership proof for the relation \Rring
which handles both $\PedVRF.\OpenKey$ and $\rVRF.\OpenKey$ efficiently.
% \vspace{-0.1in}
$$ \Rring = \Setst{ \compk, \comring }{
    \eprint{ \exists \openpk,\openring \textrm{\ s.t.\ } }{}
    \genfrac{}{}{0pt}{}{\PedVRF.\OpenKey(\compk,\openpk) \quad}{\,\, = \rVRF.\OpenRing(\comring,\openring)}
} \mathperiod $$

\begin{itemize}
\item $\rVRF.\rSign : (\sk,\openring,\msg,\aux) \mapsto \rho$ \,
 % takes a secret key \sk, a ring opening \openring, a message \msg, and \aux, and then % auxiliary data
 samples $\openpk \leftsample \F_p$ and
 % computes a proof $\piring$, a signature $\sigma$, and
 returns $\rho = (\compk,\piring,\sigma)$ where      % a ring VRF signature
 $$ \piring = \NIZK.\Prove(\Rring,\compk,\comring,\openpk,\openring) \quad\textrm{and} $$
 $$ \sigma = \PedVRF.\Sign(\sk,\openpk,\msg,\aux \doubleplus \compk \doubleplus \piring \doubleplus \comring) \mathperiod $$ % finally
\item $\rVRF.\rVerify : (\comring,\msg,\aux,\rho) \mapsto \Out \,\, \lor \perp$ \,
 parses $\rho$ as $(\compk,\piring,\sigma,)$ and returns
 $$ \PedVRF.\Verify(\compk,\msg,\aux \doubleplus \compk \doubleplus \piring \doubleplus \comring,\sigma) $$
 if $\NIZK.\Verify(\piring,\compk,\comring)$ succeeds too.
\end{itemize}

Although \PedVRF itself exhibits surprising properties, our gestalt \rVRF
satisfies sensible security definitions:
\rVRF satisfies pseudo-randomness by reduction to singleton rings.
Ring uniqueness, ring unforgeability, and ring anonymity resemble security
arguments for other ring signature involving SNARKs.
%
% \begin{proposition}\label{prop:rvrf_games}
% $\rVRF$ satisfies ring uniqueness, ring unforgeability, and ring anonymity.
% \end{proposition}
%
Appendix \ref{ap:ucproof} contains a security proof in our UC model.

\begin{theorem}\label{thm:rvrf}
$ \rVRF $  over the group structure $ (\grE,p,\genG,\genB) $ realizes $ \fgvrf $ in Figure \ref{f:gvrf} in the random oracle model assuming that NIZK is zero-knowledge and knowledge extractable, the decisional Diffie-Hellman (DDH) problem are hard in $ \grE  $. 
\end{theorem}






\endinput





TODO:  Eprint form?

\begin{itemize}
\item $\rVRF.\rSign : (\sk,\openring,\msg,\aux) \mapsto \sigma$ takes
 a secret key \sk, a ring opening \openring, a message \msg, and \aux, and then % auxiliary data
 % \begin{enumerate}
 % \item
 generates \openpk, computes a ring membership proof $\piring$
  $$ \piring = \NIZK \Setst{ \compk, \comring }{
  \exists \openpk,\openring \textrm{\ s.t.\ } 
  \genfrac{}{}{0pt}{}{\PedVRF.\OpenKey(\compk,\openpk) \quad}{\,\, = \rVRF.\OpenKey(\comring,\openring)}
  } $$
 % \item
 computes the signature
  $$ \sigma = \PedVRF.\Sign(\sk,\openpk,\msg,\aux \doubleplus \compk \doubleplus \piring \doubleplus \comring), \quad\textrm{and} $$ % finally
 % \item
 returns the ring VRF signature $\rho = (\compk,\piring,\sigma)$.
 % \end{enumerate}
\item $\rVRF.\rVerify$ takes $(\comring,\msg,\aux,\rho)$,
 parses $\rho$ as $(\compk,\piring,\sigma,)$,  and then returns
 $$ \PedVRF.\Verify(\compk,\msg,\aux \doubleplus \compk \doubleplus \piring \doubleplus \comring,\sigma) $$
 iff $\NIZK.\Verify(\piring,\compk,\comring)$ succeeds. 
\end{itemize}























\begin{lemma}\label{prop:pedersen_vrf_hiding}
$\PedVRF$ is a correct key commitment and key hiding.
\end{lemma}

Although Pedersen commitments are perfectly hiding, our $\R_\msg$ makes $\sigma$ only computationally hiding.

\begin{proposition}\label{prop:pedersen_vrf}
Assuming AGM in $\grE$, % $\ecE$ modulo $h$,
our $\PedVRF$ satisfies VRF correctness, key binding, uniqueness,
pseudo-randomness, and unforgeability. % (EUF-CMA-KC) on $(\msg,\aux)$.
\end{proposition}

We need this second verification equation in \PedVRF, but not in \ThinVRF,
because otherwise our $s_2 \genB$ term provides enough freedom to tamper
with the pre-outputs.  

We could however generalize \PedVRF to $k$ messages $\msg_1,\ldots,\msg_k$
similarly to \ThinVRF in \S\ref{subsec:vrf_thin}:  We compute for
$j=1,\ldots,k$ the $k$ distinct
points $\In_j := H_{\grE}(\msg_j)$, pre-outputs $\PreOut := \sk \, \In$,
delinearization challenges
 $c_j = H_p(\aux,\msg_1,\ldots,\msg_k,\compk,\Out_{0,1},\ldots,\Out_{0,k},j)$,
and then use the \PedVRF proof for
 $\In = \sum_j c_j \In_j$ and $\Out = \sum_j c_j \Out_j$.

% TODO: Proof correct?  Use same citations as schnorrkel.







