\section{Ring VRF construction}% {Ring VRFs from the Pedersen VRF}
\label{sec:pederson_vrf}

In this section, we introduce our ring VRF construction which has efficient instantiations. 
We instantiate parameter generation by constructing a group $\grE$ of order $ p $ and two generators $ \genG, \genB \in  \grE$.  We consider three hash functions: $ H, H_p: \{0,1\}^* \rightarrow \F_p $ and a hash-to-group function $H_{\grE} : \{0,1\}^* \rightarrow \grE$.
The first key ingredient of our protocol is a VRF that we  call the Pedersen VRF and denote \PedVRF. Therefore, we first introduce it:

\paragraph{Pedersen VRF:} We construct $ \PedVRF $ inspired from EC VRF  \cite{nsec5,VXEd25519,draft-irtf-cfrg-vrf-10}.
%If we really want to have the following explanation below, I think it should come after we explain how PedVRF works. Otherwise, it is not easy to understand it unless you know closely how EC VRF works. 
%An EC VRF  
%consists of first a verifiable unique function (VUF) given by a
%Chaum-Pedersen DLEQ proof between the signer's public key $\pk = \sk \, \genG$ and
% the VUF output $\PreOut = \sk H_{\grE}(\msg)$,
%after which a PRF \eprint{evaluation}{} yields a VRF output
% $\Out = H'(\msg, h \, \PreOut)$ ala \cite[Proposition 1]{vrf_micali}.
%
Our \PedVRF alters the EC VRF by replacing the public key by
a Pedersen commitment $\pk + \openpk \, \genB$ to the secret key \sk.  In more detail, it works as follows:

% We define security for only our ring VRF constructions, but clearly
%  \PedVRF consists of algorithms having superficially similar signatures.
% \footnote{We do not define security for \PedVRF because pseudo-randomness becomes too interesting}

%We refer readers to \S\ref{sec:ec_background} for notation,
% but highlight several points: 
% We shall need $\ecE$ to be pairing friendly in later sections, but ignore this here.
%
%COMMENT: I put these in the end. 
%Although non-essential, we prefer if \PreOut cannot be verified
%by pairings, as a miss-use resistance measure.
%We therefore suggest a ``sister'' Edwards curve $\ecE'$ with
% a subgroup $\grE'$ of the same order $p$ as $\grE_1$,
% and cofactor $h'$ divisible by 4.
%You may take $\ecE' = \ecE_1$ and $h' = 1$ if reading only for
%security proofs, but this incurs risks in deployments.
%%
%As Groth16 dominates our ring VRF verification costs, we describe only
%the non-batchable \PedVRF analogous to \cite{nsec5,VXEd25519,draft-irtf-cfrg-vrf-10},
%and ignore the cofactor $h_1$ of $\grE_1$ of $\ecE_1$, 
%but caution \cite{HdVBatchEd25519} impacts batch verifiable flavors.
%%
%We select a generator $\genG$ of $\grE$ as our public key base point
% by any desired method, but then
%fix a second generator $\genB$ of $\grE$ independent from $\genG$.

%We now describe the Pedersen VRF \PedVRF evaluation proof used by our ring VRF:
%

\begin{itemize}
\item $\PedVRF.\KeyGen$ \, returns $\sk \leftsample \F_p$ and $\pk = \sk \, \genG$.
We note that $ \pk $ has no role in this construction because we do not want to reveal it in order to obtain the anonymity property in our ring VRF construction. % % COMMENT Handan:  I guess it is true because PedVRF.Sign just shows that the eveluation value is generated with the secret key which is committed in compk

\item $\PedVRF.\Eval : (\sk,\msg) \mapsto H(\msg, \PreOut)$ where $ \PreOut =  \sk \, H_{\grE}(\msg)$
\end{itemize}
% \item $\PedVRF.\KeyGen$ selects a secret key \sk uniformly at random from $\F_p$ and computes the public key $\pk = \sk \, \genG$. 
% \item $\PedVRF.\Eval(\sk,\msg)$ takes a secret key \sk and an input $\msg$, and
%  then returns a VRF output $H'(\msg, h' \, \sk \, H_{\grE}(\msg))$.



We remark that \KeyGen works exactly like EC VRF, but \Eval differs by not including \pk as an input of $ H $.

 We add two new algorithms that differ from regular VRF constructions.
These handle a Pedersen commitment to the secret key \sk,
 together with its opening \openpk.
\begin{itemize}
\item $\PedVRF.\CommitKey(\sk)$ \, %I changed it because we commit to sk
returns a blinding factor $\openpk \leftsample \F_p$
and a commitment $\compk = \sk \genG + \openpk \, \genB$.
\item $\PedVRF.\OpenKey(\compk,\openpk)$ \,
returns $\pk = \compk - \openpk \, \genB$.
\end{itemize}
% \item $\PedVRF.\CommitKey$ selects a blinding factor $\openpk$ uniformly
%  at random from $\F_p$ and computes the commitment $\compk = \pk + \openpk \, \genB$.
% \item $\PedVRF.\OpenKey$ just returns $\pk = \compk - \openpk \, \genB$.

%These hide their public key argument \pk too, but alone do not bind \pk.
%A successful $\PedVRF.\Verify$ below proves knowledge of \compk,
 %which makes the commitment binding. % COMMENT Handan: I don't think this statement is true because Sign does not bind to pk. because there is nothing here that shows that the committed secret key multiplied by G is pk.
Next, we describe how to generate a proof for the correct evaluation of an input i.e., the honest evaluation value is generated using the committed secret key.


\begin{itemize}
\item $\PedVRF.\Sign : (\sk,\openpk,\msg,\aux) \mapsto \sigma$ \,
    % takes a secret key \sk and blinding factor \openpk, an input $\msg$, and associated data \aux, and then performs
    It generates a Chaum-Pedersen DLEQ proof  showing the following relation by running the algorithm $ \NIZK_{\mathcal{R}_{eval}}.\Prove(((\genG, \genB,\grE,\compk,\PreOut,\In); (\sk, \openpk))) $ which outputs $ \pi_{eval}$ and returns a signature $ \sigma = (\PreOut,\compk,\pi_{eval}) $
    \begin{align}
    	\mathcal{R}_{eval}= \{((\sk, \openpk); (\genG, \genB,\grE,\compk,\PreOut,\In)): \nonumber\\
    	\compk = \sk\genG + \openpk\, \genB, \PreOut = \sk \,\In \}  \label{rel:commit} 
    \end{align}
    $ \NIZK_{\mathcal{R}_{eval}}.\Prove $ first computes $\In := H_{\grE}(\msg)$ and $\PreOut := \sk \, \In$.
    Next samples $r_1,r_2 \leftsample \F_p$,
    computes $R = r_1 \genG + r_2 \genB$, and $R_m = r_1 \In$,
     $c = H_p(\aux,\msg,\compk,\PreOut,R,R_m)$,
     along with $s_1 = r_1 + c \, \sk$ and $s_2 = r_2 + c \, \openpk$.
    and finally returns the proof $\pi_{eval} = (c,s_1,s_2)$.
    %TODO We might say something about ass here
% \begin{enumerate}
%    \item compute the VRF input point $\In := H_{\grE}(\msg)$ and pre-output $\PreOut := \sk \, \In$,
%    \item Sample random $r_1,r_2 \leftarrow \F_p$ and compute $R = r_1 \genG + r_2 \genB$ and $R_m = r_1 \In$.
%    \item Compute the challenge $c = H_p(\aux,\msg,\compk,\PreOut,R,R_m)$,
%     along with $s_1 = r_1 + c \sk$ and $s_2 = r_2 + c \, \openpk$.
%    \item Return the signature $\sigma = (\PreOut,c,s_1,s_2)$.
% \end{enumerate}
\item $\PedVRF.\Verify : (\compk,\msg,\aux,\sigma) \mapsto \Out \,\, \lor \perp$ It 
    parses $\sigma = (\PreOut, \pi_{eval} =(c,s_1,s_2))$ and runs $ \NIZK_{\mathcal{R}_{eval}}.\Verify(\genG, \genB,\grE,\compk,\PreOut,\In, \pi_{eval} ) $ i.e.,\, 
    recomputes $\In := H_{\grE}(\msg)$, as well as
    $R = s_1 \genG + s_2 \genB - c \, \compk$, and
    $R_m = s_1 \In - c \, \PreOut$, and finally
    if $c = H_p(\aux,\msg,\compk,\PreOut,R,R_m)$ then it returns $H(\msg, h \, \PreOut)$, which equals $\PedVRF.\Eval(\sk,\msg)$,
         or else returns failure $\perp$ otherwise.
% \begin{enumerate}
%    \item recompute the VRF input point $\In := H_{\grE}(\msg)$,
%    \item computes $R = s_1 \genG + s_2 \genB - c \, \compk$ and $R_m = s_1 \In - c \, \PreOut$, and
%    \item returns $H'(\msg, h' \, \PreOut)$ if $c = H_p(\aux,\msg,\compk,\PreOut,R,R_M)$ or failure otherwise.
% \end{enumerate}
\end{itemize}

\noindent We remark that \PedVRF becomes EC VRF if we demand $\openpk = 0 = r_2$ in \PedVRF.\Sign.

%A signature $\sigma = (c,s_1, s_2)$ is a Fiat-Shamir transform of a
%sigma protocol that shows equality of the discrete logarithm of $\PreOut$
%in base $\In$ and the committed secret key in $\compk$.
%COMMENT Handan: We should either explain the following statement more or remove it
%We nevertheless define security only for our ring VRF constructions,
% as \PedVRF itself exhibits surprising properties.
% \footnote{We do not define security for \PedVRF because pseudo-randomness becomes too interesting}

\smallskip
% \subsection{Pedersen rVRF-AD}

Now, we are ready to introduce our ring VRF construction  $ \rVRF $.

\begin{itemize}
	\item $\rVRF.\KeyGen $ is the same as $ \PedVRF.\KeyGen$.
	\item $\rVRF.\Eval $ is the same as $ \PedVRF.\Eval$.
	\item $ \rVRF.\CommitRing: \pkring, \pk \rightarrow \comring, \openring$ given a set of public keys $ \pkring $, it returns a Merkle tree root $ \comring $ where the leaves are elements of  $ \pkring$ and Merkle tree proof $ \openring $ showing that $ \pk \in \pkring $. 
	
	We note that our construction preserves its security properties when replacing $ \comring $
	by $ \ring $. However, by using $ \comring $, we clearly improve the efficiency of our protocol
	%COMMENT Handan: We can improve the previous sentence.
	
	%COMMENT Handan: We don't need to define opring here because our construction does not need it.
\end{itemize}

%COMMENT Handan: It is better to talk about these stuffs after describing the protocol.
%As described in \S\ref{sec:overview},
%we instantiate a rVRF-AD from \PedVRF plus a ring commitment scheme
% $\rVRF.\{ \CommitRing, \OpenRing \}$.
%In our construction, $\rVRF.\Eval = \PedVRF.\Eval$.
%and $\rVRF.\KeyGen = \PedVRF.\KeyGen$.
%We alter $\rVRF.\KeyGen$ somewhat in \S\ref{subsec:rvrf_faster}
%by adopting a SNARK friendly public key,
% although the \pk here still exists under the hood.
%We discuss further nuances of this in \S\ref{subsec:AML_KYC}.

	As mentioned in \S\ref{sec:overview}, the signing algorithm of our ring VRF construction shows that the public key of the committed secret key is in the ring. Therefore, we need a zero-knowledge ring membership proof for relation $ \mathcal{R}_{ring} $; this is provided by $ \rVRF.\Sign $ that we define below.
%	 for the relation \Lring
%	which handles both $\PedVRF.\OpenKey$ and $\rVRF.\OpenKey$ efficiently.
%	% \vspace{-0.1in}
%	$$ \Lring = \Setst{ \compk, \comring }{
%		\eprint{ \exists \openpk,\openring \textrm{\ s.t.\ } }{}
%		\genfrac{}{}{0pt}{}{\PedVRF.\OpenKey(\compk,\openpk) \quad}{\,\, = \rVRF.\OpenRing(\comring,\openring)}
%	} \mathperiod $$
%	
	
	\def\tmpaux{\aux \doubleplus \compk \doubleplus \piring \doubleplus \comring}
	\def\tmpeprintaux{\eprint{\aux'}{\tmpaux}}
	\def\tmpindent{\hspace*{5pt}}
	\begin{itemize}
		\item $\rVRF.\rSign\eprint{ : }{}(\sk,\comring, \openring,\msg,\aux)\eprint{ \mapsto \rho}{}$ returns $\rho = (\compk,\piring,\sigma)$ if $ \openring $ is correct opening of $ \comring $. Here, 
		\tmpindent $(\openpk,\compk) \leftarrow \PedVRF.\CommitKey(\sk)$,  \\
		\tmpindent $\piring \leftarrow \NIZK_{\mathcal{R}_{ring}}.\Prove((\compk,\comring); \openpk)$, \\
		where 
		\begin{align}
				\mathcal{R}_{ring} = \{(\openpk;(\genG,\genB,\grE,\ring, \compk)): (\compk-\openpk \, \genB)  \in \ring\} \label{rel:snark} \nonumber
			\end{align}
		\tmpindent $\aux' \leftarrow \tmpaux$,  \\
		\tmpindent $\sigma \leftarrow \PedVRF.\Sign(\sk,\openpk,\msg, \aux')$.
		\item $\rVRF.\rVerify : (\comring,\msg,\aux,\rho) \mapsto \Out \,\, \lor \perp$ \,
		parses $\rho$ as $(\compk,\piring,\sigma)$, \eprint{next }{}sets $\aux' \leftarrow \tmpaux$,
		aborts if $\NIZK_{\mathcal{R}_{ring}}.\Verify((\compk,\comring); \piring)$ fails,
		and returns $\PedVRF.\Verify(\compk,\msg, \aux', \sigma)$.
	\end{itemize}
	



% BEGIN TODO: Oana

% % Although \PedVRF itself exhibits surprising properties, our gestalt 
% \rVRF satisfies sensible security definitions:
% Pseudo-randomness holds by reduction to singleton rings.
% Ring uniqueness, ring unforgeability, and ring anonymity resemble security
% arguments for other ring signatures built from SNARKs.

% \begin{proposition}\label{prop:rvrf_games}
% $\rVRF$ satisfies ring uniqueness, ring unforgeability, and ring anonymity.
% \end{proposition}

% END TODO: Oana

Appendix \ref{ap:ucproof} proves our ring VRF construction realizes $ \fgvrf $ in Figure \ref{f:gvrf}. Intuitively, the randomness and the determinism of $ \rVRF.\Eval $ come from the random oracles $ H' $ and $ H_{\grE'} $.  The anonymity of our ring VRF signature comes from the perfect hiding property of Pedersen commitment, the zero-knowledge property of $ \NIZK_{\mathcal{R}_{ring}} $ (Lemma \ref{lem:anonymity}) and the difficulty of DDH in  $ \grE $ (Lemma \ref{lem:honestoutput}) so that $ \PreOut $ is indistinguishable from a random element in $ \grE $. The unforgeability and uniqueness come from the fact that CDH is hard in $ \grE $ (Lemma \ref{lem:simulation-ind}), i.e., for unforgeability,  one cannot commit an honest party's secret key without breaking the CDH problem and for the uniqueness,  if one can obtain $ \PedVRF $ signatures such that $ \sigma_1 = (\PreOut_1, \pi_{\PedVRF}) $ and $ \sigma_2 = (\PreOut_2, \pi'_{\PedVRF}) $ where  $ \PreOut_1 \neq \PreOut_2 $  and verified by \compk for the message \msg, then we break a CDH problem in $ \grE $.

\begin{theorem}\label{thm:rvrfmain}
$ \rVRF $  over the group structure $ (\grE,p,\genG,\genB) $ realizes $ \fgvrf $ in Figure \ref{f:gvrf} in the random oracle model assuming that $ \NIZK_{\mathcal{R}_{eval}} $ and $ \NIZK_{\mathcal{R}_{ring}}$ are zero-knowledge and knowledge sound, the decisional Diffie-Hellman (DDH) problem are hard in $ \grE  $. 
\end{theorem}

% NOTE:  Is this redundant after the above paragraph?
% The security proof of Theorem \ref{thm:rvrfmain} is in Appendix \ref{ap:ucproof}.


\paragraph{Notes on Implementation:} Although non-essential, we prefer if \PreOut cannot be verified
by pairings, as a miss-use resistance measure.
We therefore suggest a ``sister'' Edwards curve $\ecE'$ with
a subgroup $\grE'$ of the same order $p$ as $\grE_1$,
and cofactor $h'$ divisible by 4.
You may take $\ecE' = \ecE_1$ and $h' = 1$ if reading only for
security proofs, but this incurs risks in deployments.
%
As Groth16 dominates our ring VRF verification costs, we describe only
the non-batchable \PedVRF analogous to \cite{nsec5,VXEd25519,draft-irtf-cfrg-vrf-10},
and ignore the cofactor $h_1$ of $\grE_1$ of $\ecE_1$, 
but caution \cite{HdVBatchEd25519} impacts batch verifiable flavors.
%
We select a generator $\genG$ of $\grE$ as our public key base point
by any desired method, but then
fix a second generator $\genB$ of $\grE$ independent from $\genG$.



\endinput





% TODO:  Eprint form?

\begin{itemize}
\item $\rVRF.\rSign : (\sk,\openring,\msg,\aux) \mapsto \sigma$ takes
 a secret key \sk, a ring opening \openring, a message \msg, and \aux, and then % associated data
 % \begin{enumerate}
 % \item
 generates \openpk, computes a ring membership proof $\piring$
  $$ \piring = \NIZK \Setst{ \compk, \comring }{
  \exists \openpk,\openring \textrm{\ s.t.\ } 
  \genfrac{}{}{0pt}{}{\PedVRF.\OpenKey(\compk,\openpk) \quad}{\,\, = \rVRF.\OpenKey(\comring,\openring)}
  } $$
 % \item
 computes the signature
  $$ \sigma = \PedVRF.\Sign(\sk,\openpk,\msg,\aux \doubleplus \compk \doubleplus \piring \doubleplus \comring), \quad\textrm{and} $$ % finally
 % \item
 returns the ring VRF signature $\rho = (\compk,\piring,\sigma)$.
 % \end{enumerate}
\item $\rVRF.\rVerify$ takes $(\comring,\msg,\aux,\rho)$,
 parses $\rho$ as $(\compk,\piring,\sigma,)$,  and then returns
 $$ \PedVRF.\Verify(\compk,\msg,\aux \doubleplus \compk \doubleplus \piring \doubleplus \comring,\sigma) $$
 iff $\NIZK.\Verify(\piring,\compk,\comring)$ succeeds. 
\end{itemize}

