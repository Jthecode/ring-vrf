\section{Pedersen VRF and rVRF}
\label{sec:pederson_vrf}

An EC VRF like \cite{nsec5,VXEd25519,draft-irtf-cfrg-vrf-10} consists
of a Chaum-Pedersen DLEQ proof between the signer's public key
$\pk = \sk \genG$ and a VUF output $\PreOut = \sk H_{\grE}(\msg)$,
so applying a PRF yields a VRF output
 $\Out = H'(\msg, h \PreOut)$ ala \cite[Proposition 1]{vrf_micali}.

Our {\em Pedersen VRF} \PedVRF alters this EC VRF by repalcing the
public key by a Pedersen commitment $\pk + \openpk \, \genB$, which
instantiates $\pieval$ from \S\ref{sec:overview} efficently.
\footnote{As Groth16 dominates ring VRF verification costs,
we describe only the non-batchable variant analogous to
\cite{nsec5,VXEd25519,draft-irtf-cfrg-vrf-10}, but
 batchable verifiable flavors exist.}

% We define security for only our ring VRF constructions, but clearly
%  \PedVRF consists of algorithms having superficially similar signatures.
% \footnote{We do not define security for \PedVRF because pseudo-randomness becomes too interesting}

We select a base point $\genG$ for our public key by any desired method,
but then fix a second generator $\genB$ of $\grE$ independent from $\genG$.
%
We define \KeyGen exactly like EC VRF, but
 \Eval differs by not injecting \pk into \msg:
\begin{itemize}
\item $\PedVRF.\KeyGen$ selects a secret key \sk uniformly at random from $\F_p$ and computes the public key $\pk = \sk \, \genG$. 
\item $\PedVRF.\Eval(\sk,\msg)$ takes a secret key \sk and an input $\msg$, and
 then returns a VRF output $H'(\msg, h \, \sk \, H_{\grE}(\msg))$.
\end{itemize}

We form Pedersen-like commitments to this public key \pk:
\begin{itemize}
\item $\PedVRF.\CommitKey$ selects a blinding factor $\openpk$ uniformly
 at random from $\F_p$ and computes the commitment $\compk = \pk + \openpk \, \genB$.
\item $\PedVRF.\OpenKey$ just returns $\pk = \compk - \openpk \, \genB$.
\end{itemize}
Alone these hide \pk, but they only provide a binding commitment
provided that \PedVRF.\Verify below succeeds too.

\begin{itemize}
\item $\PedVRF.\Sign(\sk,\openpk,\msg,\aux)$ takes a secret key \sk and blinding factor \openpk, an input $\msg$, and auxiliary data \aux, and then performs
\begin{enumerate}
    \item compute the VRF input point $\In := H_{\grE}(\msg)$ and pre-output $\PreOut := \sk \, \In$,
    \item Sample random $r_1,r_2 \leftarrow \F_p$ and compute $R = r_1 \genG + r_2 \genB$ and $R_\msg = r_1 \In$.
    \item Compute the challenge $c = H_q(\aux,\msg,\compk,\PreOut,R,R_m)$,
     along with $s_1 = r_1 + c \sk$ and $s_2 = r_2 + c \, \openpk$.
    \item Return the signature $(\PreOut,c,s_1,s_2)$.
\end{enumerate}
\item $\PedVRF.\Verify(\compk,\msg,\aux,\sigma)$, parses $\sigma = (\PreOut,c,s_1,s_2)$, and then 
\begin{enumerate}
    \item recompute the VRF input point $\In := H_{\grE}(\msg)$,
    \item computes $R = s_1 \genG + s_2 \genB - c \, \compk$ and $R_m = s_1 \In - c \PreOut$, and
    \item returns $H'(\msg, h \, \PreOut)$ if $c = H_q(\aux,\msg,\compk,\PreOut,R,R_\msg)$ or failure otherwise.
\end{enumerate}
\end{itemize}

We obtain EC VRF if we choose $\openpk = 0 = r_2$ in \Sign and demand $s_2 = 0$ in \Verify.

\smallskip
% \subsection{Pedersen rVRF-AD}

As described in \S\ref{sec:overview},
we instantiate a rVRF-AD from \PedVRF plus a ring commitment scheme
 $\rVRF.\{ \CommitRing, \CommitKey, \OpenKey \}$
for which some zero-knowledge ring membership proof handles both
 $\PedVRF.\OpenKey$ and $\rVRF.\OpenKey$ efficiently.
We take $\rVRF.\KeyGen = \PedVRF.\KeyGen$ and
 $\rVRF.\Eval = \PedVRF.\Eval$ of course.

We define security for only our ring VRF constructions, but clearly
 \PedVRF consists of algorithms having superficially similar signatures.
% \footnote{We do not define security for \PedVRF because pseudo-randomness becomes too interesting}

\begin{itemize}
\item $\rVRF.\rSign : (\sk,\openring,\msg,\aux) \mapsto \sigma$ takes
 a secret key \sk, a ring opening \openring, a message \msg, and \aux, and then % auxiliary data
 % \begin{enumerate}
 % \item
 generates \openpk, computes a ring membership proof $\piring$
  $$ \piring = \NIZK \Setst{ \compk, \comring }{
  \exists \openpk,\openring \textrm{\ s.t.\ } 
  \genfrac{}{}{0pt}{}{\PedVRF.\OpenKey(\compk,\openpk) \quad}{\,\, = \rVRF.\OpenKey(\comring,\openring)}
  } $$
 % \item
 computes the signature
  $$ \sigma = \PedVRF.\Sign(\sk,\openpk,\msg,\aux \doubleplus \compk \doubleplus \piring \doubleplus \comring), \quad\textrm{and} $$ % finally
 % \item
 returns the ring VRF signature $\rho = (\compk,\piring,\sigma)$.
 % \end{enumerate}
\item $\rVRF.\rVerify$ takes $(\comring,\msg,\aux,\rho)$,
 parses $\rho$ as $(\compk,\piring,\sigma,)$,  and then returns
 $$ \PedVRF.\Verify(\compk,\msg,\aux \doubleplus \compk \doubleplus \piring \doubleplus \comring,\sigma) $$
 iff $\NIZK.\Verify(\piring,\compk,\comring)$ succeeds. 
\end{itemize}

\begin{proposition}\label{prop:pedersen_rvrf}
$\rVRF$ satisfies ring uniqueness, ring unforgeability, and ring anonymity.
\end{proposition}






\endinput

















\begin{lemma}\label{prop:pedersen_vrf_hiding}
$\PedVRF$ is a correct key commitment and key hiding.
\end{lemma}

Although Pedersen commitments are perfectly hiding, our $\R_\msg$ makes $\sigma$ only computationally hiding.

\begin{proposition}\label{prop:pedersen_vrf}
Assuming AGM in $\grE$, % $\ecE$ modulo $h$,
our $\PedVRF$ satisfies VRF correctness, key binding, uniqueness,
pseudo-randomness, and unforgeability. % (EUF-CMA-KC) on $(\msg,\aux)$.
\end{proposition}

We need this second verification equation in \PedVRF, but not in \ThinVRF,
because otherwise our $s_2 \genB$ term provides enough freedom to tamper
with the pre-ouputs.  

We could however generalize \PedVRF to $k$ messages $\msg_1,\ldots,\msg_k$
similarly to \ThinVRF in \S\ref{subsec:vrf_thin}:  We compute for
$j=1,\ldots,k$ the $k$ distinct
points $\In_j := H_{\grE}(\msg_j)$, pre-outputs $\PreOut := \sk \, \In$,
delinearization challenges
 $c_j = H_p(\aux,\msg_1,\ldots,\msg_k,\compk,\Out_{0,1},\ldots,\Out_{0,k},j)$,
and then use the \PedVRF proof for
 $\In = \sum_j c_j \In_j$ and $\Out = \sum_j c_j \Out_j$.

% TODO: Proof correct?  Use same citations as schnorrkel.







