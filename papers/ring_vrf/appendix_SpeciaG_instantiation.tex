\section{$\SpecialG$ as Instantiation of $\ZKCont$}
\label{appendix_specialg}

Below we instantiate our zero-knowledge continuation notion with a scheme based on Groth16~\cite{Groth16} SNARK;
which we call  \emph{specialised Groth16} or \emph{$\mathsf{SpecialG}$}. We start by giving a reminder about   
Quadratic Arithmetic Program (QAP)~\cite{LegoSNARK}, ~\cite{GGPR13} and related $\relRQ$ in a standard way. 

\begin{definition}[QAP] 
\label{def:QAP}
A Quadratic Arithmetic Program (QAP) $\cQ = (\cA, \cB, \cC, t(X))$ of size $m$ 
and degree $d$ over a finite field $\F_q$ is defined by three sets of polynomials $\cA = \{a_i(X)\}_{i=0}^m$, 
$\cB = \{b_i(X)\}_{i=0}^m$, $\cC = \{c_i(X)\}_{i=0}^m$, each of degree less than $d-1$ and a target degree $d$ polynomial $t(X)$. Given 
$\cQ$ we define $\relRQ$ as the set of pairs $((\bary, \barx); \baromega) \in \F_q^{l} \times \F_q^{n-l} \times \F_q^{m-n}$ for which it 
holds that there exist a polynomial $h(X)$ of degree at most $d-2$ such that:
$$(\sum_{k=0}^m v_k \cdot a_k(X)) \cdot (\sum_{k=0}^m v_k \cdot b_k(X)) = (\sum_{k=0}^m v_k \cdot c_k(X)) + h(X)t(X) \ \ (\ast)$$ 
where $\barv = (v_0, \ldots, v_m) = (1, x_1, \ldots, x_n, w_1, \ldots w_{m-n})$ and $\bary = (x_1, \ldots, x_l)$ and 
$\barx = (x_{l+1}, \ldots, x_n)$ and $\baromega = (w_1, \ldots, w_{m-n})$. 
\end{definition}

\noindent Given notation provided in section~\ref{sec:background}, in particular elliptic curve $\ecE$, its pairing $e$ and 
the related source, target groups and generators, we introduce%Let $\mathbb{F}_q$ be a prime field, 
%let $G_1$, $G_2$, $G_T$ be as defined in section~\ref{??}, let $e$, $g$, $h$ be defined as $\ldots$. Let $t(X)$ and
%$\{u_i(X),v_i(X),w_i(X)\}_{i=0}^m$ be polynomials in $\F_q[X]$, let $\ldots$ be $\ldots$ such that there exists $h(X) \in \F_q[X]$ with
% $$ \sum_{i=0}^m a_iu_i(X) \cdot  \sum_{i=0}^m a_iv_i(X)  = \sum_{i=0}^m a_iw_i(X) + h(X)t(X)  \ (\ast)$$
%Then let $\relone = \{ (;) \ | \ (;)  (\ast) \}$

\begin{definition}[Specialised Groth16 ($\SpecialG$)]
\label{insta:sg16} Let $\relRQ$ be as mentioned above. We call 
specialised Groth16 for relation $\relRQ$ the following: instantiation of the zero-knowledge continuation notion from Definition~\ref{def:zk_cont}:
\begin{itemize}
\item $\SpecialG.\Setup: (1^{\lambda}, \relRQ) \mapsto (\crs, \tw,\pp)$. \\ 
\noindent Let $\alpha, \beta, \gamma, \delta, \tau, \eta  \xleftarrow{\$} \F_q^{*}$. 
Let $\tw = (\alpha, \beta, \gamma, \delta, \tau, \eta)$. \\ 
Let $\crs = (\barsig_1, \barsig_2)$ where 
\begin{align*}
\barsig_1 = (&\alpha \cdot \gone, \beta \cdot \gone, \delta \cdot \gone, \{\tau_i \cdot \gone\}_{i=0}^{d-1}, \\
& \left\{\frac{\beta a_i(\tau)+ \alpha b_i(\tau)+ c_i(\tau)}{\gamma} \cdot \gone \right\}_{i=1}^n, \frac{\eta}{\gamma} \cdot \gone, \\ 
&\left\{\frac{\beta a_i(\tau)+ \alpha b_i(\tau)+ c_i(\tau)}{\delta} \cdot \gone \right\}_{i=n+1}^m, \\
& \left\{\frac{1}{\delta}\sigma^it(\sigma)\cdot \gone \right\}_{i=0}^{d-2}, \frac{\eta}{\delta}\cdot \gone), \\
\barsig_2 = (&\beta \cdot \gtwo, \gamma \cdot \gtwo, \delta \cdot \gtwo, \{\tau^i \cdot \gtwo\}_{i=0}^{d-1}). 
\end{align*} 

%\item $\SpecialG.\Gen: (\crs, \relRQ) \mapsto (\pp, \crspk, \crsvk)$ where \\
%$\crspk = \left([\barsig_1]_1, [\beta]_2, [\delta]_2, \left\{[\tau^i]_2\right\}_{i=0}^{d-1}\right)$  \\ 
%$\crsvk = \left([\alpha]_1, \left\{ \left[ \frac{\beta a_i(\tau)+ \alpha b_i(\tau)+ c_i(\tau)}{\gamma} \right]_1 \right\}_{i=1}^{l}, 
%[\beta]_2, [\gamma]_2, [\delta]_2\right)$  \\ 
$\pp = \left( \left \{ \frac{\beta a_i(\tau)+ \alpha b_i(\tau)+ c_i(\tau)}{\gamma} \cdot \gone \right \}_{i=l+1}^{n}, \frac{\eta}{\gamma} \cdot \gone \right)$. \\
\noindent Moreover, for simplicity and later use, we call \\
$\Kgamma = \frac{\eta}{\gamma} \cdot \gone$  and $\Kdelta = \frac{\eta}{\delta} \cdot \gone$.


%{We should say what is the difference from Groth16.Setup or it is the same. I think in general in $ \SpecialG $, you should tell which part is from Groth16 or Legosnark and where we %change it while describing the algorithms. It will be much clear for the reader to verify. You have notes in the end but I think it is better to have it while describing since you can tell %more right away from the algorithm than in the end of everything}


\item $\SpecialG.\Preprove: (\crs, \bar{y}, \bar{x}, \baromega_1, \relRQ) \mapsto (X', \pi', b')$ such that \\
\begin{align*}
&\ \ \ \ \ \ \ \   b' = 0; r, s\xleftarrow{\$} \F_p; X' = \sum_{i=l+1}^{n} v_i \cdot  \frac{\beta a_i(\tau)+ \alpha b_i(\tau)+ c_i(\tau)}{\gamma} \cdot \gone;  \\
&\ \ \ \ \ \ \ \  o = \alpha + \sum_{i=0}^{m} v_i \cdot a_i(\tau) + r \cdot \delta; u = \beta + \sum_{i=0}^{m} v_i \cdot b_i(\tau) + s \cdot \delta; \\ 
&\ \ \ \ \ \ \ \   v = \frac{\sum_{i=n+1}^{m} (v_i (\beta a_i(\tau)+ \alpha b_i(\tau)+ c_i(\tau))) + h(\tau)t(\tau)}{\delta}   + \\ 
& \ \ \ \ \ \ \ \ \ \ \ \ \  \ \   + o\cdot s + u \cdot r - r \cdot s \cdot \delta; \\
&\ \ \ \ \ \ \ \ \ \pi' = (o \cdot \gone, u \cdot \gtwo, v \cdot \gone), 
\end{align*}
where $\bary = (x_1, \ldots, x_l)$, $\barx = (x_{l+1}, \ldots, x_n)$, \\
$\baromega = (w_1, \ldots, w_{m-n})$, $\barv = (1, x_1, \ldots, x_n, w_1, \ldots, w_{m-n})$ (same as per definition of QAP).

\item $\SpecialG.\Reprove: (\crs, X', \pi', b', \relRQ) \mapsto (X, \pi, b)$  such that
\begin{align*}
&\ \ \ \ \ \ \ \  b , r_1, r_2  \xleftarrow{\$} \F_p, X = X' + (b- b') \Kgamma, \pi = (O, U, V), \\
&\ \ \ \ \ \ \ \ O = \frac{1}{r_1} O', U = r_1 U' + r_1r_2 \delta \gtwo, V = V' + r_2O'  - (b - b') \Kdelta \mathperiod
\end{align*}
\noindent where $\pi' = (O', U', V')$.
 
\item $\SpecialG.\VerifyCom: (\pp, X, \barx, b) \mapsto 0/1$ where the output is 1 iff the following holds
$$X = \sum_{i=l+1}^{n} x_i \cdot  \frac{\beta a_i(\tau)+ \alpha b_i(\tau)+ c_i(\tau)}{\gamma} \cdot \gone  + b \Kgamma,$$
where $\barx = (x_{l+1}, \ldots, x_n)$, $ 0 \leq l \leq n-1$. 

\item $\SpecialG.\Verify: (\crs, \bar{y}, X, \pi, \relRQ) \mapsto 0/1$ where the output is 1 iff the following holds 
$$e(O,U) = e(\alpha \cdot \gone, \beta \cdot \gtwo) \cdot e(X + Y, \gamma \cdot \gtwo) \cdot e(V, \delta \cdot \gtwo),$$
where $\pi = (O, U, V)$, $Y = \sum_{i=1}^{l} x_i \cdot \frac{\beta a_i(\tau)+ \alpha b_i(\tau)+ c_i(\tau)}{\gamma}  \cdot \gone$ 
and $\bary = (x_1, \ldots, x_l)$.

\item $\SpecialG.\Sim: (\tw, \bary, \relRQ) \mapsto (\pi, X)$ where \\ $x, o, u \xleftarrow {\$} \F_p$ and let 
$\pi = (o \cdot \gone, u  \cdot \gtwo , v \cdot \gone)$ where \\ $v = \frac{o\cdot u - \alpha \beta - \sum_{i=1}^{l} x_i (\beta a_i(\tau)+ \alpha b_i(\tau)+ c_i(\tau))- x}{\delta}  $ and, 
by definition $\bary = (x_1, \ldots, x_l)$. Note that $\pi$ is a simulated proof for transparent input $\bary$ 
and commitment $X = x \cdot \gone$.
\end{itemize} 
\end{definition}

