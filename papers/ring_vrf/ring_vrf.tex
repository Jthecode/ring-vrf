\documentclass[runningheads,evcountsame,a4paper,11pt,orivec]{llncs}
\usepackage[margin=2.5cm,includefoot]{geometry}

%=================Begin:Packages===================
\usepackage{graphicx}
\usepackage{bmpsize}
\usepackage{algorithm}

\usepackage[noend]{algpseudocode}
% \usepackage{enumitem}
\usepackage{url}
%\usepackage{amsthm}

\usepackage{amsfonts}
\usepackage{amsmath}
\usepackage{hyperref}
% \usepackage[capitalize,nameinlink]{cleveref}
% \usepackage{framed}
% \usepackage{fancybox}
\usepackage[utf8]{inputenc}
\usepackage{mathtools}
% \usepackage{tcolorbox}
\usepackage{tikz}
% \usepackage{todonotes}
\usepackage{xspace}
\usepackage{xcolor}
\usepackage[advantage,
adversary,
asymptotics,
complexity,
ff,
mm,
operators,
probability,
]{cryptocode}
\usepackage[linewidth=1pt]{mdframed}
\usepackage{ulem}
\usetikzlibrary{arrows,chains,matrix,positioning,scopes}

% TODO: Clean up macros

% \newcommand\doubleplus{+\kern-1.3ex+\kern0.8ex}
\newcommand\doubleplus{\ensuremath{\mathbin{+\mkern-10mu+}}}


%% oracles
\newcommand{\ora}[1]{\ensuremath{\mathcal{O}\mathsf{#1}}\xspace}
\newcommand{\oramsg}[1]{\ensuremath{\mathsf{#1}}\xspace}
%% algorithm
\newcommand{\algo}[1]{{\textsc{#1}}}
%%primitive algo
\newcommand{\primalgo}[1]{{\ensuremath{\mathsf{#1}}}\xspace}
%%primitive
\newcommand{\prim}[1]{{\ensuremath{\mathsf{#1}}}\xspace}
%%set
\newcommand{\setsym}[1]{{\ensuremath{\mathcal{#1}}}}
%%array
\newcommand{\arraysym}[1]{{\ensuremath{\mathsf{#1}}}}

\newcommand\N{\mathbb{N}}
\newcommand\F{\mathbb{F}}
% \newcommand\Gr{\mathbb{G}}


\def\mathperiod{.}
\def\mathcomma{,}



\newcommand*\set[1]{\{ #1 \}} % in text, we don't want {} to grow
\newcommand*\Set[1]{\left\{ #1 \right\}}
\newcommand*\setst[2]{\{ #1 | #2 \}}
\newcommand*\Setst[2]%
        {\left\{\,#1\vphantom{#2} \;\right|\left. #2 \vphantom{#1}\,\right\}}
% ``set such that''; puts in a vertical bar of the right height


\providecommand{\bin}{\ensuremath{\{0,1\}}\xspace}


% https://tex.stackexchange.com/questions/471713/is-mathrel-always-needed
% https://tex.stackexchange.com/questions/418740/how-to-write-left-arrow-with-a-dollar-sign/626086#626086
\makeatletter
\providecommand\leftsample{\leftarrow\mathrel{\mkern-2.0mu}\pc@smalldollar}
\providecommand\rightsample{\pc@smalldollar\mathrel{\mkern-2.0mu}\rightarrow}
%
\newcommand{\pc@smalldollar}{\mathrel{\mathpalette\pc@small@dollar\relax}}
\newcommand{\pc@small@dollar}[2]{%
  \vcenter{\hbox{%
    $#1\textnormal{\fontsize{0.7\dimexpr\f@size pt}{0}\selectfont\$\hskip-0.05em
 plus 0.5em}$%
  }}%
}
\makeatother


\newcommand{\KeyGen}{\primalgo{KeyGen}}

% Why was this \prim before?
\newcommand{\VRF}{\primalgo{VRF}} 
\newcommand{\rVRF}{\primalgo{rVRF}} 

\newcommand{\Sign}{\primalgo{Sign}}
\newcommand{\Verify}{\primalgo{Verify}}
\newcommand{\Eval}{\primalgo{Eval}}
\newcommand{\Prove}{\primalgo{Prove}}
\newcommand{\Simulate}{\primalgo{Simulate}}
\newcommand{\Extract}{\primalgo{Extract}}


\newcommand{\In}{\primalgo{In}} 
\newcommand{\Out}{\primalgo{Out}} 
\newcommand{\PreOut}{\ensuremath{\primalgo{Out}_0}\xspace} 

\newcommand{\vk}{\ensuremath{\mathsf{vk}}\xspace}
\newcommand{\sk}{\ensuremath{\mathsf{sk}}\xspace}
\newcommand{\pk}{\ensuremath{\mathsf{pk}}\xspace}
\newcommand{\apk}{\ensuremath{\mathsf{apk}}\xspace}
\newcommand{\pkring}{\ensuremath{\setsym{PK}}}
\newcommand{\msg}{\ensuremath{\mathsf{msg}}\xspace}
\newcommand{\aux}{\ensuremath{\mathsf{aux}}\xspace}
\newcommand{\ctx}{\ensuremath{\mathsf{ctx}}\xspace}
\newcommand{\ringset}{\ensuremath{\mathsf{ring}}\xspace}



\newcommand\SNARK{\primalgo{SNARK}}
\newcommand\NIZK{\primalgo{NIZK}}


\newcommand{\adv}{\ensuremath{\mathcal{A}}\xspace}

\endinput

\newcommand{\evalprove}{\primalgo{EvalProve}}
\newcommand{\link}{{\primalgo{Link}}}
\newcommand{\update}{{\primalgo{Update}}}
\newcommand{\hashG}{\primalgo{H}_{\GG}}
\newcommand{\secreteval}{\primalgo{Secret}\eval}
\newcommand{\secretprove}{\primalgo{Secret}\prove}
\newcommand{\secretverify}{\primalgo{Secret}\verify}

\newcommand{\randsel}[0]{\ensuremath{\xleftarrow{\text{\$}}}}
\newcommand{\rel}{\ensuremath{\mathcal{R}}}


\endinput




\newcommand{\skvrf}{\ensuremath{\sk^{\mathsf{vrf}}}}
\newcommand{\pkvrf}{\ensuremath{\pk^{\mathsf{vrf}}}}
\newcommand{\skrvrf}{\ensuremath{\sk^{\mathsf{rvrf}}}}
\newcommand{\pkrvrf}{\ensuremath{\pk^{\mathsf{rvrf}}}}
\newcommand{\sksign}{\ensuremath{\sk^{\mathsf{sign}}}}
\newcommand{\pksign}{\ensuremath{\pk^{\mathsf{sign}}}}
\newcommand{\skksign}{\ensuremath{\sk^{\mathsf{kesign}}}}
\newcommand{\pkksign}{\ensuremath{\pk^{\mathsf{kesign}}}}
\newcommand{\pkssale}{\ensuremath{\pk^{\mathsf{ssale}}}}
\newcommand{\D}{\ensuremath{\Delta}}
\newcommand{\skzkvrf}{\ensuremath{\sk^{\mathsf{zkvrf}}}}
\newcommand{\pkzkvrf}{\ensuremath{\pk^{\mathsf{zkvrf}}}}


\newcommand{\tab}[1]{\hspace{.05\textwidth}\rlap{#1}}
\newcommand{\tabdbl}[1]{\hspace{.1\textwidth}\rlap{#1}}
\newcommand{\tabdbldbl}[1]{\hspace{.15\textwidth}\rlap{#1}}
\newcommand{\tabdbldbldbl}[1]{\hspace{.19\textwidth}\rlap{#1}}



\newcommand{\game}[3][]{\operatorname{#2}^{#1}_{#3}(\secpar)}
\newcommand{\transcript}[1]{\langle #1 \rangle}
\newcommand{\eppt}{\pccomplexitystyle{EPPT}}
\newcommand{\pt}{\pccomplexitystyle{PT}}

% \renewcommand{\pcadvstyle}[1]{\ensuremath{\mathsf{#1}}}
% \newcommand{\zdv}{\pcadvstyle{Z}}

% \newcommand{\msg}[1]{\mathsf{#1}}

\newcommand{\simulator}{\ensuremath{\mathsf{Sim}}}
%\newcommand{\minote}[1]{\todo[color=green!30,inline]{\textbf{Michele says:} #1}}

\newcommand{\fvrf}{\mathcal{F}_{\textsf{vrf}}}
\newcommand{\fgvrf}{\mathcal{F}_{\textsf{rvrf}}}
\newcommand{\fcpke}{\mathcal{F}_{\mathsf{CPKE}}}
\newcommand{\pvrf}{\mathsf{\Pi}_{\textsf{rvrf}}}
\newcommand{\svrf}{\simulator_\mathsf{gvrf}}
\newcommand{\fnizk}{\mathcal{F}_{\textsf{nizk}}}
\newcommand{\fkes}{\mathcal{F}_{\textsf{sgke}}}
\newcommand{\fcom}{\ensuremath{\mathcal{F}_{\mathsf{com}}}}
\newcommand{\fsec}{\ensuremath{\mathcal{F}_\mathsf{ED-SMT}}}
\newcommand{\frsc}{\ensuremath{\mathcal{F}_{\mathsf{rSC}}}}
\newcommand{\fsasle}{\ensuremath{\mathcal{F}_{\mathsf{sle}}}}
\newcommand{\finit}{\ensuremath{\mathcal{F}_{\mathsf{init}}}}
\newcommand{\fsig}{\mathcal{F}_{\mathsf{sig}}}
\newcommand{\fros}{\mathcal{F}_{\mathsf{ros}}}
\newcommand{\fzkvrf}{\mathcal{F}_{\mathsf{zkvrf}}}
\newcommand{\fcommit}{\mathcal{F}_{\mathsf{commit}}}
\newcommand{\gclock}{\mathcal{G}_{\mathsf{clock}}}
\newcommand{\fcrs}{\mathcal{F}_{crs}}
\newcommand{\env}{\mathcal{Z}}
\newcommand{\stake}{\mathsf{st}}
\newcommand{\stakeset}{\setsym{ST}}

\newcommand{\sid}{\textsf{sid}}
\newcommand{\pid}{\textsf{pid}}
\newcommand{\user}{\mathsf{P}}
\newcommand{\defeq}{\coloneqq}


\newcommand{\evaluationslist}{\texttt{evaluations}}
\newcommand{\evaluationsecretlist}{\texttt{secrets}}
\newcommand{\vklist}{\texttt{verification\_keys}}
\newcommand{\siglist}{\texttt{signatures}}
\newcommand{\prooflist}{\texttt{proofs}}
\newcommand{\proofzklist}{\texttt{zkproofs}}
\newcommand{\Linklist}{\texttt{links}}
\newcommand{\emptylist}{\emptyset}
\newcommand{\fail}{\mathbf{fail}}
\newcommand{\R}{\mathsf{R}}
\newcommand{\bool}{\textit{bool}}
\newcommand{\lst}{\setsym{L}}
\newcommand{\distribution}{\setsym{D}}

\newcommand{\weak}{\ensuremath{W}}
\newcommand{\inbox}{\ensuremath{\setsym{I}}}
\newcommand{\dqueue}{\ensuremath{\setsym{Q}^\D}}
\newcommand{\wqueue}{\ensuremath{\setsym{Q}^\weak}}
\newcommand{\weaklist}{\ensuremath{\setsym{\weak}}}
\newcommand{\mID}{\ensuremath{\mathsf{mid}}}
\newcommand{\plist}{\ensuremath{\setsym{P}}}
\newcommand{\timeoutlist}{\ensuremath{\setsym{T}}}
\newcommand{\anony}{\ensuremath{\mathfrak{a}}}
\newcommand{\dleqr}{\R_\textsf{dleq}}
\newcommand{\view}{\mathsf{view}} 
\renewcommand{\adv}{\ensuremath{\mathcal{A}}}
\newcommand{\preoutputlist}{\arraysym{pre\-outputs}}




\sloppy


\title{Ethical identity, ring VRFs, and zero-knowledge continuations}

\author{Jeffrey Burdges \and Handan Kilinc-Alper \and Alistair Stewart \and Sergey Vasilyev}
\date{}
\institute{Web 3.0 Foundation}


\begin{document}
	
\maketitle

\begin{abstract}
Verifiable random functions (VRFs) weld a pseudorandom function (PRF) into
a signature scheme, so that valid signatures prove correct PRF evaluations.
VRFs have diverse applications in consensus protocols like randomness
beacons, DNS via NSEC5, certificate transparency, and online games.

Anonymized {\em ring VRFs} are then ring signatures that similarly prove
correct evaluation of some authorized signer's PRF while hiding the
specific signer's identity.
Ring VRFs integrate with distributed systems better, and provide more robust
anonymity, than similar schemes like group VRFs or threshold protocols.

% Above unecessary in the abstract?

We propose ring VRFs as a natural fulcrum around which a diverse array of
zkSNARK circuits turn, which thus makes ring VRFs an ideal target
for optimization and eventually standards. 

%

Among their diverse uses, we focus upon identity-like cases in which
ring VRF outputs act like pseudonyms provably unique within some domain or
context, specified via the VRF input, but unlinkable between domains. 
These unlinkabile but unique pseudonyms provide an excellent balance between
user privacy and service provider or social interests \cite{pop2008}.
%
In fact, identity systems without unlinkability like OAuth wind up being
unethical when used outside specific narrow relationships.
% TODO: ``one person, one persona'' \cite{pop2008} https://bford.info/pub/net/sybil.pdf

%

As our core optimization tool, we explore {\em zero-knowledge continuations}
of rerandomizable zkSNARKs like \cite{groth16} to amortize expensive
ring membership proofs across many ring VRF outputs. 
%
We implement a broadly usable ring VRF that amortizes an arbitrary
ring membership proofs, with a marginal prover cost
 of eight $\ecE_1$ and two $\ecE_2$ scalar multiplications.

In other words, our ring VRF might be first ring signature with asymptotic
performance compeditive with group sigantures.
%
We expect zero-knowledge continuations of rerandomizable zkSNARKs
thuroughly trounce recusive zkSNARK techniques indefinitely.
\end{abstract}


% \section{Introduction}

\def\qaudbreak{\eprint{\quad}{\\}}

We introduce ring verifiable random functions (ring VRFs) as a natural
fulcrum around which anonymous credentials turn, in formalization,
in optimizations, in the nuances of use-cases, and in miss-use resistance.
%
Along with some formalizations, we explain portions of their unfolding
story which address three questions:
\begin{enumerate} 
\item
What are the cheapest SNARK proofs?  \qaudbreak
Ones users reuse without reproving.
% \item
% How can credentials use be contextual?  \qaudbreak
% Prove evaluation of a secret function.
\item
How can identity be safe for general use?  \qaudbreak
By revealing nothing except users' uniqueness.
\item
How can ration card issuance be transparent?  \qaudbreak
By asking users trust a public list, not certificates.
\end{enumerate}

We model the security of ring VRFs in the universally composable (UC) \cite{canetti1,canetti2} and  prove that our ring VRF protocol is UC secure. Thus, we guarantee the strongest security along with  practicality.
% First
\paragraph{Zero-knowledge continuations:}

Rerandomizable zkSNARKs like Groth16 \cite{Groth16} admit a
transformation of a valid proof into another valid but unlinkable
proof of the exact same statement.  In practice, rerandomization
was never deployed because the public inputs link the usages.

We demonstrate in \S\ref{sec:rvrf_cont} a simple transformation of
any Groth16 zkSNARK into a {\it zero-knowledge continuation} whose
public inputs become opaque Pedersen commitments, with cheaply
rerandomizable blinding factors and proofs.
These zero-knowledge continuations then prove validity of the contents
of Pedersen commitments, but can now be reused arbitrarily many times,
without linking the usages. 

As recursive SNARKs shall remain extremely slow,
we expect zero-knowledge continuations via rerandomization become
essential for zkSNARKs used outside the crypto-currency space.

% \smallskip 
\paragraph{Ring VRFs:}

A {\it ring verifiable random function} (ring VRF) is a ring signature
that proves correct evaluation of some pseudo-random function (PRF)
determined by the actual key pair used in signing. % (see \S\ref{sec:rvrf_games}).
We build extremely efficient and flexible ring VRFs by amortizing a
zero-knowledge continuation that unlinkably proves ring membership
of a secret key, and then cheaply proving individual VRF evaluations.

As the PRF output is uniquely determined by the signed message and
signers actual secret key, we can therefore link signatures by the
same signer if and only if they sign identical messages.
In effect, ring VRFs restrict anonymity similarly to but less than
 linkable ring signatures do, which makes them multi-use and contextual.

% Second
% \smallskip
\paragraph{Identity uses:}

As an identity system, ring VRFs evaluated on a specific context or
domain name output a unique identity for the user at that domain or
context (see \S\ref{sec:app_identity}), which thereby prevents
Sybil behavior and permits banning specific users.
Yet users' activities remain unlinkable across distinct contexts or
domains, which supports diverse ethical identity usages.

We contrast this ethically straightforward ring VRF based identity
with the ethically problematic case of attribute based credential
schemes like IRMA (``I Reveal My Attributes'') credentials \cite{IRMAcredentials},
 which are now marketed as an online privacy solution.
IRMA could improve privacy in narrow situations of course, but
overall attribute based credentials should {\it never} be considered
fit for general purpose usage, like the prevention of Sybil behavior.

Aside from general purpose identity, our existing offline
verification processes often better protect user privacy and human
rights than adopting online processes like IRMA.
%
In particular, there are many proposals by the W3C for attribute based
credential usage in \cite{w3c_vc_use_cases}, but broadly speaking they
all bring matching harmful uses.  % https://www.w3.org/TR/vc-use-cases/
As an example, if users could easily prove their employment online when
applying for a bank account, then job application sites could similarly
demand proof of current employment, a clear injustice.

In general, abuse risks dictate that IRMA verifiers should be tightly
controlled by legislation, which becomes difficult internationally. 
%
Ring VRFs avoid these abuse risks by being truly unlinkable, and thus
yield anonymous credentials which safely avoid legal restrictions.

{\it Any ethical general purpose identity system should be based
upon ring VRFs, not attribute based credentials like IRMA.}

We credit Bryan Ford's work on proof-of-personhood parties \cite{pop2008,pop2017}
% https://bford.info/pub/dec/pop-abs/  https://bford.info/pub/net/sybil-abs/
with first espousing the idea that anonymous credentials should produce
contextual unique identifiers, without leaking other user attributes.

As a rule, there exist simple VRF variants for all anonymous credentials
like IRMA \cite{IRMAcredentials} or group signatures \cite{group_sig_survey}.
We focus exclusively upon ring VRFs for brevity, and because alone
ring VRFs contextual linkability covers more important use cases.

% Third
% \smallskip
\paragraph{Rationing uses:}

Ring VRFs yield rate limiting or rationing systems, which work
similarly to identity applications, except their VRF inputs should also
include an approximate date and a bounded counter, and
 then their outputs should be tracked as nullifiers.
Yet, these nullifiers need only temporarily storage, which improves 
efficiency over anonymous money schemes like ZCash and blind signed tokens.

We expect a degree of fraud whenever deploying purely certificate
based systems, as witnessed by the litany of fraudulent TLS and covid
certificates.  Ring VRFs help mitigate fraudulent certificate concerns
because the ring is a database and can be audited.

We know governments have ultimately little choice but to institute
rationing in response to shortages caused by climate change, ecosystem
collapse, and peak oil.  Ring VRFs could help avoid ration card fraud,
and thereby reduce social unrest, while also protecting essential privacy.

Ring VRFs need heavier verifiers than single-use token credentials
based on OPRFs \cite{PrivacyPass} or blind signatures.
Yet, ring VRFs avoid these schemes separate issuance phase entirely,
and sometimes even their registration phase.  Instead, fresh tokens
merely require adjusting the approximate date in the VRF input.
This reduces complexity, simplifies scaling, and increases flexibility.

In particular, if governments issue ration cards based upon ring VRFs
then these credentials could safely support other use cases, like
free tiers in online services or games, and advertiser promotions,
as well as identity applications like prevention of spam and online abuse.

In this, we need authenticated domain separation of products or identity
consumers in queries to users' ring VRF credentials.  We briefly discuss
some sensible patterns in \S\ref{???} below, but overall authenticated
domain separation resemble TLS certificates except simpler in that
roots of trust can self authenticate if root keys act as domain separators.





\endinput




As a field, anonymous credentials come in myriad flavors,
many of which exist to limits the anonymity provided, ala
 attribute based credentials and group signatures. % \cite{group_sig_survey}.
% aka anonymized signatures
%
Ring VRFs by weakening anonymity only contextually provide a safer,
more private, more flexible, more powerful, and more ethical
choice for all everyday anonymous credential use cases.  % needs:  ???



% 
\section{Identity}

% “We can judge our progress by the courage of our questions and the depth of our answers, our willingness to embrace what is true rather than what feels good.” 
% - Carl Sagan

% https://twitter.com/IdentityZack/status/1480631954689216516

% bryan ford https://twitter.com/brynosaurus/status/1460094634567344133

% answer https://twitter.com/valkenburgh/status/1442894421289103361
% https://twitter.com/harryhalpin/status/1443053685219725315
% https://twitter.com/OR13b/status/1442964741022830594
% https://twitter.com/jeffburdges/status/1443539630033362948
% https://twitter.com/Steve_Lockstep/status/1448653579330342916

% https://github.com/dckc/awesome-ocap/issues/17

% https://twitter.com/smdiehl/status/1459825936757493770

% https://twitter.com/edri/status/1483818492646281225

% Zeroth law:  A robot may not harm humanity, or, by inaction, allow humanity to come to harm.
% First law:  A robot may not injure a human being or, through inaction, allow a human being to come to harm.

An identity system must not harm humanity or its human users, to do otherwise is clearly unethical.  

Identity systems for human users have three participants, an identity provider, an identity consumer, and the user being identified.  There exist two methods by which ethical identity systems avoid harming users, either 
\begin{itemize}
\item special identity systems enforce that identity consumers owe users some legal duty that prevents miss-using the user's details, or else
\item general identity systems merely constrain user activity, often only rate limiting, but avoid providing identity consumers with any user details.
\end{itemize}
In other words, identity consumers should always first prove to the identity provider that they owe the user a legal duty appropriate to the details being revealed by the identity provider.

\subsection{Legal duties}

In this paper, we discuss only cryptographic protocols for general identity systems that avoid legal entanglements by only proving user uniqueness and not providing user details.  We first in this section briefly discuss wider examples that help motivate this problem by clarifying the legal and ethical complexities that arise when revealing user details.

As an unethical example, our largest advertising companies like Google and Facebook track private users using OAuth \cite{oauth}, with the intent to waste users time with increased advertising engagement, manipulate public opinion, ensnare users into unnecessary purchases, often by harming users' psyche, and accumulate personal data users might otherwise wish kept hidden.

As an only moderately harmful example, websites often prevent abuse by demanding commenters identify themselves by email address, which creates moral hazards and should expose the website operators to legal risks.

As beneficial identity examples, financial institutions act as an identity provider for their own identity consumer logic by issuing login credentials, but then owe their customers some fiduciary duty and strongly discourage using the same login credentials elsewhere.  

As a more nuanced example, an employer identifies employees to a personel management service by way of an external OAuth service, but the employer has some legal relationship with the personel management service, the OAuth service, and the employee, so any resulting harms rest upon the employer-employee relationship.  

We think Google Single Sign-on or Facebook Connect cannot play the role of OAuth service even in this employer-employee example, and indeed cannot ever be used ethically, because they aggressively track the employee outside the employer-employee relationship.  At the same time, an employees' Github account might or might not serve this role depending upon the specific employee and how they use Github outside work.  

... passports or medical ...

\subsection{Unlinkable identity}

We now lay aside such identity systems that represent a distinguished purpose tied to onerous three-way legal relationships between the parties.  Instead we turn our attention towards the range of identity systems that avoid providing any user details.  

At present, CAPTCHAs provide a popular defense against automated abuse.  There also exist cryptographic tools that amplify defenses against automated abuse, like blind signatures or verifiable oblivious pseudorandom functions (VOPRFs), as used in Privacy Pass \cite{privacypass}.  These dispence signle-use tokens within some limits imposed by other identity sources, rate limits, payments, or CAPTCHAs.  

We think single-use tools like CAPTCHAs, blind signatures, and VOPRFs adequately deter abuse in most use cases.  Yet, there also exist situations where abusers cannot be dissuaded by solving another CAPTCHAs or spending another token, like when abuse takes a personal character, or due to a larger profit motive.  

In such harder cases, we still need an anonymous credential so that identity consumers and providers cannot collude to track users, but identity consumers banning problematic users seemingly demands that users have different stable identities with each distinct identity consumer.  
To our knowledge, this identity formulation originates with proof-of-personhood parties \cite{pop2008,pop2017}.
% https://bford.info/pub/dec/pop-abs/
% https://bford.info/pub/net/sybil-abs/

We expect stable identities arise from multi-use anonymous credentials, like group signatures or ring signatures.  In group signatures, an identity provider holds a group manger secret key, with which they both issues credentials and deanonymize users.  We only want identity consumers to recognize returning users, making the deanonymization operation unacceptable.  

Ring signatures have classically given signers' control over their anonymity set aka ``ring'', which turns out mostly useless in practice.  Instead, realistic ring signatures like Zcash's circuits \cite{zcash_prorocol} have a shared public commitment to their ``ring'', so then users need only an opening for their own public key's presence in the ring. 



% sharing economy 
% business-to-business 



%   We think identity consumers should avoid imposing unnecessary constraints upon users and that rate limiting tools usually suffice.  Yet, there exist identity consumers who depend upon stronger Sybil defenses or an ability to ban problematic users.   

\newcommand{\KeyGen}{\primalgo{KeyGen}}

% Why was this \prim before?
\newcommand{\VRF}{\primalgo{VRF}} 
\newcommand{\rVRF}{\primalgo{rVRF}} 

\newcommand{\Sign}{\primalgo{Sign}}
\newcommand{\Verify}{\primalgo{Verify}}
\newcommand{\Eval}{\primalgo{Eval}}
\newcommand{\Prove}{\primalgo{Prove}}
\newcommand{\Simulate}{\primalgo{Simulate}}
\newcommand{\Extract}{\primalgo{Extract}}


\newcommand{\In}{\primalgo{In}} 
\newcommand{\Out}{\primalgo{Out}} 
\newcommand{\PreOut}{\ensuremath{\primalgo{Out}_0}\xspace} 

\newcommand{\vk}{\ensuremath{\mathsf{vk}}\xspace}
\newcommand{\sk}{\ensuremath{\mathsf{sk}}\xspace}
\newcommand{\pk}{\ensuremath{\mathsf{pk}}\xspace}
\newcommand{\apk}{\ensuremath{\mathsf{apk}}\xspace}
\newcommand{\pkring}{\ensuremath{\setsym{PK}}}
\newcommand{\msg}{\ensuremath{\mathsf{msg}}\xspace}
\newcommand{\aux}{\ensuremath{\mathsf{aux}}\xspace}
\newcommand{\ctx}{\ensuremath{\mathsf{ctx}}\xspace}
\newcommand{\ringset}{\ensuremath{\mathsf{ring}}\xspace}



\newcommand\SNARK{\primalgo{SNARK}}
\newcommand\NIZK{\primalgo{NIZK}}


\newcommand{\adv}{\ensuremath{\mathcal{A}}\xspace}

\endinput

\newcommand{\evalprove}{\primalgo{EvalProve}}
\newcommand{\link}{{\primalgo{Link}}}
\newcommand{\update}{{\primalgo{Update}}}
\newcommand{\hashG}{\primalgo{H}_{\GG}}
\newcommand{\secreteval}{\primalgo{Secret}\eval}
\newcommand{\secretprove}{\primalgo{Secret}\prove}
\newcommand{\secretverify}{\primalgo{Secret}\verify}

\newcommand{\randsel}[0]{\ensuremath{\xleftarrow{\text{\$}}}}
\newcommand{\rel}{\ensuremath{\mathcal{R}}}


\endinput




\newcommand{\skvrf}{\ensuremath{\sk^{\mathsf{vrf}}}}
\newcommand{\pkvrf}{\ensuremath{\pk^{\mathsf{vrf}}}}
\newcommand{\skrvrf}{\ensuremath{\sk^{\mathsf{rvrf}}}}
\newcommand{\pkrvrf}{\ensuremath{\pk^{\mathsf{rvrf}}}}
\newcommand{\sksign}{\ensuremath{\sk^{\mathsf{sign}}}}
\newcommand{\pksign}{\ensuremath{\pk^{\mathsf{sign}}}}
\newcommand{\skksign}{\ensuremath{\sk^{\mathsf{kesign}}}}
\newcommand{\pkksign}{\ensuremath{\pk^{\mathsf{kesign}}}}
\newcommand{\pkssale}{\ensuremath{\pk^{\mathsf{ssale}}}}
\newcommand{\D}{\ensuremath{\Delta}}
\newcommand{\skzkvrf}{\ensuremath{\sk^{\mathsf{zkvrf}}}}
\newcommand{\pkzkvrf}{\ensuremath{\pk^{\mathsf{zkvrf}}}}


\subsection{VRF-AD-KC security}
\label{subsec:vrf_def}


Any signature scheme requires a \KeyGen algorithm of course, but we also
support hiding public keys inside a commitment scheme \CommitKey and \OpenKey.
Zero-knowledge continuations then work by running \OpenKey inside yet
another zero-knowledge proof. 

\begin{definition}
A {\em verifiable random function with auxiliary data and key commitments} (VRF-AD-KC) consists of several algorithms:
\begin{itemize}
\item $\VRF.\KeyGen$ and returns a public key \pk and a secret key \sk, which one typically instantiates via come commitment scheme. 
%
\item $\VRF.\CommitKey : (\pk,\ctx) \mapsto (\compk,\openpk)$ takes a public key \pk and a commitment context \ctx, and returns a public key commitment \compk to \sk and secret opening data \openpk.
\item $\VRF.\OpenKey : (\compk,\openpk) \mapsto \pk$ opens a public key commitment \compk given the specified opening data \openpk.
%
\item $\VRF.\Eval : (\sk,\msg) \mapsto \Out$ takes a secret key \sk and an input $\msg$, and then returns a VRF output $\Out$.
\item $\VRF.\Sign : (\sk,\openpk,\msg,\aux) \mapsto \sigma$ takes a secret key \sk, a public key opening \openpk, an input \msg, and auxiliary data \aux, and then returns a VRF signature $\sigma$.
\item $\VRF.\Verify$ takes $(\compk,\msg,\aux,\sigma)$ for a public key commitment \compk, an input \msg, and auxiliary data \aux, and then returns either an output $\Out$ or else failure $\perp$.
\end{itemize}
\end{definition}
% PPA vs DPA ?

% We typically define VRFs for secret keys 
% We say secret keys are equivalent whenever their evaluation map $F_\sk$ given by
%  $\msg \apsto \Eval([\sk],\msg)$ defines the same function.
% We also define an equivalence relation upon secret keys with classes denoted $[\sk]$
% because secret keys could contain a public key opening data with only limited impact upon the VRF output. 

% \subsection{VRF-AD-KC security}

We say a VRF-AD-KC denoted \VRF is {\em secure} if it satisfies
 correctness, uniqueness, and pseudo-randomness as defined below,
 as well as being existentially unforgeable as a signature on $(\msg,\aux)$
 and being binding in one of the senses discussed blow.
We caution that VRF security remains complex, in part due to
signer and forger each being adversarial in some security properties,
and that ring VRFs make this worse by verifiers being adversarial.

We follow \cite{agg_dkg} by distinguishing an algorithm $\VRF.\Eval$,
 instead of defining it by the equality in correctness,
which simplifies requiring that verifying honest signatures gives a well-defined function.
$\VRF.\Eval$ always has more optimized instantiations anyways.
% We merge correctness of commitment and VRF here because
% our VRF correctness invokes $\CommitKey$ by necessity.

% \subsection{VRF-AD-KC security}

\begin{definition}
We say a VRF-AD-KC satisfies {\em commitment correctness} if
 $\OpenKey \circ \CommitKey$ returns the same public key \pk.
\end{definition}

\begin{definition}
We say a VRF-AD-KC satisfies {\em VRF correctness} if
$(\pk,\sk) \leftarrow \KeyGen$ and $(\compk,\openpk) \leftarrow \CommitKey(\pk,\ctx)$
imply
$\Verify(\compk,\msg,\aux,\Sign(\sk,\openpk,\msg,\aux)) = \Eval(\sk,\msg)$.
% perhaps except with odds negligible in $\secparam$.
\end{definition}

We demand unforgability on $(\msg,\aux)$ because alone
 the usual VRF conditions only yield unforgeability for \msg.
We need the usual EUF-CMA game here, except attackers access
\CommitKey in our EUF-CMA-KC game, so if desired
 they could obtain multiple signatures under the same \compk,
 and our signing oracle \ora{Sign} enforces commitments.

\begin{definition}\label{def:vrf_sign_oracle}
We let \ora{Sign} denote a CMA oracle, which creates and stores
a key pair $(\pk,\sk) \leftarrow \KeyGen$ and thereafter
answers oracle calls $\ora{Sign}(\compk,\openpk,\msg,\aux)$ by 
logging $(\msg,\aux)$ and returning $\Sign(\sk,\openpk,\msg,\aux)$,
provided $\pk = \OpenKey(\compk,\openpk)$, or aborting otherwise.
\end{definition}

% TODO \eprint
\begin{definition}
We say a VRF-AD-KC satisfies {\em existential unforgeability (EUF-CMA-KC)} if
any PPT adversary \adv has only a negligible advantage in $\secparam$
in the usual chosen-message game adapted to key commitments:
\begin{itemize}
 \item \adv receives $\pk$ from \ora{Sign}, % of Definition \ref{def:vrf_sign_oracle}
 repeatedly queries \ora{Sign},
  and finally produces $\compk,\msg,\aux,\sigma,\openpk$.
 \item \adv wins if $\Verify(\compk,\msg,\aux,\sigma)$ succeeds,
  $\OpenKey(\compk,\openpk) = \pk$, and
   \adv never queried $\ora{Sign}(\cdot,\cdot,\msg,\aux)$.
\end{itemize}
\end{definition}

We do not demand the commitment scheme $\CommitKey$ and $\OpenKey$
be hiding in the definition of VRF-AD-KC security.
Yet, we do internally employ the usual hiding definition from
\cite[pp.8]{cryptoeprint:2019:1185} for commitment schemes however.
% We could employ a weaker chosen-message-like hiding property, but
% this full strength versions suffices.

\begin{definition}
We say a VRF-AD-KC is {\em key hiding} if any PPT adversary \adv
who creates a pair of public keys $\pk_1,\pk_2$
has only negligible advantage for identifying which lies behind a commitment
 $\compk \leftarrow \CommitKey(\pk_i,\ctx)$.
\end{definition}

We want a commitment binding property with unique openings,
analogous to \cite[pp.9]{cryptoeprint:2019:1185}.
% except weakened to require the signature verify too.

\begin{definition}\label{def:vrf_key_binding}
We say a VRF-AD-KC is {\em key binding} if no PPT adversary \adv
produces \compk, \msg, \aux and $\sk_i,\openpk_i$ for $i=1,2$
so that
 % $\CommitKey(\pk_1,\openpk_1) = \compk = \CommitKey(\pk_2,\openpk_2)$ and
 $\OpenKey(\compk,\openpk_1) \ne \OpenKey(\compk,\openpk_2)$ and
 $\Verify(\compk,\msg,\aux,\Sign(\sk_i,\openpk_i,\msg,\aux))$
both succeed for $i=1,2$, except with odds negligible in $\secparam$.
\end{definition}

We weaken binding like this because, from \S\ref{subsec:vrf_pederson}
onward, verification provides a proof-of-knowledge that turns
Pedersen commitments to secret keys into commitments to public keys.
% We always take $\ctx = \emptyset$ when using this ``unique'' key binding
% condition, like in the subsequent two sections, but
%  \ctx becomes important later in analogous ring properties.

\begin{definition}
We say a VRF-AD-KC satisfies {\em uniqueness} if anytime some PPT adversary \adv
produces $\msg$ and $\compk_i,\openpk_i,\aux_i,\sigma_i$ for $i=1,2$
 with $\OpenKey(\compk_1,\openpk_1) = \OpenKey(\compk_2,\openpk_2)$, then also
$\Verify(\compk_1,\msg,\aux_1,\sigma_1) = \Verify(\compk_2,\msg,\aux_2,\sigma_2)$
unless either $\Verify$ returns failure,
except with odds negligible in $\secparam$.
\end{definition}

If desired, one easily simplifies VRF-AD-KC to a VRF-AD by
 taking $\compk = \pk$ and fixing $\openpk = \mathtt{""}$,
 which makes $\VRF.\CommitKey$ and $\VRF.\OpenKey$ trivial.

\smallskip

We say VRFs are public key analogs of PRFs, but actually this PRF analogy
fails badly in the ``residual pseudo-randomness'' definitions by
Micali, et al. \cite{vrf_micali}, which employs \ora{Sign} in EUF-CMA-like
games, but says nothing for adversarially generated keys.
%
In \cite{praos}, there exists a UC functionality which captures the
desired PRF analogy, but brings unnecessary restrictions.

We now provide a (black-box) game-based definition which works by
counterintuitively treating \msg as the PRF key, and adversarially
supplied keys as PRF inputs, ala $\PRF_\msg : \pk \mapsto \Eval(\sk,\msg)$.

\begin{definition}
We say a VRF-AD-KC satisfies {\em pseudo-randomness} if 
any PPT adversary \adv has only a negligible advantage in $\secparam$
in this chosen-message game:
\begin{itemize}
\item[]
 Sample a random \msg, a random function $\rho$ with the same range as \Eval, and a bit $b$.
 \adv queries \ora{Verify} by providing both a public key \pk and
 a PPT (black-box) secret key algorithm $f_\sk$ such that
 if $(\openpk,\compk,\aux,\sigma) \leftarrow f_\sk(\msg)$ then
 $\pk = \OpenKey(\compk,\openpk)$ and
 $\Out \leftarrow \Verify(\compk,\msg,\aux,\sigma)$ succeeds (or aborts otherwise).
 \ora{Verify} always returns \Out and $\rho(\pk)$ with their order depending upon $b$.
 \adv wins by guessing $b$, aka by distinguish \Verify from $\rho$.
\end{itemize}
\end{definition}

Our pseudo-randomness winds up independent from residual pseudo-randomness
in \cite{vrf_micali}, even adapted to the key committing framework.
As converse to residual pseudo-randomness' weaknesses noted above,
an ordinary PRF satisfies both pseudo-randomness and uniqueness, but
without being a signature.  Yet, residual pseudo-randomness plus
uniqueness yields unforgeability on \msg.  We caution that
residual pseudo-randomness plus uniqueness says nothing about \aux, so
even if residual pseudo-randomness users require explicit unforgeability.

As in \cite{vrf_micali}, 
there exists a weaker {\em unpredictability} notion where \adv queries
only once, which defines a verifiable unpredictable function (VUF).
%
At least some works like \cite{agg_dkg} squeak by Micali's VUF and VRF's
weaknesses by threshold security arguments when generating randomness.

Also, if $H'(\cdot,k)$ is a PRF then computing $\Out = H'(\Verify(\cdots), \msg)$
transforms a VUF into a VRF, similarly to \cite[Proposition 1]{vrf_micali}.
It follows implementers should prefer VRFs over more subtle VUFs, assuming $H'$ is cheap.

% We handle cofactors explicitly in this work.  In particular, we impose
% a one-to-one map from secret keys \sk to PRFs $F_\sk$, thanks to
%  pseudo-randomness, but doing so imposes some subtleties and maybe overkill.
% TODO: Do we want this?  If so, explain better.

\smallskip


Although \cite[\S3.2 $\fvrf$]{praos} handles pseudo-randomness better,
they formalize VRFs with detached ouputs via the two algorithms
% \begin{itemize}
% \item
$\VRF.\primalgo{EvalProve}(\sk,\msg,\aux) \mapsto (\Out,\sigma)$, in which $\sigma = \VRF.\Sign(\sk,\msg,\aux)$ and $\Out = \VRF.\Eval(\sk,\msg)$, and
% \item
$\VRF.\primalgo{VerifyProof}(\pk,\msg,\aux,\Out,\sigma)$ which returns true only if $\VRF.\Verify(\pk,\msg,\aux,\sigma)$ returns $\Out$.
% \end{itemize}
We strongly prefer the \Sign and \Verify formulation from \cite{agg_dkg}
primarily because the \primalgo{EvalProve}, and \primalgo{VerifyProof}
formulation causes implementation and deployment mistakes:

EC VRF signatures have the form $\sigma = (\PreOut,\pi)$ in which some
inner proof $\pi$ proves correctness of some associated VUF output $\PreOut$. % aka ``pre-output''.  % ``pre-pseudo-random''
It follows $\VRF.\Eval$ never corresponds to $\PreOut$, but if one describes
protocols with an \primalgo{EvalProve} formulation then exposing $\PreOut$
invariably confuses developers into miss-using $\PreOut$ as the output.
% In other words, actual code never corresponds to an \primalgo{EvalProve} and \primalgo{VerifyProof} formulation.

The ``pre-output'' $\PreOut$ preserves algebraic relationships between
secret keys, so protocols described by the \primalgo{EvalProve} formulation
have implementations with broken pseudo-randomness, and perhaps
 related key vulnerabilities and mishandled cofactors.
% We need $\PreOut$ to be exposed by implementations so ...
We avoided the VUF formalism taken by \cite{agg_dkg} in part because
 VUFs obfuscate this difficulty to developers.

As a caveat, there exist UC formalisms that appear simpler with
the \primalgo{EvalProve} and \primalgo{VerifyProof} formulation, like in \cite{praos}.
We therefore propose that VRFs and protocols using VRFs should always be
described using the the \Sign and \Verify formulation, which provides
implementers with a sensible description, but then if needed adopt
 \primalgo{EvalProve} and \primalgo{VerifyProof} only inside the UC formulation itself.
We feel imposing this mental translation upon paper authors and reviewers
 beats imposing the reverse upon developers with less cryptographic knowledge.



\endinput 



\smallskip

There exist VUFs like RSA-FDH or BLS signatures that lack auxiliary data
% There even exist bespoke VRFs that relax correctness to some non-trivial
% relation on the space of secret keys and messages,
%  seemingly including some Rabin variants. 
Yet, these all suffer from either large signature sizes (RSA) or
 slow verification (BLS).
%  VRFs like single-layer XMSS, .

Instead, one prefers instantiating VRFs similarly to
 \cite{nsec5} or \cite{VXEd25519} using Chaum-Pedersen DLEQ proofs \cite{CP92} % Or should it be CP93 ??
 because they provide both small signatures and fast verification.
In these, our auxiliary data \aux can be verified for free,
by binding \aux into the challenge hash, like a Schnorr signature.
VRF protocols could often reduce bandwidth and verifier time this way,
 but some like Sassafras depend upon \aux. 





\endinput % no UC VRF discussion here






Also, pseudo-randomness in \cite{vrf_micali} merges \Eval and \Sign.















\begin{definition}
We say a VRF-AD-KC satisfies {\em residual pseudo-randomness} if 
any PPT adversary \adv has only a negligible advantage in $\secparam$
in this chosen-message game:
\begin{itemize}
\item[]
  \adv receives $\pk$ from \ora{Sign} of Definition \ref{def:vrf_sign_oracle},
  repeatedly queries \ora{Sign}, and produces $\compk,\openpk,\msg,\aux$. 
  If \adv never queried $\ora{Sign}(\cdot,\cdot,\msg,\cdot)$ then
  \adv wins by distinguishing $\Eval(\sk,\msg)$ from random.
\end{itemize}
\end{definition}
% TODO: Actually not quite right!
%












\subsection{UC}

TODO:  Should we give a relatively simple non-harmful UC functionality here?

TODO:  Can we prove this simpler UC functionality from the game?  Can our proof be close to the Praos proof?  If not then why not?











\section{Thin batchable EC VRF-AD}
\label{sec:vrf_thin}

There are VRFs built upon Chaum-Pedersen DLEQ proofs in elliptic curves,
 like \cite{nsec5} and \cite{VXEd25519}.
Yet typically their proofs have one challenge scalar and one signature
scalar, like a non-batchable Schnorr signature, while 
their verification demands two elliptic curve multi-exponentiations.
As a result, their naively batchable variant becomes ``fat'', requiring
one scalar and two separate nonces, which complicates batch verification.

We propose a new ``thin'' batchable EC VRF that merges these two nonces,
and requires only one elliptic curve multi-exponentiation, but
internally computes an extra delinearization challenge.
As such, our thin batchable variant both runs faster and simplifies heavy usage,
and also provides a less annoying interface.

Interestingly, our thin batchable EC VRF winds up literally being
a tweaked Schnorr-like signature, which opens new proof strategies and
use cases.  After this section, we abandon this thin batchable VRF
because our ``fat Pedersen'' variant introduced next in \S\ref{sec:vrf_pederson}
fits zero-knowledge continuations better.

\smallskip

\newcommand{\ThinVRF}{\primalgo{ThinVRF}} 

%PoK: We achieve this by always providing a proof-of-knowledge in the public key,
%PoK:  either separately or implicitly.

We build only a VRF-AD here which lacks the key commitments of \S\ref{sec:vrf},
meaning $\ThinVRF.\CommitKey$ and $\ThinVRF.\OpenKey$ are trivial, and $\openpk = \emptyset$.

We work solely in $\ecE$ here because we need only a basic Chaum-Pedersen DLEQ proof.
As in \S\ref{sec:lambda} and throughout,
 $\ecE$ has order $h p$ with $p \approx 2^{2\lambda}$ prime and $h$ a small cofactor.

\begin{itemize}
\item $\ThinVRF.\KeyGen$ selects a secret key \sk uniformly at random from $\F_p$ and computes the public key $\pk = \sk \, \genE$.
%PoK:  and attaches a proof-of-knowledge of $\pk$ to $\pk$ given by a Schnorr signature.  
%PoK:  All public keys must contain a valid proof-of-knowledge, or else be rejected by verifiers.
% We define equivalence $\pk_1 \equiv \pk_2$ of public keys by $h \pk_1 = h \pk_2$.
\item $\ThinVRF.\Eval(\sk,\msg)$ takes a secret key \sk and an input $\msg$, and
 then returns a VRF output $H'(\msg,\pk,h \, \sk \, H_{\grE}(\msg,\pk))$.
\item $\ThinVRF.\Sign(\sk,\msg,\aux)$ takes a secret key \sk, an input $\msg$, and auxiliary data \aux, and then performs
\begin{enumerate}
    \item compute the VRF input $\In := H_{\grE}(\msg,\pk)$ and pre-output output $\Out_0 := \sk \, \In$, 
    \item compute the delinearization challenge $c_1 = H_p(\aux,\msg,\pk,\Out_0)$,
    \item sample $r$ uniformly at random from $\F_p$ and compute $R = r (\genE + c_1 \In)$,
    \item compute the challenge $c = H_p(\aux,\msg,\pk,\Out_0,R)$, the proof $s = r + c \, \sk$, and return the signature $\sigma = (\Out_0,R,s)$.
\end{enumerate}
\item $\ThinVRF.\Verify$ takes $(\pk,\msg,\aux,\sigma)$, parses $\sigma = (\Out_0,R,s)$, and then 
\begin{enumerate}
%PoK:    \item abort unless either $\msg$ contains $\pk$ or else \pk has a valid the proof-of-knowledge,
    \item recomputes the VRF input point $\In := H_{\grE}(\msg,\pk)$,
    \item recomputes $c_1 = H_p(\aux,\msg,\pk,\Out_0)$ and $c = H_p(\aux,\msg,\pk,\Out_0,R)$, % the challenges
    \item aborts unless $s h (\genE + c_1 \In) = h R + c h (\pk + c_1 \Out_0)$ holds, and 
    \item returns $H'(\msg,\pk,h \Out_0)$ if nothing failed.
\end{enumerate}
\end{itemize}
As discussed above, if we omit this final hash $H'$ making
our output only $h \Out_0$, then we obtain only a VUF, not a VRF.
We caution that $h \ne 1$ ensures SUF-CMA fails
 by \cite[\S4.1.2]{cryptoeprint:2020:823}.

If desired, one could generalize \ThinVRF to $k$ messages by
computing for $j=1,\ldots,k$ the $k$ distinct
points $\In_j := H_{\grE}(\msg_j)$, pre-outputs $\Out_0 := \sk \, \In$,
delinearization challenges
 $c_j = H_p(\aux,\msg_1,\ldots,\msg_k,\pk,\Out_{0,1},\ldots,\Out_{0,k},j)$,
and then running our Schnorr-like signature with
 base point $\genE + \sum_{j=1}^k c_i \In_j$ and
 public key $\genE + \sum_{j=1}^k c_i \Out_j$.

% TODO: Proof correct?  Use same citations as schnorrkel.

% We define $H_\grG(\msg) = h H_\grE(\msg)$ and observe 
%
% \begin{lemma}
% If $H_\grE$ is a random oracle then $H_\grG$ is also a random oracle.
% \end{lemma}

% \begin{lemma}
% $\primalgo{PreEval}(\sk,\msg) = h \sk H_{\grE}(\msg,\pk)$
% \end{lemma}

We discuss chosen message queries against only one key in pseudo-randomness.  
% TODO: What?
In \ThinVRF, we hash the public key \pk along with the message \msg
in $H_\grE$, aka injected \pk into \msg, to prevent
related but different keys having algebraically related input points \In.
We cannot employ this trick in key committing VRFs or ring VRFs however.
Although $H'$ being a PRF mitigates problems, we still recommend caution 
when combining identical inputs \msg with related secret keys,
 like ``blockchain'' users often produce with ``soft key derivation''.

\begin{proposition}\label{prop:thin_vrf}
Assuming AGM in $\grE$, % $\ecE$ modulo $h$,
our $\ThinVRF$ satisfies
 VRF correctness, uniqueness, pseudo-randomness,
 and existential unforgeability on $(\msg,\aux)$.
\end{proposition}

\begin{proof}[Proof sketch]
TODO: ???
\end{proof}


\endinput




We expect $\ThinVRF$ to be an EUF-CMA signature scheme on $(\msg,\aux)$ too,
but proving this requires techniques outside our scope, even assuming AGM.

\begin{proof}[Proof sketch]
Correctness holds trivially.

At any fixed $\msg$ we have a Schnorr signature on $\aux$
 over the basepoint $\genE + c_1 \In$, which we name $\primalgo{VRFInner}_{\msg}$.
According to \cite[\S5]{cryptoeprint:2020:823},
 $\primalgo{VRFInner}_{\msg}$ is EUF-CMA secure,
 thanks to our random oracle assumption on $H_p$.

We consider an adversary that produces $(\pk,\msg,\aux,\sigma)$
 that pass verification, without knowing $\sk$.  
%PoK:  Ignoring the first abort path, and employing our random oracle assumption on $H_p$, 
We know from EUF-CMA security of $\primalgo{VRFInner}_{\msg}$ that
our forger knows some $w$ such that
 $h (\pk + c_1 \Out_0) = h w (\genE + c_1 \In)$.
We deduce from AGM knowledge of $x,y,u,v \in \F_p$ such that
 $\pk = x \genE + y \In$ and $\Out_0 = u \genE + v \In$
 with $x + c_1 u = w$ and $y + c_1 v = c_1 w$ in $\F_p$,
 so $c_1^2 u + c_1 (x-v) - y = 0$, except with odds negligible in $\lambda$.
At most two $c_1 \in \F_p$ satisfy this equation.
As our $c_1$ depends upon both \pk and $\Out_0$, 
it again follows from our random oracle assumption on $H_p$ that
 $u=0=y$ and $v = w = x \equiv \sk \bmod h$, meaning $\Out_0 = \sk \In$,
 except with odds negligible in $\lambda$.
%TODO: What do we cite here?
%PoK:
%PoK: We know $y=0$ if we check the proof-of-knowledge for $\pk$ of course.  
%PoK: We usually suggest that \pk appear in $\msg$ as a defense against related keys, 
%PoK: which occur if say \pk represents some account key on a blockcahin.  
%PoK: In this case, we also know $y=0$ by the random oracle assumption on $h$.  
%PoK: We even deduce $y=0$ after verifying two VRF signatures with distinct
%PoK: inputs $\msg_1$ and $\msg_2$ and hence distinct $\In_1$ and $\In_2$.
%PoK: We know $y=0$ in all cases, as desired.
%PoK: 
It follows that $\ThinVRF$ satisfies uniqueness of course. 

An unpredictability adversary $\adv$ guesses
 a \msg and corresponding pre-output $h \Out_0 := h \sk H_\grE(\msg)$,
after making chosen message queries to \Sign.
In AGM, $\adv$ must express its guess for $h \Out_0$
 in terms of $H_\grE(\msg)$ and points arising earlier.
???  So simple ???
As $H_\grE$ is a random oracle, we deduce that either
 $\adv$ solved the discrete logarithm problem, or else
 unpredictability holds for $\ThinVRF$.
As $H'$ is a PRF, we now argue pseudo-randomness for$\ThinVRF$ similarly
 to \cite[Proposition 1]{vrf_micali}.
\end{proof}
% An adversary cannot discover $\Out_0$ without querying $\msg$,
% % \cite[Theorem 6]{coron02}
% % https://eprint.iacr.org/2001/062.pdf NOT https://www.iacr.org/archive/eurocrypt2002/23320268/coron.pdf
% but our EUF-CMA game permits doing so with alternative $\aux$. 
% ...
%TODO: Actually this gets really long winded. 

%PoK:  In this, we still have a VRF if $y=0$, but not exactly the one specified.  
We caution that omitting $c_1$ only demands $x + u = v$ even if $y=0$,
which does not give a VRF.

We need two scalar multiplications in the prover and
 then four scalar multiplications in the verifier
 just like \cite{nsec5} or \cite{VXEd25519} do.
We do incur an extra hashing operation and two field multiplications,
 but they cost relatively little.
%PoK: At frist blush, we incur two more scalar multiplications when verifying
%PoK: the proof-of-knowledge for \pk too,
%PoK:  which one implements by a Schnorr signature. 
%PoK: Yet, VRF applications always require their public keys be registered in advance,
%PoK: meaning the proof-of-knowledge should be checked in advance and amortizes.
%PoK: 
We believe this approach actually reduces verifier computation because
advanced multi-scalar multiplication algorithms become more efficient when
larger, which should outweigh the extra hashing and field operations.

We also support batch verification without altering the signature or
increasing the signature size.  We think this tips the scales because
avoiding a separate batchable VRF signature type simplifies interface
over naive batch verification methods for \cite{nsec5} or \cite{VXEd25519}.

Aside from batch verification, we might simplify interaction with
other protocols by building upon Schnorr signatures.




\endinput



\section{UC}





\subsection{Pederson keyed VRF-AD-KC}
\label{sec:vrf_perderson}

Anonymized VRF signatures could use Chaum-Pedersen DLEQ proofs that
hide the true public key, which our key committing definition permits.
Zero-knowledge continuations work by opening this key comittment
 inside another zero-knowledge proof. % DUP

We only describe the non-batchable variant analogous to
 \cite{nsec5} and \cite{VXEd25519}.
At least two easy batch verifiable flavors exist, but
 they enlarge the VRF signature by 32 bytes.
We need pairings to verify the SNARKs component of our ring VRF,
which dwarfs the CPU savings from batch verification,
so saving 32 bytes sounds more helpful.

% TODO: Alternatives, like slow recursion.  Just Intro?

\newcommand{\PedVRF}{\primalgo{PedVRF}} 

We define \KeyGen like for \ThinVRF in \S\ref{sec:vrf_thin} but \Eval differs slightly:
% by selecting a secret key \sk uniformly at random from $\F_p$ and
% computing the public key $\pk = \sk \, \genG$.
\begin{itemize}
\item $\PedVRF.\KeyGen$ selects a secret key \sk uniformly at random from $\F_p$ and computes the public key $\pk = \sk \, \genG$. 
\item $\PedVRF.\Eval(\sk,\msg)$ takes a secret key \sk and an input $\msg$, and
 then returns a VRF output $H'(\msg,h \, \sk \, H_{\grE}(\msg))$.
\end{itemize}
We seleted here a base point $\genG$ for our public key
 because this avoids some confusion later.

We fix a second generator $\genB$ of $\grE$ independent from $\genG$, perhaps
 created by applying $H_\grE$ to an input outside existing usages' domain. 
We now form Pedersen-like commitments to this public key \pk as follows.
\begin{itemize}
\item $\PedVRF.\CommitKey$ selects a blinding factor $\openpk$ uniformly
 at random from $\F_p$ and computes the commitment $\compk = \pk + \openpk \, \genB$.
\item $\PedVRF.\OpenKey$ just returns $\pk = \compk - \openpk \, \genB$.
\end{itemize}
In fact, these are technically only Pedersen commitments to
the secret key \sk, not to the public key \pk, because
 \OpenKey does not enforce the structure of \pk.
Instead, our \Verify proves correctness of \compk, as
 required for our key binding condition, so
zero-knowledge continuations then use \OpenKey with only minor caveots.
% In particular, any VRF-AC-KC prevents adversaries from obtaining
% extra valid outputs, but ring VRF protocols need a proof-of-knowledge
% for the public key \pk if they demand that different \pk represent
% different functions.

\begin{itemize}
\item $\PedVRF.\Sign(\sk,\openpk,\msg,\aux)$ takes a secret key \sk and blinding factor \openpk, an input $\msg$, and auxiliary data \aux, and then performs
\begin{enumerate}
    \item compute the VRF input point $\In := H_{\grE}(\msg)$ and pre-output $\Out_0 := \sk \, \In$,
    \item Sample random $r_1,r_2 \leftarrow \F_p$ and compute $R = r_1 \genG + r_2 \genB$ and $R_\msg = r_1 \In$.
    \item Compute the challenge $c = H_q(\aux,\msg,\compk,h \Out_0,R,R_m)$,
     along with $s_1 = r_1 + c \sk$ and $s_2 = r_2 + c \, \openpk$.
    \item Return the signature $(\Out_0,c,s_1,s_2)$.
\end{enumerate}
\item $\PedVRF.\Verify(\compk,\msg,\aux,\sigma)$, parses $\sigma = (\Out_0,c,s_1,s_2)$, and then 
\begin{enumerate}
    \item recompute the VRF input point $\In := H_{\grE}(\msg)$,
    \item computes $R = s_1 \genG + s_2 \genB - c \compk$ and $R_m = s \In - c \Out_0$, and
    \item returns $H'(\msg, h \Out_0)$ if $c = H_q(\aux,\msg,\compk,\Out_0,R,R_\msg)$.
\end{enumerate}
\end{itemize}

We obtain roughly \cite{nsec5} or \cite{VXEd25519}
if we choose $\openpk = 0 = r_2$ in \Sign.

\begin{lemma}\label{prop:pedersen_vrf_hiding}
$\PedVRF$ is a correct key commitment and (perfectly) key hiding.
\end{lemma}


\begin{proposition}\label{prop:pedersen_vrf}
Assuming AGM in $\grE$, % $\ecE$ modulo $h$,
our $\PedVRF$ satisfies VRF correctness, key binding, uniqueness,
pseudorandomness, and unforgability. % (EUF-CMA-KC) on $(\msg,\aux)$.
\end{proposition}



\begin{proof}[Proof sketch]
Correctness holds trivially.

???
\end{proof}






\endinput








\section{Ring VRFs}
\label{sec:rvrf_def}

Anonymized VRFs prove a PRF output for some secret key, like VRFs do,
but without identifying the signer beyond revealing the specific PRF output.
Instead, anonymized VRFs prove only an authorization condition for
the secret key.

\def\comring{\ensuremath{\mathsf{comring}}\xspace}
\def\openring{\ensuremath{\mathsf{opring}}\xspace}
\newcommand{\CommitRing}{\primalgo{CommitRing}}
\newcommand{\OpenRing}{\primalgo{OpenRing}}

Among anonymized VRFs, ring VRFs are ring signatures broadly interpreted,
meaning they prove an associated public key lies inside some signer set
commitment \comring which anyone could construct from a set of public keys.

We focus exclusively upon ring VRFs becuase otherwise anonymity and
authorizations conditions come in too many flavours.
In particular, there are group VRFs analogous to group signatures
 \cite{group_sig_survey}, some fairly efficient.
%
We distrust group signatures however because typically some group master
has deanonymization capabilities, giving them only narrow real world uses,
like mailbox spam defense in Pond \cite{pond}.
%
In this vein, attribute based credentials are frequently instantiated
like group signatures, and thus similarly permit deanonymization by some
group master, which makes them similarly unsuitabe for general purpose
privacy.
%
There exist exceptions like credentials using Brands \cite{brands}, and
group signatures with byzantine anonymity \cite{cryptoeprint:2021:181}.

Ring VRFs avoid deanonymization by a group master, and impose coherent
structure upon authorization conditions, with all signers' keys
predating the authorization condition commitment \comring. 
% TODO: Relax in \S\ref{subsec:..}
%
Ring VRF outputs clearly link any two ring VRF signatures with
the same signer and input \msg, so in particular
 ring VRFs with a singleton input domain are linkable ring signatures.

\smallskip

A {\em ring verifiable random function with auxiliary data} (rVRF-AD)
consists of the algorithms of a VRF-AD-KC, except with
 \compk and \openpk renamed to \comring and \openring,
 plus one additional algorithm:
\begin{itemize}
\item $\rVRF.\CommitRing : \ctx \mapsto \comring$ takes a set \ctx of
 public keys and returns a public key set commitment \comring.
\end{itemize}

\def\rSign{\Sign}
\def\rVerify{\Verify}

In this, we have rename the commitment and opening to avoid confusion
when we build a rVRF-AD from a VRF-AD-KC.  This fresh notation leaves
$\rVRF.\KeyGen$ and $\rVRF.\Eval$ untouched, but
 changes the other methods' signatures:
\begin{itemize}
\item $\rVRF.\CommitKey : (\pk,\ctx) \mapsto (\comring,\openring)$
\item $\rVRF.\OpenKey : (\comring,\openring) \mapsto \pk$
\item $\rVRF.\rSign : (\sk,\openring,\msg,\aux) \mapsto \sigma$
\item $\rVRF.\rVerify : (\comring,\msg,\aux,\sigma) \mapsto \Out \, \lor \perp$
\end{itemize}

...


\subsection{rVRF-AC security}

We extend the VRF-AC-KC commitment correctness condition for \CommitRing:

\begin{definition}
We say rVRF satisfies {\em ring commitment correctness} if
commitment correctness holds, and also $\rVRF.\CommitRing$ is 
 compatable with $\rVRF.\CommitKey$ in that
  $\rVRF.\CommitRing(\ctx) = \rVRF.\CommitKey(\pk,\ctx).0$.
\end{definition}

We lack anonymity against full key exposure ala
 \cite[pp. 6 Def. 4]{cryptoeprint:2005:304} of course, due to the VRF output,
but instead demand a weaker anonymity condition similar to
 \cite[pp. 5 Def. 3]{cryptoeprint:2005:304}:

\begin{definition}
We say \rVRF satisfies {\em ring anonymity} if
any PPT adversary $\adv$ has an advantage only
 negligable in $\lambda$ to win the game:
\begin{itemize}
\item[]
 Assume a set $\ctx_0 = \{ \pk_1,\ldots,\pk_n \}$ of public keys from
 distinct key pairs $(\pk_i,\sk_i) \leftarrow \KeyGen$ for $i=1,\ldots,n$ with $n \ge 2$.
 %
 $\adv$ has signing oracles
 $\ora{Sign} : (i,\ctx',\msg',\aux') \mapsto
  \rSign(\sk_i,\CommitKey(\pk_i,\ctx').1,\msg',\aux')$
 for each secret key $\sk_i$ with $i=1,\ldots,n$, and
 their choice of $\ctx',\msg',\aux'$ with $\pk_i \in \ctx'$.
 % as well as the various algorithms. 
 %
 First $\adv$ outputs a ring $\ctx \subset \ctx_0$,
 a message \msg, associated data \aux, and
 indices $i_0,i_1 \le n$ such that $\pk_{i_0},\pk_{i_1} \in \ctx_0$ and 
 $\adv$ never invoked $\ora{Sign}$ on \msg and $i_j$ for $j=0,1$.
 Next the challenger fixes $j=0$ or $j=1$ and gives
  $\adv$ a signature $\sigma = \rSign(\sk_{i_j},\openring,\msg,\aux)$.
 Finally $\adv$ wins by guessing $j$ correctly
\end{itemize}
\end{definition}

We similarly a ring unforgability resembling
 \cite[pp. 7 Def. 7]{cryptoeprint:2005:304}:

\begin{definition}
We say \rVRF satisfies {\em ring unforgability} if
any PPT adversary $\adv$ has an advantage only
 negligable in $\lambda$ to win the game:
\begin{itemize}
\item[]
 Assume a set $\ctx_0 = \{ \pk_1,\ldots,\pk_n \}$ of public keys from
 distinct key pairs $(\pk_i,\sk_i) \leftarrow \KeyGen$ for $i=1,\ldots,n$ with $n \ge 2$.
 %
 $\adv$ has signing oracles
 $\ora{Sign} : (i,\ctx',\msg',\aux') \mapsto
  \rSign(\sk_i,\CommitKey(\pk_i,\ctx').1,\msg',\aux')$
 for each secret key $\sk_i$ with $i=1,\ldots,n$, and
  their choice of $\ctx',\msg',\aux'$ with $\pk_i \in \ctx'$.
 % as well as the various algorithms. 
 %
 Now $\adv$ wins by outputing a valid signature $\sigma$ for
 a ring $\ctx \subset \ctx_0$, a message \msg, and associated data \aux
 such that
 $\adv$ never invoked $\ora{Sign}$ on $\msg,\aux$ and $i$ with $\pk_i \in \ctx$.
\end{itemize}
\end{definition}

We also need a uniqueness condition that limits even adersaries who know the secret key.

% \begin{definition}
% We say \rVRF satisfies {\em ring uniqueness} if
% any PPT adversary $\adv$ has an advantage only
%  negligable in $\lambda$ to 
% produces a ring $\ctx$, input $\msg$, 
% and various $\aux_i,\sigma_i$ with more than $|\ctx|$
% many verifying outputs $\Out_i = \Verify(\CommitRing(\ctx),\msg,\aux_i,\sigma_i)$.
% \end{definition}

\begin{definition}
We say \rVRF satisfies {\em ring unforgability} if
any PPT adversary $\adv$ who produces
 a ring $\ctx$, input $\msg$,  and various $\aux_i,\sigma_i$
cannot produce more than $|\ctx|$ many verifying outputs
$\Out_i = \Verify(\CommitRing(\ctx),\msg,\aux_i,\sigma_i)$,
 except with odds negligible in $\lambda$.
\end{definition}

Any ring VRF becomes a non-anonymized VRF whenever
 the ring becomes a singleton $\ctx = \{ \pk \}$ of course.
In doing so this, ring uniqueness reduces to uniqueness for VRF-AC,
meaning our uniqueness with only a trivial key commitment,
but \ThinVRF is faster and simpler.

We reuse the VRF-AC-KC pseudorandomness definition for ring VRFs
because pseudorandomness is strongest for singleton rings, i.e. $|\ctx| = 1$.


\subsection{rVRF-AC instantiation}

We instantiate a rVRF-AD from a hiding VRF-AD-KC like \PedVRF plus
a ring commitment scheme
 $\rVRF.\{ \CommitRing, \CommitKey, \OpenKey \}$
for which some zero-knowledge ring membership proof handles both
 $\PedVRF.\OpenKey$ and $\rVRF.\OpenKey$
efficently.

\newcommand\piring{\ensuremath{\pi_{\mathtt{ring}}}\xspace}

\begin{itemize}
\item $\rVRF.\rSign : (\sk,\openring,\msg,\aux) \mapsto \sigma$ takes
 a secret key \sk, a ring opening \openring, a message \msg, and auxiliary data \aux, and then \\
 \begin{enumerate}
 \item computes the ring membership proof $\piring$ and associated \openpk,
  $$ \piring = \NIZK \Setst{ \compk, \comring }{
  \exists \openpk,\openring \textrm{\ s.t.\ } 
  \genfrac{}{}{0pt}{}{\PedVRF.\OpenKey(\compk,\openpk) \quad}{\,\, = \rVRF.\OpenKey(\comring,\openring)}
  } $$
 \item computes the VRF-AC-KC signature
  $\sigma = \PedVRF.\Sign(\sk,\openpk,\msg,\aux \doubleplus \compk \doubleplus \piring)$, and finally
 \item returns the ring VRF signature $\rho = (\compk,\piring,\sigma)$.
 \end{enumerate}
\item $\rVRF.\rVerify$ takes $(\comring,\msg,\aux,\rho)$,
 parses $\rho$ as $(\compk,\piring,\sigma,)$,  and then returns
 $$ \PedVRF.\Verify(\compk,\msg,\aux \doubleplus \compk \doubleplus \piring,\sigma) $$
 iff $\NIZK.\Verify(\piring,\compk,\comring)$ succeeds. 
\end{itemize}

% \noindent
In this, we tie $\sigma$ to $\piring$ by expanding $\sigma$'s auxiliary data with $\piring$.

\smallskip

We now prove security of \rVRF using that
\begin{itemize}
\item \PedVRF is a secure hiding VRF-AC-KC, and that
\item our ring commitment scheme satisfies ring commitment correctness.
\end{itemize}

Pseudorandomness holds by pseudorandomness of \PedVRF from
 Proposition \ref{prop:pedersen_vrf}.

\begin{proposition}\label{prop:pedersen_rvrf}
$\rVRF$ satisfies ring uniqueness and ring unforgability.
\end{proposition}

\begin{proof}[Proof sketch]
???
\end{proof}


TODO:  Should we give an abstract pure NIZK instaniation here?  I think later probably.


\subsection{UC}

TODO: Should we gives Handan's UC functionality or similar here? 




\endinput




\noindent 
\begin{itemize}

\item {\em Ring } - 
  Assuming ,
 a PPT adversary $\adv$ cannot distinguish a random function from
  any evaluation map $F_{\sk_i} : \msg \mapsto \Eval(\sk,\msg)$,
 even given $\ctx_0 = \{ \pk_1,\ldots,\pk_i \}$ and
  chosen-message queries to $\rVRF.\rSign(\sk_i,\openring)$,
  except with odds negligible in $\lambda$.
\end{itemize}


\begin{definition}
We say \rVRF satisfies {\em ring pseudorandomness} if 
any PPT adversary $\adv$ has only a negigable advantage in $\lambda$
in this chosen-message game:
\begin{itemize}
 \item Frist, a challenger
  generates keypairs $(\pk_1,\sk_1),\ldots,(\pk_1,\sk_1) \leftarrow \KeyGen$ and
  defines a signing oracles $\ora{Sign}_i$ given by
  $(\compk,\openpk,\msg,\aux) \mapsto \Sign(\sk_i,\openpk,\msg,\aux)$,
   except it logs $\msg$ and aborts if $\pk_i \ne \OpenKey(\compk,\openpk)$.
 \item Next $\adv$ recieves the $\pk_i$, repeatedly queries $\ora{Sign}_i $,
  and produces $\compk,\openpk,\msg,\aux$. 
 \item If $\adv$ never queried $\ora{Sign}(\cdot,\cdot,\msg,\cdot)$ then
  $\adv$ wins by distinguishing $\Eval(\sk,\msg)$ from random value.
\end{itemize}
\end{definition}




\section{Zero-knowledge continuations}
\label{sec:rvrf_cont}

\newcommand\rrSNARK{\primalgo{Groth16}\xspace}
\newcommand\pifast{\ensuremath{\pi_{\mathtt{fast}}}}
\newcommand\pisafe{\ensuremath{\pi_{\mathtt{safe}}}}

We now construct ring VRFs which achieves fast amortized prover time
by using a heavy zero-knowledge continuation for $\rVRF.\OpenKey$ but
which permits updating \openpk in the $\PedVRF.\OpenKey$ invokation
without reproving $\pi$.
$$ \pi = \NIZK \Setst{ \compk, \comring }{
 \exists \openpk,\openring \textrm{\ s.t.\ } 
 \genfrac{}{}{0pt}{}{\PedVRF.\OpenKey(\compk,\openpk) \quad}{\,\, = \rVRF.\OpenKey(\comring,\openring)}
} \mathperiod $$

% \smallskip
\subsection{Rerandomization}
% \label{sec:rvrf_groth16}

Zero-knowledge continuations need rerandomizable zkSNARKs
when being reused multiple times, meaning Groth16 \cite{groth16},
but unlinkability requires more than merely rerandomization.

In Groth16 \cite{groth16}, we have an SRS $S$ consisting of curve
points in $\grE_1$ and $\grE_2$ that encode the circuit being proven.
We follow \cite{groth16} in discussing the SRS $S$ in terms of
its ``toxic waste''
 $(\alpha,\beta,\delta,\gamma,\tau^1,\tau^2,\ldots) \in \F_q^*$.
In other words, we could write say $[ f(\tau)/\delta ]_1$ or $[\cdots]_2$
to denote an element of our SRS $S$ in $\grE_1$ or $\grE_2$ respectively,
computed by scalar multiplication from the toxic waste $\tau$ and $\delta$,
 but for which nobody knows the underlying $\tau$ or $\delta$ anymore.

In the SRS $S$, we distinguish the verifiers' string of elements
 $Y_1,\ldots,Y_k, [\alpha]_1 \in \grE_1$ and
 $[\beta]_2, [\gamma]_2, [\delta]_2 \in \grE_2$.
% as seperate from the provers' much longer string of elements in $\grE_1$ and $\grE_2$.
A Groth16 \cite{groth16} proof then takes the form 
 $\pi = (A,B,C) \in \grE_1 \times \grE_2 \times \grE_1$.
A verifier then produces a $X = \sum_i^k x_i Y_i \in \grE_1$ from
 the public inputs $x_i$ and then checks 
$$ e(A,B) = e([\alpha]_1, [\beta]_2) \cdot
 e(X, [\gamma]_2) \cdot e(C, [\delta]_2) \mathperiod $$

We need the rerandomization algorithm from \cite[Fig.~1]{RandomizationGroth16}:
% to build a zero-knowlege continuation:
% https://eprint.iacr.org/2020/811
% https://github.com/arkworks-rs/groth16/pull/16/files
% \algo{rerandomize}
An existing SNARK $(A,B,C)$ is transformed into a fresh
SNARK $(A',B',C')$ by sampling random $r_1,r_2 \in \F_p$ and computing
$$ \begin{aligned}
A' &= {1 \over r_1} A \\
B' &= r_1 B + r_1 r_2 [\delta]_2 \\
C' &= C + r_2 A \mathperiod \\
\end{aligned} $$
At this point, our $x_i$ remain identical after rerandomization,
so $X$ links $(A,B,C)$ to $(A',B',C')$.
Alone rerandomization cannot alter public inputs $x_i$, so
we instead need an opaque public input point $X$, which then becomes
part of our proof and incurs its own seperate proof of correctness.

We also need one fresh basepoint $\genB_\gamma$ independent from all others,
again perhaps created by applying $H_\grE$ to an input outside existing usages' domain.
We now give provers the additional SRS elements
$$ \genB_\delta := {\gamma\over\delta} \genB_\gamma $$
Although $\genB_\gamma$ is independent, 
we create $\genB_\delta$ during the trusted setup,
 so the toxic waste $\gamma$ and $\delta$ remain secret.
After this, subversion resistance could be checked like 
$$ e(\genB_\gamma, [\gamma]_2) = e(\genB_\delta, [\delta]_2) \mathperiod $$

We now have a zero-knowledge continuation $(X,A,B,C)$ from which
we produce an unlinkable $(X',A',B',C')$ by
 first sampling random $b,r_1,r_2 \in \F_p$ and then computing
$$ \begin{aligned}
X' &= X + b \genB_\gamma \\
A' &= {1 \over r_1} A \\
B' &= r_1 B + r_1 r_2 [\delta]_2 \\
C' &= C + r_2 A + b \genB_\delta \mathperiod \\
\end{aligned} $$
As our two $b$ terms cancel in the pairings, we wind up with the standard Groth16
 rerandomization construction above, except with $X$ opaque.

We still verify $(X',A',B',C')$ like 
$$ e(A',B') = e([\alpha]_1, [\beta]_2) \cdot
 e(X', [\gamma]_2) \cdot e(C', [\delta]_2) \mathcomma $$
As our verifier does not build $X$ themselves, we proves nothing
with this pairing equation unless the verifier seperately checks
 some proof-of-knowledge that $X' = \sum_i^k x_i Y_i$.
We foresee some $x_i$ being transperent elements that determine the
Merkle root of the ring $\ctx$, but any $x_i$ concerning the
 specific $\pk$ must be hidded by out blinding terms in $b$.

All told, our opaque rerandomization trick transforms some conventional
Groth16 zkSNARK $\pi_{\mathtt{inner}}$ for $\rVRF.\OpenKey$
into a zkSNARK $\pi$ with an opaque and unlinkable input $X'$.
We therefore explore two concrete $\pi_{\mathtt{inner}}$ proposals next.

Importantly, rerandomization requires only
 four scalar multplicaitons on $\ecE_1$ and
 two scalar multplicaitons on $\ecE_2$,
which  BLS12 curves make roughly equivlent to
 eight scalar multplicaitons on $\ecE_1$.

% Intutitively, an adversary cannot link $(X,A,B,C)$ with $(X',A',B',C')$ because 

\begin{proposition}\label{prop:unlinkable}
Let $\sigma$ and $\sigma'$ denote Chaum-Pedersen proofs-of-knowledge
 for $X$ and $X'$ respectively, with nonces chosen randomly.
Then $(X,A,B,C,\sigma)$ and $(X',A',B',C',\sigma')$ are unlinkable.
\end{proposition}

\begin{proof}[Proof stetch.]
???
\end{proof}

\subsection{Faster}
\label{subsec:rvrf_faster}

We describe the preferred faster choice $\pifast$ for $\pi_{\mathtt{inner}}$
that sets $x_1 := \sk$ and $x_0 = \CommitRing(\ctx)$ so that
taking $\genG := Y_1$, $\genB := \genB_\gamma$, and $\openpk := b$ in \PedVRF
yields and incredible fast amortized ring VRF prover.
Also, \PedVRF itself proves knowledge of $X$.
$$ X = \CommitRing(\ctx)\, Y_0 + \sk\, Y_1 + b \genB_\gamma $$

A priori, we do not know $Y_1$ during the trusted setup for $\pifast$,
which prevents computing $\pk = \sk\, Y_1$ inside $\pifast$.
Instead, we propose $\ctx$ contain commitments to $\sk$ over
some Jubjub curve $\ecJ$.  

We know $\grJ$ typically has smaller order than $\grE$,
due to $\ecJ$ being an Edwards curve, but 
if $\sk = \sk_0 + \sk_1 \, 2^{128}$ then our public key commitments could
take the form $\sk_0\, \genJ_0 + \sk_1\, \genJ_1 + d \genJ_2$,
with independent $\genJ_0,\genJ_1,\genJ_2$.
Interestingly, we avoid range proofs for $\sk_1$ and $\sk_2$
by this independence. 

$$ \pifast = \rrSNARK \Setst{ \sk_0 + \sk_1 2^{128}, \comring }{
 \exists d,\openring \textrm{\ s.t.\ }
 % 0 < \sk_0,\sk_1 < 2^{128} \textrm{\ and\ } 
 \genfrac{}{}{0pt}{}{\rVRF.\OpenKey(\comring,\openring)}{\,\, = \sk_0 \genJ_0 + \sk_1 \genJ_1 + d \genJ_2}
} $$ % \mathperiod 

We explain later in \S\ref{sec:ring_hiding} how one could
choose $Y_1$ independent before doing the trusted setup,
 and then wire $Y_1$ into $\pifast$ inside $C$.
In ths case, we could prove $\pk = \sk\, Y_1$ inside $\pifast$, but then
non-native arithmetic makes $\pifast$ far slower.

At this point, \PedVRF requires four scalar multiplications on $\ecE_1$,
so together with rerandomization our amortized prover time
 approaches 12 scalar multiplications on typical curves. 
We expect the three pairings dominate verifier time.

As an aside, one could construct a second faster curve with the same
group order as $\grE$, which speeds up one scalar multiplication
 in both the prover and verifier. 

Importantly, our fast ring VRF' amortized prover time now rivals
group signature schemes' performance.  We hope this ends the temptation
to deploy group signature like constructions where the deanonymization vectors matter.

\begin{proposition}\label{prop:pifast_anonymity}
\rVRF using \pifast satisfies ring anonymity.
\end{proposition}

\begin{proof}[Proof stetch.]
???
\end{proof}

\subsection{Safer}

Although blindingly fast, we processed $\sk$ directly inside $\pifast$,
which annoys those wanting lightweight HSM provers, and
increases side channel attack risks.

We could easily build a safer circuit in which
\PedVRF runs on, and $\pk$ lies in, the Jubjub curve $\ecJ$, like 
$$ \pisafe = \rrSNARK \Setst{ \pk, \comring }{
 \exists \openring \textrm{\ s.t.\ }
 \pk = \rVRF.\OpenKey(\comring,\openring)
} \mathperiod $$
In this, our $\genG$ and $\genB$ have no relation with $\ecE$ or \pisafe,
so our $X$ nolonger plays less nicely with \PedVRF, 
$$ X = \CommitRing(\ctx) Y_0 + \pk.x Y_1 + \pk.y Y_2 + b \genB_\gamma \mathperiod $$

Instead we strip the blinding $b \genB_\gamma$ using a second SNARK $\pisafe'$.
As our second SNARK $\pisafe'$ cannot be another zero-knowledge continuation
itself, because it knows the $\openpk$ of \PedVRF.
Yet, we keep $\pisafe'$ working on $\ecE$ by reusing the external blinding
trick again.  In other works, we construct an inner SNARK $\pisafe'$ with
its own $\delta'$ and $\gamma'$ and that processes
 transperent public inputs $\compk.x Y'_1 + \compk.y Y'_2$ and an opaque
$$ X' = \pk.x Y'_3 + \pk.y Y'_4 + b \genB_{\gamma'} \mathperiod $$

We need $\pisafe'$ to apply the $\ecJ$ blinding required by \PedVRF
ala $\pk = \PedVRF.\OpenKey(\compk,\openpk) = \compk - \openpk \genB$, so
$$ \pisafe' = \rrSNARK \Setst{ \compk, \pk }{
 \exists \openpk \textrm{\ s.t.\ } \compk = \pk + \openpk \genB
} \mathperiod $$
In this, we nolonger prove knowledge of $X$ in \PedVRF like $\pifast$ did.
Instead, we employ another Chaum-Pedersen DLEQ proof to wire
 the $X$ of $\pisafe$ to the $X'$ of $\pisafe'$,
 thus proving knowledge of both.
We cannot avoid this proof-of-knowledge, but it becomes simpler if
we choose $Y'_3 = Y_1$ and $Y'_4 = Y_2$ using the trick of \S\ref{sec:ring_hiding}.

We take $\aux \doubleplus \pisafe \doubleplus \pisafe'$
 to be the \aux of \PedVRF of course.

\begin{proposition}\label{prop:pisafe_anonymity}
\rVRF using \pisafe satisfies ring anonymity.
\end{proposition}

\begin{proof}[Proof stetch.]
???
\end{proof}

% \smallskip

As \openpk appears inside $\pisafe'$, we recompute $\pisafe'$ with
every ring VRF signature, but the elliptic curve addition requires
only 5ish constraints, and the elliptic curve scalar multiplication
requires under 750 ??? constraints. 
All told our amortized prover runs faster than a Groth16 prover
with 800 constraints.

A priori, our safer verifier requires five pairings, along with
some additional $\ecE_1$ scalar multiplications.
We conjecture $\gamma$ and $\delta$ could safely be shared between
$\pisafe$ and $\pisafe'$, thereby requiring only four pairings,
but caution this result appars non-trivial.

% \smallskip


\section{Application: Identity}
\label{sec:app_identity}

Anonymous VRFs yield anonymous identity systems:
After a user and service establish a secure channel and
the server authenticates itself with certificates, then
the user authenticates themselves by providing an anonymous
VRF signature with input \msg being the server's identity,
thus creating an anonymous or pseudo-nonymous identified session.

We expand this identified session workflow with an extra
update operation suitable for our ring VRF's amortized prover.
We discuss only \pifast here but all techniques apply to \pisk and \pipk similarly. 

\begin{itemize}
\item {\em Register} --
 Adds users' public key commitments into some public ring \ctx,
 after verifying the user does not currently exist in \ctx.
\item {\em Update} --
 User agents regenerate their stored SNARK \pifast every time \ctx changes,
 likely receiving \comring and \openring from some ring management service.
\item {\em Identify} --
 Our user agent first opens a standard TLS connection to a server \msg,
 both checking the server's name is \msg and checking certificate
 transparency logs, and then computes the shared session id \aux.
 Our user agent computes the user's identity
  $\mathtt{id} = \PedVRF.\Eval(\sk,\msg)$ on the server \msg,
 % Our user agent next rerandomizes \pifast, \compk, and \openpk, computes
 %  $\sigma = \PedVRF.\Sign(\sk,\openpk,\msg,\aux \doubleplus \compk \doubleplus \pifast)$
 and finally sends the server their ring VRF signature
 $\rVRF.\rSign(\sk,\openring,\msg,\aux)$ % $ = (\compk,\pifast,\sigma)$.
\item {\em Verify} -- 
 After receiving $(\compk,\pifast,\sigma)$ in channel \aux,
 the server named \msg checks \pifast on the input $\compk + \comring$,
 checks the VRF signature and obtains the user's identity $\mathtt{id}$.
 $$ \mathtt{id} = \PedVRF.\Verify(\compk,\msg,\aux \doubleplus \compk \doubleplus \pifast,\sigma) $$
\end{itemize}


\subsection{Browsers}

We must not link users' identities at different web sites, so user agents
must disable all cross site resource loading, referrer information, etc.
Yet, user agents could still load purely static resources, without metadata
like cookies or referrer information, especially purely content addressable
resources.

In other words, web browsers mostly fail these baseline privacy requirements.
We expect Tor browser and Brave both behave correctly however.
Apple's Safari trends towards preventing invasive cross site resources too.  
% There also do exist decentralized web aka web 3.0 projects whose stated aims
% include more privacy.
In any case, one could always specify rules against cross site privacy invasions
whenever writing ring VRF browser specifications.


\subsection{AML/KYC}\label{subsec:AML_KYC}

We shall not discuss AML/KYC in detail, because the entire field lacks
clear goals, and thus winds up being ineffective
 \cite{doi:10.1080/25741292.2020.1725366}.
% https://www.tandfonline.com/doi/full/10.1080/25741292.2020.1725366
% via https://twitter.com/ronaldpol/status/1491548352189587460
We do however observe that AML/KYC conflicts with security and privacy
laws like GDPR.  As a compromise between these regulations,
one needs a compliance party who know users' identities,
 while another separate service party knows the users' activities.
We propose this more efficient solution:

Instead our compliance party becomes an identity issuer who maintains
a ring \ctx consisting of one unique public key for each user.
Arbitrarily many service providers could ring VRF based identity proofs.
If later asked or subpoenaed, users could prove their relevant identities
to investigators, or maybe prove which services they use and do not use. 

Interestingly, \PedVRF could be run ``backwards'' to prove a specific
ring VRF output does not belong to the user, without revealing the users'
identities to investigators. 

Although our applications mostly ignore key multiplicity. 
AML/KYC demands suspects prove non-involvement using ring VRFs.

\begin{definition}\label{def:rvrf_exculpability}
We say \rVRF is {\em exculpatory} if we have an efficient algorithm
for equivalence of public keys, but a PPT adversary \adv cannot
find non-equivalent public keys $\pk_0,\pk_1$ with colliding VRF outputs.
% (perfectly or computationally)
% (either ever or with advantage negligible advantage in $\secparam$)
\end{definition}

We ad hoc rings make little sense for AML/KYC, so these ring VRFs become
exculpatory if one merely dedupliates keys during registration , like via
$\rVRF.\rVerify(\CommitRing(\{\pk\}),\mathsf{ring_name},\mathtt{""},\sigma)$.


\subsection{Moderation}
\label{subsec:moderation}

All discussion or collaboration sites have behavioral guidelines and
moderation rules that deeply impact their culture and collective values.

Our ring VRFs enables a simple blacklisting operation:
If a user misbehaves, then sites could blacklist or otherwise penalizes
their site local identity $\mathtt{id}$.
As $\mathtt{id}$ remains unlinked from other sites, we avoid thorny
questions about how such penalties impact the user elsewhere, and thus
can assess and dispense justice more precisely. 

At the same time, there exist sites who must forget users' histories
eventually, such as when users invoke GDPR compliance or to give children
room to make social mistakes.  In these cases, we suggest injecting
approximate date information into \msg along with the site name,
so \msg becomes site name along with the current year plus month or week.
In this way, users have only one stable $\mathtt{id}$ within the
approximate date range, but they obtain fresh $\mathtt{id}$s merely
by waiting until the next month or week.

As in \cite{PrivacyPass}, we could adjust \PedVRF to simultaneously
prove multiple VRF input-output pairs $(\msg_j,\mathtt{id}_j)$.
As doing so links these pairs together, sites could make users link
pseudo-nym creation date and current date, so users could have multiple
active pseudo-nyms, but only one active pseudo-nym per time period,
which prevents spam.
If instead we link only adjacent dates, then pseudo-nyms could
be abandoned and replaced, but abandoned pseudo-nyms cannot then
be reclaimed without linking to intervening dates.

In these ways, sites encode important aspects of their moderation rules
into the ring VRF inputs requested.  
% We expect this makes sites' values and culture more uniform, predictable, and transparent.


\subsection{Reduced pairings}
\label{sec:reduced_pairings}

At a high level, we distinguish moderation-like applications discussed
above, which resemble classic identity applications like AML/KYC, from
rate limiting applications discussed in the next section. 
%
In moderation-like applications, ring VRF outputs become long-term
stable identities, so users typically reidentify themselves many times
to the same sites.

As an optimization, our zero-knowledge continuation
should deterministically choose the coefficients $r_1,r_2,b$ used for
rerandomization in \S\ref{sec:rvrf_cont},
 seeded by \msg and \sk, meaning $r_1,r_2,b \leftarrow H(\sk,\msg)$. 
%
In this way, each $\mathtt{id}$ comes packages with the same unique % Groth16 SNARK
\pifast, so the verifier could cache valid pairs
$(\mathtt{id},H(\pifast),\mathtt{diffdate})$, and reaccept \pifast
without checking the Groth16 pairing equation whenever found cached.
%
We spend most verifier time checking the Groth16 pairing equation, so
this saves considerable CPU time. % assuming our cache wind up fast enough.

We still risk denial-of-service attacks by users who vary $r_1,r_2,b$ 
randomly however.  We therefore set $\mathtt{diffdate}$ to be the date
when our server last saw a different $H(\pifast)$ associated to
$\mathtt{id}$, or empty if $\mathtt{id}$ always used the same $H(\pifast)$.
We rate limit and verify more lazily if $\mathtt{diffdate}$ is non-empty,
and optionally verify somewhat lazily even if no cache entry exists.


\section{Application: Rate limiting}
\label{sec:app_rate_limits}

We showed in \S\ref{sec:app_identity} how ring VRFs give users only
one unique identity for each input \msg.  
We explained in \S\ref{subsec:moderation} that choosing \msg to be
the concatenation of a base domain and a date gives users a stream of changing identities.

We next discuss giving users exactly $n > 1$ ring VRF outputs aka
``identities'' per date, as opposed to the unique identity 


% \subsection{Implementation}

As a trivial implementation, we could include a counter $k = 1 \ldots n$
in \msg, so $\msg = \mathtt{domain} \doubleplus \mathtt{date} \doubleplus k$.


\subsection{Avoiding linkage}

Our trivial implementation leaks information about ring VRF outputs'
 ownership by revealing $k$:
%
An adversary Eve observes two ring VRF signatures with the same
$\mathtt{domain}$ and $\mathtt{date}$ so
$\msg_i = \mathtt{domain} \doubleplus \mathtt{date} \doubleplus k_i$
for $i=1,2$, but with different outputs $\Out_1$ and $\Out_2$.
If $k_1 \ne k_2$ then Eve learns nothing, but if $k_1 = k_2$ then
 Eve learns that $sk_1 \ne \sk_2$, maybe representing different users. 

We do not necessarily always care if Eve learns this much information,
but scenarios exist in which one cares.  We therefore briefly describe
several mitigations:

If $n$ remains fixed forever, then we could simply let all users
register $n$ ring VRF public keys in \ctx.
If $n$ fluctuates under an upper bound $N$, then we could create $N$
rings $\ctx_i$ for $i = 1 \ldots N$, and
 then blind \comring in \pifast similarly to \S\ref{sec:ring_hiding}.

Although simple, these two approaches require users construct $n$ or $N$
different $\pipk$ proofs every time the ring \ctx updates.

Instead of proving ring membership of one public key, $\pipk$ could
prove ring membership of a Merkle commitment to multiple keys, so
users have $\pisk^1,\ldots,\pisk^N$ for each of their multiple keys.

As a more flexible approach,
we could compute the hash-to-curve $\In := H_{\grE}(\msg)$ inside an
unamortized SNARK $\pi_{\mathtt{in}}$ and reveal only a Pedersen-like commitment
to $\In + \openpk^{-1} \genB$.  We then adjust \PedVRF to yield
a proof-of-knowledge of $\Out/\In$ subject to soundness of this
SNARK $\pi_{\mathtt{in}}$.

TOTO: Explain better!!

In all cases, we incur costs by hiding part of the input \msg, so
deployment should seriously consider if leaking $k$ suffices.


\subsection{Ration cards}

As a species, we expect $+3^{\circ}$ C or more likely $+4^{\circ}$ C
over the pre-industrial climate by 2100 \cite{IPCC}, which shall
reduce the Earth's carrying capasity below 1 billion people \cite{carrying_capasity}.
In the shorter term, we expect shortages of resources, energy, goods,
water, and food beginning during the next several decades, due to
climate change, ecosystem damage or collapse, and resource exhaustion
ala peak oil.  Invariably, nations manage shortages through rationing,
like during WWI, WWII, and the oil shocks.  

Ring VRFs support anonymous rationing:
Instead of treating ring VRF outputs like identities,
we treat them like nullifiers which could each be spent exactly once.

\def\expiry{e}
We fix a set $U$ of limited resource types, and dynamically define
an expiry date $\expiry_{u,d_0}$ and an availability $n_{u,d_0}$, 
both dependent upon the resource $u \in U$ and current date $d_0$.
We typically want a randomness beacon $r_d$ too, which prevents
anyone learning $r_d$ much before date $d$. 
% Among other usages, this reduces damages from key compromises.
As ring VRF inputs, we choose
 $\msg = u \doubleplus r_d \doubleplus d \doubleplus k$
where $u \in U$ denotes a limited resource,
 $d$ denotes an non-expired date meaning $\expiry_{u,d_0} < d \le d_0$,
 and $1 \le k \le n_{u,d_0}$.
In this way, our rationing system controls both daily consumption
via $n_{u,d_0}$ and time shifted demand via expiry time $\expiry_{u,d_0}$.

Importantly, our rationing system retains ring VRF outputs as nullifiers,
filed under their associated date $d$ and resource $u$, so nullifiers
expire once $d \le \expiry_{u,d_0}$ which permits purging old data rapidly.

We remark that fully transferable assets could have constrained lifetimes
too, which similarly eases nullifier management when implements using
blind signatures, zcash sapling, etc.  Yet, all these tokens require
an explicit issuance stage, while ring VRFs self-issue.

Among the political hurdles to rationing, we know certificates have
a considerable forgery problem, as witnessed by the long history of
fraudulent covid and TLS certificates.  It follows citizens would
justifiably protest to ration carts that operate by simple certificates.
Ring VRFs avoid this political unrest by proving membership in a public list.


\subsection{Multi-constraint rationing}

We could proving multiple \ThinVRF outputs with one signatures
in \S\ref{sec:vrf_thin}.  We needed \PedVRF to isolate the blinding 
factor when using a Pedersen commitment instead of a public key, but
exactly the same technique works for proving multiple \PedVRF outputs
in one ring VRF signature.

We could therefore impose simultanious rationing constraints for multiple
resources $u_1,\ldots,u_k$ by producing one ring VRF signature in which
\PedVRF proves correctness of pre-ouptuts for multiple messages 
 $\msg_j = u_j \doubleplus r_d \doubleplus d \doubleplus k$ for $j=1 \ldots k$.

As an example, purchasing some prepared food product could require espending
rations for multiple base food sources, like making a cake from wheat, butter,
eggs, and sugar.  


\subsection{Decommodification}

There exist many reasons to decommodify important services,
like energy, water, or internet,
 beyond rationing real physical shortage.
Ring VRFs fit these cases using similar \msg formulations.

As an example, a municipal ISP allocates some limited bandwidth capacity
among all residents.  It allocates bandwidth fairly by verifying ring VRFs
signatures on hourly \msg and then tracking nullifiers until expiry.

Aside from essential government services, commercial service providers
typically offers some free service tier, usually because doing so
familiarizes users with their intimidating technical product.

Some free and paid tier examples include DuoLingo's heats on mobile, 
continuous integration testing services, and many dating sites.

A priori, rate limiting cases benefit from unlinkability among individual
usages, not merely at some site boundary like moderation requires.
We thus use each ring VRF output only once, which prevents our cashing
trick of \S\ref{sec:reduced_pairings} from reducing verifier pairings.

Although rationing sounds valuable enough, we foresee services like ISP,
VPNs, or mixnets having many low value transactions.
In such cases, ring VRFs could authorize issuing a limited number of
fast simple single-use blind issued credentials, like blind signatures
ala GNU Taler \cite{taler} or PrivacyPass OPRF tokens \cite{PrivacyPass},
 which both solve the leakage of $k$ above too.

In principle, commercial service providers could sell the same tokens,
which avoids leaking whether the user uses the free or commercial tier.


\subsection{Delegation}

We sometimes want to delegate spending ring VRF outputs, without
creating a fully transferable asset.  In particular, parents might
delegate limited internet or streaming service access to a child,
but without making the token full transferable.

Among blind issued credential many support this, both
GNU Taler \cite{taler} and PrivacyPass \cite{PrivacyPass} could
be redeemed by a delegatee family member who trusts the original
delegator not to double spend, but transactions with untrusted spenders
risks double spending.  

Ring VRFs usage typically demands spender authenticate the specific
spending operations inside the the associated data \aux, but adjusting
\aux requires knowing \sk, perhaps unacceptable to the delegator.

We could however achieve delegation by treating the ring VRF like a
certificate that authenticates another public key held by the delegatee.
GNU Taler achieves delegation and other features like this.

We could similarly treat the ring VRF like an adaptor certificate aka
implicit certificate.  In other words, the delegatee learns the full
ring VRF signatures, but then the delegatee hides $s_1$ from downstream
recipients, and instead merely prove knowledge of $s_1$, usually via
a key exchange or another Schnorr signature.






\bibliographystyle{plain}
\bibliography{refs_misc,refs_new,refs_sass}


\end{document}





\endinput




\newpage
MERGER IN PROGRESS
\newpage


\section{Security Model of Ring VRF}


In this section, we describe the security of our new cryptographic primitive ring VRF. First, we describe the basic ring VRF in the real world and in the ideal world. Second, we extend the basic ring VRF by adding a new property that we call secret evaluation. 


\paragraph{Ring VRF (Basic Definition):} A ring VRF operates like a VRF but only proves its key comes from a specific list without giving any information about which specific key. We define the ring VRF functionality $ \fgvrf $ in Figure \ref{f:gvrf}. The functionality lets parties generate a key (Key Generation), evaluate a message with the party's key (VRF Evaluation), prove that the evaluation is executed by one of the keys (VRF evaluation and proof) and verify the evaluation without knowing the key used for the evaluation (Verification). We also define linking procedures in $ \fgvrf $ to link an evaluation and a proof with its associated key. So, if a party wants to reveal its identity at some point, it can use the linking process to prove that the evaluation is executed with its key (Linking proof). Later on, anyone can verify the linking proof (Linking verification).
%TODO Secret Evaluation
\begin{figure}
	\scriptsize
	\begin{tcolorbox}
		{  \begin{description}
				\item[Key Generation.] upon receiving a message $(\msg{keygen}, \sid)$ from a party $\user_i$, send $(\msg{keygen}, \sid, \user_i)$ to the simulator $\simulator$.
				Upon receiving a message $(\msg{verificationkey}, \sid, \pkrvrf)$ from $\simulator$, verify that $\pkrvrf$ has not been recorded before; then, store in the table $\vklist$, under $\user_i$, the value $\pkrvrf$.
				Return $(\msg{verificationkey}, \sid, \pkrvrf)$.
				
				\item[Malicious Key Generation.] upon receiving a message $(\msg{keygen}, \sid, \pkrvrf)$ from $\simulator$, verify that $\pkrvrf$ was not yet recorded, and if so record in the table $\vklist$ the value $\pkrvrf$ under $\simulator$. Else, ignore the message.
				
				\item[VRF Evaluation.] upon receiving a message $(\msg{eval}, \sid, \pkring, \pkvrf_i, m)$ from $\user_i$, verify that $\pkrvrf_i \in \pkring$ and if there exists $ \pkvrf_i $ in $\vklist $ associated with $ \user_i $. If that was not the case, just ignore the request.
				If $\evaluationslist[\pkring, m][\pkrvrf_i]$ is empty, then sample a new $y \sample \bin^{\ell(\secpar)}$ and store the pair $y$ in the array associated to the message $m$ and the ring $\pkring$. That is, store $\evaluationslist[\pkring, m][\pkrvrf_i] = y$.
				Let sets $\prooflist[\pkring, m][\pkrvrf_i] = \emptylist$ and $\Linklist[\pkring, m][\pkrvrf_i] = \emptylist$ if $ \prooflist[\pkring, m][\pkrvrf_i] $ and $\Linklist[\pkring, m][\pkrvrf_i] = \emptylist$ do not exits. Return $(\msg{evaluated}, sid, \pkring, m, y)$ to $ \user_i $.
				
				
				\item[VRF evaluation and proof.] upon receiving a message $(\msg{evalprove}, \sid, \pkring, \pkrvrf_i, m)$ from $\user_i$, verify that $\pkrvrf_i \in \pkring$ and that there exists a public key $\pkrvrf_i$ associated to $\user_i$ in the table $ \vklist $. If that wasn't the case, just ignore the request.
				Send $(\msg{evalprove}, \sid, \pkring, m)$ to $\simulator$. Upon receiving $(\msg{eval}, \sid, \pkring, m, \pi)$ as a response from $\simulator$, if $\evaluationslist[\pkring, m][\pkrvrf_i]$ is not set, select $y \sample \bin^{\ell(\secpar)}$ and set $\evaluationslist[\pkring, m][\pkrvrf_i]$ to be $y$; set $\prooflist[\pkring, m][\pkrvrf_i] = \set{\pi}$. Otherwise, let $y \defeq \evaluationslist[\pkring, m][\pkrvrf_i]$; update the list of proofs $\prooflist[\pkring, m][\pkrvrf_i]$ by adding $\pi$. Return $(\msg{evaluated\&proved}, \sid, \pkring,m, y, \pi)$ to $\user_i$.
				
				%\item[Malicious VRF evaluation.] upon receiving a message $(\msg{evalprove}, \sid, \pkring, m)$ from $\simulator$, check that $\vklist$ has a public key associated to $\simulator$. If not, ignore the request. If $\evaluationslist[\pkring, m][\simulator]$ is not set, sample $y \sample \bin^{\ell(\secpar)}$ and set $\evaluationslist[\pkring, m][\simulator] \defeq y$ (and $\prooflist[\pkring,m]$ to $\emptyset$). If $\prooflist[\pkring, m]$ contains a proof (i.e., if $\prooflist[\pkring, m]$ is not empty), return $(\msg{evaluated}, \sid, y)$ to $\simulator$. Else, ignore the request.
				
				%\item[Verification.] upon receiving a message $(\msg{verify}, \sid, \pkring, m, y, \pi)$, from any party forward the message to the simulator. If there exists a $\pkrvrf_i$ among the values of \texttt{verification\_keys}, and there exists $\pi \in \prooflist[\pkring, m]$, set $b = 1$. Else, set $b =0$. Finally, output $(\msg{verified}, \sid, \pkring, m, y, \pi, b)$.
				\item[Verification.] upon receiving a message $(\msg{verify}, \sid, \pkring, m, y, \pi)$, set $ b = 0 $ and check if there exists a  $\pkrvrf_i \in \pkring$ and  in $ \vklist $ such that $ \evaluationslist[\pkring,m][\pkvrf_i] = y $. 
				\begin{itemize}
					\item If not, set $ b = 0 $.
					\item If there exists a $ \pkrvrf_i $ such that $ \evaluationslist[\pkring,m][\pkvrf_i] = y$, $ \pkrvrf_i $ is a malicious key and   $\pi \notin \prooflist[\pkring, m][\pkvrf_i]$, forward the message to $ \simulator $. Upon receiving $ (\msg{Verified}, \sid,\pkring, m, y, \pi, b') $ from $ \simulator $. Set $ b = b' $. If $ b' = 1 $, append $ \pi $ to $ \prooflist[\pkring, m][\pkvrf_i] $.
					\item If there exists $\pi \in \prooflist[\pkring, m][\pkvrf_i]$, set $b = 1$.
				\end{itemize}
				 Finally, output $(\msg{verified}, \sid, \pkring, m, y, \pi, b)$ to the party.
			\end{description}
			\par\hrulefill\\
			We add the following linking procedures:
			\begin{description}
				\item[Linking proof.] upon receiving a message $(\msg{link}, \sid, \pkring, \pkrvrf_i, m, y,\pi)$ from $\user_i$, check that $\pkrvrf_i \in \pkring$ and that $\pkrvrf_i $ is associated to $\user_i$ in $ \vklist $. 
				Check whether $\evaluationslist[\pkring, m][\pkvrf_i] = y$ and $ \pi \in \prooflist[\pkring, m][\pkrvrf_i] $. If any of the above fails, ignore the request.
				Send $(\msg{link}, \sid, \pkring, m, y)$ to $\simulator$. Upon receiving $(\msg{linkproof}, \sid, \pkring, m, y, \hat \pi)$ from $\simulator$, append $\hat\pi$ to $\Linklist[\pkring, m, \pi][\pkvrf_i]$ and return $(\msg{linked}, \sid, y, \hat\pi)$ to $\user_i$.
				%\item[Malicious linking proof.] upon receiving a message $(\msg{link}, \sid, \pkring, m, y)$ from $\simulator$, check that $\vklist$ has a key set for $\simulator$, and that it is in $R$.
				%Check that $\evaluationslist[\pkring, m][\simulator] = y$.
				%If any of the above is not satisfied, ignore the request.
				%Return $(\msg{linked}, \sid, y)$ to $\simulator$.
				\item[Linking verification.] upon receiving a message $(\msg{verifylink}, \sid, \pkrvrf_i, \pkring, m, y,\pi,\hat\pi)$ from any party forward the message to the simulator, if there exists a $\pkrvrf_i$ among the values of $ \vklist $, and $\evaluationslist[\pkring, m][\pkvrf_i] = y$, and $\Linklist[\pkring, m, \pi][\pkvrf_i]$ contains $\hat\pi$, set $b=1$. Else, set $b=0$. Return $(\msg{verified}, \sid, \pkrvrf_i, \pkring, m, y, \hat\pi, b).$ to the party.
			\end{description}
		}
	\end{tcolorbox}
	\caption{Functionality $\fgvrf$.\label{f:gvrf}}
\end{figure}

In a nutshell, the functionality $\fgvrf$, when given as input a message $m$ and a key set $\pkring$ of participant, allows to create $n$ possible different outputs pseudo-random that appear independent from the inputs. The output can be verified to have been computed correctly by one of the participants in $\pkring$ without revealing who they are. At a later stage, the author of the VRF output can prove that the output was generated by them and no other participant could have done so.

Below, we define the real-world execution of the ring VRF.
\begin{definition}[Ring-VRF (rVRF)]\label{def:ringvrf}
	Ring VRF is a VRF with a deterministic function $ F(.,.):\{0,1\}^\kappa \times\{0,1\}^* \rightarrow \{0,1\}^{\ell_\rvrf} $ and with the following algorithms:
	
	\begin{itemize}
		\item $ \rvrf.\keygen(1^\kappa) \rightarrow (\skrvrf,\pkrvrf)$ where $ \kappa $ is the security parameter,
	\end{itemize}
	Given list of public keys $ \pkring = \set{\pkrvrf_1, \pkrvrf_2, \ldots, \pkrvrf_n}$, a message $ m \in \{0,1\}^* $
	\begin{itemize}
		\item $ \rvrf.\eval(\skrvrf_i, \pkring, m)\rightarrow y $ where $ y = F(\skrvrf,\pkring,m) $,
		\item $ \rvrf.\evalprove(\skrvrf_i, \pkring, m)\rightarrow (F(\skrvrf,\pkring,m),\pi) $ where  $ \pi $ is a proof for the evaluation.
		\item $ \rvrf.\verify(\pkring, m, y,\pi) \rightarrow  b$ where $ b \in \{0,1\} $. $ b =1 $ means verified and $ b = 0 $ means not verified.
		\item $ \rvrf.\link(\skrvrf_i, \pkring,m,y, \pi) \rightarrow \pi_{\link} $ where  $ \pi^\link $ is a proof linking the public key of the producer of $ y $. 
		\item $ \rvrf.\link\verify(\pkring, \skrvrf_i, m, y, \pi,, \pi_{\link})\rightarrow b$ where $ b \in \{0,1\} $. $ b =1 $ means verified and $ b = 0 $ means not verified.
	\end{itemize}
	
\end{definition}
\paragraph{Ring VRF with Secret Evaluation:} 


Below, we define the real-world execution of the ring VRF with secret evaluation.
\begin{definition}[Ring-VRF (rVRF)]\label{def:ringvrfse}
	Ring VRF with secret evaluation is two VRFs with a deterministic function $ F(.,.):\{0,1\}^\kappa \times\{0,1\}^* \rightarrow \{0,1\}^{\ell_\rvrf} $ and$ F_s(.,.):\{0,1\}^\kappa \times\{0,1\}^* \rightarrow \{0,1\}^{\ell_\rvrf} $. It consists of the algorithms of ring VRF defined in Definition \ref{def:ringvrf} and additionally the following algorithms:
	
	Given list of public keys $ \pkring = \set{\pkrvrf_1, \pkrvrf_2, \ldots, \pkrvrf_n}$, a message $ m \in \{0,1\}^* $
	\begin{itemize}
		\item $ \rvrf.\secreteval(\skrvrf_i, \pkring, m)\rightarrow \omega $ where $ \omega = F_s(\skrvrf,\pkring,m) $,
		\item $ \rvrf.\secretprove(\skrvrf_i, \pkring, m)\rightarrow \pi / \perp $ where  $ \pi $ is a proof for the secret evaluation. If $ (F_s(\skrvrf,\pkring,m), (\pkring, m)) \notin \rel  $, it outputs $ \perp $.
		\item $ \rvrf.\secretverify(\pkring, m,\pi) \rightarrow  b$ where $ b \in \{0,1\} $. $ b =1 $ means verified and $ b = 0 $ means not verified.
	\end{itemize}
	
\end{definition}

\begin{figure}
	\scriptsize
	\begin{tcolorbox}
	{
			%\par\hrulefill\\
			 $ \fgvrf^{zk} $ for a relation $ \mathcal{R} $ behaves exactly as $ \fgvrf $ and  it additionally does the following:
			\begin{description}
				\item[Secret Evaluation.] upon receiving a message $(\msg{secreteval}, \sid, \pkring,\pkvrf_i, m)$ from $\user_i$, verify that $\pkrvrf_i$ is in $ \pkring $ and that there exists a key $\user_i$ with an associated public key $\pkrvrf_i$ in $\vklist$. If that was not the case, just ignore the request.
				If $\evaluationslist[\pkring, m][\pkrvrf_i]$ is empty, then sample a new $y \sample \bin^{\ell(\secpar)}$ and  store $\evaluationslist[\pkring, m][\pkrvrf_i] = y$. In any case, sample a random element $ \eta  \sample \bin^{\ell(\secpar)} $ and store $ \evaluationsecretlist[\pkring, m][\pkvrf_i] = \eta $. Create a set $ \proofzklist[\pkring,m][\pkvrf_i] = \emptylist $, if it does not exist. Return $(\msg{evaluated}, sid, y, \eta)$ to $ \user_i $.
				
				\item[Secret evaluation and proof.] upon receiving a message $(\msg{secretprove}, \sid, \pkring, \pkvrf_i, m)$ from $\user_i$, verify that $\pkrvrf_i \in \pkring $ and  that there exists a key $ \user_i $ with an associated public key $ \pkrvrf_i $ in $ \vklist $. (If that wasn't the case, just ignore the request.). Obtain $ (y, \eta) $ from $\evaluationslist[\pkring, m][\pkvrf_i]$ and $\evaluationsecretlist[\pkring, m][\pkvrf_i]$. If they are not defined execute the steps in VRF evaluation and secret evaluation.
				If $ ((m, y, \pkring),(\eta,\pkvrf_i)) \notin \mathcal{R} $, ignore the request.  Otherwise, send $(\msg{zkprove}, \sid, \pkring, m, y)$ to $\simulator$. Upon receiving $(\msg{zkproof}, \sid, \pkring, m, y \pi^{zk})$ as a response from $\simulator$,  add $ \pi^{zk} $ to  the set $\proofzklist[\pkring, m][\pkvrf_i]$. Return $(\msg{zkproof}, \sid, y, \eta, \pi^{zk})$ to $\user_i$.
				
				
				\item[Secret Verification.] upon receiving a message $(\msg{secretverify}, \sid, \pkring, m, \pi^{zk})$, from a party, check whether there exists  $ \pkvrf_i \in \vklist $ such that $ \proofzklist[\pkring,m][\pkvrf_i] = \pi^{zk} $, $ \evaluationslist[\pkring,m][\pkrvrf_i] = y $ and $ \evaluationsecretlist[\pkring,m][\pkrvrf_i] $. If there exists, set $b = 1$. Else, set $b =0$. Finally, output $(\msg{verification}, \sid, \pkring, m, \pi^{zk}, b)$.
			\end{description}
		}
	\end{tcolorbox}
	\caption{Functionality  $ \fgvrf^{zk} $.\label{f:gvrfzk}}
\end{figure}

\section{Ring VRFs}
\label{sec:rvrf}

Ring VRFs are firstly ring signatures broadly interpreted, in that they
prove an associated public key lies inside some commitment \comring to
the plausible signer set,
which anyone could construct from this set of public keys.
At the same time, ring VRFs prove correct output of a PRF keyed by
the signer's actual secret key, and evaluated on a supplied message \msg,
which then links ring VRF signatures sharing the same \msg.


\section{Ring VRFs}
\label{sec:rvrf_def}

Anonymized VRFs prove a PRF output for some secret key, like VRFs do,
but without identifying the signer beyond revealing the specific PRF output.
Instead, anonymized VRFs prove only an authorization condition for
the secret key.

\def\comring{\ensuremath{\mathsf{comring}}\xspace}
\def\openring{\ensuremath{\mathsf{opring}}\xspace}
\newcommand{\CommitRing}{\primalgo{CommitRing}}
\newcommand{\OpenRing}{\primalgo{OpenRing}}

Among anonymized VRFs, ring VRFs are ring signatures broadly interpreted,
meaning they prove an associated public key lies inside some signer set
commitment \comring which anyone could construct from a set of public keys.

We focus exclusively upon ring VRFs becuase otherwise anonymity and
authorizations conditions come in too many flavours.
In particular, there are group VRFs analogous to group signatures
 \cite{group_sig_survey}, some fairly efficient.
%
We distrust group signatures however because typically some group master
has deanonymization capabilities, giving them only narrow real world uses,
like mailbox spam defense in Pond \cite{pond}.
%
In this vein, attribute based credentials are frequently instantiated
like group signatures, and thus similarly permit deanonymization by some
group master, which makes them similarly unsuitabe for general purpose
privacy.
%
There exist exceptions like credentials using Brands \cite{brands}, and
group signatures with byzantine anonymity \cite{cryptoeprint:2021:181}.

Ring VRFs avoid deanonymization by a group master, and impose coherent
structure upon authorization conditions, with all signers' keys
predating the authorization condition commitment \comring. 
% TODO: Relax in \S\ref{subsec:..}
%
Ring VRF outputs clearly link any two ring VRF signatures with
the same signer and input \msg, so in particular
 ring VRFs with a singleton input domain are linkable ring signatures.

\smallskip

A {\em ring verifiable random function with auxiliary data} (rVRF-AD)
consists of the algorithms of a VRF-AD-KC, except with
 \compk and \openpk renamed to \comring and \openring,
 plus one additional algorithm:
\begin{itemize}
\item $\rVRF.\CommitRing : \ctx \mapsto \comring$ takes a set \ctx of
 public keys and returns a public key set commitment \comring.
\end{itemize}

\def\rSign{\Sign}
\def\rVerify{\Verify}

In this, we have rename the commitment and opening to avoid confusion
when we build a rVRF-AD from a VRF-AD-KC.  This fresh notation leaves
$\rVRF.\KeyGen$ and $\rVRF.\Eval$ untouched, but
 changes the other methods' signatures:
\begin{itemize}
\item $\rVRF.\CommitKey : (\pk,\ctx) \mapsto (\comring,\openring)$
\item $\rVRF.\OpenKey : (\comring,\openring) \mapsto \pk$
\item $\rVRF.\rSign : (\sk,\openring,\msg,\aux) \mapsto \sigma$
\item $\rVRF.\rVerify : (\comring,\msg,\aux,\sigma) \mapsto \Out \, \lor \perp$
\end{itemize}

...


\subsection{rVRF-AC security}

We extend the VRF-AC-KC commitment correctness condition for \CommitRing:

\begin{definition}
We say rVRF satisfies {\em ring commitment correctness} if
commitment correctness holds, and also $\rVRF.\CommitRing$ is 
 compatable with $\rVRF.\CommitKey$ in that
  $\rVRF.\CommitRing(\ctx) = \rVRF.\CommitKey(\pk,\ctx).0$.
\end{definition}

We lack anonymity against full key exposure ala
 \cite[pp. 6 Def. 4]{cryptoeprint:2005:304} of course, due to the VRF output,
but instead demand a weaker anonymity condition similar to
 \cite[pp. 5 Def. 3]{cryptoeprint:2005:304}:

\begin{definition}
We say \rVRF satisfies {\em ring anonymity} if
any PPT adversary $\adv$ has an advantage only
 negligable in $\lambda$ to win the game:
\begin{itemize}
\item[]
 Assume a set $\ctx_0 = \{ \pk_1,\ldots,\pk_n \}$ of public keys from
 distinct key pairs $(\pk_i,\sk_i) \leftarrow \KeyGen$ for $i=1,\ldots,n$ with $n \ge 2$.
 %
 $\adv$ has signing oracles
 $\ora{Sign} : (i,\ctx',\msg',\aux') \mapsto
  \rSign(\sk_i,\CommitKey(\pk_i,\ctx').1,\msg',\aux')$
 for each secret key $\sk_i$ with $i=1,\ldots,n$, and
 their choice of $\ctx',\msg',\aux'$ with $\pk_i \in \ctx'$.
 % as well as the various algorithms. 
 %
 First $\adv$ outputs a ring $\ctx \subset \ctx_0$,
 a message \msg, associated data \aux, and
 indices $i_0,i_1 \le n$ such that $\pk_{i_0},\pk_{i_1} \in \ctx_0$ and 
 $\adv$ never invoked $\ora{Sign}$ on \msg and $i_j$ for $j=0,1$.
 Next the challenger fixes $j=0$ or $j=1$ and gives
  $\adv$ a signature $\sigma = \rSign(\sk_{i_j},\openring,\msg,\aux)$.
 Finally $\adv$ wins by guessing $j$ correctly
\end{itemize}
\end{definition}

We similarly a ring unforgability resembling
 \cite[pp. 7 Def. 7]{cryptoeprint:2005:304}:

\begin{definition}
We say \rVRF satisfies {\em ring unforgability} if
any PPT adversary $\adv$ has an advantage only
 negligable in $\lambda$ to win the game:
\begin{itemize}
\item[]
 Assume a set $\ctx_0 = \{ \pk_1,\ldots,\pk_n \}$ of public keys from
 distinct key pairs $(\pk_i,\sk_i) \leftarrow \KeyGen$ for $i=1,\ldots,n$ with $n \ge 2$.
 %
 $\adv$ has signing oracles
 $\ora{Sign} : (i,\ctx',\msg',\aux') \mapsto
  \rSign(\sk_i,\CommitKey(\pk_i,\ctx').1,\msg',\aux')$
 for each secret key $\sk_i$ with $i=1,\ldots,n$, and
  their choice of $\ctx',\msg',\aux'$ with $\pk_i \in \ctx'$.
 % as well as the various algorithms. 
 %
 Now $\adv$ wins by outputing a valid signature $\sigma$ for
 a ring $\ctx \subset \ctx_0$, a message \msg, and associated data \aux
 such that
 $\adv$ never invoked $\ora{Sign}$ on $\msg,\aux$ and $i$ with $\pk_i \in \ctx$.
\end{itemize}
\end{definition}

We also need a uniqueness condition that limits even adersaries who know the secret key.

% \begin{definition}
% We say \rVRF satisfies {\em ring uniqueness} if
% any PPT adversary $\adv$ has an advantage only
%  negligable in $\lambda$ to 
% produces a ring $\ctx$, input $\msg$, 
% and various $\aux_i,\sigma_i$ with more than $|\ctx|$
% many verifying outputs $\Out_i = \Verify(\CommitRing(\ctx),\msg,\aux_i,\sigma_i)$.
% \end{definition}

\begin{definition}
We say \rVRF satisfies {\em ring unforgability} if
any PPT adversary $\adv$ who produces
 a ring $\ctx$, input $\msg$,  and various $\aux_i,\sigma_i$
cannot produce more than $|\ctx|$ many verifying outputs
$\Out_i = \Verify(\CommitRing(\ctx),\msg,\aux_i,\sigma_i)$,
 except with odds negligible in $\lambda$.
\end{definition}

Any ring VRF becomes a non-anonymized VRF whenever
 the ring becomes a singleton $\ctx = \{ \pk \}$ of course.
In doing so this, ring uniqueness reduces to uniqueness for VRF-AC,
meaning our uniqueness with only a trivial key commitment,
but \ThinVRF is faster and simpler.

We reuse the VRF-AC-KC pseudorandomness definition for ring VRFs
because pseudorandomness is strongest for singleton rings, i.e. $|\ctx| = 1$.


\subsection{rVRF-AC instantiation}

We instantiate a rVRF-AD from a hiding VRF-AD-KC like \PedVRF plus
a ring commitment scheme
 $\rVRF.\{ \CommitRing, \CommitKey, \OpenKey \}$
for which some zero-knowledge ring membership proof handles both
 $\PedVRF.\OpenKey$ and $\rVRF.\OpenKey$
efficently.

\newcommand\piring{\ensuremath{\pi_{\mathtt{ring}}}\xspace}

\begin{itemize}
\item $\rVRF.\rSign : (\sk,\openring,\msg,\aux) \mapsto \sigma$ takes
 a secret key \sk, a ring opening \openring, a message \msg, and auxiliary data \aux, and then \\
 \begin{enumerate}
 \item computes the ring membership proof $\piring$ and associated \openpk,
  $$ \piring = \NIZK \Setst{ \compk, \comring }{
  \exists \openpk,\openring \textrm{\ s.t.\ } 
  \genfrac{}{}{0pt}{}{\PedVRF.\OpenKey(\compk,\openpk) \quad}{\,\, = \rVRF.\OpenKey(\comring,\openring)}
  } $$
 \item computes the VRF-AC-KC signature
  $\sigma = \PedVRF.\Sign(\sk,\openpk,\msg,\aux \doubleplus \compk \doubleplus \piring)$, and finally
 \item returns the ring VRF signature $\rho = (\compk,\piring,\sigma)$.
 \end{enumerate}
\item $\rVRF.\rVerify$ takes $(\comring,\msg,\aux,\rho)$,
 parses $\rho$ as $(\compk,\piring,\sigma,)$,  and then returns
 $$ \PedVRF.\Verify(\compk,\msg,\aux \doubleplus \compk \doubleplus \piring,\sigma) $$
 iff $\NIZK.\Verify(\piring,\compk,\comring)$ succeeds. 
\end{itemize}

% \noindent
In this, we tie $\sigma$ to $\piring$ by expanding $\sigma$'s auxiliary data with $\piring$.

\smallskip

We now prove security of \rVRF using that
\begin{itemize}
\item \PedVRF is a secure hiding VRF-AC-KC, and that
\item our ring commitment scheme satisfies ring commitment correctness.
\end{itemize}

Pseudorandomness holds by pseudorandomness of \PedVRF from
 Proposition \ref{prop:pedersen_vrf}.

\begin{proposition}\label{prop:pedersen_rvrf}
$\rVRF$ satisfies ring uniqueness and ring unforgability.
\end{proposition}

\begin{proof}[Proof sketch]
???
\end{proof}


TODO:  Should we give an abstract pure NIZK instaniation here?  I think later probably.


\subsection{UC}

TODO: Should we gives Handan's UC functionality or similar here? 




\endinput




\noindent 
\begin{itemize}

\item {\em Ring } - 
  Assuming ,
 a PPT adversary $\adv$ cannot distinguish a random function from
  any evaluation map $F_{\sk_i} : \msg \mapsto \Eval(\sk,\msg)$,
 even given $\ctx_0 = \{ \pk_1,\ldots,\pk_i \}$ and
  chosen-message queries to $\rVRF.\rSign(\sk_i,\openring)$,
  except with odds negligible in $\lambda$.
\end{itemize}


\begin{definition}
We say \rVRF satisfies {\em ring pseudorandomness} if 
any PPT adversary $\adv$ has only a negigable advantage in $\lambda$
in this chosen-message game:
\begin{itemize}
 \item Frist, a challenger
  generates keypairs $(\pk_1,\sk_1),\ldots,(\pk_1,\sk_1) \leftarrow \KeyGen$ and
  defines a signing oracles $\ora{Sign}_i$ given by
  $(\compk,\openpk,\msg,\aux) \mapsto \Sign(\sk_i,\openpk,\msg,\aux)$,
   except it logs $\msg$ and aborts if $\pk_i \ne \OpenKey(\compk,\openpk)$.
 \item Next $\adv$ recieves the $\pk_i$, repeatedly queries $\ora{Sign}_i $,
  and produces $\compk,\openpk,\msg,\aux$. 
 \item If $\adv$ never queried $\ora{Sign}(\cdot,\cdot,\msg,\cdot)$ then
  $\adv$ wins by distinguishing $\Eval(\sk,\msg)$ from random value.
\end{itemize}
\end{definition}



\subsection{Generic rVRF-AD instantiation}

We instantiate a rVRF-AD from a hiding VRF-AD-KC like \PedVRF plus
a ring commitment scheme
 $\rVRF.\{ \CommitRing, \CommitKey, \OpenKey \}$
for which some zero-knowledge ring membership proof handles both
 $\PedVRF.\OpenKey$ and $\rVRF.\OpenKey$
efficiently.

\begin{itemize}
\item $\rVRF.\rSign : (\sk,\openring,\msg,\aux) \mapsto \sigma$ takes
 a secret key \sk, a ring opening \openring, a message \msg, and auxiliary data \aux, and then \\
 \begin{enumerate}
 \item computes the ring membership proof $\piring$ and associated \openpk,
  $$ \piring = \NIZK \Setst{ \compk, \comring }{
  \exists \openpk,\openring \textrm{\ s.t.\ } 
  \genfrac{}{}{0pt}{}{\PedVRF.\OpenKey(\compk,\openpk) \quad}{\,\, = \rVRF.\OpenKey(\comring,\openring)}
  } $$
 \item computes the VRF-AD-KC signature
  $$ \sigma = \PedVRF.\Sign(\sk,\openpk,\msg,\aux \doubleplus \compk \doubleplus \piring), \quad\textrm{and} $$ % finally
 \item returns the ring VRF signature $\rho = (\compk,\piring,\sigma)$.
 \end{enumerate}
\item $\rVRF.\rVerify$ takes $(\comring,\msg,\aux,\rho)$,
 parses $\rho$ as $(\compk,\piring,\sigma,)$,  and then returns
 $$ \PedVRF.\Verify(\compk,\msg,\aux \doubleplus \compk \doubleplus \piring,\sigma) $$
 iff $\NIZK.\Verify(\piring,\compk,\comring)$ succeeds. 
\end{itemize}

\begin{proposition}\label{prop:pedersen_rvrf}
$\rVRF$ satisfies ring uniqueness and ring unforgeability.
\end{proposition}







\endinput










% \noindent
In this, we tie $\sigma$ to $\piring$ by expanding $\sigma$'s auxiliary data with $\piring$.

% \smallskip

We now prove security of \rVRF using that
\begin{itemize}
	\item \PedVRF is a secure hiding VRF-AD-KC, and that
	\item our ring commitment scheme satisfies ring commitment correctness.
\end{itemize}

Pseudo-randomness holds by pseudo-randomness of \PedVRF from
Proposition \ref{prop:pedersen_vrf}.

\begin{proposition}\label{prop:pedersen_rvrf}
	$\rVRF$ satisfies ring uniqueness and ring unforgeability.
\end{proposition}

\begin{proof}[Proof sketch]
	???
\end{proof}


TODO:  Should we give an abstract pure NIZK instantiation here?  I think later probably.






\newcommand{\Gen}{\ensuremath{\mathsf{Gen}}}

\newcommand{\counter}{\ensuremath{\mathsf{counter}}\xspace}
\newcommand{\anonymouskeymap}{\ensuremath{\mathtt{anonymous\_key\_map}}\xspace}



\begin{figure}\scriptsize
\begin{tcolorbox}
{ $ \fgvrf $ runs two PPT algorithms $ \Gen_W$ and $\Gen_{sign} $ during the execution.

\begin{description}
	\item[Key Generation.] Upon receiving a message $(\oramsg{keygen}, \sid)$ from a party $\user_i$, send $(\oramsg{keygen}, \sid, \user_i)$ to the simulator $\simulator$.
	Upon receiving a message $(\oramsg{verificationkey}, \sid, \pk)$ from $\simulator$, verify that $\pk$ has not been recorded before\footnote{{ \color{blue} TODO: At all in any session or just for this sid?}}; then, store in the table $\vklist$, under $\user_i$, the value $\pk$.\footnote{{ \color{blue} TODO: Can multiple keys be stored per party $P_i$ and/or per session?}}
	Return $(\oramsg{verificationkey}, \sid, \pk)$ to $ \user_i$.

	\item[Malicious Key Generation.] Upon receiving a message $(\oramsg{keygen}, \sid, \pk)$ from $\simulator$, verify that $\pk$ was not yet recorded, and if so record in the table $\vklist$ the value $\pk$ under $\simulator$. Else, ignore the message.
	
	\item [Corruption:]  Upon receiving $ (\oramsg{corrupt}, \sid, \user_i) $ from $ \simulator $, remove $ \pk_i $ from $ \vklist[\user_i] $ and store $ \pk_i $ to $ \vklist $ under $ \sim $. Return $ (\oramsg{corrupted}, \sid,\user_i) $.
	\item[Malicious Ring VRF Evaluation.] Upon receiving a message $(\oramsg{eval}, \sid, \ringset, \pk_i, W, m)$ from $\sim$, verify that $ \pk_i $ has not been recorded in $\vklist$ under an honest party's identity.
	If this is the case, record in the table $\vklist$ the value $\pk_i$ under $\simulator$. Else, ignore the request.  If $ \counter[m,\ringset] $ does not exist, initiate $ \counter[m,\ringset] = 0 $. If there exists ${\color{blue} W' \neq W }$ where 
	{\color{blue} $\anonymouskeymap[m,W',\ringset] = \pk_i$ or if $\anonymouskeymap[m,W,\ringset] \neq \pk_i$}, then abort. 
	Otherwise, check if $ \evaluationslist[m,W,\ringset] $ is defined. If it is not defined, sample a VRF output $y$ and increment $ \counter[m,\ringset] $. Then, set $ \evaluationslist[m, W, \ringset] = y$, $ \anonymouskeymap[m,W,\ringset] = \pk_i $ (if not already defined).
	Return $(\oramsg{evaluated}, \sid, \ringset, m, W, \evaluationslist[m, W, \ringset])$ to $ \user_i $.
	
	\item[Honest Ring VRF Signature] Upon receiving a message $(\oramsg{sign}, \sid, \ringset, \pk_i, m)$ from $\user_i$, verify that $\pk_i \in \ringset$ and that there exists a public key $\pk_i$ associated to $\user_i$ in the table $ \vklist $. If that is not the case, just ignore the request. 	
	If there exists no $ W' $ such that $ \anonymouskeymap[m, \ringset,W'] =  \pk_i $, run $ \Gen_W(\ringset, \pk_i, m) \rightarrow W$. Then, sample a VRF output $y$ and set $ \anonymouskeymap[m,W,\ringset] = \pk_i $ and set $ \evaluationslist[m, W,\ringset] = y$.
	If already set, obtain $ W, y $ where  $ \evaluationslist[m, W, \ringset] = y$, $ \anonymouskeymap[m,W,\ringset] =\pk_i $ and run  $ \Gen_{sign}(\ringset, W, m) \rightarrow \sigma $. Verify that $ [m, W,\ringset, \sigma, 0] $ is not recorded. {\color{blue}If it is recorded}, abort. 
	Otherwise, record $ [m, W, \ringset,\sigma, 1] $. Return $(\oramsg{signature}, \sid, \ringset,W,m, y, \sigma)$ to $\user_i$.
	
	\item[Ring VRF Verification.] Upon receiving a message $(\oramsg{verify}, \sid, \ringset,W, m, \sigma)$ from a party $P_i$, relay the message $(\oramsg{verify}, \sid, \ringset,W, m, \sigma)$ to $ \simulator $ and receive back the message $(\oramsg{verified}, \sid, \ringset,W, m, \sigma, b_{\simulator}, \pk_\simulator)$. Then run the following checks: 
	\begin{enumerate}[label={{C}}{{\arabic*}}, start = 1]
		\item If there exits a record $ [m,W,\ringset,\sigma, b'] $, set $ b = b' $. (This condition guarantees the completeness and consistency.)\footnote{ \color{blue} TODO: All parenthetical comments should become a smooth part of the text or be removed.}
		\label{cond:consistency}
	
		\item Else if $ \anonymouskeymap[m,W,\ringset]  $ is an honest verification key where  there exists a record $ [m, W,\ringset, \sigma', 1] $ for any $ \sigma' $, then let $ b= b_{\simulator} $ and record $ [m, W,\sigma, b_{\simulator}] $. (This condition guarantees that if $ m $ is signed by an honest party for the ring $ \ringset $ at some point and the signature is $ \sigma' \neq \sigma $, then the decision of verification is up to the adversary) \label{cond:differentsignature}
		
		\item Else if $\counter[m, \ringset] \geq |\ringset_{\mathit{mal}}|$, where $\ringset_{\mathit{mal}}$ is the set of malicious keys in $ \ringset $, set $ b = 0 $ and record $ [m, \ringset,W,\sigma, 0] $.
		(This condition guarantees  uniqueness meaning that the number of verifying outputs that $ \sim $ can generate for $(m, \ringset)$ 
		is at most the  number of malicious keys in $ \ringset $.)\label{cond:uniqueness}.
		
		\item Else if $ \pk_\simulator $ is an honest verification key, set $ b = 0 $ and record $ [m, \ringset,W,\sigma, 0] $. (This condition guarantees unforgeability meaning that if an honest party never signs a message $ m $ for a ring $ \ringset $)\label{cond:forgery}
		\item Else set $ b = b_\sim$. \label{cond:simulatorbit}
	\end{enumerate}
	In the end, if $ b = 0 $, let $ \Out = \perp $. Otherwise, it does the following:
	\begin{itemize}
		\item if $ \evaluationslist[m,W,\ringset] $ is not defined, set sample a VRF output $\evaluationslist[m, W, \ringset]$, $ \anonymouskeymap[m,W,\ringset]  =  \pk_\simulator$. If $ \counter[m, \ringset]  $ is not defined, set $ \counter[m, \ringset]  = 0 $. Then increment $ \counter[m, \ringset]  $. Set $ \Out = \evaluationslist[m, W, \ringset]$. 	
		\item otherwise, set $ \Out = \evaluationslist[m, W, \ringset]$. 	
	\end{itemize}
	Finally, output $(\oramsg{verified}, \sid, \ringset,W, m, \sigma, \Out, b)$ to the party.
\end{description}

}
\end{tcolorbox}
\caption{Functionality $\fgvrf$.\label{f:gvrf}}
\end{figure}



\begin{figure}\scriptsize
\begin{tcolorbox}
{ This part of $ \fgvrf $ for the parties who want to show that they generate a particular ring signature.

\begin{description}
	\item[Linking signature.] Upon receiving a message $(\oramsg{link}, \sid, \ringset, \pk_i, W, m,\sigma)$ from $\user_i$, check that $\pk_i $ is associated to $\user_i$ in $ \vklist $, $ \pk_i \in \ringset $, $ \anonymouskeymap[m,W, \ringset] = \pk_i $ and 
	check whether $ [m, W,\ringset, \sigma, 1] $ is stored. If any of them fails, ignore the request. Otherwise,
	send $(\oramsg{link}, \sid, \ringset, W, m, y)$ to $\simulator$. Upon receiving $(\oramsg{linkproof}, \sid, \ringset, W, m, y, \hat \sigma)$ from $\simulator$, verify that $ [m, \ringset, \pk_i, \sigma, \hat{\sigma}, 0] $ is not stored in $ \Linklist $. If not, abort. Otherwise,  record $\hat\sigma$ to $[m, \ringset, \pk_i,\sigma, \hat{\sigma}, 1]$ to $ \Linklist $ and return $(\oramsg{linked}, \sid, \ringset, \pk_i,W, m, y,\sigma, \hat\sigma)$ to $\user_i$.

	\item[Linking verification.] Upon receiving a message $(\oramsg{verifylink}, \sid, \pk_i, \ringset, W, m,\sigma,\hat\sigma)$ from any party forward the message to the simulator and receive back  the message $(\oramsg{verified}, \sid, \pk_i, \ringset, W,m, \sigma,\hat\sigma,  b_{\simulator})$. Then do the following:
	
	\begin{itemize}
		\item If there exits a record $ [m, \ringset,\pk_i,\sigma,\hat\sigma, 1] $ in $ \Linklist $, set $ b = 1 $ and {\color{blue} $ \pkoops = \pk $ }. (This condition guarantees the completeness.)
		\item Else if $ \pk_i $ is a key of an honest party and there exits no record such as $ [m, \ringset,\pk_i,\sigma,\hat\sigma',  1] $ for any  $  \hat\sigma'$, then set $ b = 0 $ and record $ [m, \ringset,\pk_i,\sigma,\hat\sigma,  0] $. (This condition guarantees unforgeability meaning that if an honest party never signs a message $ m $ in the linking signature, then the verification fails.)
		\item Else if there exists a record  such as $ [m, \ringset,\pk_i,\sigma,\hat\sigma,  b'] $, set $ b = b' $. 
		\item Else set $ b = b_{\simulator} $ and record $ [m, \ringset,\pk_i,\sigma,\hat\sigma,  1] $. 
	\end{itemize}
	
	Return $(\oramsg{verified}, \sid, \pk_i, \ringset, m, \hat\sigma, b).$ to the party.
\end{description}

}
\end{tcolorbox}
\caption{Functionality $\fgvrf$.\label{f:gvrf2}}
\end{figure}




% \section{Security Proof of Our Protocol in UC}
Before we start to analyse our protocol, we should define the algorithms $ \Gen_{sign} $ and $ \Gen_W $ for $ \fgvrf $. $ \fgvrf $ that \name \ realizes runs Algorithm \ref{alg:genW} to generate  anonymous keys and Algorithm \ref{alg:gensign} to generate signatures.



\begin{algorithm}
	\caption{$\Gen_{W}(\ringset,\pk,m)$}
	\label{alg:genW}	 	
	\begin{algorithmic}[1]
		\State$ W \leftsample\GG $
		%		\State \textbf{get} $ X \in \pk $
		%		\If{$\mathtt{DB}[m, \ringset] = \perp  $}		
		%		\State{$ a \leftsample \FF_p $}		
		%		\State{$\mathtt{DB}[m, \ringset] := a$}
		%		\EndIf
		%		\State$ a \leftarrow \mathtt{DB}[m, \ringset] $
		%		
		%		\State \textbf{return} $ aX $
		\State \textbf{return} $ W $
	\end{algorithmic}
	
\end{algorithm}

\begin{algorithm}
	\caption{$\Gen_{sign}(\ringset,W,\pk,m)$}
	\label{alg:gensign}	 	
	\begin{algorithmic}[1]
		\State $ c,s,\delta \leftsample \FF_p $
		\State $ \beta \leftsample \FF_p $
		\State $ C =  \pk + \beta G_2$
		%\State $ \pi_{com} \leftarrow \nizk.\mathsf{Simulate}(\rcom, (G_1, G_2, \GG,C,W,P)) $
		\State $ \pi_{com}  \leftarrow (c,s,\delta)$
		%\State \textbf{send} $(\oramsg{learn\_\tau},\sid)  $ to $ \gcrs $
		%\State \textbf{receive} $(\oramsg{trapdoor},\sid, \tau,crs)  $ from $ \gcrs $
		\State $ \pi_{MT} \leftarrow \nizk.\mathsf{Prove}(\rsnark, ((X, \beta),(G_1,G_2, \GG,\ringset, C))) $ 
		\State\Return$ \sigma = (\pi_{com},\pi_{MT},C,W) $
	\end{algorithmic}
	
\end{algorithm}

Clearly, $ \Gen_{W} $ and $ \Gen_{sign} $ satisfy the anonymity defined in Definition \ref{def:anonymity} because $ W $ and $ \sigma $ are generated independent from the public key $ \pk $.


We next show that \name \ realizes $ \fgvrf $  in the random oracle model under the assumption of the hardness of the decisional Diffie Hellman (DDH).

%The GDH problem is solving the computational DH problem by accessing the Diffie-Hellman oracle ($ \mathsf{DH}(.,.,.) $) which tells that given triple $ X,Y,Z $ is a DH-triple i.e., $ Z = xyG $ where $ X = xG $ and $ Y = yG $.

%\begin{definition}[$ n $-One-More Gap Diffie-Hellman (OM-GDH) problem]
%	Given   $ p $-order group $ \GG $ generated by $ G $, the challenges $ G, X = xG, P_1, P_2, \ldots, P_{n+1} $ and access to the DH oracle $ \mathsf{DH}(.,.,.) $ and the oracle $ \mathcal{O}_x(.) $ which returns $ xP $ given input $ P $, if a PPT adversary $ \mathcal{A} $ outputs $ xP_1, xP_2, \ldots, xP_{n+1} $ with the access of at most $ n $-times to the oracle $ \mathcal{O}_x $, then $ \mathcal{A}  $ solves the $ n $-OM-GDH problem. We say that $ n $-OM-GDH problem is hard in $ \GG $, if for all PPT adversaries, the probability of solving the $ n $-OM-GDH problem is negligible in terms of the security parameter.
%\end{definition}

\begin{theorem}
	Assuming that $ \hashG, \hash,\hash' $ are random oracles,  the DDH problem is hard in the group structure $ (\GG, G_1,G_2, p) $ and NIZK algorithms are zero-knowledge and knowledge sound, \name \ UC-realizes $\fgvrf$ according to Definition \ref{def:uc}.
\end{theorem}

\begin{proof}
	We construct a simulator $ \simulator $ that simulates the honest parties in the execution of \name \ and simulates the adversary in $ \fgvrf $. 
	\begin{itemize}
		%\item \textbf{[Simulation of $ \gcrs $:] }When simulating $ \gcrs $, it runs $ \mathsf{SNARK}.\mathsf{SetUp}(\rsnark) $ which outputs a trapdoor $ \tau $ and $ crs $ instead of picking $ crs $ randomly from the distribution $ \distribution $. Whenever a party comes to learn the $ crs $, $ \simulator $ gives $ crs $ as  $ \gcrs $.
		
		\item \textbf{[Simulation of $ \oramsg{keygen} $:]} Upon receiving $(\oramsg{keygen}, \sid, \user_i)$ from $\fgvrf$, $ \simulator $ samples $x \leftsample \FF_p$ and obtains the key $X = xG$. It adds $ xG $ to lists $ \hkeys $ and $ \vklist $ as a key of $ \user_i $. 
		In the end, $ \simulator $ returns $(\oramsg{verificationkey}, \sid, X)$ to $\fgvrf$. %Whenever the honest party $ \user $ is corrupted by $ \env, $ $ \simulator $ moves the key of $ \user $ to $ \malkeys $ from $ \hkeys $.
		
		\item \textbf{[Simulation of corruption:]} Upon receiving a message $ (\oramsg{corrupted}, \sid, \user_i) $ from $ \fgvrf $, $ \simulator $ removes the public key $ X $ from $ \hkeys $ which is stored as a key of $ \user_i $ and adds $ X $ to $ \malkeys $.
		
		\item\textbf{[Simulation of the random oracles:]} We  describe how $ \simulator $ simulates the random oracles $ \hashG, \hash $ against the real world adversaries. 	$ \simulator $ simulates  $ \hash' $  as a usual random oracle.
		
		
		$ \simulator $ simulates the random oracle $ \hashG $ as described in Figure \ref{oracle:Hg}. It selects a random element  $ h $ from $ \FF_p $ for each new input and outputs $ hG_1 $ as an output of the random oracle $ \hashG $. Thus, $ \simulator $ knows \emph{the discrete logarithm of each random oracle output of $\hashG  $}. 
		\begin{figure}
			\centering
			
			\noindent\fbox{%
				\parbox{7cm}{%
					\underline{\textbf{Oracle $ \hashG $}} \\
					\textbf{Input:} $ m, \ringset $ \\
					\textbf{if} $\mathtt{oracle\_queries\_gg}[m, \ringset] = \perp  $
					
					\tab{$ h \leftsample \FF_p $}
					
					%					\tab{\textbf{for all} $ X \in \ringset $}
					%					
					%					\tab{$ W =  hX $}
					%					
					%					\tabdbl{\textbf{if} $ W \in \anonymouskeylist $: \textsc{Abort}}
					%					
					%					\tabdbl{\textbf{else:} \textbf{add} $ W $ \textbf{to} $ \anonymouskeylist $}
					
					\tab{$ P \leftarrow hG_1 $} 
					
					\tab{$\mathtt{oracle\_queries\_gg}[m, \ringset] := h$}
					
					\textbf{else}:
					
					\tab{$ h \leftarrow \mathtt{oracle\_queries\_gg}[m, \ringset] $}
					
					\tab{$ P \leftarrow hG_1$}
					
					\textbf{return $ P $}
					
			}}	
			\caption{The random oracle $ \hashG $}
			\label{oracle:Hg}
		\end{figure}
		
		The simulation of the random oracle $ \hash $ is less straightforward (See Figure \ref{oracle:H}). Whenever an input $ (m, \ringset, W) $ is given to $ \hash $, it first obtains the discrete logarithm $ h $ of $ \hashG(m, \ringset) $ from the $ \hashG $'s database. It needs this information to obtain a public key  which could be used when running $ \rvrf.\sign $. Remark that if $ W $ is a pre-output generated as described in $ \rvrf.\sign $, it should be equal to $ x\hashG(m, \ringset)= xhG_1 $  where $ xG_1 $ is a public key. Therefore, the random oracle $ \hash $ obtains first  a  key $ X^* = h^{-1}W $ to check whether it is an honest key. 
		%If $ X^* $ has not been registered as a malicious key, it registers it to $ \fgvrf $. Thus, $ \simulator $ has a right to ask the output of the message $ m, \ringset $ to $ \fgvrf $. 
		If $ X^*$ is not an honest key generated by $ \simulator $, $ \simulator $ considers the input $ (m,\ringset,W) $ as an attempt to learn the output of the ring signature of the message $ m $, the ring $ \ringset $ and the pre-output $ W $.
		Remark that if $ X^* $ is not an honest key, $ \sim $ has a right to ask the output of $ m $ with the public key $ X^* $ and the ring $ \ringset $. Therefore, $ \simulator $ sends the message $ (\oramsg{eval}, \sid, \ringset,W,m) $ to $ \fgvrf $ and receives the message $ (\oramsg{evaluated}, \sid, \ringset,W,m,y) $.
		After learning the evaluation output $ y $, it sets $ y $ as the answer of the random oracle $ \hash $  for the input $ (m, \ringset, W) $. 
		% Remember that $ \fgvrf $ only replies to the evaluation message of $ \sim $ if $ W $ is not mapped to another message, ring and public key $ (m', \ringset', X')   $. $ W $ cannot be map to $ (m', \ringset', X')   \neq  (m, \ringset, X*)   $ because it would be aborted during the simulation $ \hashG $ if they were mapped to $ W $.
		If $ X^* $ is registered as an honest party's key, $ \simulator $ outputs a randomly selected element. 
		
		
		\begin{figure}
			\centering
			
			\noindent\fbox{%
				\parbox{9cm}{%
					\underline{\textbf{Oracle $ \hash$}} \\
					\textbf{Input:} $ m, \ringset,W $ 
					
					\textbf{if} $ \mathtt{oracle\_queries\_h}[m, \ringset, W] \neq \perp $
					
					\tab{\textbf{return $  \mathtt{oracle\_queries\_h}[m, \ringset, W] $}}
					
					$ P \leftarrow \hashG(m,\ringset) $\\			
					$ h \leftarrow \mathtt{oracle\_queries\_gg}[m, \ringset] $\\
					$ X^* := h^{-1}W $ // candidate verification key 
					
					{\textbf{if} $ X^* \notin \hkeys$ } 
					
					\tab{\textbf{send} $ (\oramsg{eval}, \sid, \ringset, W, X^*, m) $ \textbf{to} $ \fgvrf $}
					
					\tab{\textbf{receive} $ (\oramsg{evaluated}, \sid, \ringset,W, m, y) $ \textbf{from} $ \fgvrf $}
					
					\tab{$  \mathtt{oracle\_queries\_h}[m, \ringset, W]:=y $}
					
					\textbf{else}
					
					\tab{$ y \leftsample \FF_p $}
					
					\tab{$  \mathtt{oracle\_queries\_h}[m, \ringset, W]:=y $}
					%					{\textbf{else:} $ \mathtt{oracle\_queries\_h}[m, \ringset, W]  = \perp$}
					%					
					%					%\tab{\textbf{return} \textsc{Abort}}
					%					\tab{$ y \leftsample \bin^\lambda $}
					%					
					%					\tab{$\mathtt{oracle\_queries\_h}[m, \ringset, W] := y $}
					
					\textbf{return $  \mathtt{oracle\_queries\_h}[m, \ringset, W] $}
					
			}}	
			\caption{The random oracle $ \hash $}
			\label{oracle:H}
		\end{figure}
		
		%		\item \textbf{[Simulation of $ \oramsg{sign} $]} 
		%		The simulator has a table  $\preoutputlist $ to keep the pre-outputs that it selects for each input and the ring of public keys. 
		%		Upon receiving $(\oramsg{sign}, \sid, \ringset, m, y)$  from the functionality $\fgvrf$, $ \simulator $ generates the signature $ \sigma $ as follows:
		%		
		%		For the first proof, it samples $ c, s, \delta \in \FF_p $ and $ C, W \in \GG$. Then, it lets the first proof be $\pi_1 =  (c, s, \delta) $. 
		%		In addition, it sets $ R = sG_1+ \delta G_2+ cC $ and $ R_m = s \hashG(m, \ringset)+ cW $ and maps the input $ \ringset,m, W,C, R, R_m$ to $ c $ in the table of the random oracle $ \hash' $ so that $ \pi_1 $ verifies in the real-world execution.  
		%		It adds $ W $ to the list $ \preoutputlist[m, \ringset] $.
		%		
		%		$ \simulator $ gets the trapdoor $ \tau $ that it generated during the simulation of $ \gcrs $ to simulate the second proof. Then, it runs $ \mathsf{SNARK}.\mathsf{Simulate}(\rsnark,\tau, crs) $ and obtains $ \pi_2 $.
		%		
		%		In the end, $ \simulator $  responds by sending the message $(\oramsg{signature}, \sid, \ringset, m, \sigma = (\pi_1, \pi_2, C, W))$ to the $ \fgvrf $.  It also lets $ \mathtt{oracle\_queries\_h}[m, \ringset, W] $ be $ y $, if it is not defined yet. If it is defined with another value $ y' \neq y $, then it aborts.
		%TODO: Talk about this abort case happens with a negl probability. 
		
		\item \textbf{[Simulation of $ \oramsg{verify} $]} Upon receiving  $(\oramsg{verify}, \sid, \ringset,W, m, \sigma)$ from the functionality $\fgvrf$, $ \simulator $ runs the two NIZK verification algorithms run for $ \rcom, \rsnark $ with the input $ \ringset, m, \sigma, W $ described in $ \rvrf.\verify $ algorithm of \name. If  all verify, it sets $ b_{\simulator} =1 $. Otherwise it sets $ b_{\simulator} =0  $.
		
		\begin{itemize}
			\item 		If $ b_\simulator = 1 $, it sets $ X = h^{-1} W$ where $ h = \mathtt{oracle\_queries\_gg}[m, \ringset] $ and sends  $ (\oramsg{verified}, \sid, \ringset, W, m, \sigma, b_\simulator, X) $ to $ \fgvrf $ and receives back $ (\oramsg{verified}, \sid, \ringset, W, m, \sigma, y, b) $. 
			\begin{itemize}
				\item If $ b \neq b_\simulator $, it means that the signature is not a valid signature in the ideal world, while it is in the real world. So, $ \simulator $ aborts in this case.
				
				If $ \fgvrf $ does not verify a ring signature even if  it is verified in the real world, $ \fgvrf $ is in either \ref{cond:consistency}, \ref{cond:uniqueness} or \ref{cond:forgery}.
				If $ \fgvrf $ is in \ref{cond:uniqueness}, it means that $ \counter[m,\ringset] > |\ringset_m| $. If $ \fgvrf $ is in \ref{cond:forgery}, it means that $ X $ belongs to an honest party but this honest party never signs $ m $ for the ring $ \ringset $. So, $ \sigma $ is a forgery.	If $ \fgvrf $ is in  \ref{cond:consistency} and sets $ b =0 $, it means that $ m, W, \ringset, \sigma $ was recorded as invalid by $ \fgvrf $, but now $ m, W, \ringset, \sigma $ is a valid signature in the real world. 
				This case never happens because of the correctness of $ \nizk $ algorithms. 
				% In short, if $ \sim $ aborts because $ b\neq b_\sim $ it means either $ W $ of an honest party is not unique and \adv in the real world generates a forgery signature of $ (m, \ringset, \sigma) $ with $ W $ or the adversary in the real world generates anonymous keys for $ (m, \ringset) $ more than the number of adversarial keys in $ \ringset $.
				%				 
				
				\item If $ b = b_\simulator $, set $ \mathtt{oracle\_queries\_h}[m, \ringset,W] = y $. Here, if $ \sigma $ is a signature of an honest party, $ \simulator $ sets its output with respect to the output selected by $ \fgvrf $. 
				%    Remark that we do not need to set $ \mathtt{oracle\_queries\_h\_schnor} $ because it already verifies in the real world.
			\end{itemize}
			\item If $ b_\simulator = 0 $, it sets $ X = \perp $ and sends  $ (\oramsg{verified}, \sid, \ringset, W, m, \sigma, b_\simulator, X) $ to $ \fgvrf $. Then, $ \sim $ receives back $ (\oramsg{verified}, \sid, \ringset, W, m, \sigma, y, b) $. 
			\begin{itemize}
				\item If $ b \neq b_\simulator $, it means that it was a signature of an honest party and $ \nizk.\verify $ for $ \rcom $ does not validate in the real world. So, $ \simulator $ sets $ \mathtt{oracle\_queries\_h}[m, \ringset,W] = y $ and $ \mathtt{oracle\_queries\_h\_schnor}[\ringset, m, W, C, R', R_m'] = c $ where $ R' = sG_1 + \delta G_2+ cC  $, $ R_m = s \hashG(m,\ringset) + cW$. 
				Now, the signature verifies in the real world as well.
				\item If $ b = b_\simulator $, $ \simulator $ doesn't need to do anything.
			\end{itemize}
			
		\end{itemize}
		
		
		
		
		
		\item \textbf{[Simulation of $ \oramsg{link} $:]} Upon receiving $(\oramsg{link}, \sid, \ringset, W,X_i, m,  \sigma)$ from $\fgvrf$, $ \sim $ runs $ \nizk.\simulate(\rel_{dleq}, (G_1, \GG,X_i, P,W)) $ and obtains $ \hat{\sigma} $.
		
		%		it picks random $ \hat s, \hat c $ and lets $ (X_i, m, \hat sG_1-\hat cX_i, \hat s\hashG(m, \ringset)- \hat cW) $ be $ \hat c $ in the table of the random oracle $ \hash' $. It sends $ (\oramsg{link}, \sid, X_i,(\hat s,\hat c)) $ to $ \fgvrf $. Also, if $ \hash(m,\ringset, W) $ is not defined, it sets $ \mathtt{oracle\_queries\_h}[m, \ringset, W] = y$. If it is defined but not equal to $ y $, it aborts.
		
		
		\item \textbf{[Simulation of $ \oramsg{verifylink} $:]} Upon receiving  $(\oramsg{verifylink}, \sid, \pk, \ringset, W,m, \sigma, \hat \sigma)$ from the functionality $\fgvrf$,  it runs $ \rvrf.\link\verify(\pk,\ringset, m, \sigma,\hat \sigma) \rightarrow b_\simulator $ and  returns $ (\oramsg{verified}, \sid, \pk,\ringset, m, \sigma, \hat\sigma, b_\simulator) $ to  $\fgvrf  $. Then, $ \fgvrf $ replies with $ (\oramsg{verified}, \sid, \pk,\ringset,W, m, \sigma, \hat\sigma, b) $. 
		
		If $ b' \neq b_\simulator   = 1 $ and $ \pk \in \mathtt{honest\_verification\_keys} $, then $ \simulator $ aborts. If this happens, it means that a link signature for $ \sigma $ has not been yet generated for $ m, \ringset $ by an honest party with a key $ \pk $ in the ideal world.
		
		We remark that the other cases:   $ (b' \neq b_\simulator   = 1, \pk \in \mathtt{malicious\_verification\_keys})$ and  $ b \neq b_\simulator = 0 $ cannot happen thanks to  the correctness of $ \nizk $. The first case  means that $ \rvrf.\link\verify(\pk,\ringset, m, \sigma,\hat \sigma) \rightarrow 0$ before, but now $  \rvrf.\link\verify(\pk,\ringset, m, \sigma,\hat \sigma) \rightarrow 1 $.  
		The second case $ b \neq b_\simulator = 0 $ is not possible also because if $ \pk $ is an honest key, its link signature is generated by $ \simulator $ so $ \rvrf.\link\verify $ always verifies it . If $ \pk $ is a malicious key then $ \fgvrf $ outputs  $ b_\simulator $ if it was not set 0 before. 
		
		
		%If $ b = 1 $, obtain $ X^* = h^{-1}W $ where $ W \in \sigma $ and $ h = \mathtt{oracle\_queries\_gg}[m, \ringset] $. If  $ \pk \in \mathtt{malicious\_verification\_keys} $ and $ \pk \neq X^* $, then it aborts. In addition, if  $ \pk \notin \mathtt{malicious\_verification\_keys} $ and $ \pk = X^* $, then it aborts. 
		%NOTE: if the the key is malicious then W should be generated properly. If it is not generated properly i.e X^* \neq \pkoops and the linkverify outputs 1, then the soundness is broken. Also, if \pkoops is a honest key but X^* = pk, then it aborts because W is not generated properly for honest parties.
		%Otherwise, it returns $ (\oramsg{verified}, \sid, \pk,\ringset, m, \sigma, \hat\sigma, b) $ to  $\fgvrf  $. 
		
		%\item \textbf{[Simulation of outputs of ideal honest parties:]} Whatever an honest party outputs in the ideal-world, $ \simulator $ outputs the same in the real-world simulation as an output of the same honest party. 
		%If an honest party outputs a signature $ \ringset, m, \sigma $, 	it sends $ (\oramsg{verify}, \sid, \ringset, m, \sigma) $ to $ \fgvrf $. If $ \fgvrf $ verifies it, $ \simulator $ receives $ y $. Then, it sets $ \mathtt{oracle\_queries\_h}[m, \ringset, W] $ as $ y $ if $ \mathtt{oracle\_queries\_h}[m, \ringset, W] $ is not defined. If $ \mathtt{oracle\_queries\_h}[m, \ringset, W] $ was already defined with another value, $ \simulator $ aborts.  
		%NOTE: We don't need this anymore because honest evaluation value is always assigned during signing.
		%Similarly,  If an honest party outputs a link-signature $ \pk, \ringset, m, \sigma, \hat \sigma $, 	it sends $ (\oramsg{verifylink}, \sid, \pk,\ringset, m, \sigma, \hat \sigma) $ to $ \fgvrf $. If $ \fgvrf $ verifies it, $ \simulator $ obtains $ X^* = h^{-1}W  $ where $ h = \mathtt{oracle\_queries\_gg}[m, \ringset] $ and $ W \in \sigma $. If $ X^* \neq \pk $, it aborts. 
		
		%		 If the honest party's output is an evaluation value $ y $ and a proof $ \pi $ of an input $ m, \ringset $, $ \simulator $ checks whether the proof $ \pi $ of the evaluation $ y $ with the input $ m, \ringset $ is valid. For this,
		%		% it checks whether $ \prooflist[\pi] $ is assigned to an input $ m $ and a ring key set $ \ringset $. If it is the case, 
		%		it sends $ (\oramsg{verify}, \sid, \ringset, m, y, \pi) $ to $ \fgvrf $. If $ \fgvrf $ verifies it, $ \simulator $ retrieves $ W $ from $ \pi $ and sets $ \mathtt{oracle\_queries\_h}[m, \ringset, W] $ as $ y $ if $ \mathtt{oracle\_queries\_h}[m, \ringset, W] $ is not defined. $ \mathtt{oracle\_queries\_h}[m, \ringset, W] $ was already defined with another value, $ \simulator $ aborts. 
		
		%		\item \textbf{[Simulation of honest parties in the real protocol]} $ \simulator $ behaves as described in the real protocol while simulating the honest parties except the following cases: 
		%		\begin{itemize}
			%			\item When \adv sends $ \ringset,m,y, \pi $ where $ \rvrf.\verify(\ringset, m, y, \pi) \rightarrow 1$ where $ (\pi_1, \pi_2, C,W) $, $ \simulator $ checks whether it is a forgery as follows: It obtains $ h = \mathtt{oracle\_queries\_gg}[m, \ringset] $ and learns $ X^{*} = h^{-1}W $ which is supposed to be a verification key of the evaluation $ y $. If $ X^* \notin \ringset $ or $ X^* \notin \mathtt{malicious\_verification\_keys}$, it aborts. Otherwise, it continues as described in the protocol. 
			%			\item When \adv sends $ X, \ringset, m,y, \pi, \pi_{\link} $ where $ \rvrf.\link\verify(X, \ringset, m,y, \pi, \pi_{\link}) \rightarrow 1$, $ \simulator $ checks whether it is a forgery as follows: It obtains $ h = \mathtt{oracle\_queries\_gg}[m, \ringset] $ and learns $ X^{*} = h^{-1}W $ which is supposed to be a verification key of the evaluation $ y $. If $ X^* \neq X $, then it aborts.
			%		\end{itemize}		
		%		 
	\end{itemize}
	
	\begin{theorem}\label{thm:rvrf}
		\name \ over the group structure $ (\GG,p,G_1,G_2) $ realizes $ \fgvrf $ in Figure \ref{f:gvrf} in the random oracle model assuming that NIZK is zero-knowledge and knowledge extractable, the decisional Diffie-Hellman (DDH) problem are hard in $ (\GG,p,G_1,G_2)  $. 
	\end{theorem}
	
	\begin{proof}
		
		We first show that the outputs of honest parties in the ideal world are indistinguishable from the output of honest parties in the real protocol. 
		
		\begin{lemma}\label{lem:honestoutput}
			Assuming that DDH problem is hard on the group structure $ (\GG, G_1,G_2,p) $, the outputs of honest parties in the real protocol \name\ are indistinguishable from the output of the honest parties in $ \fgvrf $.
		\end{lemma}
		
		\begin{proof}
			Clearly, the link signatures and outputs of the ring signatures in the ideal world identical to the real world protocol because the link signature is generated by $ \simulator $ as in the real protocol and the outputs are randomly selected by $ \fgvrf $ as the random oracle $ \hash $ in the real protocol. The only difference is the generation of the ring signature (See Algorithm \ref{alg:gensign}) and the anonymous key (See Algorithm \ref{alg:genW}). The distribution of $ \pi_{com} = (c,s,\delta) $ and $ C $ generated by Algorithm \ref{alg:gensign} and the distribution of $ \pi_{com} = (c,s,\delta) $ and $ C $ generated by $ \rvrf.\sign $ are from uniform distribution so they are indistinguishable. Thanks to the zero-knowledge property of NIZK, $ \pi_{MT} $ generated by Algorithm \ref{alg:gensign} and $ \pi_{MT} $ generated by $ \rvrf.\sign $ are indistinguishable too.  
			
			Now, we show that the anonymous key $ W $ generated by Algorithm \ref{alg:genW} and $	 W $ generated by $ \rvrf.\sign $ are indistinguishable. For this,  we need show that selecting $ W $ randomly from $ \GG $ and computing $ W $ as $x \hashG(m, \ringset) $ are indistinguishable.
			We  show this under the assumption that the DDH problem  is hard.  In other words, we show that if there exists an adversary $ \adv' $ that distinguishes anonymous keys of honest parties in the ideal world and anonymous key of the honest parties in the real protocol then we construct another adversary $ \bdv $ which breaks the DDH problem. 
			We use the hybrid argument to show this.
			We define hybrid simulations $ H_{i} $ where  the anonymous keys of first $ i $ honest parties are computed as described in $ \rvrf.\sign $ and the rest are computed by selecting them randomly. Without loss of generality, $ \user_1, \user_2, \ldots, \user_{n_h} $ are the honest parties. Thus, $ H_0 $ is equivalent to the anonymous keys of the ideal protocol  and $ H_{n_h}  $ is equivalent to the anonymous keys of honest parties in the real world.  We construct an adversary $ \bdv $ that breaks the DDH problem given that there exists an adversary $ \adv' $ that distinguishes hybrid games $ H_i $ and $ H_{i + 1} $ for $ 0 \leq i < n_h $. $\bdv $ receives the DDH challenges $ X,Y, Z \in \GG $ from the DDH game and simulates the game against $ \adv' $ as follows: $\bdv $ generates the public key of all  honest parties' key as usual by running $ \rvrf.\keygen$ except party $ \user_{i+1} $. It lets $ \user_{i+1} $'s public key be $ X $. $ \bdv $ gives $ \GG, G_1 = Y, G_2 $ as parameters of \name. 
			
			%		\begin{figure}
				%		\centering
				%		
				%		\noindent\fbox{%
					%			\parbox{8cm}{%
						%				\underline{\textbf{Oracle $ \hashG $ in \ref{game:DDH} by the DDH adversary $\simulator $}} \\
						%				\textbf{Input:} $ m, \ringset $ \\
						%				\textbf{if} $\mathtt{oracle\_queries\_gg}[m, \ringset] = \perp  $
						%				
						%				\tab{$ h \leftsample \FF_p $}
						%				
						%				\tab{\fbox{$ P \leftarrow hY $}}
						%				
						%				\tab{$\mathtt{oracle\_queries\_gg}[m, \ringset] := h$}
						%				
						%				\textbf{else}:
						%				
						%				\tab{h $\leftarrow \mathtt{oracle\_queries\_gg}[m, \ringset]$}
						%				
						%				\tab{\fbox{$ P \leftarrow hY $}}
						%				
						%				\textbf{return $ P $}
						%				
						%		}}	
				%		\caption{The simulation of the random oracle $ \hashG $ by $\simulator $. The different steps than Figure \ref{oracle:Hg} are in the box.}
				%		\label{oracle:HgbyB}
				%		\end{figure}
			
			$\bdv $ simulates the ring signatures of first $ i $ parties as in the real protocol and the parties $ \user_{i+2}, \ldots, \user_{n_h} $ as follows: it generates a ring signature and its anonymous key by running Algorithm \ref{alg:gensign} and Algorithm \ref{alg:genW}. It generates the link signatures of $ \user_{i+2}, \ldots, \user_{n_h} $ by running $ \nizk.\simulate(\rel_{dleq}, (G_1, \GG,X_i, P,W)) $. The simulation of $ \user_{i + 1} $ is different. It lets the public key of $ \user_{i + 1} $ be $ X$. Whenever $ \user_{i+1} $ needs to sign an input $ m, \ringset $, it obtains $ P = \hashG(m, \ringset) = hY $ from $ \mathtt{oracle\_queries\_gg} $ and lets $ W = hZ $. Remark that if $ (X,Y,Z)$ is a DH triple (i.e., $  \mathsf{DH}(X,Y,Z) \rightarrow 1 $), $ \user_{i+1} $ is simulated as in \name \ because $ W = xP $ in this case. Otherwise, $ \user_{i+1} $ is simulated as in the ideal world because $ W $ is random. So, if $  \mathsf{DH}(X,Y,Z)  \rightarrow 1$, $\simulator $ simulates $ H_{i+1} $. Otherwise, it simulates $ H_{i} $. In the end of the simulation, if \adv outputs $ i $, $\simulator $ outputs $ 0 $ meaning $  \mathsf{DH}(X,Y,Z) \rightarrow 0$. Otherwise, it outputs $ i + 1 $. The success probability of $\simulator $ is equal to the success probability of $ \adv' $ which distinguishes $ H_i $ and $ H_{i +1} $. Since DDH problem is hard, $\simulator $ has negligible advantage in the DDH game. So, $ \adv' $ has a negligible advantage too. Hence, from the hybrid argument, we can conclude that $ H_0    $ which corresponds the output of honest parties in  \name\ and $ H_q  $ which corresponds to  the output of honest parties in ideal world are indistinguishable.
			
			This concludes the proof of showing the output of honest parties in the ideal world are indistinguishable from the output of the honest parties in the real protocol.
		\end{proof}	
		
		Next we show that the simulation executed by $ \simulator $ against \adv is indistinguishable from the real protocol execution.
		
		\begin{lemma} 
			The view of \adv in its interaction with the simulator $ \simulator $ is indistinguishable from the view of \adv in its interaction with real honest parties.
		\end{lemma}
		
		
		\begin{proof}
			The  simulation against the real world adversary \adv is identical to the real protocol except the output of the honest parties and cases where $ \simulator $ aborts. We have already shown in Lemma \ref{lem:honestoutput} that the output of honest parties are indistinguishable from the real protocol. Next, we show that the abort cases happen with negligible probability during the simulation. In other words, we show that if there exists an adversary \adv which makes $ \simulator $ abort during the simulation, then we construct another adversary $ \bdv $ which breaks the CDH problem. 
			
			Consider a CDH game in the group prime $ p $-order group  $ \GG $ with the challenges $ G_1,U, V \in \GG$. The CDH challenges are given to the simulator $ \bdv $. Then $ \bdv $ runs a simulated copy of $ \env $ and starts to simulate $ \fgvrf $ and $ \simulator $ for $ \env $. For this, it first runs the simulated copy of \adv as $ \simulator $ does. $ \bdv $ provides $ (\GG, p, G_1 , G_2) $ as a public parameter of the ring VRF protocol to \adv.
			
			Whenever $ \bdv $ needs to generate a ring signature for $ m, \ringset $ on behalf of an honest party with a public key $ X $, it runs  Algorithm \ref{alg:genWbdv} to generate the corresponding anonymous key of the honest party and Algorithm \ref{alg:gensignbdv}. 
			
			\begin{algorithm}
				\caption{$\Gen_{W}(\ringset,X, m)$}
				\label{alg:genWbdv}	 	
				\begin{algorithmic}[1]
					\If{$ DB_W[m,\ringset, X] = \perp $}
					\State $ W \leftsample \GG$
					\State $ DB_W[m,\ringset, X] := \perp $
					\State \textbf{add} $ W $ to list $ \anonymouskeylist[m,\ringset] $
					\EndIf
					\State \textbf{return} $ DB_W[m,\ringset, X] $
				\end{algorithmic}
			\end{algorithm}
			
			
			\begin{algorithm}
				\caption{$\Gen_{sign}(\ringset,W,X,m)$}
				\label{alg:gensignbdv}	 	
				\begin{algorithmic}[1]
					\State $ c,s,\delta \leftsample \FF_p $
					\State $ \beta \leftsample \FF_p $
					\State $ C =  X + \beta G_2$
					\State $ R' = sG_1 +\delta G_2 + cC$
					\State $ R_m = s\hashG(m, \ringset) + c W $
					\State $ \mathtt{oracle\_queries\_h\_schnor}[\ringset,m, W, C,R',R'_m] = c$						
					\State $ \pi_{com} \leftarrow (c,s,\delta) $
					%\State \textbf{send} $(\oramsg{learn\_\tau},\sid)  $ to $ \gcrs $
					%\State \textbf{receive} $(\oramsg{trapdoor},\sid, \tau,crs)  $ from $ \gcrs $
					\State $ \pi_{\pk} \leftarrow  \nizk.\prove(\rsnark,((X, \beta),(G_1,G_2, \GG,\ringset, C))) $
					\State \textbf{return} $ \sigma = (\pi_{com},\pi_{MT},C,W) $
				\end{algorithmic}
				
			\end{algorithm}
			
			Whenever $ \bdv $ needs to generate a link signature for an honest party, it runs $ \mathsf{NIZK}.\mathsf{Simulate}(\rel_{dleq},(G_1,\GG, X,P,W)) $ as $ \simulator $ does.
			
			Clearly the ring signature of an honest party outputted by $ \sim $ (remember $ \fgvrf$ generates it by Algorithm \ref{alg:gensign}) and the ring signature generated by $ \bdv $ are indistinguishable because NIZK is zero knowledge.
			
			In order to generate the public keys of honest parties, $ \bdv $ picks a random $ r_x\in \ZZ_p $ and generates the public key of each honest party as $ r_xV$.
			Remark that $ \bdv$  never needs to know the secret key of honest parties to simulate them since $ \bdv $ selects anonymous keys randomly  and generates the ring signatures and link signatures  without the secret keys. Therefore, generating the honest public keys in this way is indistinguishable. 			
			\begin{figure}
				\centering
				
				\noindent\fbox{%
					\parbox{9cm}{%
						\underline{\textbf{Oracle $ \hash$}} \\
						\textbf{Input:} $ m, \ringset,W $ 
						
						\textbf{if} $ \mathtt{oracle\_queries\_h}[m, \ringset, W] = \perp $
						
						\tab{$ y \leftsample \{0,1\}^{\ell_\rvrf} $}
						
						\tab{$  \mathtt{oracle\_queries\_h}[m, \ringset, W]:=y $}
						
						
						\textbf{return $  \mathtt{oracle\_queries\_h}[m, \ringset, W] $}
						
				}}	
				\caption{The random oracle $ \hash $}
				\label{oracle:HbyB}
			\end{figure}
			
			\begin{figure}
				\centering
				
				\noindent\fbox{%
					\parbox{7cm}{%
						\underline{\textbf{Oracle $ \hashG $}} \\
						\textbf{Input:} $ m, \ringset $ \\
						\textbf{if} $\mathtt{oracle\_queries\_gg}[m, \ringset] = \perp  $
						
						\tab{$ h \leftsample \FF_p $}
						
						
						\tab{$ P \leftarrow hU $} 
						
						\tab{$\mathtt{oracle\_queries\_gg}[m, \ringset] := h$}
						
						\textbf{else}:
						
						\tab{$ h \leftarrow \mathtt{oracle\_queries\_gg}[m, \ringset] $}
						
						\tab{$ P \leftarrow hU$}
						
						\textbf{return $ P $}
						
				}}	
				\caption{The random oracle $ \hashG $}
				\label{oracle:HgbyB}
			\end{figure}
			
			Simulation of $ \hashG $ is as described in Figure \ref{oracle:HgbyB} i.e., it returns $ hU $ instead of $ hG_1 $. The simulation of $ \hashG $ is indistinguishable from the simulation of $ \hashG $ in Figure \ref{oracle:Hg}. 
			$ \bdv $ simulates the random oracle $ \hash $ in Figure \ref{oracle:HbyB} a usual random oracle. The only difference from the simulation of $ \hash $ by $ \simulator $ is that $ \bdv $ does not ask for the output of $ \hash(m,\ringset,W) $ to $ \fgvrf $. This difference is indistinguishable from the simulation of $ \hash $ by $ \simulator $ because $ \simulator $ gets it from $ \fgvrf $ which selects it randomly as $ \bdv  $ does. Remark that since $ \hashG $ is not simulated as in Figure \ref{oracle:Hg}, $ \bdv $ cannot check whether $ W $ is an anonymous key generated by an honest key or not.  
			
			During the simulation whenever a valid signature $ \sigma = (\pi_{com},\pi_{\pk},C,W) $ of message $ m $ signed by $ \ringset $ is outputted and $ W \notin \anonymouskeylist[m,\ringset] $ (i.e., $ W $ is not generated by $ \bdv $), $ \simulator $ increments a counter $ \counter[m,\ringset] $ and adds $ W $ to $ \anonymouskeylist[m,\ringset] $.
			Then it runs $ \mathsf{Ext}(\rsnark,\pi_{\pk_j},(G_1,G_2, \GG,\ringset, C_j) ) \rightarrow X_j, \beta_j$ where $ X_j\in \ringset $ and $ C_j = X_j + \beta_j G_2 $ and $ \mathsf{Ext}(\rcom,\pi_{com_j},(G_1,G_2, \GG,\ringset, C_j,W_j,\hashG(m,\ringset))) \rightarrow (\hat{x}_j,\hat{\beta}_j )$ such that $ C_j = \hat{x}_jG_1 + \hat{\beta}_j G_2 $ and $ W_j = \hat{x}_j \hashG(m,\ringset) $. 
			
			If $ X_j  $ is an honest public key and $ X_j = \hat{x}_jG_1 $, $ \bdv $ solves the CDH problem as follows: $ W = \hat{x}_j h U $ where $ h = \mathtt{oracle\_queries\_gg}[m, \ringset] $. Since $ X_j = r_j V $, $ W = \hat{x}_jhuG_1 =r_jhuV $. So, $ \bdv $ outputs $ r_j^{-1}h^{-1}W $ as a CDH solution and simulation ends. Remark that this case happens when $ \simulator $ aborts because of \ref{cond:forgery}.
			
			If $  \counter[m,\ringset] = t \geq |\ringset_\adv| $, $ \bdv $ obtains all the signatures $ \{\sigma_i\}_{i =1}^t $ that make $ \bdv $ increment $ \counter[m,\ringset] $ and solves the CDH problem as follows: Remark that this case happens when $ \simulator $ aborts because of \ref{cond:uniqueness}.
			
			For all $ \sigma_j = (\pi_{com_j},\pi_{\pk_j},C_j,W_j) \in \{\sigma_i\}_{i =1}^t $, $ \bdv $ runs $ \mathsf{Ext}(\rsnark,\pi_{\pk_j},(G_1,G_2, \GG,\ringset, C_j) ) \rightarrow X_j, \beta_j$ where $ X_j\in \ringset $ and $ C_j = X_j + \beta_j G_2 $. One of the following cases happens:
			
			\begin{itemize}
				\item All $ X_j$'s are different: If $ \bdv $ is in this case, it means that there exists one public key $ X_a $ which is honest. Then $ \bdv $ runs $ \mathsf{Ext}(\rcom,\pi_{com_a},(G_1,G_2, \GG,\ringset, C_a,W_a,\hashG(m,\ringset))) \rightarrow (\hat{x}_a,\hat{\beta}_a )$ such that $ C_a = \hat{x}_aG_1 + \hat{\beta}_a G_2 $ and $ W_a = \hat{x}_a \hashG(m,\ringset) $.  If $ \bdv $ is in this case, $ \hat{x}_aG_1\neq X_a $ because otherwise it would solve the CDH as described before. Therefore, $ \beta_a \neq \hat{\beta}_a $. Since $ X_a + \beta_a G_2 = \hat{x}_aG_1 + \hat{\beta}_a G_2  $ and $ X_a = r_aV $ where $ r_a $ is generated by $ \bdv $ during the key generation process, $ \bdv $ obtains a representation of $ V = \gamma G_1 + \delta G_2 $ where $ \gamma = \hat{x}_ar^{-1}_a  $ and $ \delta = (\hat{\beta}_a -\beta)r_a^{-1} $. Then $ \bdv $ stores $ (\gamma, \delta) $ to a list $ \mathsf{rep} $. If $ \mathsf{rep} $ does not include another element $ (\gamma', \delta')  \neq (\gamma, \delta) $, $ \bdv $ rewinds \adv to the beginning with a new random coin.  Otherwise, it obtains $ (\gamma', \delta') $ which is another representation of $ V $ i.e., $ V = \gamma' G_1 + \delta' G_2 $. Thus, $ \bdv $ can find discrete logarithm of $ V $ on base $ G_1 $ which is $ v = \gamma + \delta \theta $ where $ \theta = (\gamma - \gamma')(\delta' - \delta)^{-1} $. $ \bdv $ outputs $ vU $ as a CDH solution and the simulation ends.
				
				
				\item There exists at least two $ X_a,X_b $ where $ X_a = X_b $. $ \bdv $ runs $ \mathsf{Ext}(\rcom,\pi_{com_a},(G_1,G_2, \GG,\ringset, C_a,W_a,\hashG(m,\ringset))) \rightarrow (\hat{x}_a,\hat{\beta}_a )$ and $ \mathsf{Ext}(\rcom,\pi_{com_b},(G_1,G_2, \GG,\ringset, C_b,W_b,\hashG(m,\ringset))) \rightarrow (\hat{x}_b,\hat{\beta}_b )$ such that $ C_a = \hat{x}_aG_1 + \hat{\beta}_a G_2, C_b = \hat{x}_bG_1 + \hat{\beta}_b G_2 $ and $ W_a = \hat{x}_a \hashG(m,\ringset), W_b = \hat{x}_b \hashG(m,\ringset) $. Since $ W_a \neq W_b $, $ \hat{x}_a \neq \hat{x}_b $.  Therefore, $ \bdv $ can obtain  two different and non trivial representation of $ X_a = X_b $ i.e., $ X_a = X_b = \hat{x}_aG_1 + (\hat{\beta}_a - \beta_a) G_2 = \hat{x}_bG_1 + (\hat{\beta}_b - \beta_b) G_2  $. Thus, $ \bdv $ finds the discrete logarithm of $ G_2 = U $ in base $ G_1 $ which is $ u = \frac{\hat{x}_a - \hat{x}_b}{\hat{\beta}_a -\beta_a -\hat{\beta}_b + \beta_b} $. $ \bdv $ outputs $ uV $ as a CDH solution.
			\end{itemize}
			
			
			
			
			
			
			
			
			
			
			
			
			%	
			%		
			%		
			%			$ \bdv $ solves CDH if $ \bdv $ is in the abort case of simulation of $ \hash$ in Figure \ref{oracle:H} by outputting $ r^{-1}h^{-1}W $ is the CDH solution of $ U,V $. $ r^{-1}h^{-1}W $ is the CDH solution because $ vU $ is a solution of $ CDH $ where $ V =  $
			%		
			%		is in this case $ \bdv $ outputs  $ r^{-1}h^{-1}W $  where $ X^* = rV $ and $ h = \mathtt{oracle\_queries\_gg}[m,\ringset] $ and simulation ends. Remark that if $ \bdv $ aborts during the simulation of $ \hash $ it means that $ X^* $ belongs to an honest party and $ X^* =  h^{-1}W = rV = rvG_1$.  Therefore, $ r^{-1}h^{-1}W $ is the CDH solution of $ U,V $.
			%				  
			%		During the simulation if $ \bdv $ sees a valid forgery ring signature  $ m, \ringset, \sigma = (\pi_{com}, \pi_{MT}, C, W) $ where $ W $ is an anonymous key generated by $ \bdv $ for $ (m',\ringset') \neq (m, \ringset) $, $ \bdv $ aborts. $ \Pr[x\hashG(m',\ringset') = W; xG_1 \in \ringset'| \ringset',W] $ is negligible because $ \hashG $ is a random oracle.
			%		%TODO exact probability
			%				  
			
			%		During the simulation if $ \bdv $ sees a forgery ring signature  $ m, \ringset, \sigma = (\pi_{com}, \pi_{MT}, C, W) $ where $ X = h^{-1}W $ is an honest key, then $ \bdv $ does the following: It runs the extractor algorithms on $ \pi_{MT} $ i.e., $ \ext(\rsnark,..) $ and obtains $ X' \in \ringset $ and $ \beta' $ where $ C = X' + \beta' G_2 $ and $ \pi_{com} $ i.e., $ \ext(\rcom,..)  \rightarrow x, \beta$ where $ C = xG_1 + \beta G_2$ and $ W = x\hashG(m, \ringset)= xhV$.  Then, $ \bdv $ outputs the CDH of $ U, V $ which is $r^{-1}xU  $. This is correct CDH solution because $ X= rV = xG_1 $, $ V= r^{-1}xG_1 $.
			%		We remark  a forgery signature corresponds to the abort case of $ \simulator $ during the verification because $ \fgvrf $ is in \ref{cond:forgerymalicious}, $ \pk_\simulator  $ is an honest party's key. 
			
			%		During the simulation if $ \bdv $ sees $  k > |\ringset_m| $-valid and malicious ring signatures $ \{\sigma_1, \sigma_2, \ldots,\sigma_k\} $ of the message $ m$ signed by $\ringset $ whose anonymous keys are $ \{W_1, W_2, \ldots, W_k\} $, respectively , it runs $ \ext(\rsnark,..) $ for each valid malicious signatures $ \sigma_i $(signatures that are not generated by $ \bdv $) and obtains $ \beta'_i, X'_i \in \ringset $. In this case, the one of following two cases must happen:
			%		
			%		\begin{itemize}
				%			\item There exists $ X'\in \ringset$ which is an honest key. In this case, $ \bdv  $ runs $ \pi_{com} $ i.e., $ \ext(\rcom,..)  \rightarrow x, \beta$ and stores $ r V = x^*G_1 + (\beta - \beta')G_2 = x^*G_1 + b G_2$ to $DB $. If $ DB $ is empty, rewind \adv to the beginning of the simulation. If it is not empty i.e., there exists $ \hat{r}X = \hat{x} G_1 + \hat{b}G_2 $, then $ \bdv $ first checks whether $ r = \hat{r} $. If it is the case, it aborts. If it is not the case, it finds the discrete logarithm of $ G_1=U $ on base $ G_2 $ which is $ t = \frac{\hat{r}^{-1}\hat{b}^*-{r}^{-1}b}{{r}^{-1}\hat{x}-\hat{r}^{-1}\hat{x}} $.  Then, it outputs the CDH of $ U,V $ which is $ tV $. 
				%			We remark that $ \bdv $ aborts after rewinding with a negligible probability because it selects $ r $ randomly.
				%			%TODO exact probability
				%			\item  There exists $ X' \in \ringset$ which is the  output of two different signatures.
				%		\end{itemize}
			%		
			%		
			%		During the simulation if $ \bdv $ sees a valid ring signature  $ m, \ringset, \sigma = (\pi_{com}, \pi_{MT}, C, W) $ where $ X = h^{-1}W \notin \ringset$, then $ \bdv $ does the following: It runs the extractor algorithms on $ \pi_{MT} $ i.e., $ \ext(\rsnark,..) $ and obtains $ X' \in \ringset $ and $ \beta' $ where $ C = X' + \beta' G_2 $ and $ \pi_{com} $ i.e., $ \ext(\rcom,..)  \rightarrow x, \beta$ where $ C = xG_1 + \beta G_2$ and $ W = x\hashG(m, \ringset)= xhV$. In this case, $ X \neq X' $, so $ \beta \neq \beta' $. 
			%				  
			%				  
			%		\begin{itemize}
				%			\item If $ X' $ is honest, then store $ r V = x^*G_1 + (\beta - \beta')G_2 = x^*G_1 + b G_2$ to $DB $. If $ DB $ is empty, rewind \adv to the beginning of the simulation. If it is not empty i.e., there exists $ \hat{r}X = \hat{x} G_1 + \hat{b}G_2 $, then $ \bdv $ first checks whether $ r = \hat{r} $. If it is the case, it aborts. If it is not the case, it finds the discrete logarithm of $ G_1=U $ on base $ G_2 $ which is $ t = \frac{\hat{r}^{-1}\hat{b}^*-{r}^{-1}b}{{r}^{-1}\hat{x}-\hat{r}^{-1}\hat{x}} $.  Then, it outputs the CDH of $ U,V $ which is $ tV $. 
				%			We remark that $ \bdv $ aborts after rewinding with a negligible probability because it selects $ r $ randomly.
				%			%TODO exact probability
				%				  	
				%			\item If $ X' $ is  a malicious key, $ \bdv $ runs the extractor algorithm on PoK proof $ \pi_{dl} $ of $ X' $ i.e., $ \ext(\rdl,..,) $ which outputs $ x' $ where $ X' = x'G_1 $. Since $ x' \neq x^* $, $ \bdv  $ has a Pedersen commitment $ C $ with two openings so it can find the discrete logarithm of $ G_2$ on base $ G_1 $ which is $ t = \frac{x^* - x'}{\beta' - \beta^*} $.  In the end, it outputs the CDH of $ X,Y $ which is $ tU $. 
				%
				%		 \end{itemize}
			%		
			%	
			So, the probability of $ \bdv $ solves the CDH problem is equal to the probability of \adv breaks the forgery or uniqueness in the real protocol. Therefore,  if there exists \adv that makes $ \sim$ aborts, then we can construct an adversary $ \bdv $ that solves the CDH problem except with a negligible probability.
			
			
			\qed	  
		\end{proof}
		
	\end{proof}
\end{proof}



\endinput
 


\subsection{UC}

TODO: Should we gives Handan's UC functionality or similar here? 




\section{A Ring VRF Construction with Secret Evaluation} 
	



