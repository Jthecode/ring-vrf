
\section{Zero-knowledge continuations}
\label{sec:rvrf_cont}

\newcommand\rrSNARK{\primalgo{Groth16}\xspace}
\newcommand\rrSNARKweak{\primalgo{Groth16/KZG}\xspace}
\newcommand\pifast{\ensuremath{\pi_{\mathtt{fast}}}\xspace}
% \newcommand\pifastdot{\ensuremath{\dot{\pi}_{\mathtt{fast}}}\xspace}
\newcommand\pisk{\ensuremath{\pi_{\mathtt{sk}}}\xspace}
\newcommand\pipk{\ensuremath{\dot{\pi}_{\mathtt{pk}}}\xspace}


We now construct ring VRFs which achieves fast amortized prover time
by using a heavy zero-knowledge continuation for $\rVRF.\OpenKey$ but
which permits updating \openpk in the $\PedVRF.\OpenKey$ invocation
without reproving $\piring$.
$$ \piring = \NIZK \Setst{ \compk, \comring }{
 \exists \openpk,\openring \textrm{\ s.t.\ } 
 \genfrac{}{}{0pt}{}{\PedVRF.\OpenKey(\compk,\openpk) \quad}{\,\, = \rVRF.\OpenKey(\comring,\openring)}
} \mathperiod $$

% \smallskip
\subsection{Rerandomization}
% \label{sec:rvrf_groth16}

Zero-knowledge continuations need rerandomizable zkSNARKs
when being reused multiple times, meaning Groth16 \cite{Groth16},
but unlinkability requires more than merely rerandomization.

In Groth16 \cite{Groth16}, we have an SRS $S$ consisting of curve
points in $\grE_1$ and $\grE_2$ that encode the circuit being proven.
We follow \cite{Groth16} in discussing the SRS $S$ in terms of
its ``toxic waste''
 $(\alpha,\beta,\delta,\gamma,\tau^1,\tau^2,\ldots) \in \F_q^*$.
In other words, we could write say $[ f(\tau)/\delta ]_1$ or $[\cdots]_2$
to denote an element of our SRS $S$ in $\grE_1$ or $\grE_2$ respectively,
computed by scalar multiplication from the toxic waste $\tau$ and $\delta$,
 but for which nobody knows the underlying $\tau$ or $\delta$ anymore.

In the SRS $S$, we distinguish the verifiers' string of elements
 $Y_1,\ldots,Y_k, [\alpha]_1 \in \grE_1$ and
 $[\beta]_2, [\gamma]_2, [\delta]_2 \in \grE_2$.
% as separate from the provers' much longer string of elements in $\grE_1$ and $\grE_2$.
A Groth16 \cite{Groth16} proof then takes the form 
 $\pi = (A,B,C) \in \grE_1 \times \grE_2 \times \grE_1$.
A verifier then produces a $X = \sum_i^k x_i Y_i \in \grE_1$ from
 the public inputs $x_i$ and then checks 
$$ e(A,B) = e([\alpha]_1, [\beta]_2) \cdot
 e(X, [\gamma]_2) \cdot e(C, [\delta]_2) \mathperiod $$

We need the rerandomization algorithm from \cite[Fig.~1]{RandomizationGroth16}:
% to build a zero-knowledge continuation:
% https://eprint.iacr.org/2020/811
% https://github.com/arkworks-rs/groth16/pull/16/files
% \algo{rerandomize}
An existing SNARK $(A,B,C)$ is transformed into a fresh
SNARK $(A',B',C')$ by sampling random $r_1,r_2 \in \F_p$ and computing
$$ \begin{aligned}
A' &= {1 \over r_1} A \\
B' &= r_1 B + r_1 r_2 [\delta]_2 \\
C' &= C + r_2 A \mathperiod \\
\end{aligned} $$
At this point, our $x_i$ remain identical after rerandomization,
so $X$ links $(A,B,C)$ to $(A',B',C')$.
Alone rerandomization cannot alter public inputs $x_i$, so
we instead need an opaque public input point $X$, which then becomes
part of our proof and incurs its own seperate proof of correctness.

We also need one fresh basepoint $\genB_\gamma$ independent from all others,
again perhaps created by applying $H_\grE$ to an input outside existing usages' domain.
We now give provers the additional SRS elements
$$ \genB_\delta := {\gamma\over\delta} \genB_\gamma $$
Although $\genB_\gamma$ is independent, 
we create $\genB_\delta$ during the trusted setup,
 so the toxic waste $\gamma$ and $\delta$ remain secret.
After this, subversion resistance could be checked like 
$$ e(\genB_\gamma, [\gamma]_2) = e(\genB_\delta, [\delta]_2) \mathperiod $$

We now have a zero-knowledge continuation $(X,A,B,C)$ from which
we produce an unlinkable $(X',A',B',C')$ by
 first sampling random $b,r_1,r_2 \in \F_p$ and then computing
$$ \begin{aligned}
X' &= X + b \genB_\gamma \\
A' &= {1 \over r_1} A \\
B' &= r_1 B + r_1 r_2 [\delta]_2 \\
C' &= C + r_2 A + b \genB_\delta \mathperiod \\
\end{aligned} $$
As our two $b$ terms cancel in the pairings, we wind up with the
standard Groth16 rerandomization construction above,
 except with $X$ an now opaque Pedersen commitment.

% TODO:  Should we be saying opaque less and Pedersen more below?

We still verify $(X',A',B',C')$ like 
$$ e(A',B') = e([\alpha]_1, [\beta]_2) \cdot
 e(X', [\gamma]_2) \cdot e(C', [\delta]_2) \mathcomma $$
As our verifier does not build $X$ themselves, we proves nothing
with this pairing equation unless the verifier seperately checks
 some proof-of-knowledge that $X' = \sum_i^k x_i Y_i$.
We foresee some $x_i$ being transperent elements that determine the
Merkle root of the ring $\ctx$, but any $x_i$ concerning the
 specific $\pk$ must be hidded by out blinding terms in $b$.

All told, our opaque rerandomization trick transforms some conventional
Groth16 zkSNARK $\pi$ for $\rVRF.\OpenKey$ into
 a zkSNARK $\pi'$ with an opaque and unlinkable input $X'$.
We therefore explore two concrete $\pi$ proposals next.

Importantly, rerandomization requires only
 four scalar multplicaitons on $\ecE_1$ and
 two scalar multplicaitons on $\ecE_2$,
which  BLS12 curves make roughly equivlent to
 eight scalar multplicaitons on $\ecE_1$.

% Intuitively, an adversary cannot link $(X,A,B,C)$ with $(X',A',B',C')$ because 

\begin{proposition}\label{prop:unlinkable}
Let $\sigma'$ and $\sigma''$ denote Chaum-Pedersen proofs-of-knowledge
 for $X'$ and $X''$ respectively, with nonces chosen randomly.
Then $(X',A',B',C',\sigma)$ and $(X'',A'',B'',C'',\sigma')$ are unlinkable.
\end{proposition}

\begin{proof}[Proof stetch.]
???
\end{proof}


\subsection{Faster}
\label{subsec:rvrf_faster}

We describe a much faster choice $\pifast$ for $\pi$
that sets $x_1 := \sk$ and $x_0 = \comring = \CommitRing(\ctx)$ so that
taking $\genG := Y_1$, $\genB := \genB_\gamma$, and $\openpk := b$ in \PedVRF
yields and incredibly fast amortized ring VRF prover.
Also, \PedVRF itself proves knowledge of $X'$,
 or more precisely of $X' - \comring\, Y_0$ since the verifier knows \comring.
$$ X' = \comring\, Y_0 + \sk\, Y_1 + b \genB_\gamma $$

% TODO: Do we mention that $Y_0$ gets omitted in the proof-of-knowledge

A priori, we do not know $Y_1$ during the trusted setup for $\pifast$,
which prevents computing $\pk = \sk\, Y_1$ inside $\pifast$.
Instead, we propose $\ctx$ contain commitments to $\sk$ over
some Jubjub curve $\ecJ$.  

We know the large prime order group $\grJ$ of $\ecJ$ typically has
smaller order than $\grE$, itself due to $\ecJ$ being an Edwards curve. 
Yet, if $\sk = \sk_0 + \sk_1 \, 2^{128}$ then our public key commitments could
take the form $\sk_0\, \genJ_0 + \sk_1\, \genJ_1 + d \genJ_2$,
with independent $\genJ_0,\genJ_1,\genJ_2$.
Interestingly, we avoid range proofs for $\sk_1$ and $\sk_2$
by this independence. 

$$ \pifast = \rrSNARK \Setst{ \sk_0 + \sk_1 2^{128}, \comring }{
 \exists d,\openring \textrm{\ s.t.\ }
 % 0 < \sk_0,\sk_1 < 2^{128} \textrm{\ and\ } 
 \genfrac{}{}{0pt}{}{ \rVRF.\OpenKey(\comring,\openring) }{ \,\, = \sk_0 \genJ_0 + \sk_1 \genJ_1 + d \genJ_2 }
} $$ % \mathperiod 

Applying our rerandomization to $\pifast$ with opaque input yields
a zkSNARK $\pifast'$ of exactly the form $\piring$.

We explain later in \S\ref{sec:ring_hiding} how one could
choose $Y_1$ independent before doing the trusted setup,
 and then wire $Y_1$ into $\pifast$ inside $C$.
In ths case, we could prove $\pk = \sk\, Y_1$ inside $\pifast$, but then
non-native arithmetic makes $\pifast$ far slower.

At this point, \PedVRF requires four scalar multiplications on $\ecE_1$,
so together with rerandomization our amortized prover time
 approaches 12 scalar multiplications on typical curves. 
We expect the three pairings dominate verifier time.

As an aside, one could construct a second faster curve with the same
group order as $\grE$, which speeds up one scalar multiplication
 in both the prover and verifier. 

Importantly, our fast ring VRF' amortized prover time now rivals
group signature schemes' performance.  We hope this ends the temptation
to deploy group signature like constructions where the deanonymization vectors matter.

\begin{proposition}\label{prop:pifast_anonymity}
\rVRF using \pifast satisfies ring anonymity.
\end{proposition}

\begin{proof}[Proof stetch.]
???
\end{proof}


\subsection{Side channel}
\label{subsec:rvrf_side_channel}

In this, we dislike exposing the secret key material inside
 the Groth16 prover for \pifast.
Adversaries could trigger \pifast recomputation only by updating the ring,
but this still presents a side channel risk.

We address this via a second zk continuation that splits \pifast into two parts: % \pisk and \pipk.
%
$$ \pipk = \rrSNARKweak \Setst{ J_\pk, \comring }{
 \exists \openring \textrm{\ s.t.\ }
  J_\pk = \rVRF.\OpenKey(\comring,\openring)
} \mathcomma \quad\textrm{and} $$ 
%
$$ \pisk = \rrSNARK \Setst{ \sk_0 + \sk_1 2^{128}, J_\pk }{ 
 \exists d \textrm{\ s.t.\ }
 % 0 < \sk_0,\sk_1 < 2^{128} \textrm{\ and\ } 
 J_\pk = \sk_0 \genJ_0 + \sk_1 \genJ_1 + d \genJ_2
} \mathperiod $$

We now prove \pisk only once {\it ever} during secret key generation,
which largely eliminates any side channel risks.
We do ask verifiers compute more pairings, but nobody cares when
the VRF verifiers are few in number or institutional,
 as in many applications.
We also ask provers rerandomize both \pisk and \pipk, but this costs relatively little.
Assuming \pipk is Groth16 then we need a proof-of-knowledge for the desired structure of $J_pk$ too.
All totaled this almost doubles the size and complexity of our ring VRF signature.

There is no ``arrow of time'' among zk continuations per se, but
as \pisk bridges between the \PedVRF and \pipk,
one might consider this zk continuation to be time reversed.

Insterestingly, our \pipk has become simple enough that openning
a KZG commitment \cite{KZG} at a secret point now suffices,
 ala Caulk \cite{caulk} or Caulk+ \cite{caulk+}.


\endinput

