
\section{Thin batchable EC VRF-AD}
\label{sec:vrf_thin}

There are VRFs built upon Chaum-Pedersen DLEQ proofs in elliptic curves,
 like \cite{nsec5} and \cite{VXEd25519}.
Yet typically their proofs have one challenge scalar and one signature
scalar, like a non-batchable Schnorr signature, while 
their verification demands two elliptic curve multi-exponentiations.
As a result, their naively batchable variant becomes ``fat'', requiring
one scalar and two separate nonces, which complicates batch verification.

We propose a new ``thin'' batchable EC VRF that merges these two nonces,
and requires only one elliptic curve multi-exponentiation, but
internally computes an extra delinearization challenge.
As such, our thin batchable variant both runs faster and simplifies heavy usage,
and also provides a less annoying interface.

Interestingly, our thin batchable EC VRF winds up literally being
a tweaked Schnorr-like signature, which opens new proof strategies and
use cases.  After this section, we abandon this thin batchable VRF
because our ``fat Pedersen'' variant introduced next in \S\ref{sec:vrf_pederson}
fits zero-knowledge continuations better.

\smallskip

\newcommand{\ThinVRF}{\primalgo{ThinVRF}} 

%PoK: We achieve this by always providing a proof-of-knowledge in the public key,
%PoK:  either separately or implicitly.

We build only a VRF-AD here which lacks the key commitments of \S\ref{sec:vrf},
meaning $\ThinVRF.\CommitKey$ and $\ThinVRF.\OpenKey$ are trivial, and $\openpk = \emptyset$.

We work solely in $\ecE$ here because we need only a basic Chaum-Pedersen DLEQ proof.
As in \S\ref{sec:lambda} and throughout,
 $\ecE$ has order $h p$ with $p \approx 2^{2\lambda}$ prime and $h$ a small cofactor.

\begin{itemize}
\item $\ThinVRF.\KeyGen$ selects a secret key \sk uniformly at random from $\F_p$ and computes the public key $\pk = \sk \, \genE$.
%PoK:  and attaches a proof-of-knowledge of $\pk$ to $\pk$ given by a Schnorr signature.  
%PoK:  All public keys must contain a valid proof-of-knowledge, or else be rejected by verifiers.
% We define equivalence $\pk_1 \equiv \pk_2$ of public keys by $h \pk_1 = h \pk_2$.
\item $\ThinVRF.\Eval(\sk,\msg)$ takes a secret key \sk and an input $\msg$, and
 then returns a VRF output $H'(\msg,\pk,h \, \sk \, H_{\grE}(\msg,\pk))$.
\item $\ThinVRF.\Sign(\sk,\msg,\aux)$ takes a secret key \sk, an input $\msg$, and auxiliary data \aux, and then performs
\begin{enumerate}
    \item compute the VRF input $\In := H_{\grE}(\msg,\pk)$ and pre-output output $\Out_0 := \sk \, \In$, 
    \item compute the delinearization challenge $c_1 = H_p(\aux,\msg,\pk,\Out_0)$,
    \item sample $r$ uniformly at random from $\F_p$ and compute $R = r (\genE + c_1 \In)$,
    \item compute the challenge $c = H_p(\aux,\msg,\pk,\Out_0,R)$, the proof $s = r + c \, \sk$, and return the signature $\sigma = (\Out_0,R,s)$.
\end{enumerate}
\item $\ThinVRF.\Verify$ takes $(\pk,\msg,\aux,\sigma)$, parses $\sigma = (\Out_0,R,s)$, and then 
\begin{enumerate}
%PoK:    \item abort unless either $\msg$ contains $\pk$ or else \pk has a valid the proof-of-knowledge,
    \item recomputes the VRF input point $\In := H_{\grE}(\msg,\pk)$,
    \item recomputes $c_1 = H_p(\aux,\msg,\pk,\Out_0)$ and $c = H_p(\aux,\msg,\pk,\Out_0,R)$, % the challenges
    \item aborts unless $s h (\genE + c_1 \In) = h R + c h (\pk + c_1 \Out_0)$ holds, and 
    \item returns $H'(\msg,\pk,h \Out_0)$ if nothing failed.
\end{enumerate}
\end{itemize}
As discussed above, if we omit this final hash $H'$ making
our output only $h \Out_0$, then we obtain only a VUF, not a VRF.
We caution that $h \ne 1$ ensures SUF-CMA fails
 by \cite[\S4.1.2]{cryptoeprint:2020:823}.

If desired, one could generalize \ThinVRF to $k$ messages by
computing for $j=1,\ldots,k$ the $k$ distinct
points $\In_j := H_{\grE}(\msg_j)$, pre-outputs $\Out_0 := \sk \, \In$,
delinearization challenges
 $c_j = H_p(\aux,\msg_1,\ldots,\msg_k,\pk,\Out_{0,1},\ldots,\Out_{0,k},j)$,
and then running our Schnorr-like signature with
 base point $\genE + \sum_{j=1}^k c_i \In_j$ and
 public key $\genE + \sum_{j=1}^k c_i \Out_j$.

% TODO: Proof correct?  Use same citations as schnorrkel.

% We define $H_\grG(\msg) = h H_\grE(\msg)$ and observe 
%
% \begin{lemma}
% If $H_\grE$ is a random oracle then $H_\grG$ is also a random oracle.
% \end{lemma}

% \begin{lemma}
% $\primalgo{PreEval}(\sk,\msg) = h \sk H_{\grE}(\msg,\pk)$
% \end{lemma}

We discuss chosen message queries against only one key in pseudo-randomness.  
% TODO: What?
In \ThinVRF, we hash the public key \pk along with the message \msg
in $H_\grE$, aka injected \pk into \msg, to prevent
related but different keys having algebraically related input points \In.
We cannot employ this trick in key committing VRFs or ring VRFs however.
Although $H'$ being a PRF mitigates problems, we still recommend caution 
when combining identical inputs \msg with related secret keys,
 like ``blockchain'' users often produce with ``soft key derivation''.

\begin{proposition}\label{prop:thin_vrf}
Assuming AGM in $\grE$, % $\ecE$ modulo $h$,
our $\ThinVRF$ satisfies
 VRF correctness, uniqueness, pseudo-randomness,
 and existential unforgeability on $(\msg,\aux)$.
\end{proposition}

\begin{proof}[Proof sketch]
TODO: ???
\end{proof}


\endinput




We expect $\ThinVRF$ to be an EUF-CMA signature scheme on $(\msg,\aux)$ too,
but proving this requires techniques outside our scope, even assuming AGM.

\begin{proof}[Proof sketch]
Correctness holds trivially.

At any fixed $\msg$ we have a Schnorr signature on $\aux$
 over the basepoint $\genE + c_1 \In$, which we name $\primalgo{VRFInner}_{\msg}$.
According to \cite[\S5]{cryptoeprint:2020:823},
 $\primalgo{VRFInner}_{\msg}$ is EUF-CMA secure,
 thanks to our random oracle assumption on $H_p$.

We consider an adversary that produces $(\pk,\msg,\aux,\sigma)$
 that pass verification, without knowing $\sk$.  
%PoK:  Ignoring the first abort path, and employing our random oracle assumption on $H_p$, 
We know from EUF-CMA security of $\primalgo{VRFInner}_{\msg}$ that
our forger knows some $w$ such that
 $h (\pk + c_1 \Out_0) = h w (\genE + c_1 \In)$.
We deduce from AGM knowledge of $x,y,u,v \in \F_p$ such that
 $\pk = x \genE + y \In$ and $\Out_0 = u \genE + v \In$
 with $x + c_1 u = w$ and $y + c_1 v = c_1 w$ in $\F_p$,
 so $c_1^2 u + c_1 (x-v) - y = 0$, except with odds negligible in $\lambda$.
At most two $c_1 \in \F_p$ satisfy this equation.
As our $c_1$ depends upon both \pk and $\Out_0$, 
it again follows from our random oracle assumption on $H_p$ that
 $u=0=y$ and $v = w = x \equiv \sk \bmod h$, meaning $\Out_0 = \sk \In$,
 except with odds negligible in $\lambda$.
%TODO: What do we cite here?
%PoK:
%PoK: We know $y=0$ if we check the proof-of-knowledge for $\pk$ of course.  
%PoK: We usually suggest that \pk appear in $\msg$ as a defense against related keys, 
%PoK: which occur if say \pk represents some account key on a blockcahin.  
%PoK: In this case, we also know $y=0$ by the random oracle assumption on $h$.  
%PoK: We even deduce $y=0$ after verifying two VRF signatures with distinct
%PoK: inputs $\msg_1$ and $\msg_2$ and hence distinct $\In_1$ and $\In_2$.
%PoK: We know $y=0$ in all cases, as desired.
%PoK: 
It follows that $\ThinVRF$ satisfies uniqueness of course. 

An unpredictability adversary $\adv$ guesses
 a \msg and corresponding pre-output $h \Out_0 := h \sk H_\grE(\msg)$,
after making chosen message queries to \Sign.
In AGM, $\adv$ must express its guess for $h \Out_0$
 in terms of $H_\grE(\msg)$ and points arising earlier.
???  So simple ???
As $H_\grE$ is a random oracle, we deduce that either
 $\adv$ solved the discrete logarithm problem, or else
 unpredictability holds for $\ThinVRF$.
As $H'$ is a PRF, we now argue pseudo-randomness for$\ThinVRF$ similarly
 to \cite[Proposition 1]{vrf_micali}.
\end{proof}
% An adversary cannot discover $\Out_0$ without querying $\msg$,
% % \cite[Theorem 6]{coron02}
% % https://eprint.iacr.org/2001/062.pdf NOT https://www.iacr.org/archive/eurocrypt2002/23320268/coron.pdf
% but our EUF-CMA game permits doing so with alternative $\aux$. 
% ...
%TODO: Actually this gets really long winded. 

%PoK:  In this, we still have a VRF if $y=0$, but not exactly the one specified.  
We caution that omitting $c_1$ only demands $x + u = v$ even if $y=0$,
which does not give a VRF.

We need two scalar multiplications in the prover and
 then four scalar multiplications in the verifier
 just like \cite{nsec5} or \cite{VXEd25519} do.
We do incur an extra hashing operation and two field multiplications,
 but they cost relatively little.
%PoK: At frist blush, we incur two more scalar multiplications when verifying
%PoK: the proof-of-knowledge for \pk too,
%PoK:  which one implements by a Schnorr signature. 
%PoK: Yet, VRF applications always require their public keys be registered in advance,
%PoK: meaning the proof-of-knowledge should be checked in advance and amortizes.
%PoK: 
We believe this approach actually reduces verifier computation because
advanced multi-scalar multiplication algorithms become more efficient when
larger, which should outweigh the extra hashing and field operations.

We also support batch verification without altering the signature or
increasing the signature size.  We think this tips the scales because
avoiding a separate batchable VRF signature type simplifies interface
over naive batch verification methods for \cite{nsec5} or \cite{VXEd25519}.

Aside from batch verification, we might simplify interaction with
other protocols by building upon Schnorr signatures.




\endinput



\section{UC}




