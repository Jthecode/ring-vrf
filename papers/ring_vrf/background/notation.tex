
\section{Notation}
\label{sec:notation}

% As our ring VRF is built by composing them, 
% We briefly recall the primitives and security assumptions underlying
% both Chaum-Pedersen DLEQ proofs and pairing based zkSNARKs. 

We briefly establish elliptic curve notion here.  
Appendix \ref{sec:background} contains the standard security definitions
for zero-knowledge proofs and the universal composability model.

\smallskip 

We obey mathematical and cryptographic implementation convention by 
adopting additive notation for elliptic curve and multiplicative notation
for elliptic curve scalar multiplications.
%
All objects implicitly depend a security parameter \secparam.

We have an elliptic curve $\ecE$ over a field of characteristic $q$,
equipped with a type III pairing $e : \grE_1 \times \grE_2 \to \grE_T$,
where the groups  $\grE_1 \le \ecE[\F_q]$, $\grE_2 \le \ecE[\F_{q^2}]$, and
$\grE_T \le \F^*_{q^{12}}$ all have prime order $p \approx 2^{2\secparam}$.

We write $\grE$ when discussing the Chaum-Pedersen DLEQ proofs, which do
not employ pairings, but $\grE$ always denotes $\grE_1$ eventually.
We avoid pairing unfriendly assumptions like DDH of course, but really
we employ the algebraic group model (AGM) throughout.

We sweep cofactor concerns under the rug when discussing Groth16,
where pairings demand deserialization prove group membership in $\grE_1$
or $\grE_2$.  We explicitly multiply by the cofactor $h$ when doing
Chaum-Pedersen DLEQ proofs though, as not doing so risks miss-readings by implementers.

We let $H_p : \{0,1\}^* \to \F_p$ and $H_\grE : \{0,1\}^* \to \grE_1$
denote hash-to-scalar and hash-to-curve random oracles (RO) with ranges
$\F_p$ and $\grE$, respectively.  

We also let $\ecJ$ denote a ZCash Sapling style ``JubJub'' Edwards curve
over $\F_p$, with distinguished subgroup $\grJ$ of prime order, so that
SNARKs on $\ecE$ prove $\grJ$ arithmetic relatively cheaply.

