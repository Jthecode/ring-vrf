\subsection{Ring VRFs based on SpecialG}
\label{subsec:rvrf_faster}

We can apply the results of the previous subsections to construct a ring VRF using $\SpecialG$ that allows a\eprint{n incredibly }{ }fast amortized ring VRF prover. 
First, \PedVRF has a sigma protocol which proves the relation $\mathcal{R}_{eval}$, 
where $\mathcal{R}_{eval}$ is the instantiation of relation $\reltwo'(\pp)$ (for some appropriately chosen public parameters $\pp$). Second, we can use $\SpecialG$ 
for a relation $\relone$ similar to $\Rring$. In fact, we will instantiate $\relone$ with $\Rring^{\mathtt{inner}}$. 


We now present an appropriate $ \mathsf{Com}.\mathsf{Commit}(\sk) $ algorithm that together with $ \SpecialG $ efficiently instantiate the NIZK for $ \Rring^{\mathtt{inner}} $. This works on the Jubjub curve $\ecJ$ which contains a large subgroup $\grJ$ of prime order $p_\grJ$. Here, $p_\grJ < p$ where $ p $ is the order of $\grE$ used in our ring VRF construction\footnote{This condition can be satisfied if $\ecJ$ is an Edwards curve with a cofactor.}. We let $\genJ_0,\genJ_1,\genJ_2 \in \grJ$ be independent generators. We also fix a parameter $ \kappa $ where $(\log_2 p)/2 < \kappa < \log_2 p_\grJ$. $ \mathsf{Com}.\mathsf{Commit}(\sk) $ first samples $\sk_1,\sk_2 \in 2^\kappa$  where $\sk = \sk_0 + \sk_1 \, 2^{\lambda} \mod p$ and samples a blinding factor $d \leftsample \F_{p_\grJ} $. In the end, it outputs $ \sk_0, \sk_1,d $ as an opening and the commitment $\pk=\sk_0\, \genJ_0 + \sk_1\, \genJ_1 + d \genJ_2$ as a public key of our ring VRF construction. This commitment scheme is binding and perfectly hiding as our ring VRF construction requires because $ \pk $ is, in fact, a Pedersen commitment. Indeed, $\pk$ is a Pedersen commitment to $\sk$ because we can represent $ \sk = \sk_0\, \genJ_0 + \sk_1 \mod p$ since we have selected $ \kappa $ accordingly.

%TODO MAKE THIS CONSISTENT WITH THE DESCRIPTION SINCE WE DEFINE EVERYTHING ON ONE GROUP

We thus have a fairly efficient instantiation for \relone given by

$$  \Rring^{\inner} = \Setst{ (\comring, \sk ; \sk_0, \sk_1, d, \openring) }{
  \sk = \sk_0+ 2^{\lambda} \sk_1 \wedge
 \genfrac{}{}{0pt}{}{ \eprint{\rVRF.}{}\OpenRing(\comring,\openring) }{ \,\, = \sk_0 \genJ_0 + \sk_1 \genJ_1 + d \genJ_2 }
} \mathperiod $$

Combining $\SpecialG$ for $\relone$ and the sigma protocol which is part of $\PedVRF$ gives a 
NIZK for relation $\rel_{\mathtt{rvrf}}$ (i.e., an instantiation of $\rel$ from previous subsections):

$$ \rel_{\mathtt{rvrf}} = \Setst{ \Out, \msg, \comring ; \sk_0,\sk_1, d, \openring }{
    \eprint{
        \eprint{\rVRF.}{}\OpenRing(\comring,\openring)    \textrm{ and }
        \,\, = \sk_0 \genJ_0 + \sk_1 \genJ_1 + d \genJ_2, \quad
        \Out = \rVRF.\Eval(\sk_0+ 2^{\lambda} \sk_1,\msg)
    }{
        \begin{aligned}
        & \eprint{\rVRF.}{}\OpenRing(\comring,\openring) \\
        & \,\, = \sk_0 \genJ_0 + \sk_1 \genJ_1 + d \genJ_2,\\
        & \Out = \rVRF.\Eval(\sk_0+ 2^{\lambda} \sk_1,\msg)
        \end{aligned}
    }
} $$


\paragraph{Efficiency:} If we have a $\SpecialG$ proof for $\relone$ for our $\pk$ in a ring defined by $\comring$, 
to generate a ring VRF proof for the same ring, we need to run $\SpecialG.\Reprove$ and $\PedVRF.\Sign$. $\PedVRF.\Sign$ 
requires two scalar multiplications on $\ecE_1$
and two on the same or faster $\ecE'$,
so together with $\SpecialG.\Reprove$ costing four scalar multiplications
on $\ecE_1$ and two on $\ecE_2$, our amortised prover time
runs faster than 12 scalar multiplications on typical $\ecE_1$ curves. 
We expect the three pairings dominate verifier time, but
verifiers also need five scalar multiplications on $\ecE_1$.
Importantly, our fast ring VRF's amortised prover time now rivals
group signature schemes' performance \cite{group_sig_survey}.
We hope this ends the temptation to deploy group signature like
 constructions where the deanonymisation vectors matter. 

\begin{comment}
\paragraph{Efficiency:} If we have a $\SpecialG$ proof for $\relinner$ for our $\pk$ in a ring defined by $\comring$, 
to generate a ring VRF proof for the same ring, we need to run $\SpecialG.\Reprove$ and $\NIZK_{\mathcal{R}_{eval}}.\Prove$. $\NIZK_{\mathcal{R}_{eval}}.\Prove$ 
requires two scalar multiplications on $\grE_1$
and two on the same or faster $\grE'$,
so together with $\SpecialG.\Reprove$ costing four scalar multiplications
on $\grE_1$ and two on $\grE_2$, our amortised prover time
runs faster than 12 scalar multiplications on typical $\grE_1$ curves. 
We expect the three pairings dominate verifier time, but
verifiers also need five scalar multiplications on $\grE_1$.
\end{comment}

%TODO WHAT DOES IT MEAN
Importantly, our fast ring VRF's amortised prover time now rivals
group signature schemes' performance \cite{group_sig_survey}.
We hope this ends the temptation to deploy group signature like
constructions where the deanonymisation vectors matter. 

\begin{comment}
% TODO \PedVRF.\OpenKey(\compk,\openpk)

\def\longeq{=\mathrel{\mkern-10mu}=}% {=\joinrel=} % https://tex.stackexchange.com/questions/35404/is-there-a-wider-equal-sign
We describe a much faster choice \pifast for \piring with
opaque inputs $x_1 \longeq \sk$ and transparent inputs $y_1 \longeq \comring$
 so that taking
 $\genG \longeq \chi_1$, $\genB \longeq \genB_\gamma$, and $\openpk \longeq b$
in \PedVRF yields an incredibly fast amortised ring VRF prover.
Also \PedVRF itself proves knowledge of $X' =  \sk\, \chi_1 + b \genB_\gamma $,
 as required by $\SpecialG.\Verify$.
% $$ X' + Y = \comring\, \Upsilon_1 + \sk\, \chi_1 + b \genB_\gamma $$

A priori, we do not know $\chi_1$ during the trusted setup for $\pifast$,
which prevents computing $\pk = \sk\, \chi_1$ inside $\pifast$.
Instead, we propose $\ring$ contain commitments to $\sk$ over
some Jubjub curve $\ecJ$, while $\sk \in \F_p$ remains a scale for $\grJ$.

We know the large subgroup $\grJ$ of $\ecJ$ typically has smaller prime
order $p_\grJ$ than $\grE$, itself due to $\ecJ$ being an Edwards curve.
%
We thus choose $\sk_0,\sk_1 < p_\grJ$ with at least $\lambda$ bits
so that
 $\PedVRF.\sk = \sk_0 + \sk_1 \, 2^{\lambda} \mod p_\grE$
becomes our secret key.
\footnote{If $\lambda \approx 128$ then $p, p_\grJ > 2^{2\lambda-3}$.}
Our $\rVRF.\KeyGen$ \eprint{returns}{shall now return}
a secret key of the form $\rVRF.\sk = (\sk_0,\sk_1,d)$
 with $d \leftsample \F_{p_\grJ}$ and
a public key of the form
 $\rVRF.\pk = \sk_0\, \genJ_0 + \sk_1\, \genJ_1 + d \genJ_2$,
for some independent $\genJ_0,\genJ_1,\genJ_2 \in \grJ$. % (see \S\ref{subsec:AML_KYC}).
\footnote{Interestingly we avoid range proofs for $\sk_1$ and $\sk_2$ by this independence.}
We thus have a fairly efficient instantiation for $\Lring^\inner$ give by

$$ \Lfast^\inner = \Setst{ \sk_0 + 2^{128} \sk_1, \comring }{
 \eprint{ \exists d,\openring \textrm{\ s.t.\ } }{}
 % 0 < \sk_0,\sk_1 < 2^{128} \textrm{\ and\ } 
 \genfrac{}{}{0pt}{}{ \eprint{\rVRF.}{}\OpenRing(\comring,\openring) }{ \,\, = \sk_0 \genJ_0 + \sk_1 \genJ_1 + d \genJ_2 }
} \mathperiod $$

Applying our rerandomization \Reprove to $\pifast^\inner$ with opaque input
yields a zkSNARK $\pifast$ with the extra $\PedVRF.\OpenKey$ arithmetic to
have exactly the form $\piring$.

We explain later in \S\ref{sec:ring_hiding} how one could
choose $\chi_1$ independent before doing the trusted setup,
 and then wire $\chi_1$ into $\pifast$ inside $C$.
We could then prove $\pk = \sk\, \chi_1$ directly inside $\pifast^\inner$,
but doing so here requires slow non-native field arithmetic.

At this point, $\PedVRF.\Sign$ requires two scalar multiplications on $\ecE_1$
 and two on the somewhat faster $\ecE'$,
so together with rerandomization costing four scalar multiplications
on $\ecE_1$ and two on $\ecE_2$, our amortized prover time
 runs faster than 12 scalar multiplications on typical $\ecE_1$ curves. 
We expect the three pairings dominate verifier time, but
 verifiers also need five scalar multiplications on $\ecE_1$.

As an aside, one could construct a second faster curve with the same
group order as $\grE$, which speeds up two scalar multiplications
 in both the prover and verifier. 

Importantly, our fast ring VRF's amortised prover time now rivals
group signature schemes' performance \cite{group_sig_survey,}.
We hope this ends the temptation to deploy group signature like
 constructions where the deanonymization vectors matter.

% BEGIN TODO: Oana

\begin{theorem}\label{thm:knowledge_soundness}
\rVRF instantiated with \pifast and \PedVRF satisfies knowledge soundness.
\end{theorem}

\begin{proof}[Proof stetch.]
An extractor for \PedVRF reveals the opening of $X$ for us,
so our result follows from Lemma \ref{lem:knowledge_soundness}.
\end{proof}

% \begin{corollary}\label{cor:???}
% Our Pedersen ring VRF instantiated with \pifast satisfies ring unforgability and uniqueness.
% \end{corollary}

% \begin{theorem}\label{thm:pifast_anonymity}
% \rVRF instantiated with \pifast and \PedVRF satisfies zero-knowledge.
% \end{theorem}
%
% \begin{proof}[Proof stetch.]
% Assuming the same \comring, we know the zero-knowledge continuations
% are identically distributed by Lemma \ref{lem:unlinkable},
% even when reusing a zero-knowledge continuation $(X,A,B,C)$.
% It follows the typical simulator for \PedVRF ... WHAT???
% \end{proof}

% \begin{corollary}\label{cor:???}
% Our Pedersen ring VRF instantiated with \pifast satisfies ring anonymity.
% \end{corollary}

% END TODO: Oana

\end{comment}
