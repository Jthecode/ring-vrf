\section{Introduction}

\def\qaudbreak{\eprint{\quad}{\\}}


We introduce an anonymous credential flavor called
 ring verifiable random functions (ring VRFs),
in essence ring signatures that anonymize signers but
 also prove evaluation of the signers' PRFs.
Ring VRFs provide a better foundation for anonymous credentials
across a range of concerns, including formalization, optimizations,
the nuances of use-cases, and miss-use resistance.

Along with some formalizations, we address three questions within
the unfolding ring VRF story:
\begin{enumerate} 
% \item
% How can credentials use be contextual?  \qaudbreak
% Prove evaluation of a secret function.
\item
What are the cheapest SNARK proofs?  \qaudbreak
Ones users reuse without reproving.
\item
How can identity be safe for general use?  \qaudbreak
By revealing nothing except users' uniqueness.
\item
How can ration card issuance be transparent?  \qaudbreak
By asking users trust a public list, not certificates.
\end{enumerate}

% After briefly introducing ring VRFs, we discuss these three questions,
% which we later elaborate upon in
%  \S\ref{sec:rvrf_cont}, \S\ref{sec:app_identity}, and \S\ref{sec:app_rate_limits} 

% \smallskip 
\paragraph{Ring VRFs:}

A ring signature proves only that its actual signer lies in a ``ring''
of public keys, without revealing which signer really signed the message.
A {\it verifiable random function} (VRF) is a signature that proves
correct evaluation of a PRF defined by the signer's key.
% but nuances exist .

A {\it ring verifiable random function} (ring VRF) is a ring signature, in
that it anonymizes its actual signer within a ring of plausible signers,
but also proves correct evaluation of a pseudo-random function (PRF)
defined by the actual signer's key. % (see \S\ref{sec:rvrf_games}).
%
Ring VRF outputs then provide linking proofs between different signatures
iff the signatures have identical inputs, as well as pseudo-randomness.

As this pseudo-random output is uniquely determined by the signed message
and signer's actual secret key, we can therefore link signatures by the
same signer if and only if they sign identical messages.
In effect, ring VRFs restrict anonymity similarly to but less than
linkable ring signatures do, which makes them multi-use and contextual.

We define the security of ring VRFs in both the standard model
and in the universally composable (UC) \cite{canetti1,canetti2} model. 
We show that our ring VRF protocol is secure in the UC model.

In \S\ref{sec:rvrf_cont}, we build extremely efficient and flexible
ring VRFs by amortizing a ``zero-knowledge continuation'' that unlinkably
proves ring membership of a secret key, and then cheaply proving
individual VRF evaluations.

% We discuss the applications to identity in \S\ref{sec:app_identity} and
% to rationing in \S\ref{sec:app_rate_limits}.
% As a highly multi-use primitive, ring VRFs also present a multi-use

% First
% \smallskip 
\paragraph{Zero-knowledge continuations:}

Rerandomizable zkSNARKs like Groth16 \cite{Groth16} admit a
transformation of a valid proof into another valid but unlinkable
proof of the exact same statement.  In practice, rerandomization
never gets deployed because the public inputs link different usages,
breaking privacy.

We demonstrate in \S\ref{sec:rvrf_cont} a simple transformation of
any Groth16 zkSNARK into a {\it zero-knowledge continuation} whose
public inputs involve opaque Pedersen commitments, with cheaply
rerandomizable blinding factors and proofs.
These zero-knowledge continuations then prove validity of the contents
of Pedersen commitments, but can now be reused arbitrarily many times,
without linking the usages.

In brief, we adjust the trusted setup of the Groth16 to additionally
produce an independent blinding factor base for the Groth16 public input, 
along with an absorbing base that cancels out this blinding factor in the
Groth16 verification.
As our public inputs involve opaque Pedersen commitments,
they now require proofs-of-knowledge resentment of to \cite{LegoSNARK}. 
% In essence, this specializes the Groth16 

As recursive SNARKs might remain slow,
we expect zero-knowledge continuations via rerandomization become
essential for zkSNARKs used in identity and elsewhere outside the crypto-currency space.

% Second
% \smallskip
\paragraph{Identity uses:}

An identity system can be based upon ring VRFs in an natural way:
After verifying an identity requesting domain name in TLS,
our user agent signs into the session by returning a ring VRF
signature whose input is the requesting domain name, so their
ring VRF output becomes their unique identity at that domain
(see \S\ref{sec:app_identity}).

At this point, our requesting domain knows each users represents
distinct ring members, which prevents Sybil behavior, and permits
banning specific users.
At the same time, users' activities remain unlinkable across different
domains

In essence, ring VRF based credentials, if correctly deployed, only
prevent users being Sybil, but leak nothing more about users.  We argue
this yields diverse legally and ethically straightforward identity usages.

As a problematic contrast, attribute based credential schemes like
IRMA (``I Reveal My Attributes'') credentials \cite{IRMAcredentials}
are being marketed as an online privacy solution, but cannot prevent
users being Sybil unless they first reveal numerous attributes.
Attribute based credentials therefore provide little or no privacy
when used to prevent abuse.

Abuse and Sybil prevention is not merely the most common use cases for
anonymous credentials, but in fact define the ``general'' use cases for
anonymous credentials.
IRMA might improve privacy when used as ``special purpose'' credential 
in narrower situations of course, but overall attribute based credentials
should {\it never} be considered fit for general purpose usage.

Aside from general purpose identity being problematic for attribute based
credentials, our existing offline processes often better protect users'
privacy and human rights than adopting online processes like IRMA.
%
In particular, there are many proposals by the W3C for attribute based
credential usage in \cite{w3c_vc_use_cases}, but broadly speaking they
all bring matching harmful uses.  % https://www.w3.org/TR/vc-use-cases/

As an example, the W3C wants users to be able to easily prove their
employment status, ostensibly so users could open bank accounts purely
online.  Yet, job application sites could similarly demand these same
proofs of current employment, a discriminatory practice.
Average users apply for jobs far more often than they open bank accounts,
so credentials that prove current employment do more harm than good.

An IRMA deployment should prevent this abusive practice by making
verifiers prove some legal authorization to request employment status,
or other attributes, before user agents prove their attributes.
Indeed IRMA deployments need to regulate IRMA verifiers, certainly by
government privacy laws, or ideally by some more aggressive ethics board,
but this limits their flexibility and becomes hard internationally.

Ring VRFs avoid these abuse risks by being truly unlinkable, and thus
yield anonymous credentials which safely avoid legal restrictions.

{\it Any ethical general purpose identity system should be based
upon ring VRFs, not attribute based credentials like IRMA.}

We credit proof-of-personhood parties by Bryan Ford, et al. \cite{pop2008,pop2017}
% https://bford.info/pub/dec/pop-abs/  https://bford.info/pub/net/sybil-abs/
with first espousing the idea that anonymous credentials should produce
contextual unique identifiers, without leaking other user attributes.

As a rule, there exist simple VRF variants for all anonymous credentials,
including IRMA \cite{IRMAcredentials} or group signatures \cite{group_sig_survey}.
We focus exclusively upon ring VRFs for brevity, and because alone
ring VRFs contextual linkability covers the most important use cases.
% and our optimizations make ring VRFs extremely efficient.

% Third
% \smallskip
\paragraph{Rationing uses:}

A rate limiting or rationing system should provide users with a stream
of single-use anonymous tokens that each enable consuming some resource.
As a rule, cryptographers always construct these either
 from blind signatures ala \cite{chaum83}, or else
 from OPRFs like PrivacyPass \cite{PrivacyPass},
both of which have an $O(n)$ issuance phase.

Ring VRFs yield rate limiting or rationing systems with no issuance phase:\quad 
We first place into the ring the public keys for all users permitted to
consume resources, perhaps all legal residents within some country.  
We define single-use tokens to be ring VRF signatures whose VRF input
consists of a resource name, an approximate date, and a bounded counter.
Now merchants reports each anonymous token back to some authority who
enforces rate limits by rejecting duplicate ring VRF outputs.
(See \S\ref{sec:app_rate_limits})

In other words, our rate limiting authority treats outputs like the
``nullifiers'' in anonymous payment schemes.
Yet, ring VRF nullifiers need only temporarily storage, as eventually one
expires the date in the VRF input.  Asymptotically we thus only need
$O(\mathtt{users})$ storage vs the $O(\mathtt{history})$ storage
required by anonymous payment schemes like ZCash and blind signed tokens.

We further benefit from the ``ring'' credential format too,
 as opposed to certificate based designs like group signatures:\quad 
We expect a degree of fraud whenever deploying purely certificate
based systems, as witnessed by the litany of fraudulent TLS and covid
certificates.  Ring VRFs help mitigate fraudulent certificate concerns
because the ring is a database and can be audited.

We know governments have ultimately little choice but to institute
rationing in response to shortages caused by climate change, ecosystem
collapse, and peak oil.  Ring VRFs could help avoid ration card fraud,
and thereby reduce social opposition, while also protecting essential privacy.

As an important caveat, ring VRFs need heavier verifiers than single-use
tokens based on OPRFs \cite{PrivacyPass} or blind signatures, but
those credentials' heavy issuance phase represents a major adoption hurdle.
A ring VRF systems issue fresh tokens almost non-interactively merely by
adjusting allowed VRF input on resource names, dates, and bounds.
This reduces complexity, simplifies scaling, and increases flexibility.

In particular, if governments issue ration cards based upon ring VRFs
then these credentials could safely support other use cases, like
free tiers in online services or games, and advertiser promotions,
as well as identity applications like prevention of spam and online abuse.

In this, we need authenticated domain separation of products or identity
consumers in queries to users' ring VRF credentials.  We briefly discuss
some sensible patterns in \S\ref{subsec:app_ration_carts} below, but
overall authenticated domain separation resemble TLS certificates except
simpler in that roots of trust can self authenticate if root keys act as
domain separators.


%TODO Our contributions should be listed here: https://crypto.iacr.org/2023/papersubmission.php says that 'The introduction should summarize the contributions of the paper at the level understandable for a non-expert reader.'




\endinput




As a field, anonymous credentials come in myriad flavors,
many of which exist to limits the anonymity provided, ala
 attribute based credentials and group signatures. % \cite{group_sig_survey}.
% aka anonymized signatures
%
Ring VRFs by weakening anonymity only contextually provide a safer,
more private, more flexible, more powerful, and more ethical
choice for all everyday anonymous credential use cases.  % needs:  ???


