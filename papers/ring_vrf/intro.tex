% \section{Introduction}

We answer three questions:
\begin{enumerate}
\item
What are the cheapest SNARK proofs?  \quad
Ones users reuse without reproving.
\item
How can identity be safe for online use?  \quad
By revealing nothing except users' uniqueness.
\item
How can ration card issuance be trusted?  \quad
By asking users trust a public list, not certificates.
\end{enumerate}

Rerandomizable zkSNARKs like Groth16 \cite{Groth16} admit a
transformation of a valid proof into another valid but unlinkable
proof of the exact same statement.  In practice, rerandomization
was never depoyed because the public inputs link the usages.

We demonstrate in \S\ref{sec:rvrf_cont} a simple transformation of
any Groth16 zkSNARK into a {\it zero-knowledge continuation} whose
public inputs become opaque Pedersen commitments, with cheaply
rerandomizable blinding factors and proof.
These zero-knowledge continuations then prove validity of the contents
of Pedersen commitments, but can now be reused arbitrarily many times,
without linking the usages. 

A {\it ring verifiable random function} (ring VRF) is a ring signature
that proves correct evaluation of some pseudo-random function (PRF)
determined by the actual key pair used in signing (see \S\ref{sec:rvrf_def}).
We build extremely efficent and flexible ring VRFs by amortizing a
zero-knowledge continuation that unlinkably proves ring membership
of a secret key, and then cheaply proving individual VRF evaluations.

Ring VRFs then underpin a diverse array of anonymous credential schemes:

\smallskip
% \subsection{Identity}

As an identity system, ring VRFs evaluated on a specific context or
domain name output a unique identity for the user at that domain or
context (see \S\ref{sec:app_identity}), which thereby prevents
Sybil behavior and permits banning specific users.
Yet users' activities remain unlinkable across distinct contexts or
domains, which supports diverse ethical identity usages.

We contrast this ethically straightforward ring VRF based identity
with the ethically problematic case of attribute based credential
schemes like IRMA \cite{IRMA_credentials},
 which are now marketed as an online privacy solution.
IRMA could improve privacy in narrow situations of course, but
overall attribute based credentials should {\it never} be considered
fit for general purpose usage, like the prevention of Sybil behavior.
Aside from general purpose identity, our existing offline verification
processes often better protect user privacy and human rights than
adopting online processes like IRMA.

In particular, there are many proposals by the W3C for attribute based
credential usage in \cite{w3c_vc_use_cases}, but broadly speaking they
all bring matching harmful uses.  % https://www.w3.org/TR/vc-use-cases/
As an example, if users could easily prove their employment online when
applying for a bank account, then job application sites could similarly
demand proof of current employment, a clear injustice.

In general, abuse risks dictate that IRMA verifiers should be tighly
controlled by legislation, which becomes difficult internationally. 
%
Ring VRFs avoid these abuse risks by being truly unlinkable, and thus
yield anonymous credentials which safely avoid legal restrictions.

{\it Any ethical general purpose identity system should be based
upon ring VRFs, not attribute based credentials like IRMA.}

We credit Bryan Ford's work on proof-of-personhood parties \cite{pop2008,pop2017}
% https://bford.info/pub/dec/pop-abs/  https://bford.info/pub/net/sybil-abs/
with first expousing the idea that anonymous credentials should produce
contextual unique identifiers, without leaking other user attributes.

% TODO: AML/KYC

\smallskip
% \subsection{Rationing}

Ring VRFs yield rate limiting or rationing systems, which work
similarly to identity applications, except the VRF input should also
include a course date and a counter, and
 their outputs should be tracted as nullifiers, but only temporarily.

We must expect a degree of fraud whenever deploying purely certificate
based systems, as witnessed by the litany of fraudulent TLS and covid
certificates.  Ring VRFs avoid fraudulent certificates concerns because
they permit direct public audits of ring membership.

We know governments shall institute ration cards in responce to shortages
caused by climate change, ecosystem collapse, and peak oil.  Ring VRFs
could then help avoid ration card fraud and thus reduce social unrest.


\endinput























We first in \S\ref{sec:rvrf_glue} unify a simple fast VRF signature given by a DLEQ proof together with a SNARK that proves ring membership by the public key, but here both must handle the public key as a Jubjub Pedersen commitment.  ZCash sapling uses a similar Schnorr signature scheme, without the Jubjub Pedersen commitment (TODO). 

We next in \S\ref{sec:rvrf_cont} split this SNARK into a slower preliminary proof extracting the public key from the ring, and a continuations by a second faster SNARK that prepares the Jubjub Pedersen commitment.  We prefer this version whenever rings have low churn, but our first version has faster verifier time by one miller loop, making it preferable for very high churn rings.





We first in \S\ref{sec:rvrf_dleq} unify a simple fast VRF signature given by a DLEQ proof together with a SNARK that proves ring membership by the public key, but here both must handle the public key as a Jubjub Pedersen commitment.  ZCash sapling uses a similar Schnorr signature scheme, without the Jubjub Pedersen commitment (TODO). 

We next in \S\ref{sec:rvrf_two_snarks} split this SNARK into a slower preliminary proof extracting the public key from the ring, and a continuations by a second faster SNARK that prepares the Jubjub Pedersen commitment.  We prefer this version whenever rings have low churn, but our first version has faster verifier time by one miller loop, making it preferable for very high churn rings.




Also, zero-knowlede continuations wind up so much faster than bruit force techniques like recursion. 
