\subsection{Pederson keyed VRF-AD-KC}
\label{sec:vrf_perderson}

Anonymized VRF signatures could use Chaum-Pedersen DLEQ proofs that
hide the true public key, which our key committing definition permits.
Zero-knowledge continuations work by opening this key comittment
 inside another zero-knowledge proof. % DUP

We only describe the non-batchable variant analogous to
 \cite{nsec5} and \cite{VXEd25519}.
At least two easy batch verifiable flavors exist, but
 they enlarge the VRF signature by 32 bytes.
We need pairings to verify the SNARKs component of our ring VRF,
which dwarfs the CPU savings from batch verification,
so saving 32 bytes sounds more helpful.

% TODO: Alternatives, like slow recursion.  Just Intro?

\newcommand{\PedVRF}{\primalgo{PedVRF}} 

We define \KeyGen as in \ThinVRF in \S\ref{sec:vrf_thin} but \Eval differs
slightly by not injecting \pk into \msg:
% by selecting a secret key \sk uniformly at random from $\F_p$ and
% computing the public key $\pk = \sk \, \genG$.
\begin{itemize}
\item $\PedVRF.\KeyGen$ selects a secret key \sk uniformly at random from $\F_p$ and computes the public key $\pk = \sk \, \genG$. 
\item $\PedVRF.\Eval(\sk,\msg)$ takes a secret key \sk and an input $\msg$, and
 then returns a VRF output $H'(\msg,h \, \sk \, H_{\grE}(\msg))$.
\end{itemize}
We select here an arbitrary base point $\genG$ for our public key
 because this avoids some confusion later.

We fix a second generator $\genB$ of $\grE$ independent from $\genG$,
perhaps created by applying $H_\grE$ to an input outside existing usages'
domain and also containing $\genG$. 
We now form Pedersen-like commitments to this public key \pk as follows.
\begin{itemize}
\item $\PedVRF.\CommitKey$ selects a blinding factor $\openpk$ uniformly
 at random from $\F_p$ and computes the commitment $\compk = \pk + \openpk \, \genB$.
\item $\PedVRF.\OpenKey$ just returns $\pk = \compk - \openpk \, \genB$.
\end{itemize}
In fact, these are technically only Pedersen commitments to
the secret key \sk, not to the public key \pk, because
 \OpenKey does not enforce the structure of \pk.
Instead, our \Verify proves correctness of \compk, as
 required for our key binding condition, so
zero-knowledge continuations then use \OpenKey with only minor caveots.
% In particular, any VRF-AC-KC prevents adversaries from obtaining
% extra valid outputs, but ring VRF protocols need a proof-of-knowledge
% for the public key \pk if they demand that different \pk represent
% different functions.

\begin{itemize}
\item $\PedVRF.\Sign(\sk,\openpk,\msg,\aux)$ takes a secret key \sk and blinding factor \openpk, an input $\msg$, and auxiliary data \aux, and then performs
\begin{enumerate}
    \item compute the VRF input point $\In := H_{\grE}(\msg)$ and pre-output $\Out_0 := \sk \, \In$,
    \item Sample random $r_1,r_2 \leftarrow \F_p$ and compute $R = r_1 \genG + r_2 \genB$ and $R_\msg = r_1 \In$.
    \item Compute the challenge $c = H_q(\aux,\msg,\compk,h \Out_0,R,R_m)$,
     along with $s_1 = r_1 + c \sk$ and $s_2 = r_2 + c \, \openpk$.
    \item Return the signature $(\Out_0,c,s_1,s_2)$.
\end{enumerate}
\item $\PedVRF.\Verify(\compk,\msg,\aux,\sigma)$, parses $\sigma = (\Out_0,c,s_1,s_2)$, and then 
\begin{enumerate}
    \item recompute the VRF input point $\In := H_{\grE}(\msg)$,
    \item computes $R = s_1 \genG + s_2 \genB - c \compk$ and $R_m = s \In - c \Out_0$, and
    \item returns $H'(\msg, h \Out_0)$ if $c = H_q(\aux,\msg,\compk,\Out_0,R,R_\msg)$ or failure otherwise.
\end{enumerate}
\end{itemize}

We obtain roughly \cite{nsec5} or \cite{VXEd25519}
if we choose $\openpk = 0 = r_2$ in \Sign.

\begin{lemma}\label{prop:pedersen_vrf_hiding}
$\PedVRF$ is a correct key commitment and (perfectly) key hiding.
\end{lemma}


\begin{proposition}\label{prop:pedersen_vrf}
Assuming AGM in $\grE$, % $\ecE$ modulo $h$,
our $\PedVRF$ satisfies VRF correctness, key binding, uniqueness,
pseudorandomness, and unforgability. % (EUF-CMA-KC) on $(\msg,\aux)$.
\end{proposition}



\begin{proof}[Proof sketch]
Correctness holds trivially.

???
\end{proof}






\endinput






