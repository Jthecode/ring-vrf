 \documentclass[runningheads,evcountsame,a4paper,11pt,orivec]{llncs}
% \documentclass[a4paper,11pt]{amsart}

\usepackage[margin=2.5cm,includefoot]{geometry}

%=================Begin:Packages===================
\usepackage{graphicx}
% \usepackage{enumitem}
\usepackage{url}
% \usepackage{amsthm}
\usepackage{amsfonts}
\usepackage{amsmath}
\usepackage{hyperref}
% \usepackage[capitalize,nameinlink]{cleveref}
% \usepackage{framed}
% \usepackage{fancybox}
\usepackage[utf8]{inputenc}
\usepackage{mathtools}
% \usepackage{tcolorbox}
\usepackage{tikz}
% \usepackage{todonotes}
\usepackage{xspace}
\usepackage{xcolor}
\usepackage[linewidth=1pt]{mdframed}
\usepackage{ulem}
\usetikzlibrary{arrows,chains,matrix,positioning,scopes}

% TODO: Clean up macros

% \newcommand\doubleplus{+\kern-1.3ex+\kern0.8ex}
\newcommand\doubleplus{\ensuremath{\mathbin{+\mkern-10mu+}}}


%% oracles
\newcommand{\ora}[1]{\ensuremath{\mathcal{O}\mathsf{#1}}\xspace}
\newcommand{\oramsg}[1]{\ensuremath{\mathsf{#1}}\xspace}
%% algorithm
\newcommand{\algo}[1]{{\textsc{#1}}}
%%primitive algo
\newcommand{\primalgo}[1]{{\ensuremath{\mathsf{#1}}}\xspace}
%%primitive
\newcommand{\prim}[1]{{\ensuremath{\mathsf{#1}}}\xspace}
%%set
\newcommand{\setsym}[1]{{\ensuremath{\mathcal{#1}}}}
%%array
\newcommand{\arraysym}[1]{{\ensuremath{\mathsf{#1}}}}

\newcommand\N{\mathbb{N}}
\newcommand\F{\mathbb{F}}
% \newcommand\Gr{\mathbb{G}}


\def\mathperiod{.}
\def\mathcomma{,}



\newcommand*\set[1]{\{ #1 \}} % in text, we don't want {} to grow
\newcommand*\Set[1]{\left\{ #1 \right\}}
\newcommand*\setst[2]{\{ #1 | #2 \}}
\newcommand*\Setst[2]%
        {\left\{\,#1\vphantom{#2} \;\right|\left. #2 \vphantom{#1}\,\right\}}
% ``set such that''; puts in a vertical bar of the right height


\providecommand{\bin}{\ensuremath{\{0,1\}}\xspace}


% https://tex.stackexchange.com/questions/471713/is-mathrel-always-needed
% https://tex.stackexchange.com/questions/418740/how-to-write-left-arrow-with-a-dollar-sign/626086#626086
\makeatletter
\providecommand\leftsample{\leftarrow\mathrel{\mkern-2.0mu}\pc@smalldollar}
\providecommand\rightsample{\pc@smalldollar\mathrel{\mkern-2.0mu}\rightarrow}
%
\newcommand{\pc@smalldollar}{\mathrel{\mathpalette\pc@small@dollar\relax}}
\newcommand{\pc@small@dollar}[2]{%
  \vcenter{\hbox{%
    $#1\textnormal{\fontsize{0.7\dimexpr\f@size pt}{0}\selectfont\$\hskip-0.05em
 plus 0.5em}$%
  }}%
}
\makeatother


\newcommand{\KeyGen}{\primalgo{KeyGen}}

% Why was this \prim before?
\newcommand{\VRF}{\primalgo{VRF}} 
\newcommand{\rVRF}{\primalgo{rVRF}} 

\newcommand{\Sign}{\primalgo{Sign}}
\newcommand{\Verify}{\primalgo{Verify}}
\newcommand{\Eval}{\primalgo{Eval}}
\newcommand{\Prove}{\primalgo{Prove}}
\newcommand{\Simulate}{\primalgo{Simulate}}
\newcommand{\Extract}{\primalgo{Extract}}


\newcommand{\In}{\primalgo{In}} 
\newcommand{\Out}{\primalgo{Out}} 
\newcommand{\PreOut}{\ensuremath{\primalgo{Out}_0}\xspace} 

\newcommand{\vk}{\ensuremath{\mathsf{vk}}\xspace}
\newcommand{\sk}{\ensuremath{\mathsf{sk}}\xspace}
\newcommand{\pk}{\ensuremath{\mathsf{pk}}\xspace}
\newcommand{\apk}{\ensuremath{\mathsf{apk}}\xspace}
\newcommand{\pkring}{\ensuremath{\setsym{PK}}}
\newcommand{\msg}{\ensuremath{\mathsf{msg}}\xspace}
\newcommand{\aux}{\ensuremath{\mathsf{aux}}\xspace}
\newcommand{\ctx}{\ensuremath{\mathsf{ctx}}\xspace}
\newcommand{\ringset}{\ensuremath{\mathsf{ring}}\xspace}



\newcommand\SNARK{\primalgo{SNARK}}
\newcommand\NIZK{\primalgo{NIZK}}


\newcommand{\adv}{\ensuremath{\mathcal{A}}\xspace}

\endinput

\newcommand{\evalprove}{\primalgo{EvalProve}}
\newcommand{\link}{{\primalgo{Link}}}
\newcommand{\update}{{\primalgo{Update}}}
\newcommand{\hashG}{\primalgo{H}_{\GG}}
\newcommand{\secreteval}{\primalgo{Secret}\eval}
\newcommand{\secretprove}{\primalgo{Secret}\prove}
\newcommand{\secretverify}{\primalgo{Secret}\verify}

\newcommand{\randsel}[0]{\ensuremath{\xleftarrow{\text{\$}}}}
\newcommand{\rel}{\ensuremath{\mathcal{R}}}


\endinput




\newcommand{\skvrf}{\ensuremath{\sk^{\mathsf{vrf}}}}
\newcommand{\pkvrf}{\ensuremath{\pk^{\mathsf{vrf}}}}
\newcommand{\skrvrf}{\ensuremath{\sk^{\mathsf{rvrf}}}}
\newcommand{\pkrvrf}{\ensuremath{\pk^{\mathsf{rvrf}}}}
\newcommand{\sksign}{\ensuremath{\sk^{\mathsf{sign}}}}
\newcommand{\pksign}{\ensuremath{\pk^{\mathsf{sign}}}}
\newcommand{\skksign}{\ensuremath{\sk^{\mathsf{kesign}}}}
\newcommand{\pkksign}{\ensuremath{\pk^{\mathsf{kesign}}}}
\newcommand{\pkssale}{\ensuremath{\pk^{\mathsf{ssale}}}}
\newcommand{\D}{\ensuremath{\Delta}}
\newcommand{\skzkvrf}{\ensuremath{\sk^{\mathsf{zkvrf}}}}
\newcommand{\pkzkvrf}{\ensuremath{\pk^{\mathsf{zkvrf}}}}


\newcommand{\tab}[1]{\hspace{.05\textwidth}\rlap{#1}}
\newcommand{\tabdbl}[1]{\hspace{.1\textwidth}\rlap{#1}}
\newcommand{\tabdbldbl}[1]{\hspace{.15\textwidth}\rlap{#1}}
\newcommand{\tabdbldbldbl}[1]{\hspace{.19\textwidth}\rlap{#1}}



\newcommand{\game}[3][]{\operatorname{#2}^{#1}_{#3}(\secpar)}
\newcommand{\transcript}[1]{\langle #1 \rangle}
\newcommand{\eppt}{\pccomplexitystyle{EPPT}}
\newcommand{\pt}{\pccomplexitystyle{PT}}

% \renewcommand{\pcadvstyle}[1]{\ensuremath{\mathsf{#1}}}
% \newcommand{\zdv}{\pcadvstyle{Z}}

% \newcommand{\msg}[1]{\mathsf{#1}}

\newcommand{\simulator}{\ensuremath{\mathsf{Sim}}}
%\newcommand{\minote}[1]{\todo[color=green!30,inline]{\textbf{Michele says:} #1}}

\newcommand{\fvrf}{\mathcal{F}_{\textsf{vrf}}}
\newcommand{\fgvrf}{\mathcal{F}_{\textsf{rvrf}}}
\newcommand{\fcpke}{\mathcal{F}_{\mathsf{CPKE}}}
\newcommand{\pvrf}{\mathsf{\Pi}_{\textsf{rvrf}}}
\newcommand{\svrf}{\simulator_\mathsf{gvrf}}
\newcommand{\fnizk}{\mathcal{F}_{\textsf{nizk}}}
\newcommand{\fkes}{\mathcal{F}_{\textsf{sgke}}}
\newcommand{\fcom}{\ensuremath{\mathcal{F}_{\mathsf{com}}}}
\newcommand{\fsec}{\ensuremath{\mathcal{F}_\mathsf{ED-SMT}}}
\newcommand{\frsc}{\ensuremath{\mathcal{F}_{\mathsf{rSC}}}}
\newcommand{\fsasle}{\ensuremath{\mathcal{F}_{\mathsf{sle}}}}
\newcommand{\finit}{\ensuremath{\mathcal{F}_{\mathsf{init}}}}
\newcommand{\fsig}{\mathcal{F}_{\mathsf{sig}}}
\newcommand{\fros}{\mathcal{F}_{\mathsf{ros}}}
\newcommand{\fzkvrf}{\mathcal{F}_{\mathsf{zkvrf}}}
\newcommand{\fcommit}{\mathcal{F}_{\mathsf{commit}}}
\newcommand{\gclock}{\mathcal{G}_{\mathsf{clock}}}
\newcommand{\fcrs}{\mathcal{F}_{crs}}
\newcommand{\env}{\mathcal{Z}}
\newcommand{\stake}{\mathsf{st}}
\newcommand{\stakeset}{\setsym{ST}}

\newcommand{\sid}{\textsf{sid}}
\newcommand{\pid}{\textsf{pid}}
\newcommand{\user}{\mathsf{P}}
\newcommand{\defeq}{\coloneqq}


\newcommand{\evaluationslist}{\texttt{evaluations}}
\newcommand{\evaluationsecretlist}{\texttt{secrets}}
\newcommand{\vklist}{\texttt{verification\_keys}}
\newcommand{\siglist}{\texttt{signatures}}
\newcommand{\prooflist}{\texttt{proofs}}
\newcommand{\proofzklist}{\texttt{zkproofs}}
\newcommand{\Linklist}{\texttt{links}}
\newcommand{\emptylist}{\emptyset}
\newcommand{\fail}{\mathbf{fail}}
\newcommand{\R}{\mathsf{R}}
\newcommand{\bool}{\textit{bool}}
\newcommand{\lst}{\setsym{L}}
\newcommand{\distribution}{\setsym{D}}

\newcommand{\weak}{\ensuremath{W}}
\newcommand{\inbox}{\ensuremath{\setsym{I}}}
\newcommand{\dqueue}{\ensuremath{\setsym{Q}^\D}}
\newcommand{\wqueue}{\ensuremath{\setsym{Q}^\weak}}
\newcommand{\weaklist}{\ensuremath{\setsym{\weak}}}
\newcommand{\mID}{\ensuremath{\mathsf{mid}}}
\newcommand{\plist}{\ensuremath{\setsym{P}}}
\newcommand{\timeoutlist}{\ensuremath{\setsym{T}}}
\newcommand{\anony}{\ensuremath{\mathfrak{a}}}
\newcommand{\dleqr}{\R_\textsf{dleq}}
\newcommand{\view}{\mathsf{view}} 
\renewcommand{\adv}{\ensuremath{\mathcal{A}}}
\newcommand{\preoutputlist}{\arraysym{pre\-outputs}}




% \sloppy


\title{Thin VRF and pseudo-random frenemies}

\author{Jeffrey Burdges \and Alistair Stewart}
% \and Handan Kilinc-Alper
% Sergey Vasilyev

\date{}
% \institute{Web 3.0 Foundation}


\begin{document}
	
\maketitle

\begin{abstract}
We build a new thin(er batchable EC) VRF signature from a Schnorr
signature by replacing the base point with a pseudo-random linear
combination of the base point and the hashed-to-curve input.
%
As a Schnorr signature, thin VRF supports efficient batch verification,
without widening the signature like other proposals.

We also discuss auxiliary data signed by VRFs, multiple inputs,
and randomness concerns in other proposals.
\end{abstract}


% \section{Introduction}

\def\qaudbreak{\eprint{\quad}{\\}}

We introduce ring verifiable random functions (ring VRFs) as a natural
fulcrum around which anonymous credentials turn, in formalization,
in optimizations, in the nuances of use-cases, and in miss-use resistance.
%
Along with some formalizations, we explain portions of their unfolding
story which address three questions:
\begin{enumerate} 
\item
What are the cheapest SNARK proofs?  \qaudbreak
Ones users reuse without reproving.
% \item
% How can credentials use be contextual?  \qaudbreak
% Prove evaluation of a secret function.
\item
How can identity be safe for general use?  \qaudbreak
By revealing nothing except users' uniqueness.
\item
How can ration card issuance be transparent?  \qaudbreak
By asking users trust a public list, not certificates.
\end{enumerate}

We model the security of ring VRFs in the universally composable (UC) \cite{canetti1,canetti2} and  prove that our ring VRF protocol is UC secure. Thus, we guarantee the strongest security along with  practicality.
% First
\paragraph{Zero-knowledge continuations:}

Rerandomizable zkSNARKs like Groth16 \cite{Groth16} admit a
transformation of a valid proof into another valid but unlinkable
proof of the exact same statement.  In practice, rerandomization
was never deployed because the public inputs link the usages.

We demonstrate in \S\ref{sec:rvrf_cont} a simple transformation of
any Groth16 zkSNARK into a {\it zero-knowledge continuation} whose
public inputs become opaque Pedersen commitments, with cheaply
rerandomizable blinding factors and proofs.
These zero-knowledge continuations then prove validity of the contents
of Pedersen commitments, but can now be reused arbitrarily many times,
without linking the usages. 

As recursive SNARKs shall remain extremely slow,
we expect zero-knowledge continuations via rerandomization become
essential for zkSNARKs used outside the crypto-currency space.

% \smallskip 
\paragraph{Ring VRFs:}

A {\it ring verifiable random function} (ring VRF) is a ring signature
that proves correct evaluation of some pseudo-random function (PRF)
determined by the actual key pair used in signing. % (see \S\ref{sec:rvrf_games}).
We build extremely efficient and flexible ring VRFs by amortizing a
zero-knowledge continuation that unlinkably proves ring membership
of a secret key, and then cheaply proving individual VRF evaluations.

As the PRF output is uniquely determined by the signed message and
signers actual secret key, we can therefore link signatures by the
same signer if and only if they sign identical messages.
In effect, ring VRFs restrict anonymity similarly to but less than
 linkable ring signatures do, which makes them multi-use and contextual.

% Second
% \smallskip
\paragraph{Identity uses:}

As an identity system, ring VRFs evaluated on a specific context or
domain name output a unique identity for the user at that domain or
context (see \S\ref{sec:app_identity}), which thereby prevents
Sybil behavior and permits banning specific users.
Yet users' activities remain unlinkable across distinct contexts or
domains, which supports diverse ethical identity usages.

We contrast this ethically straightforward ring VRF based identity
with the ethically problematic case of attribute based credential
schemes like IRMA (``I Reveal My Attributes'') credentials \cite{IRMAcredentials},
 which are now marketed as an online privacy solution.
IRMA could improve privacy in narrow situations of course, but
overall attribute based credentials should {\it never} be considered
fit for general purpose usage, like the prevention of Sybil behavior.

Aside from general purpose identity, our existing offline
verification processes often better protect user privacy and human
rights than adopting online processes like IRMA.
%
In particular, there are many proposals by the W3C for attribute based
credential usage in \cite{w3c_vc_use_cases}, but broadly speaking they
all bring matching harmful uses.  % https://www.w3.org/TR/vc-use-cases/
As an example, if users could easily prove their employment online when
applying for a bank account, then job application sites could similarly
demand proof of current employment, a clear injustice.

In general, abuse risks dictate that IRMA verifiers should be tightly
controlled by legislation, which becomes difficult internationally. 
%
Ring VRFs avoid these abuse risks by being truly unlinkable, and thus
yield anonymous credentials which safely avoid legal restrictions.

{\it Any ethical general purpose identity system should be based
upon ring VRFs, not attribute based credentials like IRMA.}

We credit Bryan Ford's work on proof-of-personhood parties \cite{pop2008,pop2017}
% https://bford.info/pub/dec/pop-abs/  https://bford.info/pub/net/sybil-abs/
with first espousing the idea that anonymous credentials should produce
contextual unique identifiers, without leaking other user attributes.

As a rule, there exist simple VRF variants for all anonymous credentials
like IRMA \cite{IRMAcredentials} or group signatures \cite{group_sig_survey}.
We focus exclusively upon ring VRFs for brevity, and because alone
ring VRFs contextual linkability covers more important use cases.

% Third
% \smallskip
\paragraph{Rationing uses:}

Ring VRFs yield rate limiting or rationing systems, which work
similarly to identity applications, except their VRF inputs should also
include an approximate date and a bounded counter, and
 then their outputs should be tracked as nullifiers.
Yet, these nullifiers need only temporarily storage, which improves 
efficiency over anonymous money schemes like ZCash and blind signed tokens.

We expect a degree of fraud whenever deploying purely certificate
based systems, as witnessed by the litany of fraudulent TLS and covid
certificates.  Ring VRFs help mitigate fraudulent certificate concerns
because the ring is a database and can be audited.

We know governments have ultimately little choice but to institute
rationing in response to shortages caused by climate change, ecosystem
collapse, and peak oil.  Ring VRFs could help avoid ration card fraud,
and thereby reduce social unrest, while also protecting essential privacy.

Ring VRFs need heavier verifiers than single-use token credentials
based on OPRFs \cite{PrivacyPass} or blind signatures.
Yet, ring VRFs avoid these schemes separate issuance phase entirely,
and sometimes even their registration phase.  Instead, fresh tokens
merely require adjusting the approximate date in the VRF input.
This reduces complexity, simplifies scaling, and increases flexibility.

In particular, if governments issue ration cards based upon ring VRFs
then these credentials could safely support other use cases, like
free tiers in online services or games, and advertiser promotions,
as well as identity applications like prevention of spam and online abuse.

In this, we need authenticated domain separation of products or identity
consumers in queries to users' ring VRF credentials.  We briefly discuss
some sensible patterns in \S\ref{???} below, but overall authenticated
domain separation resemble TLS certificates except simpler in that
roots of trust can self authenticate if root keys act as domain separators.





\endinput




As a field, anonymous credentials come in myriad flavors,
many of which exist to limits the anonymity provided, ala
 attribute based credentials and group signatures. % \cite{group_sig_survey}.
% aka anonymized signatures
%
Ring VRFs by weakening anonymity only contextually provide a safer,
more private, more flexible, more powerful, and more ethical
choice for all everyday anonymous credential use cases.  % needs:  ???



% 
\section{Background}
\label{sec:background}

\def\secparam{\ensuremath{\lambda}\xspace}

\def\ecE{{\mathbb{E}}}
\def\grE{{\mathbf{E}}}
\def\genE{E}
\def\genG{G}
\def\genB{K} %{\genE_{\mathrm{bind}}}

\def\ecJ{{\mathbb{J}}}
\def\grJ{{\mathbf{J}}}
\def\genJ{J}

% As our ring VRF is built by composing them, 
We briefly recall the primitives and security assumptions underlying
both Chaum-Pedersen DLEQ proofs and pairing based zkSNARKs. 


\subsection{Elliptic curves}

We obey mathematical and cryptographic implementation convention by using additive notation for elliptic curve and multipicative notation for eliptic curve scalar multiplications. 

We always implicitly have a paramater generation procedure $\mathtt{Params}$ that takes a security level $\secparam \in \N$ and returns elliptic curve paramaters including some prime numbers and efficient algorithms for computing elliptic curve operations.  As customary, we treat $\secparam$ and the output of $\mathtt{Params}$ as fixed paramaters, which makes sense because humans run $\mathtt{Params}$ manually in practice. 

As implicit outputs of $\mathtt{Params}$, we work with an elliptic curve $\ecE[\F]$ over some base field $\F$ of (prime) characteristic $q_{\grE}$, along with a distinguished subgroup $\grE \le \ecE[\F]$ of prime order $p_{\grE} \approx 2^{2\secparam}$.  As $\grE$ distinguishes $\ecE[\F]$, we let $h_{\grE}$ denote the cofactor of $\grE$ in $\ecE[\F]$, meaning $\ecE[\F]$ has $h_{\grE} p_{\grE}$ points.
% but abbreviate $h = h_{\grE}$, $p = p_{\grE}$, and $q = q_{\grE}$ when $\grE$ is clear from context.
We write $\grE$ without subscript, and abbreviate $h = h_{\grE}$, $p = p_{\grE}$, and $q = q_{\grE}$, when $\ecE$ is either our uinque pairing friendly curve or else the only curve in view.

We let $H_p : \{0,1\}^* \to \F_p$ or $H_q : \{0,1\}^* \to \F_q$ denote random oracles (RO) with a range $\F_p$ or $\F_q$.  We let $H_\ecE : \{0,1\}^* \to \ecE$ or $H_\grE : \{0,1\}^* \to \grE$ denote a hash-to-curve for $\ecE$ or hash-to-group for $\grE$, which we model as a random oracles too.  We note $H_\grE(x) = h H_\ecE(x)$ always works, although more efficent exist.

\smallskip

Almost all SNARKs like \cite{Groth16} or \cite{plonk} employ a pairing friendly elliptic curve $\ecE$ over a field $\F_q$ of characteristic $q \approx 2^{2\secparam}$, which comes equipped with a type III pairing on subgroups of prime order $p \approx 2^{2\secparam}$:  We let $q_1,q_2,q_T$ denote small powers of $q$, and let $\grE_1 \le \ecE[\F_{q_1}]$ and $\grE_2 \le \ecE[\F_{q_2}]$ and $\grE_T \le \F_{q_T}^\times$ denote subgroups all of prime order $p$.  We also let $e : \grE_1 \times \grE_2 \to \grE_T$ denote a type III pairing, meaning a computable bilinear map without known efficiently computable maps between $\grE_1$ and $\grE_2$.  Also $q_i = q_{\grE_i}$ for $i=1,2$ in our above notation.  

Any pairing friendly elliptic curve $\ecE$ provides solutions to the decisional Diffie-Hellman problem (DDH).  We do however assume the computational Diffie-Hellman problem (CDH) remains hard in $\ecE$.  We remark that $H_\grE$ being a random oracle plus CDH hardness prevents computable relationships between $H_\grE$ outputs.

% TODO: Pairing assumptions required by Groth16

\smallskip

% We shall require ZCash Sapling style ``Jubjub'' Edwards curves, whose base field characteristic divides of the order of a pairing friendly elliptic curve, thereby making Jubjub base field arithmetic SNARK friendly, and hence Jubjub elliptic curve operations as well \cite{}.

We let $\ecJ$ denote a ZCash Sapling style ``JubJub'' Edwards curve associated to the pairing friendly elliptic curve $\ecE$, meaning $\ecJ$ has base field $\F_p$ whose characteristic $q_{\grJ} = p$ matches the group order $p$ of $\grE_1 \cong \grE_2 \cong \grE_T$.  As in ZCash Sapling, we now prove statements about elliptic curve operations inside $\ecJ$ by proving base field arithmetic in $\F_p$, which our $q_{\grJ} = p$ makes relatively inexpensive inside SNARKs on $\ecE$.

As above, $\grJ \le \ecJ[\F_p]$ has large prime order $p_{\grJ}$ and a small cofactor $h_{\grJ}$.  We always support $4 p_{\grJ} < p$ because if $\ecJ$ is an Edwards curve then $h_{\grJ} \ge 4$ which imposes this by the Hasse bound.

\smallskip

We ask that deserialization prove that putative elements of $\grE$ lie in
$\ecE[\F]$ by verifying curve equations, perhaps including twist checks.

Anytime $\ecE$ represents a pairing friendly curve then we ask that
deserialization prove elements of $\grE_1$, $\grE_2$, and $\grE_T$
lie inside the correct subgroup of order $p$,
 which typically requires checking whether $|\grE| X = 1$ or similar.
As our SNARKs works with points in $\ecJ$ directly, we conversely
prefer writing $\grJ$ equations in $\ecJ[\F_p]$ and explicitly describe
where one clears the cofactor $h_{\grJ}$.  We handled $\grE$ withr
$\ecE$ not necessarily pairing friendly similarly to $\ecJ$.
We scrape by with only CDH hardness for $\grJ$ for convenience,
although DDH winds up hard in $\grJ$.


\subsection{Zero-knowledge proofs}

\newcommand\rel{\ensuremath{\mathcal{R}}\xspace}
\newcommand\lang{\ensuremath{\mathcal{L}}\xspace}

% refs.
% https://people.csail.mit.edu/silvio/Selected%20Scientific%20Papers/Zero%20Knowledge/Noninteractive_Zero-Knowkedge.pdf
%   Alright but kinda poorly phrases
% https://inst.eecs.berkeley.edu/~cs276/fa20/notes/Multiple%20NIZK%20from%20general%20assumptions.pdf
%   Addresses the ZK definitions better
% 

We let \rel denote a polynomial time decidable relation, so the language
 $\lang = \{x \mid \exists \omega (\omega,x) \in \rel \}$ lies in NP.
All non-interactive zero-knowledge proof systems have some setup procedure $\mathtt{Setup}$ that takes our parameters generated by $\mathtt{Params}$ and some ``circuit'' description of \rel, and produces a structured reference string (SRS).

A non-interactive proof system for $\rel$ consists of \Prove and \Verify PPT algorithms
\begin{itemize}
%\item $\NIZK.\setup(\rel) \rightarrow (crs, \tau)$ ---- Given the relation $ \rel $ it outputs a common reference string $ crs $ and a trapdoor $ \tau $ for $ \rel $.
\item $\NIZK_\rel.\Prove(\omega, x) \mapsto \pi$ creates a proof $\pi$ for a witness and statement pair $(\omega,x) \in \rel$.
\item $\NIZK_\rel.\Verify(x, \pi)$ returns either true of false, depending upon whether $\pi$  proves $x$.
\end{itemize}	
which satisfy the following completeness, zero-knowledge, and knowledge soundness definitions.

\begin{definition}\label{def:nizk_completeness}
We say $\NIZK_\rel$ is {\em complete} if $\Verify(x, \Prove(\omega,x)$ succeeds for all $(\omega,x) \in \rel$.  % with high probability
\end{definition}

\def\advV{\ensuremath{V^*}\xspace} % Why not use \adv here?

\begin{definition}\label{def:nizk_zero_knowledge}
We say $\NIZK_\rel$ is {\em zero-knowledge} if
there exists a PPT simulator $\NIZK_\rel.\Simulate(x) \mapsto \pi$
that outputs proofs for statement $x \in L$ alone, which are
computationally indistinguishable from legitimate proofs by \Prove,
i.e.\ any non-uniform PPT adversary \advV cannot distinguish pairs $(x,\pi)$
generated by \Simulate or by \Prove except with odds negligible in \secparam
(see \cite[Def. 9, \S A, pap. 29]{RandomizationGroth16}). %  or ...
\end{definition}

\def\advP{\ensuremath{P^*}\xspace} % Why not use \adv here?

\begin{definition}\label{def:nizk_knowledge_sound}
We say $\NIZK_\rel$ is {\em (white-box) knowledge sound} if
for any non-uniform PPT adversary \adv who outputs a statement $x \in \lang$ and proof $\pi$
there exists a PPT extractor algorithm $\Extract$ that white-box observes $\advP$ and
if $\Verify(x,\pi)$ holds then $\Extract$ returns an $\omega$ for which $(\omega,x) \in \rel$
(see \cite[Def. 7, \S A, pap. 29]{RandomizationGroth16}).
\end{definition}

Our zero-knowledge continuations in \S\ref{sec:rvrf_cont} demand
rerandomizing existing zkSNARKs, which only Groth16 supports \cite{Groth16}.
We therefore introduce some details of Groth16 \cite{Groth16} there,
when we tamper with Groth16's SRS and $\mathtt{Setup}$ to create zero-knowledge continuations. 
% TODO: Do we describe Groth16 \cite{Groth16} enough?

% In this, we exploit several arguments given by \cite{RandomizationGroth16},
% but for now we recall that \cite{RandomizationGroth16} proves that Groth16
% satisfies: % white-box weak simulation-extractablity .
%
% \begin{definition}\label{def:nizk_weak_simulation_extractable}
% We say $\NIZK_\rel$ is {\em white-box weak simulation-extractable} if
% for any non-uniform PPT adversary \advP with oracle access to \Simulate
% who outputs a statement $x \in \lang$ and proof $\pi$,
% there exists a PPT extractor algorithm $\Extract$ that white-box observes $\advP$ and
% if \advP never queried $x$ and $\Verify(x,\pi)$ holds
% then $\Extract$ returns an $\omega$ for which $(\omega,x) \in \rel$
% (see \cite[Def. 7, \S 2.3, pap. 29]{RandomizationGroth16}).
% \end{definition}

TODO: AGM and Groth16 here?


\subsection{Universal Composable (UC) Model}

TODO: Chat on why UC is here?

A protocol $ \phi $ in the UC model is an execution between distributed interactive Turing machines (ITM). Each ITM has a storage to collect the incoming messages from other ITMs, adversary \adv or the environment $ \env $. $ \env $ is an entity to represent the external world outside of the protocol execution.  The environment $ \env $ initiates ITM instances (ITIs) and the adversary \adv with arbitrary inputs and then terminates them to collect the outputs.
% An ITM that is initiated by $ \env $ is called ITM instance (ITI). 
We identify an ITI with its session identity $ \sid $ and its ITM's identifier $ \pid $. In this paper, when we call an entity as a party in the UC model we mean an ITI with the identifier $ (\sid, \pid) $.

We define the ideal world where there exists an ideal functionality $ \mathcal{F} $ and the real world where a protocol $ \phi $ is run as follows:

\paragraph{Real world:} $ \env $ initiates ITMs and \adv to run the protocol instance with some input $ z \in \{0,1\}^* $  and a security parameter $ \secparam $. After $ \env $ terminates the protocol instance, we denote the output of the real world by the random variable $ \mathsf{EXEC}(\secparam, z)_{\phi, \adv, \env} \in \{0,1\} $. Let $ \mathsf{EXEC}_{\phi, \adv, \env} $ denote the ensemble $ \{\mathsf{EXEC}(\secparam, z)_{\phi, \adv, \env} \}_{z \in \{0,1\}^*} $.

\paragraph{Ideal world:} $ \env $ initiates ITMs and a simulator $ \sim $ to contact with the ideal functionality $ \mathcal{F} $ with some input $ z \in \{0,1\}^* $  and a security parameter $ \secparam $. $ \mathcal{F} $ is trusted meaning that it cannot be corrupted.
$ \sim $ forwards all messages forwarded by $ \env $ to $ \mathcal{F} $. The output of execution with $ \mathcal{F} $ is denoted by a random variable $ \mathsf{EXEC}(\secparam, z)_{\mathcal{F},\sim, \env} \in \{0,1\}$.  Let $ \mathsf{EXEC}_{\mathcal{F},\sim, \env} $ denote the ensemble $ \{\mathsf{EXEC}(\secparam, z)_{\mathcal{F}, \sim, \env} \}_{z \in \{0,1\}^*} $.

TODO: \secparam should likely be implicit, especially since it appears in both worlds.

\begin{definition}[UC-Security of $ \phi $] \label{def:uc}
Given a real world protocol $ \phi $ and an ideal functionality $ \mathcal{F} $ for the protocol $ \phi $, we call that $ \phi $ is UC-secure if $ \phi $ UC-realizes $ \mathcal{F} $ if for all PPT adversaries \adv, there exists a simulator $ \sim  $ such that for any environment $ \env $,
 $\mathsf{EXEC}_{\phi, \adv, \env}$ indistinguishable from $\mathsf{EXEC}_{\mathcal{F},\sim, \env}$
\end{definition}

TODO: if ... if makes no sense.  These definitions need much clearer explanation, or more likely citations to places with clear explanations. 

\begin{definition}[UC-Security of $ \phi $ in the hybrid world]
Given a real world protocol $ \phi $ which runs some (polynomially many) functionalities $ \{\mathcal{F}_1, \mathcal{F}_2, \ldots, \mathcal{F}_k\} $ in the ideal world and an ideal functionality $ \mathcal{F} $ for the protocol $ \phi $, we call that $ \phi $ is UC-secure in the hybrid model $ \{\mathcal{F}_1, \mathcal{F}_2, \ldots, \mathcal{F}_k\} $ if $ \phi $ UC-realizes $ \mathcal{F} $ if for all PPT adversaries \adv, there exists a simulator $ \sim  $ such that for any environment $ \env $,
 $\mathsf{EXEC}_{\phi, \adv, \env}$ is indistinguishable from $\mathsf{EXEC}_{\mathcal{F},\sim, \env}$.
\end{definition}

% REMARKS:  Removed excessive notation $\approx$.














\endinput



BROKEN BOLOW THIS




We fix $J \in \ecJ$ as a generator for public keys.  Any $\KeyGen$ algorithm randomly samples a secret keys $\sk \in \F_q$ and then computes its associate public keys $\pk = \sk J$.  We shall not discuss infrastructure that authorizes public keys.  Yet although our results do not require proof-of-knowledge on $\pk$ per se, we still strongly recommend that back certifications accompany any certificates that authorize $\pk$.

\smallskip






\section{Thin batchable EC VRF-AD}
\label{sec:thin_vrf}

\def\secparam{\ensuremath{\lambda}\xspace}

\def\ecE{{\mathbb{E}}}
\def\grE{{\mathbf{E}}}
\def\genE{E}
\def\genG{G}
\def\genB{K} %{\genE_{\mathrm{bind}}}

\def\ecJ{{\mathbb{J}}}
\def\grJ{{\mathbf{J}}}
\def\genJ{J}

\newcommand{\ThinVRF}{\primalgo{ThinVRF}} 

We describe our protocol first, and then discuss the security zoo later in \S\ref{sec:security}.

\begin{definition}
a {\em verifiable random function with auxiliary data} (VRF-AD) consists of several algorithms:
\begin{itemize}
\item $\VRF.\KeyGen$ and returns a public key \pk and a secret key \sk, which one typically instantiates via come commitment scheme. 
% \item $\VRF.\Eval : (\sk,\msg) \mapsto \Out$ takes a secret key \sk and an input $\msg$, and then returns a VRF output $\Out$.
\item $\VRF.\Sign : (\sk,\msg,\aux) \mapsto \sigma$ takes a secret key \sk, an input \msg, and auxiliary data \aux, and then returns a VRF signature $\sigma$.
\item $\VRF.\Verify$ takes $(\pk,\msg,\aux,\sigma)$ for a public key \pk, an input \msg, and auxiliary data \aux, and then returns either an output $\Out$ or else failure $\perp$.
\end{itemize}
\end{definition}

% \smallskip

We obey mathematical and cryptographic implementation convention by using additive notation for elliptic curve and multiplicative notation for elliptic curve scalar multiplications.  We let \secparam denote our security parameter choose an elliptic curve $\ecE[\F]$ denote over some base field $\F$ of (prime) characteristic $q$, along with a distinguished subgroup $\grE \le \ecE[\F]$ of prime order $p \approx 2^{2\secparam}$, and let $h$ denote the cofactor of $\grE$ in $\ecE[\F]$, meaning $\ecE[\F]$ has $h_{\grE} p_{\grE}$ points.

We let $H_p : \{0,1\}^* \to \F_p$ or $H_q : \{0,1\}^* \to \F_q$ denote random oracles (RO) with a range $\F_p$ or $\F_q$.  We let $H_\ecE : \{0,1\}^* \to \ecE$ or $H_\grE : \{0,1\}^* \to \grE$ denote a hash-to-curve for $\ecE$ or hash-to-group for $\grE$, which we model as a random oracles too.  We note $H_\grE(x) = h H_\ecE(x)$ always works, although more efficient exist.

We work solely in $\ecE$ here because we need only a basic Chaum-Pedersen DLEQ proof.
As in \S\ref{sec:background} and throughout,
 $\ecE$ has order $h p$ with $p \approx 2^{2\secparam}$ prime and $h$ a small cofactor.

\begin{itemize}
\item $\ThinVRF.\KeyGen$ selects a secret key \sk uniformly at random from $\F_p$ and computes the public key $\pk = \sk \, \genE$.
 We define equivalence $\pk_1 \equiv \pk_2$ of public keys by $h \pk_1 = h \pk_2$.
% \item $\ThinVRF.\Eval(\sk,\msg)$ takes a secret key \sk and an input $\msg$, and
%  then returns a VRF output $H'(\msg,\pk,h \, \sk \, H_{\grE}(\msg,\pk))$.
\item $\ThinVRF.\Sign(\sk,\msg,\aux)$ takes a secret key \sk, an input $\msg$, and auxiliary data \aux, and then performs
\begin{enumerate}
    \item compute the VRF input $\In := H_{\grE}(\msg,\pk)$ and pre-output output $\PreOut := \sk \, \In$, 
    \item compute the delinearization challenge $c_1 = H_p(\aux,\msg,\pk,\PreOut)$,
    \item sample $r$ uniformly at random from $\F_p$ and compute $R = r (\genE + c_1 \In)$,
    \item compute the challenge $c = H_p(\aux,\msg,\pk,\PreOut,R)$, the proof $s = r + c \, \sk$, and return the signature $\sigma = (\PreOut,R,s)$.
\end{enumerate}
\item $\ThinVRF.\Verify$ takes $(\pk,\msg,\aux,\sigma)$, parses $\sigma = (\PreOut,R,s)$, and then 
\begin{enumerate}
%PoK:    \item abort unless either $\msg$ contains $\pk$ or else \pk has a valid the proof-of-knowledge,
    \item recomputes the VRF input point $\In := H_{\grE}(\msg,\pk)$,
    \item recomputes $c_1 = H_p(\aux,\msg,\pk,\PreOut)$ and $c = H_p(\aux,\msg,\pk,\PreOut,R)$, % the challenges
    \item aborts unless $s h (\genE + c_1 \In) = h R + c h (\pk + c_1 \PreOut)$ holds, and 
    \item returns $H'(\msg,\pk,h \PreOut)$ if nothing failed.
\end{enumerate}
\end{itemize}
Implicitly, we have $\ThinVRF.\Eval(\sk,\msg) \mapsto \Out$ given by
 $H'(\msg,\pk,h \, \sk \, H_{\grE}(\msg,\pk))$ too, but it could be
defined using \Sign and \Verify.

As discussed below, if we omit this final hash $H'$ making
our output only $h \PreOut$, then we obtain only a VUF, not a VRF.
We caution that $h \ne 1$ ensures SUF-CMA fails
 by \cite[\S4.1.2]{cryptoeprint:2020:823}.

If desired, one could generalize \ThinVRF to $k$ messages by
computing for $j=1,\ldots,k$ the $k$ distinct
points $\In_j := H_{\grE}(\msg_j)$, pre-outputs $\PreOut := \sk \, \In$,
delinearization challenges
 $c_j = H_p(\aux,\msg_1,\ldots,\msg_k,\pk,\Out_{0,1},\ldots,\Out_{0,k},j)$,
and then running our Schnorr-like signature with
 base point $\genE + \sum_{j=1}^k c_i \In_j$ and
 public key $\genE + \sum_{j=1}^k c_i \Out_j$.

\smallskip
% \subsection{Performance}

\ThinVRF needs two scalar multiplications in the prover and
then four scalar multiplications in the verifier
just like fat EC VRFs like \cite{nsec5} or \cite{VXEd25519} do.
We do incur an extra hashing operation and two field multiplications,
but they cost relatively little.
\ThinVRF reduces verifier computation by merging the these two
multi-scalar multiplication algorithms, which outweighs the extra hashing and field operations.

\ThinVRF supports batch verification or half-aggregation \cite{???}
without altering the signature or increasing the signature size.
We think this tips the scales as avoiding a separate batchable VRF
signature simplifies interface over naive batch verification methods
for \cite{nsec5} or \cite{VXEd25519}.

% In principle, both \ThinVRF and EC VRF could support many Schnorr features,
% such as adapter signatures \cite{???}, blind signatures \cite{???},
% and 2-round multi-signatures \cite{???,???}.
% Yet, \ThinVRF shares more code with Schnorr

\endinput 












% TODO: Proof correct?  Use same citations as schnorrkel.

% We define $H_\grG(\msg) = h H_\grE(\msg)$ and observe 
%
% \begin{lemma}
% If $H_\grE$ is a random oracle then $H_\grG$ is also a random oracle.
% \end{lemma}

% \begin{lemma}
% $\primalgo{PreEval}(\sk,\msg) = h \sk H_{\grE}(\msg,\pk)$
% \end{lemma}

We discuss chosen message queries against only one key in pseudo-randomness.  
% TODO: What?
In \ThinVRF, we hash the public key \pk along with the message \msg
in $H_\grE$, aka injected \pk into \msg, to prevent
related but different keys having algebraically related input points \In.
We cannot employ this trick in key committing VRFs or ring VRFs however.
Although $H'$ being a PRF mitigates problems, we still recommend caution 
when combining identical inputs \msg with related secret keys,
 like ``blockchain'' users often produce with ``soft key derivation''.

\begin{proposition}\label{prop:thin_vrf}
Assuming AGM in $\grE$, % $\ecE$ modulo $h$,
our $\ThinVRF$ satisfies
 VRF correctness, uniqueness, pseudo-randomness,
 and existential unforgeability on $(\msg,\aux)$.
\end{proposition}



\endinput







\begin{proof}[Proof sketch]
	TODO: ???
\end{proof}
















We expect $\ThinVRF$ to be an EUF-CMA signature scheme on $(\msg,\aux)$ too,
but proving this requires techniques outside our scope, even assuming AGM.

\begin{proof}[Proof sketch]
Correctness holds trivially.

At any fixed $\msg$ we have a Schnorr signature on $\aux$
 over the basepoint $\genE + c_1 \In$, which we name $\primalgo{VRFInner}_{\msg}$.
According to \cite[\S5]{cryptoeprint:2020:823},
 $\primalgo{VRFInner}_{\msg}$ is EUF-CMA secure,
 thanks to our random oracle assumption on $H_p$.

We consider an adversary that produces $(\pk,\msg,\aux,\sigma)$
 that pass verification, without knowing $\sk$.  
%PoK:  Ignoring the first abort path, and employing our random oracle assumption on $H_p$, 
We know from EUF-CMA security of $\primalgo{VRFInner}_{\msg}$ that
our forger knows some $w$ such that
 $h (\pk + c_1 \PreOut) = h w (\genE + c_1 \In)$.
We deduce from AGM knowledge of $x,y,u,v \in \F_p$ such that
 $\pk = x \genE + y \In$ and $\PreOut = u \genE + v \In$
 with $x + c_1 u = w$ and $y + c_1 v = c_1 w$ in $\F_p$,
 so $c_1^2 u + c_1 (x-v) - y = 0$, except with odds negligible in $\secparam$.
At most two $c_1 \in \F_p$ satisfy this equation.
As our $c_1$ depends upon both \pk and $\PreOut$, 
it again follows from our random oracle assumption on $H_p$ that
 $u=0=y$ and $v = w = x \equiv \sk \bmod h$, meaning $\PreOut = \sk \In$,
 except with odds negligible in $\secparam$.
%TODO: What do we cite here?
%PoK:
%PoK: We know $y=0$ if we check the proof-of-knowledge for $\pk$ of course.  
%PoK: We usually suggest that \pk appear in $\msg$ as a defense against related keys, 
%PoK: which occur if say \pk represents some account key on a blockcahin.  
%PoK: In this case, we also know $y=0$ by the random oracle assumption on $h$.  
%PoK: We even deduce $y=0$ after verifying two VRF signatures with distinct
%PoK: inputs $\msg_1$ and $\msg_2$ and hence distinct $\In_1$ and $\In_2$.
%PoK: We know $y=0$ in all cases, as desired.
%PoK: 
It follows that $\ThinVRF$ satisfies uniqueness of course. 

An unpredictability adversary \adv guesses
 a \msg and corresponding pre-output $h \PreOut := h \sk H_\grE(\msg)$,
after making chosen message queries to \Sign.
In AGM, \adv must express its guess for $h \PreOut$
 in terms of $H_\grE(\msg)$ and points arising earlier.
???  So simple ???
As $H_\grE$ is a random oracle, we deduce that either
 \adv solved the discrete logarithm problem, or else
 unpredictability holds for $\ThinVRF$.
As $H'$ is a PRF, we now argue pseudo-randomness for$\ThinVRF$ similarly
 to \cite[Proposition 1]{vrf_micali}.
\end{proof}
% An adversary cannot discover $\PreOut$ without querying $\msg$,
% % \cite[Theorem 6]{coron02}
% % https://eprint.iacr.org/2001/062.pdf NOT https://www.iacr.org/archive/eurocrypt2002/23320268/coron.pdf
% but our EUF-CMA game permits doing so with alternative $\aux$. 
% ...
%TODO: Actually this gets really long winded. 

%PoK:  In this, we still have a VRF if $y=0$, but not exactly the one specified.  
We caution that omitting $c_1$ only demands $x + u = v$ even if $y=0$,
which does not give a VRF.





\endinput




\section{VRF-AD security}
\label{sec:games}

We say a VRF-AD-KC denoted \VRF is {\em secure} if it satisfies
 correctness, uniqueness, and pseudo-randomness as defined below,
 as well as being existentially unforgeable as a signature on $(\msg,\aux)$.
%
We caution that VRF security remain subtle, in part due to
signer and forger each being adversarial in some security properties.
%
% At a high level however VRF security assumptions boil down to translating the PRF definition into the public key setting.
% TODO: What of the above two lines?  Merge?

% We follow \cite{agg_dkg} by distinguishing an algorithm $\VRF.\Eval$,
%  instead of defining it by the equality in correctness,
% which simplifies requiring that verifying honest signatures gives a well-defined function.
% $\VRF.\Eval$ always has more optimized instantiations anyways.

We demand unforgability on $(\msg,\aux)$ because alone
the usual VRF conditions only yield unforgeability for \msg.

\begin{definition}\label{def:vrf_sign_oracle}
We let \ora{Sign} denote a CMA oracle, which creates and stores
a key pair $(\pk,\sk) \leftarrow \KeyGen$, returning \pk, and
thereafter answers oracle calls $\ora{Sign}(\msg,\aux)$ by 
logging $(\msg,\aux)$ and returning $\Sign(\sk,\msg,\aux)$.
\end{definition}

\begin{definition}
We say a VRF-AD satisfies {\em existential unforgeability (EUF-CMA-KC)} if
any PPT adversary \adv has only a negligible advantage in $\secparam$
in the usual chosen-message game adapted to key commitments:
\begin{itemize}
  \item \adv receives $\pk$ from \ora{Sign}, % of Definition \ref{def:vrf_sign_oracle}
  repeatedly queries \ora{Sign},
  and finally produces $\pk,\msg,\aux,\sigma$.
  \item \adv wins if $\Verify(\pk,\msg,\aux,\sigma)$ succeeds, and
  \adv never queried $\ora{Sign}(\msg,\aux)$.
\end{itemize}
\end{definition}

% TODO: Any chat here?

\begin{definition}
We say a VRF-AD satisfies {\em VRF correctness} if
 $\Out = \Verify(\pk,\msg,\aux,\Sign(\sk,\msg,\aux))$ succeeds
whenever $(\pk,\sk) \leftarrow \KeyGen$, and
$\Eval : (\sk,\msg) \mapsto \out$ is a well-defined function.
\end{definition}
% TODO: Is the second condition supurfluous?

We recast the uniqueness as VRFs being well-defined functions of
their public key too, at least up to cryptographic assumptions,
but our definition is clearly equivalent to uniqueness given in
\cite[Def. 2 \S3.2, pp. 4]{vrf_micali} or \cite[Def. 3, pp. 8]{agg_dgk}.

\begin{definition}
We say a VRF-AD satisfies {\em uniqueness} if
if anytime some PPT adversary \adv produces $\msg$, $\pk$, and $\aux_i$, $\sigma_i$ for $i=1,2$, then
$\Verify(\pk,\msg,\aux_1,\sigma_1) = \Verify(\pk,\msg,\aux_2,\sigma_2)$
unless either $\Verify$ returns failure, except with odds negligible in $\secparam$.
\end{definition}

\begin{definition}
We say a VRF-AD satisfies {\em strong uniqueness} if
there exists a (not efficiently computable) function
 $F : (\msg,\pk) \mapsto \Out$ such that
anytime some PPT adversary \adv produces $\msg$, $\pk$, $\aux$, and $\sigma$
then $\Verify(\pk,\msg,\aux,\sigma) \in \{ F(\msg,\pk), \perp \}$
except with odds negligible in $\secparam$.
\end{definition}
% TODO: Keep?

We say VRFs are public key analogs of PRFs, but actually this PRF analogy
fails in the ``residual pseudo-randomness'' definitions by
Micali, et al. \cite[Def. VRF (3) \S3.2, pp. 4]{vrf_micali},
 which employs \ora{Sign} in EUF-CMA-like games,
 but says nothing for adversarially generated keys.

\begin{definition}
We say a VRF-AD-KC satisfies {\em public keyed} or {\em residual pseudo-randomness} if 
any PPT adversary \adv has only a negligible advantage in $\secparam$
in this chosen-message game:
\begin{itemize}
	\item[]
	\adv receives $\pk$ from \ora{Sign} of Definition \ref{def:vrf_sign_oracle},
	repeatedly queries \ora{Sign}, and produces $\msg,\aux$.
	If \adv never queried $\ora{Sign}(\msg,\cdot)$ then
	\adv wins by distinguishing $\msg \mapsto \Eval(\sk,\msg)$ from a random.
\end{itemize}
\end{definition}

In \cite{praos}, there exists a UC functionality which captures the
desired PRF analogy, but brings unnecessary restrictions.

We know a function family $\{ F_\msg : \pk \mapsto F(\msg,\pk) \}$ exists
by strong uniqueness, although not efficiently computable, so intuitively
our VRF-AD is {\em pseudo-random} if an adversary cannot distinguish
$F_\msg$ from a random function.
% TODO: Keep?

\bigskip

MISTAKES BELOW THIS POINT

\bigskip 

As a formalization, we provide a black-box game-based definition which
treats \msg as the PRF key, and handles adversarially supplied keys as
PRF inputs by not necessarily terminating.

\begin{definition}
We say a VRF-AD-KC satisfies {\em message keyed pseudo-randomness} if 
any PPT adversary \adv for whom the following black-box game always
terminates has only a negligible advantage in $\secparam$ of winning.
\begin{itemize}
	\item[]
	Sample a random \msg, a random function $\rho$ with the same range as \Eval, and a bit $b$.
	\adv queries \ora{Verify} by providing both a public key \pk and
	a PPT (black-box) secret key algorithm $f_\sk : () \mapsto (\aux,\sigma)$
	such that repeatedly trying $\Out \leftarrow \Verify(\pk,\msg,f_\sk(\msg))$
	eventually succeeds.
	\ora{Verify} always returns \Out and $\rho(\pk)$ but with their order depending upon $b$.
	\adv wins by guessing $b$, aka by distinguish \Verify from $\rho$.
\end{itemize}
\end{definition}

There are also verifiable unpredictable function (VUF), which replace
pseudo-randomness by the weaker {\em unpredictability} definition from
\cite[Def. VUF (3) \S3.2, pp. 5]{vrf_micali} or \cite[Def. 4, pp. 8]{agg_dgk}.
Interestingly VUFs often suffice threshold security assumptions \cite{agg_dkg}.

\begin{definition}
We say a VRF-AD-KC satisfies {\em residual unpredictability} if 
any PPT adversary \adv has only a negligible advantage in $\secparam$
in this chosen-message game:
\begin{itemize}
	\item[]
	\adv receives $\pk$ from \ora{Sign} of Definition \ref{def:vrf_sign_oracle},
    repeatedly queries \ora{Sign}, and produces $\msg,\aux$.
    If \adv never queried $\ora{Sign}(\msg,\cdot)$ then
    \adv wins by guessing $\Eval(\sk,\msg)$ for an unqueried \msg.
\end{itemize}
\end{definition}

Also, if $H'(\cdot,k)$ is a PRF then \cite[Proposition 1]{vrf_micali}
shows computing $\Out = H'(\Verify(\cdots), \msg)$ transforms
 residual unpredictability into a residual pseudo-randomness.
As $H'$ is cheap, we conclude implementers should prefer VRFs over more subtle VUFs.

\begin{definition}
We say a VRF-AD-KC satisfies {\em message keyed unpredictability} if 
any PPT adversary \adv for whom the following black-box game always
terminates has only a negligible advantage in $\secparam$ of winning.
\begin{itemize}
	\item[]
	Sample a random \msg.
	\adv queries \ora{Verify} by providing both a public key \pk and
	a PPT (black-box) secret key algorithm $f_\sk : () \mapsto (\aux,\sigma)$ such that
	repeatedly trying $\Out \leftarrow \Verify(\pk,\msg,f_\sk(\msg))$ eventually succeeds.
	\ora{Verify} always returns \Out.
	\adv wins by correctly guessing $\Out = F(\msg,\pk)$ for an unqueried \pk. 
\end{itemize}
\end{definition}

TODO: Justify?

TODO: Relationships?  


\subsection{Confusion}
% \smallskip

Although \cite[\S3.2 $\fvrf$]{praos} handles pseudo-randomness better than \cite{vrf_micali},
they formalize VRFs with detached outputs via the two algorithms:
% \begin{itemize}
% \item
$\VRF.\primalgo{EvalProve}(\sk,\msg,\aux) \mapsto (\Out,\sigma)$, in which $\sigma = \VRF.\Sign(\sk,\msg,\aux)$ and $\Out = \VRF.\Eval(\sk,\msg)$, and
% \item
$\VRF.\primalgo{VerifyProof}(\pk,\msg,\aux,\Out,\sigma)$ which returns true only if $\VRF.\Verify(\pk,\msg,\aux,\sigma)$ returns $\Out$.
% \end{itemize}
We strongly prefer the \Sign and \Verify formulation from \cite{agg_dkg}
primarily because the \primalgo{EvalProve}, and \primalgo{VerifyProof}
formulation causes implementation and deployment mistakes:

EC VRF signatures have the form $\sigma = (\PreOut,\pi)$ in which some
inner proof $\pi$ proves correctness of some associated VUF output $\PreOut$. % aka ``pre-output''.  % ``pre-pseudo-random''
It follows $\VRF.\Eval$ never corresponds to $\PreOut$, but if one describes
protocols with an \primalgo{EvalProve} formulation then exposing $\PreOut$
invariably confuses developers into miss-using $\PreOut$ as the output.
% In other words, actual code never corresponds to an \primalgo{EvalProve} and \primalgo{VerifyProof} formulation.

The ``pre-output'' $\PreOut$ preserves algebraic relationships between
secret keys, so protocols described by the \primalgo{EvalProve} formulation
have implementations with broken pseudo-randomness, and perhaps
 related key vulnerabilities and mishandled cofactors.
% We need $\PreOut$ to be exposed by implementations so ...
We avoided the VUF formalism taken by \cite{agg_dkg} in part because
 VUFs obfuscate this difficulty to developers.

As a caveat, there exist UC formalisms that appear simpler with
the \primalgo{EvalProve} and \primalgo{VerifyProof} formulation, like in \cite{praos}.
We therefore propose that VRFs and protocols using VRFs should always be
described using the the \Sign and \Verify formulation, which provides
implementers with a sensible description, but then if needed adopt
 \primalgo{EvalProve} and \primalgo{VerifyProof} only inside the UC formulation itself.
We feel imposing this mental translation upon paper authors and reviewers
 beats imposing the reverse upon developers with less cryptographic knowledge.



\endinput 



\smallskip

There exist VUFs like RSA-FDH or BLS signatures that lack auxiliary data
% There even exist bespoke VRFs that relax correctness to some non-trivial
% relation on the space of secret keys and messages,
%  seemingly including some Rabin variants. 
Yet, these all suffer from either large signature sizes (RSA) or
 slow verification (BLS).
%  VRFs like single-layer XMSS, .

Instead, one prefers instantiating VRFs similarly to
 \cite{nsec5} or \cite{VXEd25519} using Chaum-Pedersen DLEQ proofs \cite{CP92} % Or should it be CP93 ??
 because they provide both small signatures and fast verification.
In these, our auxiliary data \aux can be verified for free,
by binding \aux into the challenge hash, like a Schnorr signature.
VRF protocols could often reduce bandwidth and verifier time this way,
 but some like Sassafras depend upon \aux. 





\endinput % no UC VRF discussion here









\bibliographystyle{plain}
\bibliography{../commit,../anoncred,../sassafras,../vrf,../zkp}


\end{document}





\endinput



