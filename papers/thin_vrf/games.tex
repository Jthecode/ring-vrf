
\section{VRF-AD security}
\label{sec:games}

We say a VRF-AD-KC denoted \VRF is {\em secure} if it satisfies
 correctness, uniqueness, and pseudo-randomness as defined below,
 as well as being existentially unforgeable as a signature on $(\msg,\aux)$.
%
We caution that VRF security remain subtle, in part due to
signer and forger each being adversarial in some security properties.
%
% At a high level however VRF security assumptions boil down to translating the PRF definition into the public key setting.
% TODO: What of the above two lines?  Merge?

% We follow \cite{agg_dkg} by distinguishing an algorithm $\VRF.\Eval$,
%  instead of defining it by the equality in correctness,
% which simplifies requiring that verifying honest signatures gives a well-defined function.
% $\VRF.\Eval$ always has more optimized instantiations anyways.

We demand unforgability on $(\msg,\aux)$ because alone
the usual VRF conditions only yield unforgeability for \msg.

\begin{definition}\label{def:vrf_sign_oracle}
We let \ora{Sign} denote a CMA oracle, which creates and stores
a key pair $(\pk,\sk) \leftarrow \KeyGen$, returning \pk, and
thereafter answers oracle calls $\ora{Sign}(\msg,\aux)$ by 
logging $(\msg,\aux)$ and returning $\Sign(\sk,\msg,\aux)$.
\end{definition}

\begin{definition}
We say a VRF-AD satisfies {\em existential unforgeability (EUF-CMA-KC)} if
any PPT adversary \adv has only a negligible advantage in $\secparam$
in the usual chosen-message game adapted to key commitments:
\begin{itemize}
  \item \adv receives $\pk$ from \ora{Sign}, % of Definition \ref{def:vrf_sign_oracle}
  repeatedly queries \ora{Sign},
  and finally produces $\pk,\msg,\aux,\sigma$.
  \item \adv wins if $\Verify(\pk,\msg,\aux,\sigma)$ succeeds, and
  \adv never queried $\ora{Sign}(\msg,\aux)$.
\end{itemize}
\end{definition}

% TODO: Any chat here?

\begin{definition}
We say a VRF-AD satisfies {\em VRF correctness} if
 $\Out = \Verify(\pk,\msg,\aux,\Sign(\sk,\msg,\aux))$ succeeds
whenever $(\pk,\sk) \leftarrow \KeyGen$, and
$\Eval : (\sk,\msg) \mapsto \out$ is a well-defined function.
\end{definition}
% TODO: Is the second condition supurfluous?

We recast the uniqueness as VRFs being well-defined functions of
their public key too, at least up to cryptographic assumptions,
but our definition is clearly equivalent to uniqueness given in
\cite[Def. 2 \S3.2, pp. 4]{vrf_micali} or \cite[Def. 3, pp. 8]{agg_dgk}.

\begin{definition}
We say a VRF-AD satisfies {\em uniqueness} if
if anytime some PPT adversary \adv produces $\msg$, $\pk$, and $\aux_i$, $\sigma_i$ for $i=1,2$, then
$\Verify(\pk,\msg,\aux_1,\sigma_1) = \Verify(\pk,\msg,\aux_2,\sigma_2)$
unless either $\Verify$ returns failure, except with odds negligible in $\secparam$.
\end{definition}

\begin{definition}
We say a VRF-AD satisfies {\em strong uniqueness} if
there exists a (not efficiently computable) function
 $F : (\msg,\pk) \mapsto \Out$ such that
anytime some PPT adversary \adv produces $\msg$, $\pk$, $\aux$, and $\sigma$
then $\Verify(\pk,\msg,\aux,\sigma) \in \{ F(\msg,\pk), \perp \}$
except with odds negligible in $\secparam$.
\end{definition}
% TODO: Keep?

We say VRFs are public key analogs of PRFs, but actually this PRF analogy
fails in the ``residual pseudo-randomness'' definitions by
Micali, et al. \cite[Def. VRF (3) \S3.2, pp. 4]{vrf_micali},
 which employs \ora{Sign} in EUF-CMA-like games,
 but says nothing for adversarially generated keys.

\begin{definition}
We say a VRF-AD-KC satisfies {\em public keyed} or {\em residual pseudo-randomness} if 
any PPT adversary \adv has only a negligible advantage in $\secparam$
in this chosen-message game:
\begin{itemize}
	\item[]
	\adv receives $\pk$ from \ora{Sign} of Definition \ref{def:vrf_sign_oracle},
	repeatedly queries \ora{Sign}, and produces $\msg,\aux$.
	If \adv never queried $\ora{Sign}(\msg,\cdot)$ then
	\adv wins by distinguishing $\msg \mapsto \Eval(\sk,\msg)$ from a random.
\end{itemize}
\end{definition}

In \cite{praos}, there exists a UC functionality which captures the
desired PRF analogy, but brings unnecessary restrictions.

We know a function family $\{ F_\msg : \pk \mapsto F(\msg,\pk) \}$ exists
by strong uniqueness, although not efficiently computable, so intuitively
our VRF-AD is {\em pseudo-random} if an adversary cannot distinguish
$F_\msg$ from a random function.
% TODO: Keep?

\bigskip

MISTAKES BELOW THIS POINT

\bigskip 

As a formalization, we provide a black-box game-based definition which
treats \msg as the PRF key, and handles adversarially supplied keys as
PRF inputs by not necessarily terminating.

\begin{definition}
We say a VRF-AD-KC satisfies {\em message keyed pseudo-randomness} if 
any PPT adversary \adv for whom the following black-box game always
terminates has only a negligible advantage in $\secparam$ of winning.
\begin{itemize}
	\item[]
	Sample a random \msg, a random function $\rho$ with the same range as \Eval, and a bit $b$.
	\adv queries \ora{Verify} by providing both a public key \pk and
	a PPT (black-box) secret key algorithm $f_\sk : () \mapsto (\aux,\sigma)$
	such that repeatedly trying $\Out \leftarrow \Verify(\pk,\msg,f_\sk(\msg))$
	eventually succeeds.
	\ora{Verify} always returns \Out and $\rho(\pk)$ but with their order depending upon $b$.
	\adv wins by guessing $b$, aka by distinguish \Verify from $\rho$.
\end{itemize}
\end{definition}

There are also verifiable unpredictable function (VUF), which replace
pseudo-randomness by the weaker {\em unpredictability} definition from
\cite[Def. VUF (3) \S3.2, pp. 5]{vrf_micali} or \cite[Def. 4, pp. 8]{agg_dgk}.
Interestingly VUFs often suffice threshold security assumptions \cite{agg_dkg}.

\begin{definition}
We say a VRF-AD-KC satisfies {\em residual unpredictability} if 
any PPT adversary \adv has only a negligible advantage in $\secparam$
in this chosen-message game:
\begin{itemize}
	\item[]
	\adv receives $\pk$ from \ora{Sign} of Definition \ref{def:vrf_sign_oracle},
    repeatedly queries \ora{Sign}, and produces $\msg,\aux$.
    If \adv never queried $\ora{Sign}(\msg,\cdot)$ then
    \adv wins by guessing $\Eval(\sk,\msg)$ for an unqueried \msg.
\end{itemize}
\end{definition}

Also, if $H'(\cdot,k)$ is a PRF then \cite[Proposition 1]{vrf_micali}
shows computing $\Out = H'(\Verify(\cdots), \msg)$ transforms
 residual unpredictability into a residual pseudo-randomness.
As $H'$ is cheap, we conclude implementers should prefer VRFs over more subtle VUFs.

\begin{definition}
We say a VRF-AD-KC satisfies {\em message keyed unpredictability} if 
any PPT adversary \adv for whom the following black-box game always
terminates has only a negligible advantage in $\secparam$ of winning.
\begin{itemize}
	\item[]
	Sample a random \msg.
	\adv queries \ora{Verify} by providing both a public key \pk and
	a PPT (black-box) secret key algorithm $f_\sk : () \mapsto (\aux,\sigma)$ such that
	repeatedly trying $\Out \leftarrow \Verify(\pk,\msg,f_\sk(\msg))$ eventually succeeds.
	\ora{Verify} always returns \Out.
	\adv wins by correctly guessing $\Out = F(\msg,\pk)$ for an unqueried \pk. 
\end{itemize}
\end{definition}

TODO: Justify?

TODO: Relationships?  


\subsection{Confusion}
% \smallskip

Although \cite[\S3.2 $\fvrf$]{praos} handles pseudo-randomness better than \cite{vrf_micali},
they formalize VRFs with detached outputs via the two algorithms:
% \begin{itemize}
% \item
$\VRF.\primalgo{EvalProve}(\sk,\msg,\aux) \mapsto (\Out,\sigma)$, in which $\sigma = \VRF.\Sign(\sk,\msg,\aux)$ and $\Out = \VRF.\Eval(\sk,\msg)$, and
% \item
$\VRF.\primalgo{VerifyProof}(\pk,\msg,\aux,\Out,\sigma)$ which returns true only if $\VRF.\Verify(\pk,\msg,\aux,\sigma)$ returns $\Out$.
% \end{itemize}
We strongly prefer the \Sign and \Verify formulation from \cite{agg_dkg}
primarily because the \primalgo{EvalProve}, and \primalgo{VerifyProof}
formulation causes implementation and deployment mistakes:

EC VRF signatures have the form $\sigma = (\PreOut,\pi)$ in which some
inner proof $\pi$ proves correctness of some associated VUF output $\PreOut$. % aka ``pre-output''.  % ``pre-pseudo-random''
It follows $\VRF.\Eval$ never corresponds to $\PreOut$, but if one describes
protocols with an \primalgo{EvalProve} formulation then exposing $\PreOut$
invariably confuses developers into miss-using $\PreOut$ as the output.
% In other words, actual code never corresponds to an \primalgo{EvalProve} and \primalgo{VerifyProof} formulation.

The ``pre-output'' $\PreOut$ preserves algebraic relationships between
secret keys, so protocols described by the \primalgo{EvalProve} formulation
have implementations with broken pseudo-randomness, and perhaps
 related key vulnerabilities and mishandled cofactors.
% We need $\PreOut$ to be exposed by implementations so ...
We avoided the VUF formalism taken by \cite{agg_dkg} in part because
 VUFs obfuscate this difficulty to developers.

As a caveat, there exist UC formalisms that appear simpler with
the \primalgo{EvalProve} and \primalgo{VerifyProof} formulation, like in \cite{praos}.
We therefore propose that VRFs and protocols using VRFs should always be
described using the the \Sign and \Verify formulation, which provides
implementers with a sensible description, but then if needed adopt
 \primalgo{EvalProve} and \primalgo{VerifyProof} only inside the UC formulation itself.
We feel imposing this mental translation upon paper authors and reviewers
 beats imposing the reverse upon developers with less cryptographic knowledge.



\endinput 



\smallskip

There exist VUFs like RSA-FDH or BLS signatures that lack auxiliary data
% There even exist bespoke VRFs that relax correctness to some non-trivial
% relation on the space of secret keys and messages,
%  seemingly including some Rabin variants. 
Yet, these all suffer from either large signature sizes (RSA) or
 slow verification (BLS).
%  VRFs like single-layer XMSS, .

Instead, one prefers instantiating VRFs similarly to
 \cite{nsec5} or \cite{VXEd25519} using Chaum-Pedersen DLEQ proofs \cite{CP92} % Or should it be CP93 ??
 because they provide both small signatures and fast verification.
In these, our auxiliary data \aux can be verified for free,
by binding \aux into the challenge hash, like a Schnorr signature.
VRF protocols could often reduce bandwidth and verifier time this way,
 but some like Sassafras depend upon \aux. 





\endinput % no UC VRF discussion here




