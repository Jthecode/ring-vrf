\section{Security Model of Ring VRF}


In this section, we describe the security of our new cryptographic primitive ring VRF. First, we describe the basic ring VRF in the real world and in the ideal world. Second, we extend the basic ring VRF by adding a new property that we call secret evaluation. 


\paragraph{Ring VRF (Basic Definition):} A ring VRF operates like a VRF but only proves its key comes from a specific list without giving any information about which specific key. We define the ring VRF functionality $ \fgvrf $ in Figure \ref{f:gvrf}. The functionality lets parties generate a key (Key Generation), evaluate a message with the party's key (VRF Evaluation), prove that the evaluation is executed by one of the keys (VRF evaluation and proof) and verify the evaluation without knowing the key used for the evaluation (Verification). We also define linking procedures in $ \fgvrf $ to link an evaluation and a proof with its associated key. So, if a party wants to reveal its identity at some point, it can use the linking process to prove that the evaluation is executed with its key (Linking proof). Later on, anyone can verify the linking proof (Linking verification).
%TODO Secret Evaluation
\begin{figure}
	\scriptsize
	\begin{tcolorbox}
		{  \begin{description}
				\item[Key Generation.] upon receiving a message $(\msg{keygen}, \sid)$ from a party $\user_i$, send $(\msg{keygen}, \sid, \user_i)$ to the simulator $\simulator$.
				Upon receiving a message $(\msg{verificationkey}, \sid, \pkrvrf)$ from $\simulator$, verify that $\pkrvrf$ has not been recorded before; then, store in the table $\vklist$, under $\user_i$, the value $\pkrvrf$.
				Return $(\msg{verificationkey}, \sid, \pkrvrf)$.
				
				\item[Malicious Key Generation.] upon receiving a message $(\msg{keygen}, \sid, \pkrvrf)$ from $\simulator$, verify that $\pkrvrf$ was not yet recorded, and if so record in the table $\vklist$ the value $\pkrvrf$ under $\simulator$. Else, ignore the message.
				
				\item[VRF Evaluation.] upon receiving a message $(\msg{eval}, \sid, \pkring, \pkvrf_i, m)$ from $\user_i$, verify that $\pkrvrf_i \in \pkring$ and if there exists $ \pkvrf_i $ in $\vklist $ associated with $ \user_i $. If that was not the case, just ignore the request.
				If $\evaluationslist[\pkring, m][\pkrvrf_i]$ is empty, then sample a new $y \sample \bin^{\ell(\secpar)}$ and store the pair $y$ in the array associated to the message $m$ and the ring $\pkring$. That is, store $\evaluationslist[\pkring, m][\pkrvrf_i] = y$.
				Let sets $\prooflist[\pkring, m][\pkrvrf_i] = \emptylist$ and $\Linklist[\pkring, m][\pkrvrf_i] = \emptylist$ if $ \prooflist[\pkring, m][\pkrvrf_i] $ and $\Linklist[\pkring, m][\pkrvrf_i] = \emptylist$ do not exits. Return $(\msg{evaluated}, sid, \pkring, m, y)$ to $ \user_i $.
				
				
				\item[VRF evaluation and proof.] upon receiving a message $(\msg{evalprove}, \sid, \pkring, \pkrvrf_i, m)$ from $\user_i$, verify that $\pkrvrf_i \in \pkring$ and that there exists a public key $\pkrvrf_i$ associated to $\user_i$ in the table $ \vklist $. If that wasn't the case, just ignore the request.
				Send $(\msg{evalprove}, \sid, \pkring, m)$ to $\simulator$. Upon receiving $(\msg{eval}, \sid, \pkring, m, \pi)$ as a response from $\simulator$, if $\evaluationslist[\pkring, m][\pkrvrf_i]$ is not set, select $y \sample \bin^{\ell(\secpar)}$ and set $\evaluationslist[\pkring, m][\pkrvrf_i]$ to be $y$; set $\prooflist[\pkring, m][\pkrvrf_i] = \set{\pi}$. Otherwise, let $y \defeq \evaluationslist[\pkring, m][\pkrvrf_i]$; update the list of proofs $\prooflist[\pkring, m][\pkrvrf_i]$ by adding $\pi$. Return $(\msg{evaluated\&proved}, \sid, \pkring,m, y, \pi)$ to $\user_i$.
				
				%\item[Malicious VRF evaluation.] upon receiving a message $(\msg{evalprove}, \sid, \pkring, m)$ from $\simulator$, check that $\vklist$ has a public key associated to $\simulator$. If not, ignore the request. If $\evaluationslist[\pkring, m][\simulator]$ is not set, sample $y \sample \bin^{\ell(\secpar)}$ and set $\evaluationslist[\pkring, m][\simulator] \defeq y$ (and $\prooflist[\pkring,m]$ to $\emptyset$). If $\prooflist[\pkring, m]$ contains a proof (i.e., if $\prooflist[\pkring, m]$ is not empty), return $(\msg{evaluated}, \sid, y)$ to $\simulator$. Else, ignore the request.
				
				%\item[Verification.] upon receiving a message $(\msg{verify}, \sid, \pkring, m, y, \pi)$, from any party forward the message to the simulator. If there exists a $\pkrvrf_i$ among the values of \texttt{verification\_keys}, and there exists $\pi \in \prooflist[\pkring, m]$, set $b = 1$. Else, set $b =0$. Finally, output $(\msg{verified}, \sid, \pkring, m, y, \pi, b)$.
				\item[Verification.] upon receiving a message $(\msg{verify}, \sid, \pkring, m, y, \pi)$, set $ b = 0 $ and check if there exists a  $\pkrvrf_i \in \pkring$ and  in $ \vklist $ such that $ \evaluationslist[\pkring,m][\pkvrf_i] = y $. 
				\begin{itemize}
					\item If not, set $ b = 0 $.
					\item If there exists a $ \pkrvrf_i $ such that $ \evaluationslist[\pkring,m][\pkvrf_i] = y$, $ \pkrvrf_i $ is a malicious key and   $\pi \notin \prooflist[\pkring, m][\pkvrf_i]$, forward the message to $ \simulator $. Upon receiving $ (\msg{Verified}, \sid,\pkring, m, y, \pi, b') $ from $ \simulator $. Set $ b = b' $. If $ b' = 1 $, append $ \pi $ to $ \prooflist[\pkring, m][\pkvrf_i] $.
					\item If there exists $\pi \in \prooflist[\pkring, m][\pkvrf_i]$, set $b = 1$.
				\end{itemize}
				 Finally, output $(\msg{verified}, \sid, \pkring, m, y, \pi, b)$ to the party.
			\end{description}
			\par\hrulefill\\
			We add the following linking procedures:
			\begin{description}
				\item[Linking proof.] upon receiving a message $(\msg{link}, \sid, \pkring, \pkrvrf_i, m, y,\pi)$ from $\user_i$, check that $\pkrvrf_i \in \pkring$ and that $\pkrvrf_i $ is associated to $\user_i$ in $ \vklist $. 
				Check whether $\evaluationslist[\pkring, m][\pkvrf_i] = y$ and $ \pi \in \prooflist[\pkring, m][\pkrvrf_i] $. If any of the above fails, ignore the request.
				Send $(\msg{link}, \sid, \pkring, m, y)$ to $\simulator$. Upon receiving $(\msg{linkproof}, \sid, \pkring, m, y, \hat \pi)$ from $\simulator$, append $\hat\pi$ to $\Linklist[\pkring, m, \pi][\pkvrf_i]$ and return $(\msg{linked}, \sid, y, \hat\pi)$ to $\user_i$.
				%\item[Malicious linking proof.] upon receiving a message $(\msg{link}, \sid, \pkring, m, y)$ from $\simulator$, check that $\vklist$ has a key set for $\simulator$, and that it is in $R$.
				%Check that $\evaluationslist[\pkring, m][\simulator] = y$.
				%If any of the above is not satisfied, ignore the request.
				%Return $(\msg{linked}, \sid, y)$ to $\simulator$.
				\item[Linking verification.] upon receiving a message $(\msg{verifylink}, \sid, \pkrvrf_i, \pkring, m, y,\pi,\hat\pi)$ from any party forward the message to the simulator, if there exists a $\pkrvrf_i$ among the values of $ \vklist $, and $\evaluationslist[\pkring, m][\pkvrf_i] = y$, and $\Linklist[\pkring, m, \pi][\pkvrf_i]$ contains $\hat\pi$, set $b=1$. Else, set $b=0$. Return $(\msg{verified}, \sid, \pkrvrf_i, \pkring, m, y, \hat\pi, b).$ to the party.
			\end{description}
		}
	\end{tcolorbox}
	\caption{Functionality $\fgvrf$.\label{f:gvrf}}
\end{figure}

In a nutshell, the functionality $\fgvrf$, when given as input a message $m$ and a key set $\pkring$ of participant, allows to create $n$ possible different outputs pseudo-random that appear independent from the inputs. The output can be verified to have been computed correctly by one of the participants in $\pkring$ without revealing who they are. At a later stage, the author of the VRF output can prove that the output was generated by them and no other participant could have done so.

Below, we define the real-world execution of the ring VRF.
\begin{definition}[Ring-VRF (rVRF)]\label{def:ringvrf}
	Ring VRF is a VRF with a deterministic function $ F(.,.):\{0,1\}^\kappa \times\{0,1\}^* \rightarrow \{0,1\}^{\ell_\rvrf} $ and with the following algorithms:
	
	\begin{itemize}
		\item $ \rvrf.\keygen(1^\kappa) \rightarrow (\skrvrf,\pkrvrf)$ where $ \kappa $ is the security parameter,
	\end{itemize}
	Given list of public keys $ \pkring = \set{\pkrvrf_1, \pkrvrf_2, \ldots, \pkrvrf_n}$, a message $ m \in \{0,1\}^* $
	\begin{itemize}
		\item $ \rvrf.\eval(\skrvrf_i, \pkring, m)\rightarrow y $ where $ y = F(\skrvrf,\pkring,m) $,
		\item $ \rvrf.\evalprove(\skrvrf_i, \pkring, m)\rightarrow (F(\skrvrf,\pkring,m),\pi) $ where  $ \pi $ is a proof for the evaluation.
		\item $ \rvrf.\verify(\pkring, m, y,\pi) \rightarrow  b$ where $ b \in \{0,1\} $. $ b =1 $ means verified and $ b = 0 $ means not verified.
		\item $ \rvrf.\link(\skrvrf_i, \pkring,m,y, \pi) \rightarrow \pi_{\link} $ where  $ \pi^\link $ is a proof linking the public key of the producer of $ y $. 
		\item $ \rvrf.\link\verify(\pkring, \skrvrf_i, m, y, \pi,, \pi_{\link})\rightarrow b$ where $ b \in \{0,1\} $. $ b =1 $ means verified and $ b = 0 $ means not verified.
	\end{itemize}
	
\end{definition}
\paragraph{Ring VRF with Secret Evaluation:} 


Below, we define the real-world execution of the ring VRF with secret evaluation.
\begin{definition}[Ring-VRF (rVRF)]\label{def:ringvrfse}
	Ring VRF with secret evaluation is two VRFs with a deterministic function $ F(.,.):\{0,1\}^\kappa \times\{0,1\}^* \rightarrow \{0,1\}^{\ell_\rvrf} $ and$ F_s(.,.):\{0,1\}^\kappa \times\{0,1\}^* \rightarrow \{0,1\}^{\ell_\rvrf} $. It consists of the algorithms of ring VRF defined in Definition \ref{def:ringvrf} and additionally the following algorithms:
	
	Given list of public keys $ \pkring = \set{\pkrvrf_1, \pkrvrf_2, \ldots, \pkrvrf_n}$, a message $ m \in \{0,1\}^* $
	\begin{itemize}
		\item $ \rvrf.\secreteval(\skrvrf_i, \pkring, m)\rightarrow \omega $ where $ \omega = F_s(\skrvrf,\pkring,m) $,
		\item $ \rvrf.\secretprove(\skrvrf_i, \pkring, m)\rightarrow \pi / \perp $ where  $ \pi $ is a proof for the secret evaluation. If $ (F_s(\skrvrf,\pkring,m), (\pkring, m)) \notin \rel  $, it outputs $ \perp $.
		\item $ \rvrf.\secretverify(\pkring, m,\pi) \rightarrow  b$ where $ b \in \{0,1\} $. $ b =1 $ means verified and $ b = 0 $ means not verified.
	\end{itemize}
	
\end{definition}

\begin{figure}
	\scriptsize
	\begin{tcolorbox}
	{
			%\par\hrulefill\\
			 $ \fgvrf^{zk} $ for a relation $ \mathcal{R} $ behaves exactly as $ \fgvrf $ and  it additionally does the following:
			\begin{description}
				\item[Secret Evaluation.] upon receiving a message $(\msg{secreteval}, \sid, \pkring,\pkvrf_i, m)$ from $\user_i$, verify that $\pkrvrf_i$ is in $ \pkring $ and that there exists a key $\user_i$ with an associated public key $\pkrvrf_i$ in $\vklist$. If that was not the case, just ignore the request.
				If $\evaluationslist[\pkring, m][\pkrvrf_i]$ is empty, then sample a new $y \sample \bin^{\ell(\secpar)}$ and  store $\evaluationslist[\pkring, m][\pkrvrf_i] = y$. In any case, sample a random element $ \eta  \sample \bin^{\ell(\secpar)} $ and store $ \evaluationsecretlist[\pkring, m][\pkvrf_i] = \eta $. Create a set $ \proofzklist[\pkring,m][\pkvrf_i] = \emptylist $, if it does not exist. Return $(\msg{evaluated}, sid, y, \eta)$ to $ \user_i $.
				
				\item[Secret evaluation and proof.] upon receiving a message $(\msg{secretprove}, \sid, \pkring, \pkvrf_i, m)$ from $\user_i$, verify that $\pkrvrf_i \in \pkring $ and  that there exists a key $ \user_i $ with an associated public key $ \pkrvrf_i $ in $ \vklist $. (If that wasn't the case, just ignore the request.). Obtain $ (y, \eta) $ from $\evaluationslist[\pkring, m][\pkvrf_i]$ and $\evaluationsecretlist[\pkring, m][\pkvrf_i]$. If they are not defined execute the steps in VRF evaluation and secret evaluation.
				If $ ((m, y, \pkring),(\eta,\pkvrf_i)) \notin \mathcal{R} $, ignore the request.  Otherwise, send $(\msg{zkprove}, \sid, \pkring, m, y)$ to $\simulator$. Upon receiving $(\msg{zkproof}, \sid, \pkring, m, y \pi^{zk})$ as a response from $\simulator$,  add $ \pi^{zk} $ to  the set $\proofzklist[\pkring, m][\pkvrf_i]$. Return $(\msg{zkproof}, \sid, y, \eta, \pi^{zk})$ to $\user_i$.
				
				
				\item[Secret Verification.] upon receiving a message $(\msg{secretverify}, \sid, \pkring, m, \pi^{zk})$, from a party, check whether there exists  $ \pkvrf_i \in \vklist $ such that $ \proofzklist[\pkring,m][\pkvrf_i] = \pi^{zk} $, $ \evaluationslist[\pkring,m][\pkrvrf_i] = y $ and $ \evaluationsecretlist[\pkring,m][\pkrvrf_i] $. If there exists, set $b = 1$. Else, set $b =0$. Finally, output $(\msg{verification}, \sid, \pkring, m, \pi^{zk}, b)$.
			\end{description}
		}
	\end{tcolorbox}
	\caption{Functionality  $ \fgvrf^{zk} $.\label{f:gvrfzk}}
\end{figure}
