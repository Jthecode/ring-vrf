\section{Ring updates}
\label{sec:ring_updates}

We discuss \pifast, or \pisk and \pipk, representing public keys
 in $\grE$ in $\grJ$ already,
% along with circuit implementation details of $\PedVRF.\{ \CommitKey, \OpenKey \}$,
but otherwise mostly treated the ring commitment scheme
$\rVRF.\{ \CommitRing, \CommitKey, \OpenKey \}$ like a black box.

Although our $\rVRF.\rSign$ runs fast, all users must update their
stored zkSNARK \pipk every time the ring $\ctx$ changes.
Almost any circuit works for \pipk though,
 which permits diverse optimizations depending upon usecase.


\subsection{Merkle trees} % {Poseidon}

Our $\rVRF.\{ \CommitRing, \CommitKey, \OpenKey \}$ could implement a
Merkle tree using zkSNARK friendly hash functions like Poseidon \cite{poseidon}.
%
All users need $O(\log |\ctx|)$ data with every update, which sounds
reasonable but not free.  There is a fast moving literature on securing
and optimizing zkSNARK friendly hash functions, with different techniques
being better suited to different zkSNARKs or even curves.

TODO: Arity 9 for 300 constraints?   % \cite{Groth16} vs plookup \cite{plookup}.

We leave deeper discussion of zkSNARK friendly Merkle to the literature.
Instead we spend this section focusing upon the diversity of circuit
designs that fit our framework.


\subsection{Vector commitments}

As noted in \S\ref{subsec:rvrf_side_channel}, our zkSNARK \pipk could use
polynomial based vector commitments \cite{KZG}. % or so called ``Verkle trees'' \cite{??Verkle??} too.
We need hidden opening locations of the sort discussed in Caulk \cite{caulk} and Caulk+ \cite{caulk+}.

TODO: Discuss Caulk

Interesstingly, we might avoid the extra proof-of-knowledge added
in \S\ref{subsec:rvrf_side_channel} because the KZG commitment itself
can provide the strucural proof.

TODO:  Is this still true with Caulk?


\subsection{Certificates} % \& revokation}

If an authority grants ring membership, then ring membership proofs
could simply verify some certificate by the authority, likely using
a signature on JubJub.

In this, we prefer a SNARK friendly random oracle,
because conventional random oracles cost like 30k constraints.
We also need a variable base scalar multiplication, which costs like
4k constraints, as well as a couple fixed base scalar multiplication.
A priori, these fixed base scalar multiplications cost roughly 700
constraints each, but ocasionally they cost only half this.   

We conjecture one fixed based scalar multiplication could be replaced
by adapting implicit certificate scheme technqiues,
 instead of simply a signature on a user provided key.

We typically need expiration dates in certificates, likely demanding
a range proof and maybe requiring that \pipk be recomputed more often.

% \subsection{Revokation}

As a rule, one needs some revokation path for certificates,
despite the underlying signature not being revokable. 
%
We suggest maitaining a seperate revokation list and then inside
\pipk prove non-membership in the revokation list.
% perhaps via \cite{???}.
In this way, we update \pipk only when the revokation list updates.
We expect this represents a significatn savings because the revokation
list could update far less often than the full ring \ctx itself.
% perhaps corresponding with expiration checks
% especially since ring membership cannot be traced across site so easily.

We already trust an authority with issuing certificates, so we trust
them with managing therevokation list too.  As such, our revokation list
non-membership proofs merely requires proving adjacency of the revoked
public keys lexicographically before and after our own public key.
If the revokation list requires secrecy, then VRFs could hide its ordering,
similarly to NSEC5 \cite{nsec5}.


\subsection{Append only logs}

If an append only log grants ring membership, then a recursive SNARK
could validate ring membership with each recursive addition being
relatively inexpensive.

In this, we need a $\pipk^0$ similar to \pipk as well as a
$\pipk^n$ that proves some $\comring_{n-1}$ to be the ancestor of
its own $\comring_n$ and recursively proves some $\pifast^{n-1}$ with
its own $\comring_{n-1}$ and the same $\sk$.
We expect half pairing cycles fit this usage nicely, although they complicate provers.

Append only logs still depand \pipk be reproven whenever
\ctx updates however, so they only reduce bandwidth they not CPU usage.
We shy away from such append only log optimizations, due to this
prover complexity and our desire for revokation, but
 they remain an interesting corner of the design space.
