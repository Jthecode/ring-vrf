
\subsection{Pedersen rVRF-AD}

We instantiate a rVRF-AD from a hiding VRF-AD-KC like \PedVRF plus
a ring commitment scheme
 $\rVRF.\{ \CommitRing, \CommitKey, \OpenKey \}$
for which some zero-knowledge ring membership proof handles both
 $\PedVRF.\OpenKey$ and $\rVRF.\OpenKey$
efficiently.

\begin{itemize}
\item $\rVRF.\rSign : (\sk,\openring,\msg,\aux) \mapsto \sigma$ takes
 a secret key \sk, a ring opening \openring, a message \msg, and \aux, and then % auxiliary data
 % \begin{enumerate}
 % \item
 generates \openpk, computes a ring membership proof $\piring$
  $$ \piring = \NIZK \Setst{ \compk, \comring }{
  \exists \openpk,\openring \textrm{\ s.t.\ } 
  \genfrac{}{}{0pt}{}{\PedVRF.\OpenKey(\compk,\openpk) \quad}{\,\, = \rVRF.\OpenKey(\comring,\openring)}
  } $$
 % \item
 computes the VRF-AD-KC signature
  $$ \sigma = \PedVRF.\Sign(\sk,\openpk,\msg,\aux \doubleplus \compk \doubleplus \piring \doubleplus \comring), \quad\textrm{and} $$ % finally
 % \item
 returns the ring VRF signature $\rho = (\compk,\piring,\sigma)$.
 % \end{enumerate}
\item $\rVRF.\rVerify$ takes $(\comring,\msg,\aux,\rho)$,
 parses $\rho$ as $(\compk,\piring,\sigma,)$,  and then returns
 $$ \PedVRF.\Verify(\compk,\msg,\aux \doubleplus \compk \doubleplus \piring \doubleplus \comring,\sigma) $$
 iff $\NIZK.\Verify(\piring,\compk,\comring)$ succeeds. 
\end{itemize}

\begin{proposition}\label{prop:pedersen_rvrf}
$\rVRF$ satisfies ring uniqueness and ring unforgeability.
\end{proposition}







\endinput










% \noindent
In this, we tie $\sigma$ to $\piring$ by expanding $\sigma$'s auxiliary data with $\piring$.

% \smallskip

We now prove security of \rVRF using that
\begin{itemize}
	\item \PedVRF is a secure hiding VRF-AD-KC, and that
	\item our ring commitment scheme satisfies ring commitment correctness.
\end{itemize}

Pseudo-randomness holds by pseudo-randomness of \PedVRF from
Proposition \ref{prop:pedersen_vrf}.

\begin{proposition}\label{prop:pedersen_rvrf}
	$\rVRF$ satisfies ring uniqueness and ring unforgeability.
\end{proposition}

\begin{proof}[Proof sketch]
	???
\end{proof}


TODO:  Should we give an abstract pure NIZK instantiation here?  I think later probably.


