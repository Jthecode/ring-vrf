Verifiable random functions (VRFs) weld a pseudo-random function (PRF) into
a signature scheme, so that valid signatures prove correct PRF evaluations.
VRFs have diverse applications in consensus protocols, randomness
beacons, DNS via NSEC5, certificate transparency, and online games.

\smallskip

Anonymized {\em ring VRFs} are then ring signatures that similarly prove
correct evaluation of some authorized signer's PRF while hiding the
specific signer's identity within some set of possible signers,
 known as the ring.
% Ring VRFs integrate with distributed systems better, and provide more robust
% anonymity, than similar schemes like group VRFs or threshold protocols.

% Above necessary in the abstract?

\smallskip % 

We propose ring VRFs as a natural fulcrum around which a diverse array of
zkSNARK circuits turn, which thus makes ring VRFs an ideal target
for optimization and eventually standards. 

\smallskip

Among their diverse uses, we discuss first identity-like cases in which
ring VRF outputs act like pseudonyms provably unique within some domain or
context, specified via the VRF input, but unlinkable between domains. 
These unlinkable but unique pseudonyms provide an excellent balance between
user privacy and service provider or social interests \cite{pop2008}.
%
In fact, identity systems without unlinkability like OAuth wind up being
unethical when used outside specific narrow relationships.
% TODO: ``one person, one persona'' \cite{pop2008} https://bford.info/pub/net/sybil.pdf

\smallskip

We similarly discuss how ring VRFs make anonymously rationing or
rate limiting resource consumption to be vastly more efficent than
purchases via monitary transactions.

\smallskip %

As our core optimization tool, we explore {\em zero-knowledge continuations}
of rerandomizable zkSNARKs like \cite{Groth16} to amortize expensive
ring membership proofs across many ring VRF outputs. 
%
We implement a broadly usable ring VRF that amortizes an arbitrary
ring membership proofs, with a marginal prover cost
 of eight $\mathbb{E}_1$ and two $\mathbb{E}_2$ scalar multiplications.

\smallskip

In other words, our ring VRF might be first ring signature with asymptotic
performance competitive with group signatures.
%
We expect zero-knowledge continuations of rerandomizable zkSNARKs
thoroughly trounce recursive zkSNARK techniques indefinitely.

